\chapter{Introduction to the constructive gravity programme}

\section{The conception of general relativity}

\begin{itemize}
\item general relativity as closure of Maxwell electrodynamics
\item uniqueness (different derivations and proofs)
\item modified gravity, how this is compatible to uniqueness
\end{itemize}

\section{Modified gravity from refined matter theories}

\begin{itemize}
\item motto: matter tells gravity how to curve, gravity tells matter how to move -> lift to theory level
\item 2d plot of theories?
\item what could hint at the existence of refined matter theories?
\item dark matter
\end{itemize}

\section{Canonical and covariant approaches to constructive gravity}

\begin{itemize}
\item canonical approach: implementing the constraint algebra
\item a success story, but does it keep its promises?
\item covariant view: equivalent and complementary
\item simpler?
\end{itemize}

Two kinds of inconsistencies: Within the SMs -> quantum gravity inevitable? But also pheno that does not match SM -> closure?

We only ever so slightly deviate from the established models about matter and gravity. For example, where the standard model is restricted to field equations of second derivative, we keep this restriction. This is not because other efforts are not deemed worthwile---they certainly are, but should be explored \emph{ceteris paribus}, one at a time. Our focus lies on novel matter theories and their gravitational implications \emph{within the existing meta-theory of classical physics}.

\textbf{game plan: structure of the thesis}
