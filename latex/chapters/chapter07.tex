\chapter{Outlook: symmetry-reduced constructive gravity}\label{chapter_cosmo}

\textit{In the previous chapters, we reduced the complexity of the covariant construction procedure by considering a perturbative equivalent. Consequently, the result was an approximation of the exact gravitational theory, valid for sufficiently weak gravitational fields. A second approach towards reducing the complexity of the equivariance equations is symmetry reduction, which assumes that the gravitational field exhibits certain symmetries. Ideally, these symmetries bring the construction equations into a much simpler form. The solutions of the reduced equations should be theories of gravity valid in this symmetry-reduced sector, comparable to the Friedmann equations for spatially homogeneous and isotropic Einstein gravity. In this chapter, we explore a possible approach towards symmetry-reduced covariant constructive gravity by reducing the bundle on which the procedure operates. Our main result will be that the FLRW model can be recovered without the need to know the full Einstein-Hilbert Lagrangian beforehand. The corresponding area metric equivalent will not be solved, only the construction equations are derived. For a more in-depth study of symmetry reduction in the context of canonical constructive gravity, see Ref.~\cite{Duell_2020}.}

\section{The cosmological bundle}
For the purpose of developing a symmetry reduction strategy, let us consider the cosmological symmetry, which assumes spacetime to be spatially homogeneous and isotropic (see e.g.~\cite{Weinberg_1972,Wald_1984}). It is well known what this entails for the metric field: implicitly, this symmetry comes with the assumption of a sliced spacetime, i.e.\ $M \cong \mathbb R \times \Sigma$. In appropriate coordinates on the product manifold, the covariant metric tensor then reads \cite{Weinberg_1972,Katanaev_2016}
\begin{equation}
  g = dt \otimes dt - a(t)^2 \gamma_{\alpha\beta} dx^\alpha \otimes dx^\beta,
\end{equation}
where the spatial part is given by the positive \emph{scale factor} $a(t) > 0$ and a constant curvature metric $\gamma$ on $\Sigma$.\footnote{Restricting the topology of the spatial manifold $\Sigma$ to either $\mathbb R^3$ or $S^3$.}

More formally, we have a slicing $\phi\colon\mathbb R\times\Sigma\rightarrow M$, which induces embeddings
\begin{equation}
  \begin{aligned}
    \phi_\lambda\colon\Sigma\rightarrow {} & M \\
    p \mapsto {} & \phi_\lambda(p) \vcentcolon= \phi(\lambda,p)
  \end{aligned}
\end{equation}
of the spatial hypersurface $\Sigma$ into the spacetime manifold $M$. Each slicing introduces a time coordinate $t \vcentcolon= \pi_{\mathbb R} \circ \phi^{-1}$. The corresponding vector field $\partial_t$ defines the spatial and spatio-temporal components of the metric tensor by virtue of the conditions
\begin{equation}
  g(\partial_t,\partial_t) = 1\quad\text{and}\quad dt(X) = 0\Longrightarrow g(\partial_t,X) = 0.
\end{equation}
For the spatial components, we consider the pullback of the metric tensor onto the spatial slice $\Sigma$. This yields Riemannian 3-manifolds
\begin{equation}
  (\Sigma, \gamma_\lambda \vcentcolon= -\phi_\lambda^\ast g)
\end{equation}
that are of constant curvature. For simplicity, let us restrict to \emph{zero} curvature manifolds, such that the spatial volume is determined only by the scale factor
\begin{equation}
  a(\lambda) \vcentcolon= \sqrt{\operatorname{det} \gamma_\lambda}^\frac{1}{3}.
\end{equation}

From these insights, we define the \emph{cosmological bundle} for metric gravity.
\begin{definition}[metric cosmological bundle]
  The cosmological bundle over a manifold $M$ which captures the information of a spatially homogeneous and isotropic metric spacetime $(M,g)$ is defined as
  \begin{equation}
    TM \oplus_M \operatorname{Vol}^{\frac{1}{3}}(M),
  \end{equation}
  i.e.\ the sum of the tangent bundle and the bundle of densities with weight $\frac{1}{3}$.
\end{definition}
A similar definition can be given for the area metric bundle. It has been shown \cite{Duell_2020} that a spatially homogeneous and isotropic area metric manifold is determined by \emph{two} spatial degrees of freedom, which are a density-valued scale factor and a second scalar-valued factor.\footnote{This already follows quite intuitively from the $3+1$ decomposition in Sect.~\ref{sect_three_plus_one}. As opposed to metric gravity, where a single three-metric determines all spatial components, we now have \emph{two} spatial metrics and one tracefree endomorphism which parameterize the 17 spatial degrees of freedom. If the fields are to be isotropic, they must be given by two scale factors for the metrics; the tracefree endomorphism must be zero. Instead of working with two density-valued functions, it is more convenient to use a redefinition where one of the function becomes scalar-valued.} Consequently, the area metric cosmological bundle can be defined as follows.
\begin{definition}[area metric cosmological bundle]
  The cosmological bundle over a manifold $M$ which captures the information of a spatially homogeneous and isotropic area metric spacetime $(M,G)$ is defined as
  \begin{equation}
    TM \oplus_M \operatorname{Vol}^{\frac{1}{3}}(M) \oplus_M M\times\mathbb R,
  \end{equation}
  i.e.\ the sum of the tangent bundle, the bundle of $\frac{1}{3}$-densities, and the line bundle.
\end{definition}

\section{Recovering the FLRW model}
Having defined the metric cosmological bundle $TM \oplus_M \operatorname{Vol}^{\frac{1}{3}}(M)$, setting up the equivariance equations \eqref{equivariance_eqn_1}--\eqref{equivariance_eqn_4} is just a matter of deriving the Gotay-Marsden coefficients. For vector fields, Proposition \ref{prop_gmc_intertwiner} yields
\begin{equation}
  \gmc{a}{b}{n}{m} = \delta^a_m \delta^n_b,
\end{equation}
while $\frac{1}{3}$-densities transform according to the Gotay-Marsden coefficients
\begin{equation}
  \gmc{}{}{n}{m} = -\frac{1}{3} \delta^n_m.
\end{equation}
Using the right coordinates, the scale factor $a$ depends only on coordinate time, such that the only nonvanishing derivatives are $\dot a$. The vector field, called $U$, depends on all coordinates.

\newpage

\section{Construction equations for area metric cosmology}
\begin{itemize}
\item set up equivariance equations for area metric cosmological bundle
\item proof: matter causality is metric
\end{itemize}

