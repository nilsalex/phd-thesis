\chapter{Application: gravitational radiation from birefringent matter dynamics}
\label{chapter_weak_area}

\textit{So far, we have developed the general framework of covariant constructive gravity and derived a perturbative equivalent. A few examples illustrated the constructions, but the presentation focused on broad applicability to various geometries, without any specific bundle or matter theory in mind. In this chapter, we shift our focus and consider in depth the application of the framework to generalized linear electrodynamics, a birefringent generalization of Maxwell electrodynamics introduced in Chapter \ref{chapter_construction_algorithm}. Applying the perturbative construction procedure to third order yields gravitational field equations to second order. We will carefully analyze a $3+1$ split for the linear part of this theory and restrict to a certain sector with, in a very specific sense, physically sane phenomenology. Afterwards, we solve the two-body problem to first order and obtain the orbits of a binary system in area metric gravity. Building up on this solution, the second order of the field equations is used to derive the emission of gravitational radiation from the binary system and the radiative loss, which causes spin-up of the system. The binary star subject to area metric gravity turns out to exhibit qualitatively new behaviour as compared to Einstein gravity, e.g.~additional massive modes of gravitational radiation and a modification of Kepler's third law.}

\textit{To a large extent, the work presented in this chapter has been published as Ref.~\cite{Alex_2020_2}. The results on radiation loss are not part of this publication.}

\section{Construction of third-order area metric Lagrangians}\label{section_area_construction}
The matter theory in question is generalized linear electrodynamics (GLED) as defined in Def.~\ref{def_gled} with the Lagrangian density
\begin{equation*}
  L_\text{GLED} = \omega_G G^{abcd} F_{ab} F_{cd},
\end{equation*}
where we choose without loss of generality the scalar density
\begin{equation}\label{area_density}
  \omega_G = \left(\frac{1}{24}\epsilon_{abcd}G^{abcd}\right)^{-1}.
\end{equation}
The principal polynomial of GLED is quartic and takes the form
\begin{equation}
  \mathcal P_\text{GLED}(k) = -\frac{1}{24} \omega_G^2 \epsilon_{mnpq} \epsilon_{rstu} G^{mnra} G^{bpsc} G^{dqtu} k_a k_b k_c k_d.
\end{equation}
As appropriate Lorentz invariant expansion point constructed from the Minkowski metric $\eta$, we already determined in Example \ref{example_expansion_points}
\begin{equation}\label{area_expansion_point}
  N^A = J_{abcd}^A (\eta^{ac} \eta^{bd} - \eta^{ad} \eta^{bc} + \epsilon^{abcd}).
\end{equation}
Before solving the system of equivariance equations perturbatively around $N$, let us reconsider the reduced power series ansatz \eqref{ansatz_reduced}. In addition to dropping terms with a total number of derivatives that is odd or greater than 2, and dropping non-Lorentz invariant expansion coefficients, we can also discard the linear term $a_A H^A$. This term would yield a constant in the Euler-Lagrange equations, causing the flat expansion point $N$ to no longer constitute a solution to the vacuum field equations. However, the perturbation ansatz stipulates that we perturb around a solution of the field equations. Since it is obvious that Equation \eqref{prolong_0} implies from vanishing coefficients $a_A$ that also the coefficient $a$ vanishes, we readily drop both and make the further reduced ansatz
\begin{equation}\label{ansatz_area_gravity}
  \begin{aligned}
    L &{} = a_A^{\hphantom AI} H^A_{\hphantom AI} \\
      &{} \hphantom{=} + a_{AB} H^A H^B + a_{A\hphantom pB}^{\hphantom Ap\hphantom Bq} H^A_{\hphantom Ap} H^B_{\hphantom Bq} + a_{AB}^{\hphantom{AB}I} H^A H^B_{\hphantom BI} \\
      &{} \hphantom{=} + a_{ABC} H^A H^B H^C + a_{AB\hphantom pC}^{\hphantom{AB}p\hphantom Cq} H^A H^B_{\hphantom Bp} H^C_{\hphantom Cq} + a_{ABC}^{\hphantom{ABC}I} H^A H^B H^C_{\hphantom CI} \\
      &{} \hphantom{=} + \mathcal O(H^4).
  \end{aligned}
\end{equation}

\subsection{Solving axiom 1}\label{sect_area_lagrangian}
Step 1 of the perturbative construction algorithm \ref{perturbative_algorithm} consists in computing the Gotay-Marsden coefficients for the gravitational bundle. For area metric gravity, we found in Sect.~\ref{section_gled}
\begin{equation}
  \gmc{A}{B}{n}{m} = 4 I^{pqrn}_B J^A_{pqrm},
\end{equation}
which followed from the general result \eqref{gmc_contra} for purely contravariant tensor bundles.

Proceeding with step 2, we need to construct a basis for the Lorentz invariant expansion coefficients
\begin{equation}
  (a_A^{\hphantom AI}, a_{AB}, a_{A\hphantom pB}^{\hphantom Ap\hphantom Bq}, a_{AB}^{\hphantom{AB}I}, a_{ABC}, a_{AB\hphantom pC}^{\hphantom{AB}p\hphantom Cq}, a_{ABC}^{\hphantom{ABC}I})
\end{equation}
in the ansatz \eqref{ansatz_area_gravity}. This task is solved using the Haskell library \texttt{sparse-tensor} \cite{Reinhart_2019_sparse-tensor} discussed in Chapter \ref{}. The result is a basis of dimension 237, enumerated in full in Appendix \ref{} and summarized in Table \ref{}. It should be emphasized that the requirement of Lorentz invariance, which is not a direct stipulation but follows via the equivariance equations from a physically motivated assumption about the expansion point, drastically reduces the dimensionality of the ansatz from
\begin{equation}
  210 + \frac{21\cdot 22}{2} + 21\cdot 210 + \frac{84\cdot 85}{2} + \frac{21\cdot 22\cdot 23}{6} + \frac{21\cdot 22}{2} \cdot  210 + 21\cdot\frac{84\cdot 85}{2} = 133672
\end{equation}
to only 237. In principle, the correctness of the ansatz can be verified by showing that it is the most generic solution to the ansatz equations \ref{}. All we have to show is that the dimensionality of the ansatz equals the corank of the linear system of ansatz equations. For the ansatz including third-order coefficients, the system is quite large---considering that the coefficient space is already of dimension 133672---such that, on standard hardware, the rank cannot be computed naively by storing the matrix in memory and using methods like singular value decomposition or fraction-free gaussian elimination. It is rather easy, however, to use the aforementioned methods and work out the corank of the linear system determining the Lorentz invariant ansatz coefficients to second order, as the dimension of this ansatz space is only $210 + \frac{21\cdot 22}{2} + 21 \cdot 210 + \frac{84\cdot 85}{2} = 8421$. Confirming the number of obtained basis ansätze up to second order, the corank of the corresponding system is indeed 40. The computer code for this computation is published as Ref.~\cite{Alex_2020_area-metric-gravity}.
\begin{table}
  \centering
  \begin{tabular}{l r r}
    \toprule
    coefficient & dimension & gravitational constants \\
    \midrule
    $a_A^{\hphantom AI}$ & 3 & $(e_{38},\dots ,e_{40})$ \\ \addlinespace[2pt]
    $a_{AB}$ & 6 & $(e_{1},\dots ,e_{6})$ \\ \addlinespace[2pt]
    $a_{A\hphantom pB}^{\hphantom Ap\hphantom Bq}$ & 15 & $(e_{7},\dots ,e_{21})$ \\ \addlinespace[2pt]
    $a_{AB}^{\hphantom{AB}I}$ & 16 & $(e_{22},\dots ,e_{37})$ \\ \addlinespace[2pt]
    $a_{ABC}$ & 15 & $(e_{41},\dots ,e_{55})$ \\ \addlinespace[2pt]
    $a_{AB\hphantom pC}^{\hphantom{AB}p\hphantom Cq}$ & 110 & $(e_{56},\dots ,e_{165})$ \\ \addlinespace[2pt]
    $a_{ABC}^{\hphantom{ABC}I}$ & 72 & $(e_{166},\dots ,e_{237})$ \\ \addlinespace[2pt]
    \bottomrule 
  \end{tabular}
  \caption{Summary of the Lorentz invariant expansion coefficients for the area metric gravity ansatz \eqref{ansatz_area_gravity} obtained from the Haskell library \texttt{sparse-tensor} \cite{}. The dimension is the number of linearly independent basis tensors returned from the computer program. Assigning labels from $1$ to $237$ to all basis tensors, an ansatz is represented by real numbers $e_1\dots e_{237}$ using its unique basis decomposition. These numbers parameterize the gravitational theory and are thus referred to as \emph{gravitational constants}. For a complete picture of the decomposition of ansätze using basis tensors, refer to Appendix \ref{} or the computer code in Ref.~\cite{Alex_2020_area-metric-gravity}.}
\end{table}

With the 237 ansatz coefficients at hand, solving the equivariance equations as required for step 5 is only a matter of inserting the ansatz in the system and its first two prolongations as displayed in Eqns.~\ref{prolong_0}--\ref{prolong_2}, extracting a system of linear equations for the gravitational constants, and solving this system. This task is again performed using efficient computer algebra, implemented in the Haskell library \texttt{safe-tensor}, which is introduced in Chapter \ref{}. The procedure is roughly as follows: A compatibility layer with \texttt{sparse-tensor} is used in order to construct the ansatz tensors and make them available as \texttt{Tensor} types with generalized rank (see Sect.~\ref{}). Together with predefined tensors like Kronecker deltas, intertwiners, Gotay-Marsden coefficients, or the Minkowski metric, the ansatz tensors are used in order to construct the (prolonged) equivariance equations evaluated at $N$ (Eqns.~\ref{prolong_0}--\ref{prolong_2}). Each tensorial equation is a value of type \texttt{Tensor} and, as such, can be evaluated into a list of its components. Every component is a linear equation for the 237 gravitational constants. Collecting all components for all tensorial equations, we obtain a matrix representing the linear system for the constants $e_1\dots e_{237}$. The system is small enough to be brought into reduced row echelon form applying fraction-free gaussian elimination and backward substitution using 64-bit integers\footnote{Exploiting the observation we made earlier that, using intertwiners with purely rational components, all coefficients in the system remain rational.}, which yields a solution that parameterizes the constants with a few remaining indeterminate gravitational constants. As an example for the process, let us walk through the solution for the linear expansion coefficient $a_A^{\hphantom AI}$.

\begin{example}[solution of the equivariance equations to first order]
  Having set $a_A = 0$, the remaining expansion coefficient for the linear order is $a_A^{\hphantom AI}$, which is determined in part by the second unprolonged equation \eqref{prolong_0}. A suitable basis for this coefficient is
  \begin{equation}
    a_A^{\hphantom AI} = J_A^{abcd} J_{pq}^I\lbrack e_1 \cdot \eta_{ac} \eta_{bd} \eta^{pq} + e_2 \cdot \eta_{ac} \delta_b^p \delta_d^q + e_3 \cdot \epsilon_{abcd} \eta^{pq} \rbrack
  \end{equation}
  with three gravitational constants $e_1,e_2,e_3$. Inserting this ansatz into the unprolonged equation
  \begin{equation}
    0 = a_A^{\hphantom AI} \gmc{A}{B}{(n}{m} J^{pq)}_I N^B =\vcentcolon T^{npq}_m
  \end{equation}
  yields a tensorial equation $0 = T^{npq}_m$ with 256 components. Each component is of the form
  \begin{equation}
    0 = c_1\cdot e_1  + c_2\cdot e_2 + c_3\cdot e_3.
  \end{equation}
  The collection of all components is a system of 256 linear equations for three variables. A lot of these equations are redundant, because they are trivial or linearly dependent. A naive reduction by eliminating trivial equations and choosing only one representative for equations that are multiples of each other already reduces the system to the single equation
  \begin{equation}
    0 = 2 e_1 + e_2 + 4 e_3.
  \end{equation}
  Setting e.g.~$e_2 = -2 e_1 -4 e_3$ solves the equivariance equation for the coefficient $a_A^{\hphantom AI}$, leaving it parameterized by two gravitational constants $e_1$ and $e_3$.
\end{example}

Applied to the whole system of equivariance equations, we obtain a parameterization of the solution (displayed in full in Appendix \ref{}) by 50 independent gravitational constants. A subset of 16 constants governs \emph{linearized} area metric gravity via the quadratic Lagrangian density, from which---as we will encounter later---only 11 independent linear combinations play a role for the Euler-Lagrange equations. The procedure outlined here is implemented in Haskell using the library \texttt{sparse-tensor} for ansatz generation as well as the library \texttt{safe-tensor} for constructing and solving the equivariance equations, with the source code and results published as Ref.~\cite{Alex_2020_area-metric-gravity}.

\subsection{Solving axiom 2}
The pedestrian approach towards implementing causal compatibility of the just constructed gravitational theory with GLED is to carefully execute steps 6--12 of the perturbative construction algorithm. This way, we obtain an approximation of the area metric gravity principal polynomial and have to match the causal structure with a first-order expansion of the GLED principal polynomial. While entirely feasible, this approach is less illustrative than the \emph{constructive} approach we employ instead. The underlying realization behind this technique is that the diffeomorphism invariance of the gravitational theory dramatically restricts the possible principal polynomials. In fact, we will see that for third-order area metric Lagrangians, the admissible principal polynomials are already causally compatible with the corresponding expansion of the GLED polynomial. \emph{There is no causality mismatch left to be fixed.}

To this end, recall the GLED polynomial \eqref{gled_polynomial}, which using the scalar density \eqref{area_density} assumes the form
\begin{equation}
  \mathcal P_\text{GLED}(k) = -\frac{1}{\frac{1}{24}(\epsilon_{abcd}G^{abcd})^2} \epsilon_{mnpq} \epsilon_{rstu} G^{mnra} G^{bpsc} G^{dqtu} k_a k_b k_c k_d.
\end{equation}
Expanding this expression to linear order in the perturbation yields
\begin{equation}\label{gled_poly_first_order}
  \begin{aligned}
    \mathcal P_\text{GLED}(k) &{} = \left\{ \left\lbrack 1 - \frac{1}{24} \epsilon(H) \right\rbrack \eta(k,k) + \frac{1}{2} H(k,k) \right\}^2 + \mathcal O(H^2) \\
                              &{} = \lbrack P^{(\leq 1)}_\text{GLED} \rbrack^2 + \mathcal O(H^2),
  \end{aligned}
\end{equation}
where the abbreviations
\begin{equation}
  \epsilon(H) = \epsilon_{abcd} H^{abcd}\quad\text{and}\quad H(k,k) = \eta_{ac} H^{abcd} k_b k_d
\end{equation}
have been introduced. In the following, we will also make use of the contraction
\begin{equation}
  \eta(H) = \eta_{ac} \eta_{bd} H^{abcd}.
\end{equation}
Up to first order, we find that the GLED polynomial factors into the square of a metric polynomial $P^{(\leq 1)}_\text{GLED}$. This has a remarkable consequence: For weak gravitational fields, where the approxmiation to first order is sufficiently good, the physics of point particles adhering to GLED dynamics is indistinguishable from the Maxwellian setting with a metric perturbation $h$ by virtue of the identification
\begin{equation}
  h^{ab} = \left\lbrack 1 - \frac{1}{24} \epsilon(H) \right\rbrack \eta^{ab} + \frac{1}{2} \eta_{cd}H^{acbd} = (P^{(\leq 1)}_\text{GLED})^{ab}.
\end{equation}
This effect only holds in the limit of geometric optics---the GLED field equations do \emph{not} reduce to Maxwell equations with a metric perturbation. Consequently, even to first order in the area metric perturbation, nonmetric effects can be observed. An in-depth study of classical and quantum electrodynamics on weakly birefringent backgrounds based on exactly this realization has been conducted in Ref.~\cite{GrosseHolz_2017}.

We will now proceed to show that the possible principal polynomials arising from third-order area metric gravity Lagrangians as constructed in the previous section are only mildly more general than the effectively quadratic first-order GLED polynomial \eqref{gled_poly_first_order}. This issue is approached by first considering the corresponding Euler-Lagrange equations.

\begin{proposition}\label{prop_euler_tensor}
  Let $E\overset{\pi}{\longrightarrow}M$ be a sub-bundle of some tensor bundle over $M$. Consider a Lagrangian field theory on $J^2\pi$ that is degenerate in the sense that the Euler-Lagrange equations are of second derivative order, i.e.~are also defined on $J^2\pi$. If the Lagrangian field theory is diffeomorphism invariant with respect to the diffeomorphism action on the second jet bundle, it follows that the Euler-Lagrange equations are diffeomorphism equivariant. In particular, a local representation of the Euler-Lagrange equations
  \begin{equation}\label{euler_lagrange_local_repeat}
  E_A = L_{:A} - D_p L_{:A}^{\hphantom{:A}p} + I_I^{pq} D_p D_q L_{:A}^{\hphantom{:A}I}
  \end{equation}
  exhibits the transformation behaviour
  \begin{equation}\label{trafo_euler_lagrange}
    \delta_\xi E_A = -E_A \xi^m_{,m} - E_B \gmc{B}{A}{n}{m} \xi^m_{,n},
  \end{equation}
  where $\gmc{B}{A}{n}{m}$ are the Gotay-Marsden coefficients corresponding to the field bundle. In other words, the Euler-Lagrange equations transform as tensor density of weight 1.
\end{proposition}
\begin{proof}
  The claim follows from expanding the left hand side of Eq.~\ref{trafo_euler_lagrange} as
  \begin{equation}
    \delta_\xi E_A = E_{A:B} \delta_\xi u^B + E_{A:B}^{\hphantom{A:B}p} \delta_\xi u^B_{\hphantom Bp} + E_{A:B}^{\hphantom{A:B}I} \delta_\xi u^B_{\hphantom BI},
  \end{equation}
  then replacing $E_A$ with its definition \eqref{euler_lagrange_local_repeat} and simplifying the result using the equivariance of the Lagrangian density $L$. Rather than performing this tedious calculation, we can alternatively consider the geometric definition \ref{global_lagrange_form} of the Euler-Lagrange form and deduce that it must transform covariantly (for a contravariant tensor bundle) with density weight of one, i.e.~according to the local expression \eqref{trafo_euler_lagrange}.
\end{proof}
This transformation behaviour carries over to the principal symbol of the Euler-Lagrange equations, which is also a tensor density of weight 1.
\begin{proposition}
  Consider the same Lagrangian field theory as in Prop.~\ref{prop_euler_tensor}. The principal symbol
  \begin{equation}\label{trafo_symbol}
    T_{AB}(k) = E_{A:B}^{\hphantom{A:B}I} J_I^{pq} k_p k_q
  \end{equation}
  of the corresponding Euler-Lagrange equations $E_A$, where $k \in T^\ast M$ denotes a covector, transforms as a tensor density of weight one, i.e.~an infinitesimal diffeomorphism acts as
  \begin{equation}
    \delta_\xi T_{AB}(k) = - T_{AB}(k) \xi^m_{,m} - T_{CB}(k) \gmc{C}{A}{n}{m} \xi^m_{,n} - T_{AC}(k) \gmc{C}{B}{n}{m} \xi^m_{,n}.
  \end{equation}
\end{proposition}
\begin{proof}
  The idea of the proof is as before: We insert the just proven transformation behaviour of the Euler-Lagrange equations $E_A$ and of covectors $k$, which is
  \begin{equation}
    \delta_\xi k_a = - k_m \xi^m_{,a},
  \end{equation}
  into the transformation
  \begin{equation}
    \begin{aligned}
      \delta_\xi T_{AB}(k) &{} = (T_{AB}(k))_{:C} \delta_\xi u^C + (T_{AB}(k))_{:C}^{\hphantom{:C}p} \delta_\xi u^C_{\hphantom Cp} + (T_{AB}(k))_{:C}^{\hphantom{:C}I} \delta_\xi u^C_{\hphantom CI} \\
                           &{} \hphantom{=} + \frac{\partial T_{AB}}{\partial k_a}(k) \delta_\xi k_a.
    \end{aligned}
  \end{equation}
  This time, the calculation is rather trivial and the claim \eqref{trafo_symbol} follows almost immediately.
\end{proof}
We are now in a position to prove the first part of the central result, which is that the principal polynomial of area metric gravity is a scalar density. Note that we restrict our considerations to the case of a principal symbol that is independent from the derivatives of the derivatives of the gravitational field, as otherwise the causality could not be matched anyway (see Sect.~\ref{section_axiom2_perturb}).
\begin{theorem}\label{thm_poly_trafo}
  Let $\pi$ be the area metric bundle. Conside a degenerate Lagrangian field theory with a principal symbol that is independent from the derivatives of the area metric field. The principal polynomial $\mathcal P(k)$ corresponding to the symbol, as defined in Def.~\ref{def_principal_polynomial} is a scalar density of weight 23, i.e.~transforms locally under infinitesimal spacetime diffeomorphisms as\footnote{Here, we correct a numerical---but inconsequential---error in the calculation presented in Ref.~\cite{Alex_2020_2}, where the factor was calculated to be 57 instead of 23.}
  \begin{equation}
    \delta_\xi \mathcal P(k) = -23 \cdot \mathcal P(k) \xi^m_{,m}.
  \end{equation}
\end{theorem}
\begin{proof}
  From the transformation behaviour of the area metric field and covectors, it follows that an infinitesimal diffeomorphism acts on generators $\chi^A_{(i)}(k) = \gmc{A}{B}{n}{i} u^B k_n$ of gauge transforms as
  \begin{equation}
    \delta_\xi \chi^A_{(i)}(k) = \gmc{A}{B}{n}{m} \chi^B_{(i)}(k) \xi^m_{,n} - \chi^A_{(m)}(k)\xi^m_{,i}.
  \end{equation}
  Now calculating the transformation behaviour of the principal polynomial numerator $Q^{(A_1\dots A_4)(B_1\dots B_4)}$ (dropping the covector $k$ from the notation) we obtain
  \begin{equation}\label{Q_trafo}
    \begin{aligned}
      \delta_\xi Q^{(A_1\dots A_4)(B_1\dots B_4)} &{} = \delta_\xi \frac{\partial^4 \operatorname{det} T}{\partial T_{A_1B_1}\dots\partial T_{A_4B_4}} \\
                                                  &{} = \delta_\xi \left\lbrack \frac{4}{21!} \epsilon^{A_1\dots A_{21}} \epsilon^{B_1\dots B_{21}} T_{A_5B_5} \dots T_{A_{21}B_{21}}\right\rbrack \\
                                                  &{} = \frac{4\cdot 17}{21!} \epsilon^{A_1\dots A_{21}} \epsilon^{B_1\dots B_{21}} \lbrack\delta_\xi T_{A_5B_5}\rbrack T_{A_6B_6}\dots T_{A_{21}B_{21}} \\
                                                  &{} = -17\cdot \delta_\xi Q^{(A_1\dots A_4)(B_1\dots B_4)} \xi^m_{,m} \\
                                                  &{} \hphantom{=} -\frac{4\cdot 17}{21!} \epsilon^{A_1\dots A_{21}} \gmc{A}{A_5}{n}{m} \epsilon^{B_1\dots B_{21}} T_{AB_5} \dots T_{A_{21}B_{21}} \xi^m_{,n} \\
                                                  &{} \hphantom{=} -\frac{4\cdot 17}{21!} \epsilon^{A_1\dots A_{21}} \epsilon^{B_1\dots B_{21}} \gmc{B}{B_5}{n}{m} T_{A_5B} \dots T_{A_{21}B_{21}} \xi^m_{,n}.
    \end{aligned}
  \end{equation}
  This is further simplified using the identity $0 = \epsilon^{\lbrack A_1\dots A_{21}} X^{A\rbrack\dots}$, from which we derive after a few index relabellings
  \begin{equation}
    \begin{aligned}
      0 &{} = 22 \cdot \epsilon^{\lbrack A_1\dots A_{21}} \gmc{A\rbrack}{A_5}{n}{m} \epsilon^{B_1\dots B_21} T_{AB_5}T_{A_6B_6}\dots T_{A_{21}B_{21}} \xi^m_{,n} \\
        &{} = 17 \cdot \epsilon^{A_1\dots A_{21}} \gmc{A}{A_5}{n}{m} \epsilon^{B_1\dots B_{21}} T_{AB_5}T_{A_6B_6}\dots T_{A_{21}B_{21}} \xi^m_{,n} \\
        &{} \hphantom{=} - \gmc{A}{A}{n}{m} \epsilon^{A_1\dots A_{21}}\epsilon^{B_1\dots B_{21}} T_{A_5B_5}\dots T_{A_{21}B_{21}} \xi^m_{,n} \\
        &{} \hphantom{=} + \epsilon^{AA_2A_3A_4\dots A_{21}} \gmc{A_1}{A}{n}{m} \epsilon^{B_1\dots B_{21}} T_{A_5B_5}T_{A_6B_6}\dots T_{A_{21}B_{21}} \xi^m_{,n} \\
        &{} \hphantom{=} + \epsilon^{A_1AA_3A_4\dots A_{21}} \gmc{A_2}{A}{n}{m} \epsilon^{B_1\dots B_{21}} T_{A_5B_5}T_{A_6B_6}\dots T_{A_{21}B_{21}} \xi^m_{,n} \\
        &{} \hphantom{=} + \epsilon^{A_1A_2AA_4\dots A_{21}} \gmc{A_3}{A}{n}{m} \epsilon^{B_1\dots B_{21}} T_{A_5B_5}T_{A_6B_6}\dots T_{A_{21}B_{21}} \xi^m_{,n} \\
        &{} \hphantom{=} + \epsilon^{A_1A_2A_3A\dots A_{21}} \gmc{A_4}{A}{n}{m} \epsilon^{B_1\dots B_{21}} T_{A_5B_5}T_{A_6B_6}\dots T_{A_{21}B_{21}} \xi^m_{,n}. \\
    \end{aligned}
  \end{equation}
  Applying the same technique to the index set $[B_1\dots B_{21} B]$ and carrying out the contraction $\gmc{A}{A}{n}{m} = 4\delta^{n}_{m}$, the identity can be applied to the second and third terms in Eq.~\ref{Q_trafo}, such that we finally obtain
  \begin{equation}
    \begin{aligned}
      \delta_\xi Q^{(A_1\dots A_4)(B_1\dots B_4)} &{} = -25 \cdot Q^{(A_1\dots A_4)(B_1\dots B_4)} \xi^m_{,m} \\
                                                  &{} \hphantom{=} + \gmc{A_1}{A}{n}{m} Q^{(AA_2A_3A_4)(B_1\dots B_4)} \xi^m_{,n} + \gmc{A_2}{A}{n}{m} Q^{(A_1AA_3A_4)(B_1\dots B_4)} \xi^m_{,n} \\
                                                  &{} \hphantom{=} + \gmc{A_3}{A}{n}{m} Q^{(A_1A_2AA_4)(B_1\dots B_4)} \xi^m_{,n} + \gmc{A_4}{A}{n}{m} Q^{(A_1A_2A_3A)(B_1\dots B_4)} \xi^m_{,n} \\
                                                  &{} \hphantom{=} + \gmc{B_1}{B}{n}{m} Q^{(A_1\dots A_4)(BB_2B_3B_4)} \xi^m_{,n} + \gmc{B_2}{B}{n}{m} Q^{(A_1\dots A_4)(B_1BB_3B_4)} \xi^m_{,n} \\
                                                  &{} \hphantom{=} + \gmc{B_3}{B}{n}{m} Q^{(A_1\dots A_4)(B_1B_2BB_4)} \xi^m_{,n} + \gmc{B_4}{B}{n}{m} Q^{(A_1\dots A_4)(B_1B_2B_3B)} \xi^m_{,n}.
    \end{aligned}
  \end{equation}
  A similar calculation, this time using the identity $0 = \epsilon^{\lbrack a_1a_2a_3a_4} X^{a\rbrack\dots}$, yields the transformation of the denominator $f^{(A_1\dots A_4)(B_1\dots B_4)}$,
  \begin{equation}
    \begin{aligned}
      \delta_\xi f^{(A_1\dots A_4)(B_1\dots B_4)} &{} = \delta_\xi\left\lbrack \epsilon^{a_1\dots a_4} \epsilon^{b_1\dots b_4} \prod_{i=1}^4 \chi^{A_i}_{(a_i)} \chi^{B_i}_{(b_i)} \right\rbrack \\
                                                  &{} = -2 \cdot f^{(A_1\dots A_4)(B_1\dots B_4)} \xi^m_{,m} \\
                                                  &{} \hphantom{=} + \gmc{A_1}{A}{n}{m} f^{(AA_2A_3A_4)(B_1\dots B_4)} \xi^m_{,n} + \gmc{A_2}{A}{n}{m} f^{(A_1AA_3A_4)(B_1\dots B_4)} \xi^m_{,n} \\
                                                  &{} \hphantom{=} + \gmc{A_3}{A}{n}{m} f^{(A_1A_2AA_4)(B_1\dots B_4)} \xi^m_{,n} + \gmc{A_4}{A}{n}{m} f^{(A_1A_2A_3A)(B_1\dots B_4)} \xi^m_{,n} \\
                                                  &{} \hphantom{=} + \gmc{B_1}{B}{n}{m} f^{(A_1\dots A_4)(BB_2B_3B_4)} \xi^m_{,n} + \gmc{B_2}{B}{n}{m} f^{(A_1\dots A_4)(B_1BB_3B_4)} \xi^m_{,n} \\
                                                  &{} \hphantom{=} + \gmc{B_3}{B}{n}{m} f^{(A_1\dots A_4)(B_1B_2BB_4)} \xi^m_{,n} + \gmc{B_4}{B}{n}{m} f^{(A_1\dots A_4)(B_1B_2B_3B)} \xi^m_{,n}.
    \end{aligned}
  \end{equation}
  Putting both numerator and denominator together proves the claim
  \begin{equation}
    \delta_\xi \mathcal P(k) = -23 \cdot \mathcal P(k) \xi^m_{,m}.
  \end{equation}
\end{proof}
An equivalent formulation of the fact that $\mathcal P(k)$ is a density of weight 23 is that the symmetric coefficients\footnote{Recall that the principal polynomial for area metric gravity is homogeneous and of degree 26.} $P^{a_1\dots a_{26}}$ constitute a tensor density of the same weight, i.e.~live of the bundle of symmetric tensor densities of contravariant rank 26 with weight 23. For this geometry, the equivariance equations on the ``zeroth jet bundle'' (since the polynomial must not depend on derivatives of the geometry) are
\begin{equation}\label{polynomial_equivariance_eqns}
  \begin{aligned}
    P^{a_1\dots a_{26}}_{\hphantom{a_1\dots a_{26}},m} &{} = 0, \\
    P^{a_1\dots a_{26}}_{\hphantom{a_1\dots a_{26}}:A} \gmc{A}{B}{n}{m} u^B &{} = - 23\cdot P^{a_1\dots a_{26}} \delta^n_m + 26\cdot P^{n(a_1\dots a_{25}} \delta^{a_{26})}_m.
  \end{aligned}
\end{equation}

The second part of the central result follows from these equations. All we have to do is construct the perturbative solution to first order and see that it is impossible \emph{not} to have the causality match GLED causality to the same order.

\begin{theorem}
  Let $\mathcal P_\text{area}$ be the principal polynomial of area metric gravity as considered in Thm.~\ref{thm_poly_trafo}. To first order in the expansion $G = N + H$ of the area metric field, where $N$ is the Lorentz invariant expansion point \eqref{area_expansion_point}, $\mathcal P_\text{area}$ is equivalent to the GLED principal polynomial $\mathcal P_\text{GLED}$ in the sense that
  \begin{equation}\label{eq_thm_poly}
    \mathcal P_\text{area} = \lbrack \omega P_\text{GLED}^{(\leq 1)}\rbrack^{13} + \mathcal O(H^2),
  \end{equation}
  where $\omega$ denotes a density of weight $\frac{23}{13}$ on the area metric bundle and $P_\text{GLED}^{(\leq 1)}$ is the expansion of the GLED polynomial to first order. In particular, to first order in the perturbation, both principal polynomials describe the same null surfaces and hyperbolicity cones.
\end{theorem}
\begin{proof}
  Knowing that the principal polynomial of area metric gravity transforms as a density of weight 23, we can construct possible candidates by solving the equivariance equations \eqref{polynomial_equivariance_eqns}. To this end, we make the ansatz
  \begin{equation}\label{poly_ansatz}
    \begin{aligned}
      \mathcal P_\text{area}(k) &{} = \eta(k,k)^{13} \\
                                &{} \hphantom{=} + A\cdot \epsilon(H) \eta(k,k)^{13} + B\cdot \eta(H) \eta(k,k)^{13} + C\cdot H(k,k)\eta(k,k)^{12} \\
                                &{} \hphantom{=} + \mathcal O(H^2).
    \end{aligned}
  \end{equation}
An overall factor would be irrelevant, so it has already been dropped when setting the coefficient of the constant term to 1. The generality of the ansatz can, as always, be verified by calculating the corank of the ansatz equations, which will yield 4---the number of ansatz tensors in Eq.~\eqref{poly_ansatz}. Evaluating the equivariance equation at the ansatz and contracting the 26 symmetric indices with covector components, for the sake of a cleaner presentation, yields an equation where we can cancel a common factor of $\eta(k,k)^{(12)}$. The remaining equation has a covariant and a contravariant spacetime index, such that a decomposition into the trace
\begin{equation}
  0 = \lbrack 24 A + 12 B + 3 C + 23 - \frac{13}{2}\rbrack \delta^n_m
\end{equation}
and the tracefree part
\begin{equation}
  0 = \lbrack 4 C - 26\rbrack [\eta^{na}\delta^b_m k_a k_b - \frac{1}{4} \delta^n_m\eta(k,k)]
\end{equation}
lends itself for a first attempt in order to retrieve scalar equations for the system. As it turns out, these two equations are already maximal. Parameterizing the solution with $B$ yields
\begin{equation}\label{general_area_metric_gravity_poly}
  \begin{aligned}
    \mathcal P_\text{area}(k) &{} = \eta(k,k)^{13} \\
                              &{} \hphantom{=} - \frac{3}{2} \epsilon(H)\eta(k,k)^{13} + B(\eta(H)-\frac{1}{2}\epsilon(H))\eta(k)^{13} + \frac{13}{2} H(k,k)\eta(k,k)^{12} \\ 
                              &{} \hphantom{=} + \mathcal O(H^2) \\
                              &{} = \left\{ \left\lbrack 1 - \frac{3}{2\cdot 13} \epsilon(H) + \frac{B}{13}\left(\eta(H) - \frac{1}{2}\epsilon(H) \right)\right\rbrack \eta(k,k) + \frac{1}{2} H(k,k) \right\}^{13} \\
                              &{} \hphantom{=} + \mathcal O(H^2),
  \end{aligned}
\end{equation}
where for the last equality we completed the thirteenth power as
\begin{equation}
  1 + \epsilon = \left( 1 + \frac{1}{13}\epsilon\right)^{13} + \mathcal O(\epsilon^2).
\end{equation}

In order to relate the quadratic polynomial that determines the first order of $\mathcal P_\text{area}(k)$ to $\mathcal P_\text{GLED}^{(\leq 1)}$ via a scalar density, as claimed in Eq.~\eqref{eq_thm_poly}, we consider the equivariance equations
\begin{equation}
  \begin{aligned}
    \omega_{,m} &{} = 0, \\
    \omega_{:A} \gmc{A}{B}{n}{m} u^B &{} = -\frac{23}{13}\,\omega \delta^n_m
  \end{aligned}
\end{equation}
for such a density $\omega$ of weight $\frac{23}{13}$. This time, the Lorentz invariant ansatz is just
\begin{equation}
  \omega = 1 + A\cdot \epsilon(H) + B\cdot \eta(H) + \mathcal O(H^2)
\end{equation}
and reduces the equivariance equations to the single condition
\begin{equation}
  24A + 12B = -\frac{23}{13},
\end{equation}
such that the most general scalar density of weight $\frac{23}{13}$ is to first order given by
\begin{equation}
  \omega = 1 - \frac{23}{13\cdot 24} \epsilon(H) + B \lbrack \eta(H) - \frac{1}{2} \epsilon(H)\rbrack + \mathcal O(H^2).
\end{equation}
The result now follows from multiplication of $\mathcal P_\text{area}^{(\leq 1)}$ with $\omega$, which yields exactly the area metric gravity polynomial \eqref{general_area_metric_gravity_poly}. To first order, the principal polynomial of area metric gravity is determined by a quadratic polynomial which reduces to the quadratic first-order GLED polynomial \emph{up to a factor}. Because such an overall factor is irrelevant for vanishing sets and hyperbolicity cones, the polynomials must be considered identical for the purpose of comparing their causal structure.
\end{proof}

Having fixed the causality of third-order perturbative area metric gravity---by proof, rather than by explicit calculation---the construction procedure up to this order is completed. Third-order area metric gravity\footnote{With second-order field equations and, therefore, a principal polynomial of first order.} is determined by the ansatz \eqref{ansatz_area_gravity} which is constructed from the Lorentz-invariant basis tensors \eqref{???}. From the 237 gravitational constants---the coefficients in the basis expansion---50 constants turn out to be independent, 11 of which govern the linearized field equations. The relations between gravitational constants are collected in Appendix \ref{}. In the following, we will examine the linear theory, which forms the basis for predicting first-order and, later on, second-order effects of area metric gravity.

\subsection{3+1 decomposition}
\label{sect_three_plus_one}
As remarked in Sect.~\ref{section_gled}, the expansion point should be an area metric of a certain subclass in order to guarantee hyperbolicity of the GLED principal polynomial---which encompasses, by the previously proven result, hyperbolicity of third-order area metric gravity. Indeed, $N$ is of subclass I according to the classification in Ref.~\cite{Schuller_2010}. Thus, we can turn to a $3+1$ formulation, starting with the definition of a slicing.
\begin{definition}[slicing]
  Consider a spacetime manifold $M$ of dimension four. Any diffeomorphism
  \begin{equation}
    \phi\colon \Sigma\times\mathbb R \rightarrow M
  \end{equation}
  from a three-dimensional spatial manifold $\Sigma$ and the reals to $M$ is called a slicing of $M$.
\end{definition}
Such a slicing always exists, as we only consider matter theories that have a well-defined initial value problem. It is, however, not unique: Any diffeomorphism $\psi\colon M\rightarrow M$ yields a new slicing $\tilde\phi = \psi\circ\phi$. Since the spatial manifold is of dimension three and not four, working with slicings comes with new indices running from one to three. These will be denoted with lowercase greek letters, while lowercase latin letters represent spacetime indices running from zero to three.

Every tangent space $T_{\phi(s,\lambda)}M$ has a holonomic basis
\begin{equation}
  \frac{\partial}{\partial x^a} = \left( \frac{\partial}{\partial t},\frac{\partial}{\partial x^\alpha}\right),
\end{equation}
where the vectors on the right are understood as pushforwards of holonomic basis vectors on $T_s\Sigma$ and $T_\lambda\mathbb R$. The same construction yields a holonomic basis
\begin{equation}
  \mathrm dx^a = (\mathrm dt, \mathrm dx^\alpha)
\end{equation}
for the cotangent spaces $T_{\phi(s,\lambda)}^\ast M$. The bundle $\pi_\text{area}$, constructed as subbundle of $T^4_0M$, inherits a $3+1$ split from the decomposition of tangent and cotangent spaces, and so does the second jet bundle of $\pi_\text{area}$.

Based on a slicing, we now introduce an observer definition\footnote{see also Ref.~\cite{Giesel_2012}} for arbitrary tensor theories. Only the principal polynomial is needed for this notion.
\begin{definition}[observer frame, lapse and shift]
  Let $P$ be the principal polynomial of a field theory on a tensor bundle. An observer frame consists of a nonholonomic frame
  \begin{equation}
    (T, e_\alpha = \frac{\partial}{\partial x^\alpha})
  \end{equation}
  and a dual coframe
  \begin{equation}
    (n = \lambda\cdot \mathrm dt, \epsilon^{\alpha}),
  \end{equation}
  where the temporal direction and codirection must satisfy\footnote{$DP$ denotes the formal derivative of $P$ as a polynomial.}
  \begin{equation}\label{frame_conditions}
    P(n) = 1\quad\text{and}\quad T=\frac{1}{\operatorname{deg}P}\frac{DP(n)}{P(n)}.
  \end{equation}
  In the following, we assume $P(n) = 1$ to be solved by choosing an appropriate basis on $T\mathbb R$ and setting $\lambda = 1$.

  The holonomic time direction $\frac{\partial}{\partial t}$ decomposes in the observer frame as
  \begin{equation}
    \frac{\partial}{\partial t} = NT + N^\alpha \frac{\partial}{\partial x^\alpha}
  \end{equation}
  with the lapse $N$ and shift $N^\alpha$.
\end{definition}

Essential for the $3+1$ split is the parameterization of the geometry with quantities an observer can measure in her frame, as well as lapse and shift. For example, using the completeness relation
\begin{equation}
  \mathrm{id} = T\otimes n + e_\alpha \otimes \epsilon^\alpha = \frac{1}{N} \frac{\partial}{\partial t} \otimes n - \frac{1}{N} N^\alpha e_\alpha \otimes n + e_\alpha \otimes \epsilon^\alpha,
\end{equation}
a vector field $v$ decomposes as
\begin{equation}
  v = v\circ \mathrm{id} = v(n)\, T + v(\epsilon^\alpha)\, e_\alpha.
\end{equation}
The holonomic components are thus determined by lapse $N$, shift $N^\alpha$, and the observer quantities $v(n)$ and $v(\epsilon^\alpha)$ as
\begin{equation}
  v(\mathrm dt) = v(n)\quad\text{and}\quad v(\mathrm dx^\alpha) = -\frac{1}{N}N^\alpha v(n) + v(\epsilon^\alpha).
\end{equation}
Obviously, the information contained in $N$, $N^\alpha$, $v(n)$, and $v(\epsilon^\alpha)$ is redundant---four holonomic components are represented using 8 observer quantities. This is where the frame conditions \eqref{frame_conditions} come into play: Consider the decomposition of the area metric field into \cite{Giesel_2012}
\begin{gather}
  G(\mathrm dt,\mathrm dx^\alpha,\mathrm dt,\mathrm dx^\beta) = \frac{1}{N^2} G(n,\epsilon^\alpha,n,\epsilon^\beta), \\
  G(\mathrm dt,\mathrm dx^\alpha, \mathrm dx^\beta, \mathrm dx^\gamma) = -\frac{2}{N^2} G(n,\epsilon^\alpha,n,\epsilon^{\lbrack\gamma})N^{\beta\rbrack} + \frac{1}{N} G(n,\epsilon^\alpha,\epsilon^\beta,\epsilon^\gamma), \\
  \begin{aligned}
    G(\mathrm dx^\alpha, \mathrm dx^\beta, \mathrm dx^\gamma, \mathrm dx^\delta) &{} = \frac{4}{N^2} N^{\lbrack\alpha}G(n,\epsilon^{\beta\rbrack},n,\epsilon^{\lbrack\delta})N^{\gamma\rbrack} + \frac{2}{N} N^{\lbrack\alpha}G(n,\epsilon^{\beta\rbrack},\epsilon^\gamma,\epsilon^\delta) \\
                                                                                 &{} \hphantom{=} + \frac{2}{N} N^{\lbrack\gamma}G(n,\epsilon^{\delta\rbrack},\epsilon^\alpha,\epsilon^\beta) + G(\epsilon^\alpha,\epsilon^\beta,\epsilon^\gamma,\epsilon^\delta).
  \end{aligned}
\end{gather}
So far, the situation seems to be similar---21 area metric components are determined by 21 observer quantities plus lapse and shift. The difference to the decomposition of a vector is that the frame conditions \eqref{frame_conditions} depend---via the principal polynomial---on the area metric, which introduces dependences among area metric, lapse, and shift. To formulate these conditions, it is more convenient to redefine the observer quantities as \cite{Giesel_2012}
\begin{equation}
  \begin{aligned}
    \hat{G}^{\alpha\beta}                  &{} = -G(n,\epsilon^\alpha,n,\epsilon^\beta), \\
    \hat{G}^\alpha_{\hphantom\alpha\beta}  &{} = \frac{1}{2} (\omega_{\hat G})^{-1} \epsilon_{\beta\mu\nu} G(n, \epsilon^\alpha, \epsilon^\mu, \epsilon^\nu) - \delta^\alpha_{\hphantom\alpha\beta}, \\
    \hat{G}_{\alpha\beta}                  &{} = \frac{1}{4} (\omega_{\hat G})^{-2} \epsilon_{\alpha\mu\nu} \epsilon_{\beta\rho\sigma} G(\epsilon^\mu, \epsilon^\nu, \epsilon^\rho, \epsilon^\sigma),
  \end{aligned}
\end{equation}
with the spatial density
\begin{equation}
  \omega_{\hat G} = \sqrt{\operatorname{det}\hat G^{\cdot\cdot}}.
\end{equation}
By definition, $\hat G^{\alpha\beta}$ and $\hat G_{\alpha\beta}$ are symmetric. The frame conditions \eqref{frame_conditions} translate into the two additional properties\cite{Giesel_2012}
\begin{equation}
  0 = \hat G^\alpha_{\hphantom\alpha\alpha}\quad\text{and}\quad0 = \hat G^{\mu\lbrack\alpha} \hat G^{\beta\rbrack}_{\hphantom\beta\mu},
\end{equation}
i.e.~$\hat G^\alpha_{\hphantom\alpha\beta}$ is tracefree and symmetric with respect to $\hat G^{\alpha\beta}$. In total, lapse and shift and the observer quantities $\hat G^{\alpha\beta}$, $\hat G^{\alpha}_{\hphantom\alpha\beta}$, $\hat G_{\alpha\beta}$ have $1+3+6+5+6 = 21$ degrees of freedom, such that they are in one-to-one correspondence with the area metric field $G^{abcd}$. Note the similarity to the $3+1$ decomposition of the metric tensor $g^{ab}$ into shift $N^\alpha$, lapse $N$, and spatial metric $\hat g^{\alpha\beta}$---the purely temporal and spatio-temporal components of the metric are parameterized only by shift and lapse, due to the frame conditions \eqref{frame_conditions}.

Around the perturbation point $N$, the area metric observer quantities expand as
\begin{equation}
  \begin{aligned}
    N &{} = 1 + A, \\
    N^\alpha &{} = b^\alpha, \\
    \hat G^{\alpha\beta} &{} = \gamma^{\alpha\beta} + h^{\alpha\beta}, \\
    \hat G^\alpha_{\hphantom\alpha\beta} &{} = k^\alpha_{\hphantom\alpha\beta}, \\
    \hat G_{\alpha\beta} &{} = \gamma_{\alpha\beta} + l_{\alpha\beta}.
  \end{aligned}
\end{equation}
With $\gamma$ we denote the positive-definite spatial part of the Minkowski metric, i.e.~$\eta^{\alpha\beta} = - \gamma^{\alpha\beta}$. From now on, spatial indices are raised and lowered at will using $\gamma$ and its inverse. The perturbations $A$, $b$, $h$, $k$, and $l$ are again in one-to-one correspondence with the 21 perturbations $H$, by virtue of
\begin{equation}\label{area_metric_perturbation}
  \begin{aligned}
    H^{0\alpha0\beta} &{} = 2 A \gamma^{\alpha\beta} - h^{\alpha\beta}, \\
    H^{0\alpha\beta\gamma} &{} = -A \epsilon^{\alpha\beta\gamma} + 2 b^{\lbrack\beta} \gamma^{\gamma\rbrack\alpha} + \frac{1}{2} \epsilon^{\alpha\beta\gamma} \gamma_{\mu\nu} h^{\mu\nu} + \epsilon^{\mu\beta\gamma} k^\alpha_{\hphantom\alpha\mu}, \\
    H^{\alpha\beta\gamma\delta} &{} = 2\gamma^{\alpha\lbrack\gamma}\gamma^{\delta\rbrack\beta} \gamma_{\mu\nu} h^{\mu\nu} + \epsilon^{\mu\alpha\beta} \epsilon^{\nu\gamma\delta} l_{\mu\nu}.
  \end{aligned}
\end{equation}
A set of perturbations that is more convenient to work with is given by the linear combinations
\begin{equation}
  u^{\alpha\beta} = h^{\alpha\beta} - l^{\alpha\beta},\quad v^{\alpha\beta} = h^{\alpha\beta} + l^{\alpha\beta},\quad w^{\alpha\beta} = 2 k^{\alpha\beta}.
\end{equation}
Using these fields rather then the original ones, the field equations assume a particularly simple form. In fact, we find in Sect.~\ref{sect_area_linear_eom} that this choice yields \emph{decoupled} equations for the individual fields.

Area metric gravity as constructed in the framework of covariant constructive gravity is---by the first axiom---diffeomorphism invariant. For the linear theory, this invariance manifests itself in the presence of a gauge symmetry
\begin{equation}\label{area_gauge_transform}
  H^{\prime A} = H^A + \gmc{A}{B}{n}{m} N^B \xi^m_{,n}
\end{equation}
generated by vector fields $\xi \in \Gamma(TM)$. As a result, the Euler-Lagrange equations are underdetermined, as solutions can only be obtained up to a gauge transform.

In order to have a determined system for our following analysis, we fix the gauge by reducing the number of perturbation fields in a way that can always be reproduced using appropriate gauge transforms. The tool that makes the gauge fixing quite straightforward is Helmholtz' theorem\footnote{The Helmholtz theorem is only valid for certain classes of functions. Applicability to linearized area metric gravitay, i.e.~sufficiently well-behaved perturbations, is assumed.}, which allows us to decompose the spatial vector field $b$ into a so-called longitudinal scalar $B$ and a divergence-free transverse vector $B^\alpha$ satisfying $\partial_\alpha B^\alpha = 0$ as
\begin{equation}
  b^\alpha = \partial^\alpha B + B^\alpha.
\end{equation}
Applied to a tensor of rank 2, the Helmholtz theorem yields a decomposition
\begin{equation}
  u^{\alpha\beta} = U^{\alpha\beta} + 2\partial^{(\alpha}U^{\beta)} + \gamma^{\alpha\beta} \tilde U + \Delta^{\alpha\beta} U.
\end{equation}
In this decomposition, $U^{\alpha\beta}$ is the transverse traceless (TT) tensor satisfying $\partial_\alpha U^{\alpha\beta} = 0$ and $\gamma_{\alpha\beta}U^{\alpha\beta}=0$. The vector $U^\alpha$ is again a transverse vector, $U$ and $\tilde U$ are scalars, and $\Delta_{\alpha\beta} = \partial_\alpha\partial_\beta - \frac{1}{3}\gamma_{\alpha\beta}\Delta$, with the Laplacian $\Delta$, denotes the traceless Hessian. The same decomposition
\begin{equation}
  v^{\alpha\beta} = V^{\alpha\beta} + 2\partial^{(\alpha}V^{\beta)} + \gamma^{\alpha\beta} \tilde V + \Delta^{\alpha\beta} V.
\end{equation}
applies to $v^{\alpha\beta}$. Being traceless, the field $w^{\alpha\beta}$ is missing the trace scalar $\tilde W$, but otherwise admits a similar deconstruction into transverse traceless tensor $W^{\alpha\beta}$, transverse vector $W^\alpha$, and longitudinal scalar $W$. At last, we have the lapse perturbation $A$, which is already a scalar.

Explicitly carrying out the gauge transform \eqref{area_gauge_transform} and carefully inspecting the components of $H^{\prime A}$, we find that the vector field $\xi$ can always be chosen such that the four gauge conditions
\begin{equation}
  0 = B,\quad 0 = U^\alpha - V^\alpha,\quad 0 = U + V
\end{equation}
are satisfied (see \cite{Schneider_2017}). This choice reduces the degrees of freedom to 17, wich are summarized in Table \ref{table_area_dof}.
\begin{table}
  \centering
  \begin{tabular}{l c c c}
    \toprule
    perturbation kind & dof per field & fields & total dof \\
    \midrule
    scalar & 1 & $A,\tilde U,\tilde V,V,W$ & 5 \\
    transverse vector & 2 & $B^\alpha,U^\alpha,W^\alpha$ & 6 \\
    transverse traceless tensor & 2 & $U^{\alpha\beta},V^{\alpha\beta},W^{\alpha\beta}$ & 6 \\
    \bottomrule 
  \end{tabular}
  \caption{The 17 gauge-fixed degrees of freedom (dof) in linearized area metric gravity. Transverse vectors are divergence free, i.e.~satisfy $0 = \partial_\alpha U^\alpha$. Transverse traceless vectors are symmetric, tracefree, and divergence free, i.e.~$0 = U^{\lbrack\alpha\beta\rbrack}$, $0 = \gamma_{\alpha\beta} U^{\alpha\beta}$, and $0 = \partial_\alpha U^{\alpha\beta}$. Together with the four gauge-fixed fields $B=0$, $V^\alpha = U^\alpha$, and $U=-V$, the area metric perturbation in this particular gauge is reproduced using Eq.~\eqref{area_metric_perturbation}.}
  \label{table_area_dof}
\end{table}

Let us briefly collect the results of a similar decomposition and gauge fixing for metric gravity perturbed around the Minkowski metric. This will be of use later when we compare area metric gravity with metric gravity and highlight the differences. The metric tensor has 10 degrees of freedom and, as already remarked, decomposes into shift $N^\alpha$, lapse $N$, and spatial metric $\hat g^{\alpha\beta}$ by virtue of the relations
\begin{equation}\label{metric_three_plus_one}
  \begin{aligned}
    g(\mathrm dt,\mathrm dt) &{} = \frac{1}{N^2}, \\
    g(\mathrm dt,\mathrm dx^\alpha) &{} = -\frac{N^\alpha}{N^2}, \\
    g(\mathrm dx^\alpha,\mathrm dx^\beta) &{} = \frac{N^\alpha N^\beta}{N^2} - \hat g^{\alpha\beta}.
  \end{aligned}
\end{equation}
Around $\eta$, the observer quantities expand as
\begin{equation}\label{metric_expansion}
  \begin{aligned}
    N &{} = 1 + A, \\
    N^\alpha &{} = b^\alpha, \\
    \hat g^{\alpha\beta} &{} = \gamma^{\alpha\beta} + \varphi^{\alpha\beta}.
  \end{aligned}
\end{equation}
Like before, we use the Helmholtz theorem to write
\begin{equation}
  b^\alpha = \partial^\alpha B + B^\alpha
\end{equation}
and
\begin{equation}\label{metric_split}
  \varphi^{\alpha\beta} = E^{\alpha\beta} + 2\partial^{(\alpha}V^{\beta)} + C \gamma^{\alpha\beta} + \Delta^{\alpha\beta} D.
\end{equation}
A possible choice of gauge conditions is to set $B$, $D$, and $V^\alpha$ to zero, leaving us with 6 degrees of freedom in the fields $A$, $B^\alpha$, $C$, and $E^{\alpha\beta}$.

\subsection{Linearized field equations}\label{sect_area_linear_eom}
Applying the $3+1$ decomposition of the area metric field to the Lagrangian density constructed in Sect.~\ref{sect_area_lagrangian} yields an expression that is determined only by lapse, shift, and observer quantities. The corresponding field equations are obtained by the variations
\begin{equation}
  \frac{\delta L}{\delta N},\ \frac{\delta L}{\delta N^\alpha},\ \frac{\delta L}{\delta G^{\alpha\beta}},\ \frac{\delta L}{\delta G^\alpha_{\hphantom\alpha\beta}},\ \frac{\delta L}{\delta G_{\alpha\beta}}
\end{equation}
with respect to all of these fields---as opposed to the ``single'' variation
\begin{equation}
  \frac{\delta L}{\delta G^{abcd}}
\end{equation}
with respect to the area metric in the spacetime picture.

For the linearized field equations, we automatically obtain a Helmholtz decomposition of the Euler-Lagrange equations: The variation with respect to the lapse $N$ is a scalar and contains only contributions from scalars, the variation with respect to the shift $N^\alpha$ is a vector and contains only contributions from vectors. The same holds for the scalar and vector constituents of the observer quantities $\hat G$. Also the variations with respect to transverse traceless tensors are again tensors and only comprised of tensors. As a result, the field equations already decouple to a large extent. For this reason, the terminology of the individual scalar, vector, and tensor fields as \emph{modes} of the gravitational field is justified.

Performing the $3+1$ split is a computationally heavy task. Essentially, the perturbation \eqref{area_metric_perturbation} has to be inserted into the Ansätze (see Appendix \ref{appendix}), the resulting expression must be simplified, then varied with respect to the different modes, and simplified again. In order to gain confidence in the result, speed up the computation, and---very importantly---have a calculation that can be reproduced and amended, the task has been offloaded to the computer algebra system \texttt{cadabra}\cite{Peeters2007,Peeters_2018}. The code is available at Ref.~\cite{Alex_2020_area-metric-gravity} and, in parts, in Appendix \ref{}.

The result of this computation finally yields the field equations of perturbative area metric gravity in a gauge-fixed $3+1$ setting. Of the 16 undetermined gravitational constants $k_i$ that determine the expansion coefficients $e_i$ (see Appendix \ref{appendix_ansaetze}), ten independent linear combinations $s_i$ (listed in Appendix \ref{appendix_reduction}) make up the scalar equations
\begingroup\allowdisplaybreaks
\begin{align}\footnotesize
  \bigg\lbrack\frac{\delta L}{\delta u^{\alpha\beta}}\bigg\rbrack^\text{S-TF} = {} & \Delta_{\alpha\beta} \bigg\lbrack s_1 A -\frac{s_1}{4} \tilde U + s_3 \tilde V + s_4 \ddot V - \frac{s_4}{3} \Delta V + s_6 \ddot W - \frac{s_6}{3} \Delta W\bigg\rbrack, \nonumber \\
  \bigg\lbrack\frac{\delta L}{\delta v^{\alpha\beta}}\bigg\rbrack^\text{S-TF} = {} & \Delta_{\alpha\beta} \bigg\lbrack (s_1 + 4s_4) A + (\frac{s_1}{4} + s_4) \tilde U + (\frac{3s_1}{4} + 3 s_4) \tilde V \nonumber \\ & + s_{11} \ddot V - (\frac{s_1}{3} + \frac{4s_4}{3} + s_{11}) \Delta V + s_{13} V + s_{14} \Box W + s_{16} W \bigg\rbrack, \nonumber \\
  \bigg\lbrack\frac{\delta L}{\delta w^{\alpha\beta}}\bigg\rbrack^\text{S-TF} = {} & \Delta_{\alpha\beta} \bigg\lbrack 4s_6 A + s_6 \tilde U + 3 s_6 \tilde V \nonumber \\ & + (-s_6 + s_{14}) \ddot V - (\frac{s_6}{3} + s_{14}) \Delta V + s_{16} V - (\frac{s_1}{4} + s_4 + s_{11}) \Box W - s_{13} W \bigg\rbrack, \nonumber \\
  \bigg\lbrack\frac{\delta L}{\delta u^{\alpha\beta}}\bigg\rbrack^\text{S-TR} = {} & \gamma_{\alpha\beta} \bigg\lbrack -\frac{2s_1}{3} \Delta A -\frac{s_1}{2} \ddot{\tilde U} + \frac{s_1}{6} \Delta\tilde U + (-\frac{3s_1}{4} + s_3) \ddot{\tilde V} -\frac{2s_3}{3} \Delta\tilde V \nonumber \\ & + \frac{s_1}{3} \Delta\ddot V + \frac{2s_4}{9} \Delta\Delta V + \frac{2s_6}{9} \Delta\Delta W\bigg\rbrack, \nonumber \\
  \bigg\lbrack\frac{\delta L}{\delta v^{\alpha\beta}}\bigg\rbrack^\text{S-TR} = {} & \gamma_{\alpha\beta} \bigg\lbrack (-s_1 + \frac{4s_3}{3}) \Delta A + (-\frac{3s_1}{4} + s_3) \ddot{\tilde U} - \frac{2s_3}{3} \Delta\tilde U \nonumber \\ & + s_{37} \ddot{\tilde V} - (\frac{3s_1}{2} - 2s_3 + s_{37}) \Delta\tilde V + s_{39} \tilde V \nonumber \\ & + (\frac{s_1}{2} - \frac{2s_3}{3}) \Delta\ddot V + (\frac{s_1}{6} + \frac{2s_3}{9} + \frac{2s_4}{3}) \Delta\Delta V + \frac{2s_6}{3} \Delta\Delta W\bigg\rbrack, \nonumber \\
  \bigg\lbrack\frac{\delta L}{\delta b^{\alpha}}\bigg\rbrack^\text{S} = {} & \partial_\alpha \partial_t \bigg\lbrack -2s_1\tilde U + (-3s_1 + 4s_3) \tilde V + (\frac{4s_1}{3} + \frac{8s_4}{3}) \Delta V + \frac{8s_6}{3} \Delta W\bigg\rbrack, \nonumber \\
  \frac{\delta L}{\delta A} = {} & -2s_1 \Delta\tilde U + (-3s_1 + 4s_3) \Delta\tilde V + (\frac{4s_1}{3} + \frac{8s_4}{3}) \Delta\Delta V + \frac{8s_6}{3} \Delta\Delta W. \label{scalar_equations}
\end{align}%
\endgroup
The label $(\text{S-TF})$ denotes the projection of a tensor onto the tracefree scalar, $(\text{S-TR})$ means the projection onto the trace.

A subset of seven constants out of the ten constants $s_i$ parameterizes the vector field equations
\begingroup\allowdisplaybreaks
\begin{align}\footnotesize
 \bigg\lbrack\frac{\delta L}{\delta u^{\alpha\beta}}\bigg\rbrack^\text{V} = {} & \partial_t \partial_{(\alpha} \bigg\lbrack s_1 B_{\beta)} - 2s_4 \dot U_{\beta)} - 2s_6 \epsilon_{\beta)}^{\hphantom{\beta)}\mu\nu} U_{\mu,\nu} + 2s_6 \dot W_{\beta)} + (-\frac{s_1}{2} - 2s_4) \epsilon_{\beta)}^{\hphantom{\beta)}\mu\nu} W_{\mu,\nu} \bigg\rbrack, \nonumber \\
 \bigg\lbrack\frac{\delta L}{\delta v^{\alpha\beta}}\bigg\rbrack^\text{V} = {} & \partial_{(\alpha} \bigg\lbrack (-s_1 - 4s_4) \dot B_{\beta)} + 4s_6 \epsilon_{\beta)}^{\hphantom{\beta)}\mu\nu} B_{\mu,\nu} \nonumber \\ & + (s_1 + 4s_4 + 2s_{11}) \ddot U_{\beta)} + (-\frac{3s_1}{2} - 6s_4 - 2s_{11}) \Delta U_{\beta)} + 2s_6 \epsilon_{\beta)}^{\hphantom{\beta)}\mu\nu} \dot U_{\mu,\nu} + 2s_{13} U_{\beta)} \nonumber \\ & + 2 s_{14} \Box W_{\beta)} + 2 s_{16} W_{\beta)} \bigg\rbrack, \nonumber \\
 \bigg\lbrack\frac{\delta L}{\delta w^{\alpha\beta}}\bigg\rbrack^\text{V} = {} & \partial_{(\alpha} \bigg\lbrack 4s_6 \dot B_{\beta)} + (s_1 + 4s_4) \epsilon_{\beta)}^{\hphantom{\beta)}\mu\nu} B_{\mu,\nu} \nonumber \\ & + (2s_6 + 2 s_{14}) \ddot U_{\beta)} - 2 s_{14} \Delta U_{\beta)} + (\frac{s_1}{2} + 2 s_4) \epsilon_{\beta)}^{\hphantom{\beta)}\mu\nu} \dot U_{\mu,\nu} + 2 s_{16} U_{\beta)} \nonumber \\ & + (-\frac{3s_1}{2} - 6s_4 - 2s_{11}) \Box W_{\beta)} - 2s_{13} W_{\beta)} \bigg\rbrack, \nonumber \\
 \bigg\lbrack\frac{\delta L}{\delta b^\alpha}\bigg\rbrack^\text{V} = {} & \Delta\bigg\lbrack 2s_1 B_{\alpha} - 4s_4 \dot U_{\alpha} - 4s_6 \epsilon_{\alpha}^{\hphantom{\alpha}\mu\nu} U_{\mu,\nu} + 4s_6 \dot W_{\alpha} + (-s_1 - 4s_4) \epsilon_{\alpha}^{\hphantom{\alpha}\mu\nu} W_{\mu,\nu} \bigg\rbrack, \label{vector_equations}
\end{align}%
\endgroup
as well as the transverse traceless tensor field equations
\begingroup\allowdisplaybreaks
\begin{align}\footnotesize
 \bigg\lbrack\frac{\delta L}{\delta u^{\alpha\beta}}\bigg\rbrack^\text{TT} = {} & \frac{s_1}{4} \Box U_{\alpha\beta} \nonumber \\ &  + (\frac{s_1}{4} + s_4) \ddot V_{\alpha\beta} + (\frac{s_1}{4} + s_4) \Delta V_{\alpha\beta} - 2s_6 \epsilon_{(\alpha}^{\hphantom{(\alpha}\mu\nu} \dot V_{\beta)\mu,\nu} \nonumber \\ & + s_6 \ddot W_{\alpha\beta} + s_6 \Delta W_{\alpha\beta} + (\frac{s_1}{2} + 2 s_4) \epsilon_{(\alpha}^{\hphantom{(\alpha}\mu\nu} \dot W_{\beta)\mu,\nu}, \nonumber \\
 \bigg\lbrack\frac{\delta L}{\delta v^{\alpha\beta}}\bigg\rbrack^\text{TT} = {} & (\frac{s_1}{4} + s_4) \ddot U_{\alpha\beta} + (\frac{s_1}{4} + s_4) \Delta U_{\alpha\beta} + 2s_6 \epsilon_{(\alpha}^{\hphantom{(\alpha}\mu\nu} \dot U_{\beta)\mu,\nu} \nonumber \\ &  + (\frac{s_1}{4} + s_4 + s_{11}) \Box V_{\alpha\beta} + s_{13} V_{\alpha\beta} + s_{14} \Box W_{\alpha\beta} + s_{16} W_{\alpha\beta}, \nonumber \\
 \bigg\lbrack\frac{\delta L}{\delta w^{\alpha\beta}}\bigg\rbrack^\text{TT} = {} & s_6 \ddot U_{\alpha\beta} + s_6 \Delta U_{\alpha\beta} - (\frac{s_1}{2} + 2 s_4) \epsilon_{(\alpha}^{\hphantom{(\alpha}\mu\nu} \dot U_{\beta)\mu,\nu}\nonumber \\ &  + s_{14} \Box V_{\alpha\beta} + s_{16} V_{\alpha\beta} - (\frac{s_1}{4} + s_4 + s_{11}) \Box W_{\alpha\beta} - s_{13} W_{\alpha\beta}. \label{tensor_equations}
\end{align}

The second Noether theorem \eqref{second_noether}
\begin{equation}
  0 = D_n \mathcal T^n_m - \frac{\delta L}{\delta u^A}u^A_{\hphantom Am} = - D_n \big\lbrack\frac{\delta L}{\delta u^A} \gmc{A}{B}{n}{m} u^B\big\rbrack - \frac{\delta L}{\delta u^A}u^A_{\hphantom Am} 
\end{equation}
provides a sanity check for the constructed field equations by virtue of its expansion around $N$
\begin{equation}
  0 = -\big\lbrack D_n \frac{\delta L}{\delta u^A}\big\rbrack_{N+H} \gmc{A}{B}{n}{m} N^B + \mathcal O(H^2).
\end{equation}
Inverting the relation \eqref{area_metric_perturbation} between spacetime area metric and observer fields, we can make use of the chain rule in order to express the variations with respect to the area metric in terms of variations with respect to the observer quantities. This renders the perturbative expansion of the Noether theorem in the particularly simple form
\begin{equation}\label{second_noether_area_linear}
  0 = \partial_t \frac{\delta L}{\delta A} - \partial_\alpha \frac{\delta L}{\delta b_\alpha}\quad\text{and}\quad 0 = \partial_t\frac{\delta L}{\delta b^\alpha} - 4\partial_\beta\frac{\delta L}{\delta u_{\alpha\beta}},
\end{equation}
which is indeed satisfied by the system \eqref{scalar_equations}--\eqref{tensor_equations}. As a consequence of the diffeomorphism invariance of the theory, the field equations have four dependences among themselves. This is, of course, expected---not only from the Noether theorem, but also from the fact that gauge-fixing the observer quantities by constraining four fields reduces the 21 unknowns by four. In order for the system of 21 field equations not to be overdetermined, it must express additional dependences. These considerations are reminiscent of the rich field of constraint analysis\footnote{See e.g.~\cite{}}, which is predominantly studied in the Hamiltonian picture and also plays a role in canonical constructive gravity. For some results in the context of covariant constructive gravity, limited to first-derivative-order theories, see Ref.~\cite{}.

While the Noether identities are expected and, in fact, indispensable, a thorough analysis of the linearized field equations reveals further properties that are impossible to reconcile with our premises. After all, the axioms of covariant constructive gravity are only \emph{necessary} conditions for a theory to be viable. Any such constructed theory needs to be further specified by finding appropriate values for the gravitational constants. This also applies to Einstein gravity---the Newtonian and cosmological constants only match observations for specific ranges, where some possibilities like a negative Newtonian constant can be dismissed outright.

The first restriction of the area metric gravity parameter range we will make is to match the weak gravitational field sourced by a point mass with a modest generalization of the Einstein equivalent. More specifically, we consider the gravitational field sourced by a point mass $M$ which is at rest at the coordinate origin and thus describes the world line
\begin{equation}\label{stationary_worldline}
  \gamma^a(\lambda) = \lambda \delta^a_0.
\end{equation}
If the point particle $M$ is an idealization of a matter field that obeys GLED dynamics, its action is given by \cite{R_tzel_2011,Rivera_2012}
\begin{equation}
  S_\text{matter}\lbrack\gamma\rbrack = -M \int \mathrm d\lambda \mathcal P_\text{GLED}(\mathcal L^{-1}(\dot\gamma(\lambda)))^{-\frac{1}{4}},
\end{equation}
where $\mathcal L^{-1}$ is the inverse of the Legendre map associated with the principal polynomial. In the Einstein equivalent, this action coincides with the common notion of the length of the particle worldline as measured using the covariant metric tensor. The full expansion for arbitrary curves $\gamma$ is employed in the following section, it suffices here to consider the special case \eqref{stationary_worldline} and find the only nonvanishing contribution
\begin{equation}\label{scalar_source}
  \frac{\delta S_\text{matter}}{\delta A(x)} = -M\delta^{(3)}(x).
\end{equation}
With the matter distribution being stationary, we consider a stationary ansatz for the solution to the field equations by assuming that the time derivatives of the gravitational field vanish. Using the source \eqref{scalar_source} as the right-hand side of the linearized field equations \eqref{scalar_equations}--\eqref{tensor_equations} yields vector and tensor equations that are trivially sourced by zero and as such only admit the trivial solution $B^\alpha = U^\alpha = W^\alpha = 0$ and $U^{\alpha\beta} = V^{\alpha\beta} = W^{\alpha\beta} = 0$.\footnote{\textbf{calculate?}} The scalar equations take the form
\begin{equation}
  E_i^{(\text{scalar})} = M \delta^{(3)}(x) \delta^0_i + \sum_j\lbrack a_{ij} S_j + b_{ij} \Delta S_j + c_{ij} \Delta\Delta S_j\rbrack
\end{equation}
for constant coefficients $a_{ij}, b_{ij}, c_{ij}$ and scalar fields $S^{(i)}$.

As solution to the scalar equations we obtain\footnote{\textbf{calculate?}} certain combinations of long-ranging Coulomb solutions $\propto \frac{1}{r}$ and short-ranging Yukawa solutions $\propto \frac{1}{r} \mathrm e^{-\mu r}$. While the coefficients of these combinations depend in an intricate way on the gravitational constants and are impossible to present in general, it \emph{is} feasible to make a generic argument concerning the phenomenology of the linearized result: \emph{The solution to the scalar field equations corresponds to the linearized Schwarzschild solution of general relativity for a central mass $M$ corrected by short-ranging Yukawa potentials if and only if two linear conditions on the gravitational constants $s_i$ hold.}

This statement concerns the metric limit of area metric gravity, which is reached using the metrically induced area metric \eqref{metric_induced_area}. Inserting the metric $3+1$ decomposition \eqref{metric_three_plus_one} and its perturbative expansion \eqref{metric_expansion} in the expression for the induced area metric yields
\begin{equation}
  \begin{aligned}
    \hat G^{\alpha\beta} &{} = \hat g^{\alpha\beta} = \gamma^{\alpha\beta} + \varphi^{\alpha\beta}, \\
    \hat G^\alpha_{\hphantom\alpha\beta} &{} = 0, \\
    \hat G_{\alpha\beta} &{} = (\hat g^{-1})_{\alpha\beta} \approx \gamma_{\alpha\beta} - \varphi_{\alpha\beta}, 
  \end{aligned}
\end{equation}
from which we read off the induced perturbations
\begin{equation}\label{induced_perturbations}
  u^{\alpha\beta} = 2\varphi^{\alpha\beta},\quad v^{\alpha\beta} = 0,\quad w^{\alpha\beta} = 0.
\end{equation}
If the metric perturbation is now given by the expansion of the Schwarzschild solution\cite{Schwarzschild_1916} to first order,
\begin{equation}
  A \propto \frac{1}{r}\quad\text{and}\quad \varphi^{\alpha\beta} = 2A\gamma^{\alpha\beta},
\end{equation}
the metrically induced area metric scalar fields amount to first order to
\begin{equation}
  \begin{gathered}
    V = W = \tilde V = 0, \\
    \tilde U = 4 A, \\
    A \propto \frac{1}{r}.
  \end{gathered}
\end{equation}
The condition stated above requires that the area metric deviations from these fields amount to short-ranging Yukawa corrections, i.e.~informally
\begin{equation}
  \begin{aligned}
    4A - \tilde U &{} = (\text{Yukawa corrections}), \\
                V &{} = (\text{Yukawa corrections}), \\
                W &{} = (\text{Yukawa corrections}), \\
        \tilde{V} &{} = (\text{Yukawa corrections}). \\
  \end{aligned}
\end{equation}
These conditions are equivalent to the vanishing of the linear combinations
\begin{equation}
  s_1 + 4 s_4 = 0\quad\text{and}\quad s_6 = 0,
\end{equation}
which we from now on implement, reducing the number of first-order gravitational constants by two to eight. Thus, we have ruled out the possibility of deviating \emph{too much}\footnote{In the specific sense explained above.} from Einstein gravity already in the regime of weak birefringence and restricted perturbative area metric gravity to a phenomenologically plausible sector. In this subtheory, the scalar fields around a point mass reduce to
\begin{equation}\label{linearized_schwarzschild_solution}
  \begin{aligned}
    V(x) &{} = 0, \\
    W(x) &{} = 0, \\
    \tilde U(x) &{} = \frac{M}{4\pi r} \big\lbrack \alpha - (\beta + \frac{3}{4}\gamma)\mathrm e^{-\mu r}\big\rbrack, \\
    \tilde V(x) &{} = \frac{M}{4\pi r} \big\lbrack \frac{1}{4}\gamma \mathrm e^{-\mu r}\big\rbrack, \\
    A (x) &{} = \frac{M}{4\pi r} \big\lbrack \frac{1}{4} \alpha + \frac{1}{4} \beta \mathrm e^{-\mu r}\big\rbrack,
  \end{aligned}
\end{equation}
where we redefined the relevant gravitational constants using the more convenient set
\begin{equation}\label{schwarzschild_constants}
  \begin{aligned}
    \mu^2 &{} = \frac{8s_1s_{39}}{9s_1^2 - 24s_1s_3 + 8s_1s_{37} + 16s_3^2}, \\
    \alpha &{} = \frac{1}{2s_1}, \\
    \beta &{} = \frac{(3s_1 + 4s_3)^2}{6s_1(9s_1^2 - 24s_1s_3 + 8s_1s_{37} + 16s_3^2)}, \\
    \gamma &{} = \frac{-8(3s_1 + 4s_3)}{6(9s_1^2 - 24s_1s_3 + 8s_1s_{37} + 16s_3^2)}.
  \end{aligned}
\end{equation}

With the reduction from ten to eight gravitational constants, the linearized field equations assume a simpler form. There are reduced scalar field equations
\begingroup\allowdisplaybreaks
\begin{align}\footnotesize
  \bigg\lbrack\frac{\delta L}{\delta u^{\alpha\beta}}\bigg\rbrack^\text{S-TF} = {} & \Delta_{\alpha\beta} \bigg\lbrack s_1 A -\frac{s_1}{4} \tilde U + s_3 \tilde V - \frac{s_1}{4} \ddot V + \frac{s_1}{12} \Delta V \bigg\rbrack, \nonumber \\
  \bigg\lbrack\frac{\delta L}{\delta v^{\alpha\beta}}\bigg\rbrack^\text{S-TF} = {} & \Delta_{\alpha\beta} \bigg\lbrack s_{11} \Box V + s_{13} V + s_{14} \Box W + s_{16} W \bigg\rbrack, \nonumber \\
  \bigg\lbrack\frac{\delta L}{\delta w^{\alpha\beta}}\bigg\rbrack^\text{S-TF} = {} & \Delta_{\alpha\beta} \bigg\lbrack s_{14} \Box V + s_{16} V - s_{11} \Box W - s_{13} W \bigg\rbrack, \nonumber \\
  \bigg\lbrack\frac{\delta L}{\delta u^{\alpha\beta}}\bigg\rbrack^\text{S-TR} = {} & \gamma_{\alpha\beta} \bigg\lbrack -\frac{2s_1}{3} \Delta A -\frac{s_1}{2} \ddot{\tilde U} + \frac{s_1}{6} \Delta\tilde U + (-\frac{3s_1}{4} + s_3) \ddot{\tilde V} -\frac{2s_3}{3} \Delta\tilde V \nonumber \\ & + \frac{s_1}{3} \Delta\ddot V - \frac{s_1}{18} \Delta\Delta V \bigg\rbrack, \label{scalar_equations_reduced} \\
  \bigg\lbrack\frac{\delta L}{\delta v^{\alpha\beta}}\bigg\rbrack^\text{S-TR} = {} & \gamma_{\alpha\beta} \bigg\lbrack (-s_1 + \frac{4s_3}{3}) \Delta A + (-\frac{3s_1}{4} + s_3) \ddot{\tilde U} - \frac{2s_3}{3} \Delta\tilde U \nonumber \\ & + s_{37} \ddot{\tilde V} - (\frac{3s_1}{2} - 2s_3 + s_{37}) \Delta\tilde V + s_{39} \tilde V \nonumber \\ & + (\frac{s_1}{2} - \frac{2s_3}{3}) \Delta\ddot V + \frac{2s_3}{9} \Delta\Delta V\bigg\rbrack, \nonumber \\
  \bigg\lbrack\frac{\delta L}{\delta b^{\alpha}}\bigg\rbrack^\text{S} = {} & \partial_\alpha \partial_t \bigg\lbrack -2s_1\tilde U + (-3s_1 + 4s_3) \tilde V + \frac{2s_1}{3} \Delta V\bigg\rbrack, \nonumber \\
  \frac{\delta L}{\delta A} = {} & -2s_1 \Delta\tilde U + (-3s_1 + 4s_3) \Delta\tilde V + \frac{2s_1}{3} \Delta\Delta V, \nonumber
\end{align}
\endgroup
vector field equations
\begingroup\allowdisplaybreaks
\begin{align}\footnotesize
  \bigg\lbrack\frac{\delta L}{\delta u^{\alpha\beta}}\bigg\rbrack^\text{V} = {} & \frac{s_1}{2} \partial_t \partial_{(\alpha} \bigg\lbrack 2 B_{\beta)} + \dot U_{\beta)}\bigg\rbrack, \nonumber \\
  \bigg\lbrack\frac{\delta L}{\delta v^{\alpha\beta}}\bigg\rbrack^\text{V} = {} & 2 \partial_{(\alpha} \bigg\lbrack  s_{11} \Box U_{\beta)} + s_{13} U_{\beta)} +  s_{14} \Box W_{\beta)} +  s_{16} W_{\beta)} \bigg\rbrack, \nonumber \\
  \bigg\lbrack\frac{\delta L}{\delta w^{\alpha\beta}}\bigg\rbrack^\text{V} = {} & 2 \partial_{(\alpha} \bigg\lbrack s_{14} \Box U_{\beta)} + s_{16} U_{\beta)} - s_{11} \Box W_{\beta)} - s_{13} W_{\beta)} \bigg\rbrack, \label{vector_equations_reduced} \\
  \bigg\lbrack\frac{\delta L}{\delta b^\alpha}\bigg\rbrack^\text{V} = {} & s_1 \Delta\bigg\lbrack 2 B_{\alpha} + \dot U_{\alpha} \bigg\rbrack, \nonumber
\end{align}
\endgroup
and traceless tensor field equations
\begingroup\allowdisplaybreaks
\begin{align}\footnotesize
  \bigg\lbrack\frac{\delta L}{\delta u^{\alpha\beta}}\bigg\rbrack^\text{TT} = {} & \frac{s_1}{4} \Box U_{\alpha\beta}, \nonumber \\
  \bigg\lbrack\frac{\delta L}{\delta v^{\alpha\beta}}\bigg\rbrack^\text{TT} = {} & s_{11} \Box V_{\alpha\beta} + s_{13} V_{\alpha\beta} + s_{14} \Box W_{\alpha\beta} + s_{16} W_{\alpha\beta}, \label{tensor_equations_reduced} \\
  \bigg\lbrack\frac{\delta L}{\delta w^{\alpha\beta}}\bigg\rbrack^\text{TT} = {} & s_{14} \Box V_{\alpha\beta} + s_{16} V_{\alpha\beta} - s_{11} \Box W_{\alpha\beta} - s_{13} W_{\alpha\beta}. \nonumber
\end{align}
\endgroup

The second observation we want to make concerns the subset
\begin{equation}
  \begin{aligned}
  \bigg\lbrack\frac{\delta L}{\delta v^{\alpha\beta}}\bigg\rbrack^\text{S-TF} = {} & \Delta_{\alpha\beta} \bigg\lbrack s_{11} \Box V + s_{13} V + s_{14} \Box W + s_{16} W \bigg\rbrack, \\
  \bigg\lbrack\frac{\delta L}{\delta w^{\alpha\beta}}\bigg\rbrack^\text{S-TF} = {} & \Delta_{\alpha\beta} \bigg\lbrack s_{14} \Box V + s_{16} V - s_{11} \Box W - s_{13} W \bigg\rbrack
  \end{aligned}
\end{equation}
of the reduced scalar equations \eqref{scalar_equations_reduced}, whose pattern is repeated in the vector equations \eqref{vector_equations_reduced} for the modes $U^\alpha$ and $W^\alpha$ as well as in the tensor equations \eqref{tensor_equations_reduced} for the modes $V^{\alpha\beta}$ and $W^{\alpha\beta}$. Linear combinations of these equations \emph{in vacuo} yield the equivalent system
\begin{equation}\label{coupled_wave_equations}
  \begin{aligned}
    0 = {} & \Box V + \nu^2 V + \sigma W, \\
    0 = {} & \Box W + \nu^2 W - \sigma V,
  \end{aligned}
\end{equation}
with constants
\begin{equation}
  \nu^2 = \frac{s_{11}s_{13}+s_{14}s_{16}}{s_{11}^2+s_{14}^2}\quad\text{and}\quad\sigma = \frac{s_{11}s_{16} - s_{13} s_{14}}{s_{11}^2 + s_{14}^2}.
\end{equation}
Performing a spatial Fourier transform of the vacuum scalar equations \ref{coupled_wave_equations}, we can translate them into a system of linear, first-order ordinary differential equations for the modes $\tilde v(t,k)$ and $\tilde w(t,k)$
\begin{equation}
  \frac{\mathrm d}{\mathrm dt}\begin{pmatrix}\tilde v \\ \tilde w \\ \dot{\tilde v} \\ \dot{\tilde w}\end{pmatrix} = \begin{pmatrix}0 & 0 & 1 & 0 \\ 0 & 0 & 0 & 1 \\ -(k^2 + \nu^2) & -\sigma & 0 & 0 \\ \sigma & -(k^2 + \nu^2) & 0 & 0\end{pmatrix}\begin{pmatrix}\tilde v \\ \tilde w \\ \dot{\tilde v} \\ \dot{\tilde w}\end{pmatrix}.
\end{equation}
What is now interesting about this system are the eigenvalues of the time evolution, which are the four complex roots
\begin{equation}
  \lambda_k = \pm \mathrm i \sqrt{(k^2 + \nu^2) \pm \mathrm i\sigma}.
\end{equation}
Most importantly, there are always $\lambda_k$ such that $\operatorname{Re}(\lambda_k)>0$ \emph{unless} $\sigma$ vanishes. As a consequence, there will always be diverging modes under time evolution if $\sigma$ is not zero. This is not restricted to the scalar modes we analyzed, but also holds for the vector and transverse traceless tensor modes that are coupled in the same way. Such a theory would not only be physically \emph{implausible}, it would be fundamentally broken. We set $\sigma$ to zero by imposing the additional condition
\begin{equation}
  s_{11} s_{16} - s_{13} s_{14} = 0
\end{equation}
and have thus reduced linearized area metric gravity to a theory parameterized by seven remaining gravitational constants, of which there are five combinations that determine the results obtained above: The two constants $\mu$ and $\nu$ appear as \emph{masses} in wave equations and screened Poisson equations, respectively, and three constants $\alpha$, $\beta$, and $\gamma$ further parameterize the linearized Schwarzschild solution.

Note that with $\sigma=0$ the wave equations for $W$, $V$, $U^\alpha$, $V^\alpha$, $U^{\alpha\beta}$, $V^{\alpha\beta}$, and $W^{\alpha\beta}$ \emph{decouple}, e.g.~the system of transverse traceless tensor equations can be transformed by taking linear combinations into
\begingroup\allowdisplaybreaks
\begin{align}\footnotesize
  \bigg\lbrack\frac{\delta L}{\delta u^{\alpha\beta}}\bigg\rbrack^\text{TT} = {} & \frac{s_1}{4} \Box U_{\alpha\beta}, \nonumber \\
  \frac{s_{11}}{s_{11}^2 + s_{14}^2} \bigg\lbrack\frac{\delta L}{\delta v^{\alpha\beta}}\bigg\rbrack^\text{TT} + \frac{s_{14}}{s_{11}^2 + s_{14}^2} \bigg\lbrack\frac{\delta L}{\delta w^{\alpha\beta}}\bigg\rbrack^\text{TT} = {} & \Box V_{\alpha\beta} + \nu^2 V_{\alpha\beta}, \label{tensor_equations_decoupled} \\
  \frac{s_{14}}{s_{11}^2 + s_{14}^2} \bigg\lbrack\frac{\delta L}{\delta v^{\alpha\beta}}\bigg\rbrack^\text{TT} - \frac{s_{11}}{s_{11}^2 + s_{14}^2} \bigg\lbrack\frac{\delta L}{\delta w^{\alpha\beta}}\bigg\rbrack^\text{TT} = {} & \Box W_{\alpha\beta} + \nu^2 W_{\alpha\beta}. \nonumber
\end{align}
\endgroup
Similar decoupled wave equations are obtained for the mentioned vector and scalar modes. It is also possible to find a linear combination of scalar field equations \eqref{scalar_equations_reduced} such that the mode $\tilde V$ obeys a massive wave equation\footnote{Not denoting linear combinations of Lagrangian variations explicitly but just referring to them as \emph{source terms}.}
\begin{equation}\label{trace_massive_wave}
  \text{(source terms)} = \Box \tilde V + \mu^2 \tilde V.
\end{equation}
Counting the wave equations we already found, there are at least 13 propagating degrees of freedom. This is already the maximum number, because our system for 17 degrees of freedom must exhibit four constraint equations arising from the gauge symmetry\cite{}. In fact, the four remaining degrees $B^\alpha$, $\tilde U$, and $A$ are determined by field equations with less than two time derivatives, as can be read off from Eqns.~\eqref{scalar_equations_reduced}--\eqref{tensor_equations_reduced}. Such equations as part of an initial value problem are usually associated with constraints, as they are not capable to \emph{evolve} initial data, but only to \emph{constrain} it.\cite{}

Summing up, the phenomenologically relevant subsector of linearized area metric gravity admits two massless propagating degrees of freedom in the form of the tensor mode $U^{\alpha\beta}$. Furthermore, there are 11 massive propagating degrees of freedom with mass $\mu$, represented by the fields $W$, $V$, $\tilde V$, $U^\alpha$, $W^\alpha$, $V^{\alpha\beta}$, and $W^{\alpha\beta}$. The remaining four degrees of freedom $A$, $\tilde U$, and $B^\alpha$ do not propagate but follow from constraints.

This again constitutes an important sanity check: The count of propagating degrees of freedom is as expected and yields $21 - 2\times 4 = 13$, just like in general relativity where we have $10 - 2\times 4 = 2$ degrees of freedom. In the latter theory, only the transverse traceless part of the spatial metric tensor propagates and does so according to a massless wave equation. For area metric gravity, the only massless propagating modes turn out to be the transverse traceless tensor $U^{\alpha\beta}$, \emph{which is exactly the perturbation induced by the propagating metric modes} (see \eqref{induced_perturbations}).

All other modes, which are not inducible by the propagating metric modes, follow massive wave equations with mass $\mu$. In the next section, it will become clear that the generation of such modes from matter distributions is suppressed, e.g.~a binary star only radiates on nonmetric tensor modes or on vector or scalar modes when its angular velocity exceeds a certain threshold. This is another realization of the correspondence principle, which demands that Einstein gravity approximate area metric gravity in certain limits.

\section{The binary star}\label{section_binary_star}

\subsection{Iterative solution strategy for gravitational field equations}\label{section_iterative_solution}
Covariant constructive gravity closes matter theories by providing previously unknown dynamics for geometry to which the matter field couples. Let $\phi$ be the matter field in question, coupling locally to a geometric field $G$. Starting from the matter action\footnote{Round parentheses indicate local dependences.} $S_\text{matter}\lbrack\phi, G)$, the closure procedure yields the joint action
\begin{equation}
  S\lbrack G,\phi\rbrack = S_\text{gravity}\lbrack G\rbrack + \kappa S_\text{matter}\lbrack\phi, G),
\end{equation}
where $S_\text{gravity}$ is the action of the constructed theory compatible with the matter theory. The constant $\kappa$ controls the scale of coupling between both fields. Abbreviated as
\begin{equation}
  e\lbrack G\rbrack = \frac{\delta S_\text{grav}}{\delta G},\quad T\lbrack\phi, G) = \frac{\delta S_\text{mat}}{\delta G},\quad f\lbrack\phi,G) = \frac{\delta S_\text{mat}}{\delta \phi},
\end{equation}
the variations with respect to the matter field and the gravitational field yield the Euler-Lagrange equations
\begin{equation}\label{coupled_euler_lagrange}
  e\lbrack G\rbrack = -\kappa T\lbrack\phi, G)\quad\text{and}\quad f\lbrack\phi,G) = 0.
\end{equation}
Such a tightly coupled system is hard to solve in general. Fortunately, it is not our objective to obtain exact solutions---we have expanded the field equations up to second order and only seek to derive effects up to this finite order. Proceeding similarly as in Ref.~\cite{poisson2014gravity}, a solution is constructed iteratively by expanding the geometry formally as
\begin{equation}
  G = N + \sum_{k=1}^\infty\kappa^k H_{(k)}.
\end{equation}
Truncations of the expansion at order $k$ yield approxmiations $G_{(k)}$ of the geometry. We expand the constituents $e$ and $T$ of the Euler-Lagrange equations expand as
\begin{equation}
  \begin{aligned}
    e\lbrack N + H\rbrack = {} & e_{(0)} + e_{(1)}\lbrack H\rbrack + e_{(2)}\lbrack H\rbrack H + \mathcal O(H^3), \\
    T\lbrack\phi, N + H) = {} & T_{(0)}\lbrack\phi\rbrack + T_{(1)}\lbrack\phi,H) + \mathcal O(H^2),
  \end{aligned}
\end{equation}
where $H$ contributes linearly to the first-order terms and quadratically to the second-order terms. We now solve the equations for the gravitational field up to second order by considering the orders zero to two in $\kappa$.

For the \emph{zeroth} iteration, the Euler-Lagrange equations \eqref{coupled_euler_lagrange} are evaluated at $G_{(0)} = N$, resulting in the equation
\begin{equation}\label{zeroth_order}
  e\lbrack N\rbrack = e_{(0)} = 0.
\end{equation}
This just enforces that the expansion point $N$ must solve the gravitational field equations \emph{in vacuo}. Since we explicitly consider this condition when perturbatively constructing theories, Eq.~\eqref{zeroth_order} is solved trivially.

Proceeding with the \emph{first} iteration, we evaluate at $G_{(1)} = N + \kappa H_{(1)}$. Since $e_{(0)} = 0$ already holds from the previous iteration, the first of the two equations simplifies to
\begin{equation}
  e_{(1)}\lbrack H_{(1)}\rbrack = -T_{(0)} \lbrack\phi\rbrack.
\end{equation}
Figuratively speaking, the first correction of the gravitational field is sourced by the matter content on a flat background. Having solved this equation for $H_{(1)}$, the perturbation may be used in order to solve the second equation
\begin{equation}\label{fix_phi}
  f\lbrack\phi,G_{(1)}) = 0 + \mathcal O(\kappa^2).
\end{equation}
The interpretation is similar: A deviation from the flat gravitational field, caused by the presence of matter, makes the matter field deviate from its unperturbed configuration.

The \emph{second} iteration yields an equation for the second-order perturbation $H_{(2)}$ by inserting $G_{(2)} = N + \kappa H_{(1)} + \kappa^2 H_{(2)}$ in the first field equation and simplifying using the lower-order equations. We obtain the result
\begin{equation}\label{second_order}
  e_{(1)}\lbrack H_{(2)}\rbrack = -\kappa^{-1}T_{(0)}\lbrack\phi\rbrack - T_{(1)}\lbrack\phi,H_{(1)}) - e_{(2)}\lbrack H_{(1)}\rbrack + \mathcal O(k),
\end{equation}
where it has to be noted that $\phi$, having been fixed in Eq.~\eqref{fix_phi}, has a dependence on $\kappa H_{(1)}$. Therefore, contributions from $T_{(0)}\lbrack\phi\rbrack$ must only be considered up to order $\kappa^1$ and contributions from $T_{(1)}\lbrack\phi,H_{(1)})$ only up to order $\kappa^0$.

The second-order perturbation $H_{(2)}$ is thus sourced by both the first-order deviations of the gravitational field and the induced motion of the matter field, as will become clear when explicitly solving the binary star in the following section. Aborting the iterative solution procedure at this point, we have found the approximation
\begin{equation}
  G_{(2)} = N + \kappa H_{(1)} + \kappa^2 H_{(2)}
\end{equation}
of the geometry G coupled to $\phi$ and, as a side effect, the trajectory of the matter field $\phi$ on the linearized background $G_{(1)}$.

\subsection{Solution in Einstein gravity}\label{section_einstein_waves}
Before proceeding to make use of the iterative solution strategy and solve the binary star in area metric gravity, let us consider the same problem in Einstein gravity. We will, of course, only reproduce well-established results, but also gain confidence in the approach and become acquainted with the calculations. It is also advantageous to have the metric theory at hand in order to distinguish the uniquely area metric features later on. State-of-the-art methods derived from Einstein gravity (see e.g.~Ref.~\cite{poisson2014gravity}) extend to higher perturbation orders and much more complex matter configurations than the relatively simple case considered here, but they are not applicable to area metric gravity. Rather, we make use of our hand-crafted approach that accommodates nonmetric geometries just as well.

A binary star consists of two slowly moving point masses $m_i$ describing two world lines $\gamma_i\colon\mathbb R\rightarrow M$. The metric field is a section $g$ of the metric bundle and defines the matter action $S_\text{matter}$ via the length functional\footnote{From now on, we do not use geometrized units but state every occurence of the speed of light $c$ and Newton's constant $G$ explicitly.}
\begin{equation}\label{metric_matter}
  S_\text{matter}\lbrack\gamma_{(1)},\gamma_{(2)},g) = \sum_{i=1,2} m_i c \int \mathrm d\lambda \sqrt{g^{-1}(\dot\gamma_{(i)}(\lambda),\dot\gamma_{(i)}(\lambda)}.
\end{equation}
Einstein gravity completes Eq.~\eqref{metric_matter} to a predictive model by providing dynamics for the metric $g$ in terms of the Einstein-Hilbert action
\begin{equation}
  S_\text{gravity}\lbrack g\rbrack = \frac{c^3}{16\pi G}\int\mathrm d^4x \sqrt{-\operatorname{det}g}R.
\end{equation}
We use the parameterization $\gamma_{(i)}^0(\lambda) = ct$ and obtain by variation the Euler-Lagrange equations
\begin{equation}\label{einstein_equation}
  \sqrt{-\operatorname{det}g}\left\lbrack R^{ab} - \frac{1}{2} g^{ab} R\right\rbrack = \frac{8\pi G}{c^3} \sum_{i=1,2} m_i \delta^{(3)}(\vec x-\vec\gamma_{(i)}(t))\frac{\dot\gamma^a_{(i)}\dot\gamma^b_{(i)}}{\sqrt{g^{-1}(\dot\gamma_{(i)},\dot\gamma_{(i)})}}
\end{equation}
and
\begin{equation}\label{geodesic_equation}
  0 = \ddot\gamma^a_{(i)} + \Gamma^a_{\hphantom abc}\dot\gamma^b_{(i)}\gamma^c_{(i)}.
\end{equation}
The first equation \eqref{einstein_equation} consists of the densitized Einstein tensor on the right-hand side and the stress-energy-momentum tensor of the point particle on the left-hand side. Eq.~\eqref{geodesic_equation} is the geodesic equation on a pseudo-Riemannian manifold with the Christoffel symbols $\Gamma^a_{\hphantom abc}$. Using the slow-motion condition
\begin{equation}
  \frac{1}{c} \dot\gamma^\alpha_{(i)} \ll 1,
\end{equation}
the geodesic equation simplifies to
\begin{equation}
  \dot\gamma^0_{(i)} = c\quad\text{and}\quad\frac{1}{c^2}\ddot\gamma^\alpha_{(i)} = - \Gamma^\alpha_{\hphantom\alpha 00}.
\end{equation}

In order to construct the second-order solution for the metric tensor in this setting, we expand $g$ as
\begin{equation}
  g^{ab} = \eta^{ab} + h^{ab} = \eta^{ab} + G h^{ab}_{(1)} + G^2 h^{ab}_{(2)} + \mathcal O(G^3),
\end{equation}
using the Newtonian constant $G$ as coupling constant. Adopting the $3+1$ decomposition \eqref{metric_three_plus_one}--\eqref{metric_split} for the metric tensor as well as the gauge $B=D=0$ and $V^\alpha=0$, the perturbation is given by
\begin{equation}
  h^{00} = -2A, \quad h^{0\alpha} = B^\alpha,\quad h^{\alpha\beta} = - E^{\alpha\beta} - \gamma^{\alpha\beta} C,
\end{equation}
with scalar modes $A$ and $C$, vector modes $B^\alpha$, and transverse traceless tensor modes $E^{\alpha\beta}$.

Since the variation $e\lbrack g\rbrack$ of the Einstein-Hilber Lagrangian with respect to the metric tensor is given by the Einstein tensor, which is derived from the Riemann curvature tensor, the zeroth-order equation $e\lbrack \eta\rbrack = 0$ is already solved---the flat Minkowski metric $\eta$ has zero curvature.

The first-order equation
\begin{equation}
  e_{(1)}\lbrack h_{(1)}\rbrack = -T_{(0)}\lbrack\gamma_{(1)},\gamma_{(2)}\rbrack
\end{equation}
is obtained from the full Euler-Lagrange equations \eqref{einstein_equation} using the well-known expansion of the Einstein tensor to linear order\footnote{See e.g.~\ref{}. The prefactor given by the metric determinant is irrelevant: It expands as $1 + \frac{1}{2}\eta_{\alpha\beta} h^{\alpha\beta}$ and contributes only to zeroth order, because the expansion of the Einstein tensor has \emph{no} zeroth order.} for the right-hand side and---since the left-hand side already contains a factor $G$---the zeroth order of the matter distribution. Split into spatial and temporal contributions, we get
\begin{equation}\label{three_plus_one_metric_lhs}
  \begin{aligned}
    e^{00}_{(1)}\lbrack h\rbrack = {} & \Delta C, \\
    e^{0\alpha}_{(1)}\lbrack h\rbrack = {} & -\frac{1}{2} \Delta B^\alpha - \partial^\alpha \dot C, \\
    e^{\alpha\beta}_{(1)}\lbrack h\rbrack = {} & -\frac{1}{2} \Box E^{\alpha\beta} + \partial^{(\alpha} \dot B^{\beta)} + \gamma^{\alpha\beta}\left\lbrack \ddot C - \frac{2}{3}\Delta(-A + \frac{1}{2} C)\right\rbrack + \Delta^{\alpha\beta}\left\lbrack -A + \frac{1}{2} C\right\rbrack,
  \end{aligned}
\end{equation}
and the only nonzero contribution\footnote{Implementing the slow-motion condition.}
\begin{equation}
  \begin{aligned}
    T^{00}\lbrack\gamma_{(1)},\gamma_{(2)}\rbrack =\hphantom{\vcentcolon} {} & -\frac{8\pi}{c^2}\sum_{i=1,2}m_i\delta^{(3)}(\vec x-\vec \gamma_{(i)}(t)) \\
    =\vcentcolon {} & -\frac{8\pi}{c^2} \rho(\vec x,t),
  \end{aligned}
\end{equation}
such that the first iteration boils down to the Poisson equation
\begin{equation}\label{poisson_equation}
  \Delta C_{(1)} = \frac{8\pi}{c^2}\sum_{i=1,2}m_i\delta^{(3)}(\vec x-\vec \gamma_{(i)}(t)).
\end{equation}
The remaining equations are not sourced by matter and, thus, yield the trivial results $A_{(1)} = \frac{1}{2} C_{(1)}$ and $B_{(1)}^\alpha = 0$. For $E_{(1)}^{\alpha\beta}$, we obtain the massless wave equation \emph{in vacuo},
\begin{equation}
  0 = \Box E_{(1)}^{\alpha\beta},
\end{equation}
which we solve by setting $E_{(1)}^{\alpha\beta}$ to zero.\footnote{Allowing for nonvanishing solutions would place the binary star not on a flat background but on a background filled with gravitational radiation. As long as this radiation is weak enough in order not to interfere with the second-order field equations, they can be included without affecting the phenomenology. For simplicity, however, it is customary to choose the zero solution.}

Solving Eq.~\eqref{poisson_equation} yields the linearized solution
\begin{equation}\label{metric_linear_solution}
  E^{\alpha\beta}_{(1)} = 0,\quad B^\alpha_{(1)} = 0,\quad A_{(1)} = \frac{1}{c^2}\phi,\quad C_{(1)} = \frac{2}{c^2} \phi,
\end{equation}
effectively composed of one scalar field, the Newtonian potential
\begin{equation}\label{newtonian_potential}
  \begin{aligned}
    \phi(\vec x,t) = {} & - \int \mathrm d^3\vec y\frac{\rho(\vec y,t)}{\lvert\vec x-\vec y\rvert} \\
    = {} & - \frac{m_1}{\lvert\vec x-\vec\gamma_{(1)}(t)\rvert} - \frac{m_2}{\lvert\vec x-\vec\gamma_{(2)}(t)\rvert}.
  \end{aligned}
\end{equation}

According to the iterative solution procedure, the world lines $\gamma_{(i)}$ can now be fixed by solving their equations of motion \eqref{geodesic_equation} on the linearized background \eqref{metric_linear_solution}. These equations are governed by the Christoffel symbols, which expand as
\begin{equation}
  \Gamma^\alpha_{\hphantom\alpha 00} = -\frac{1}{2}\partial^\alpha h^{00} - \dot h^{\alpha 0} + \mathcal O(h^2),
\end{equation}
such that on the linearized background provided by the first iteration
\begin{equation}\label{newton_equation}
  \begin{aligned}
    \ddot \gamma^\alpha_{(i)} = {} & - c^2 \Gamma^\alpha_{\hphantom\alpha 00} \\
    = {} & -G \partial^\alpha \phi + \mathcal O(G^2).
  \end{aligned}
\end{equation}
After all, slowly moving matter obeys---to first order---the Newtonian laws of gravity!

This comes with the same inconsistencies that plague Newtonian gravity: As is obvious from the formula \eqref{newtonian_potential}, the potential sourced by a point mass diverges at the very location of the mass itself. Consequently, whenever a particle ``feels'' its own field, which is certainly the case in Eq.~\eqref{newton_equation}, infinities are involved. The culprit is the idealization of the matter distribution as point masses. One of the remedies pointed out in Ref.~\cite{poisson2014gravity} is to forgo this idealization and model the stars as extended fluids---taking the limit of negligible extension where necessary. Alternatively, the diverging integrals may be regularized, which has the same impact on the results. Effectivley, both approaches are implemented the same way: We keep the point mass idealization but discard diverging integrals, i.e.~when solving for the trajectory of the first particle, the diverging term
\begin{equation}
  m_1 \int\mathrm d^3\vec y \frac{\delta^{(3)}(\vec y - \vec\gamma_{(1)}(t))}{\lvert\vec\gamma_{(1)}(t) - \vec y\rvert}
\end{equation}
does not contribute. The same holds \emph{mutatis mutandis} for the second particle.

With this regularization, the stars are subject to the equations of motion
\begin{equation}
  \ddot\gamma^\alpha_{(i)} = -G \sum_{j\neq i} m_j \frac{\gamma^\alpha_{(i)} - \gamma^\alpha_{(j)}}{\prescript{\hphantom{3}}{}\lvert\vec\gamma_{(i)} - \vec\gamma_{(j)}\rvert^3},
\end{equation}
which is the centuries-old Kepler problem. The solutions are given by the various conic sections, depending on the initial conditions. We will consider the bound states and within this sector the configurations with exactly circular orbits. As it will turn out, the additional complexity introduced by excentricities is immaterial for at least some of the new effects that come with the area metric generalization. In this configuration, the bodies have constant separation $r$ and move on trajectories
\begin{equation}\label{circular_orbit}
  \vec\gamma_{(1)}(t) = \frac{m_2}{m} r\vec n,\quad \vec\gamma_{(2)}(t) = -\frac{m_1}{m} r\vec n,
\end{equation}
where $m = m_1 + m_2$ denotes the total mass. The vector $\vec n$ is one of the three basis vectors
\begin{equation}
  \vec n = \begin{pmatrix} \operatorname{cos}\omega t \\ \operatorname{sin}\omega t \\ 0 \end{pmatrix},\quad \vec \lambda = \begin{pmatrix} -\operatorname{sin}\omega t \\ \operatorname{cos}\omega t \\ 0 \end{pmatrix},\quad \vec e_z = \begin{pmatrix}0 \\ 0 \\ 1\end{pmatrix}
\end{equation}
that span the \emph{orbit-adapted} frame\cite{poisson2014gravity}. Both masses reside in the orbital plane spanned by $\vec n$ and $\vec\lambda$, to which $\vec e_z$ is perpendicular. The frame rotates around the axis $\vec e_z$ with angular frequency $\omega$ according to Kepler's third law
\begin{equation}
  \omega^2 = \frac{Gm}{r^3}.
\end{equation}

Based on this configuration of matter content and gravitational field, the second iteration yields the corrections sourced by both the first-order gravitational field itself and by the influence of the gravitational field on the masses. We are only concerned with the propagating degrees of freedom, as our interest lies in radiation emitted into the far zone, so it suffices to consider the purely spatial part from Eq.~\eqref{second_order}\footnote{Note that the labels on the world lines $\gamma_{(i)}$ do not denote perturbation orders but the individual stars.}
\begin{equation}\label{second_iteration_spatial}
  e^{\alpha\beta}_{(1)}\lbrack h_{(2)}\rbrack = -G^{-1} T^{\alpha\beta}_{(0)}\lbrack\gamma_{(1)},\gamma_{(2)}\rbrack - T^{\alpha\beta}_{(1)}\lbrack\gamma_{(1)},\gamma_{(2)},h_{(1)}) - e^{\alpha\beta}_{(2)}\lbrack h_{(1)}\rbrack + \mathcal O(G).
\end{equation}
Again the functionals $e^{\alpha\beta}_{(.)}$ can be read off from the left-hand side of the full Euler-Lagrange equations \eqref{einstein_equation} and the functionals $T^{\alpha\beta}_{(.)}$ follow from the right-hand side.

The contribution from $e^{\alpha\beta}_{(1)}\lbrack h_{(2)}\rbrack$ is already known from Eq.~\eqref{three_plus_one_metric_lhs}. Its projection to the transverse traceless tensor part is given by
\begin{equation}
  e^{\alpha\beta}_{(1)}\lbrack h_{(2)}\rbrack^{\text{TT}} = -\frac{1}{2} \Box E^{\alpha\beta}.
\end{equation}
We find that the first-order matter functional $T^{\alpha\beta}_{(1)}\lbrack\gamma_{(1)},\gamma_{(2)},h_{(1)})$ does not contribute, because each derivative of a spatial trajectory comes with a factor $\omega$, such that the whole functional is proportional to $\omega^2\propto G$. This is already of higher order than considered in the second iteration equation \eqref{second_iteration_spatial}. Reading off the term $T^{\alpha\beta}_{(0)}\lbrack\gamma_{(1)},\gamma_{(2)}\rbrack$ and projecting onto the transverse traceless tensor mode, we arrive at the intermediate expression
\begin{equation}\label{metric_second_order_wave_equation}
  \Box E^{\alpha\beta} = -\frac{16\pi}{Gc^4}\left\lbrack\sum_{i=1,2}m_i\delta^{(3)}(\vec x-\vec\gamma_{(i)}(t))\dot\gamma^\alpha_{(i)}\dot\gamma^\beta_{(i)}\right\rbrack^\text{TT} + 2 e^{\alpha\beta}_{(2)}\lbrack h_{(1)}\rbrack^\text{TT}.
\end{equation}

It remains to derive the contribution from $e^{\alpha\beta}_{(2)}\lbrack h_{(1)}\rbrack$. This is the first and only time where the second order of the Einstein field equations is needed. Thankfully, the field equations have to be evaluated at the result of the first iteration, $h_{(1)}$, which assumes a particularly simple form where all fields are derived from only the Newtonian potential $\phi$. Evaluation of the Einstein field equations at this solution\footnote{Using \texttt{cadabra}, for the corresponding code see Appendix \ref{appendix_cadabra}.} yields the transverse traceless tensor part
\begin{equation}
  e^{\alpha\beta}_{(2)}\lbrack h_{(1)}\rbrack^\text{TT} = -\frac{2}{c^4}\left\lbrack\partial^\alpha\phi\partial^\beta\phi - 2\partial^\alpha(\phi\partial^\beta\phi) \right\rbrack^\text{TT}.
\end{equation}

We are thus left with the wave equation \eqref{metric_second_order_wave_equation}, which is of the kind
\begin{equation}
  \Box\psi(\vec x,t) = 4\pi\varphi(\vec x,t).
\end{equation}
Such an equation is solved by convolution of the source with the retarded Green's function\cite{}
\begin{equation}
  \psi(\vec x,t) = \int\mathrm d^3\vec y\frac{\varphi(\tau,\vec y)}{\lvert\vec x-\vec y\rvert},
\end{equation}
where the source is evaluated at the retarded time
\begin{equation}
  \tau = t - \frac{1}{c}\lvert\vec x-\vec y\rvert.
\end{equation}
For radiation into the far zone, we are only interested in the result at points in spacetime with $R=\lvert\vec x\rvert\gg r$. A first approximation in this regime is given by the zeroth order $\vec x - \vec y \approx R$, which yields the simplified integral
\begin{equation}
  \psi(\vec x,t) = \frac{1}{R}\int\mathrm d^3\vec y\varphi(\tau,\vec y),
\end{equation}
where from now on $\tau = t-R/c$. This approximation is valid to lowest order, because for the first part of the source (the first summand in Eq.~\eqref{metric_second_order_wave_equation}), the integration variable $\vec y$ is confined to the matter distribution, a region of radius $r$, such that
\begin{equation}
  \lvert\vec x-\vec y\rvert \leq \lvert\vec x\rvert + \lvert\vec y\rvert \leq \lvert\vec x\rvert + r = R(1 + \frac{r}{R}) \xrightarrow{\frac{r}{R}\to 0} R.
\end{equation}
For the second part, the source occupies an unbounded region but decreases in magnitude with $\lvert\vec y\rvert^{-4}$, allowing for a similar argument.

The integrals that remain to be evaluated are
\begin{equation}
  K^{\alpha\beta} = \int\mathrm d^3\vec y\sum_{i=1,2}m_i\delta^{(3)}(\vec y-\vec\gamma_{(i)}(\tau))\dot\gamma^\alpha_{(i)}\dot\gamma^\beta_{(i)}
\end{equation}
and, after dropping a boundary term,
\begin{equation}
    U^{\alpha\beta} = \int\mathrm d^3\vec y\partial^\alpha\phi\partial^\beta\phi.
\end{equation}
Evaluating $K^{\alpha\beta}$, whose integrand is a simple delta distribution, gives
\begin{equation}
  K^{\alpha\beta} = \frac{G\eta m^2}{r}\lambda^\alpha\lambda^\beta.
\end{equation}
Here, the reduced mass
\begin{equation}
  \eta = \frac{m_1m_2}{m^2}
\end{equation}
makes its first appearance. In order to evaluate the second integral, we first substitute the Newtonian potential with the unevaluated integral expression \eqref{newtonian_potential}, such that
\begin{equation}
  U^{\alpha\beta} = \int\mathrm d^3\vec y\int\mathrm d^3\vec{y^\prime}\int\mathrm d^3\vec{y^{\prime\prime}} \frac{\rho(\vec{y^\prime})\rho(\vec{y^{\prime\prime}})}{\lvert\vec y-\vec{y^\prime}\rvert^3 \lvert\vec y-\vec{y^{\prime\prime}}\rvert^3} (y^\alpha - y^{\prime\alpha}) (y^\beta - y^{\prime\prime\beta}).
\end{equation}
The integration over $\vec y$ now yields
\begin{equation}
  U^{\alpha\beta} = 2\pi \int\mathrm d^3\vec{y^\prime} \rho(\vec{y^\prime}) \int\mathrm d^3\vec{y^{\prime\prime}}\frac{\rho(\vec{y^{\prime\prime}})}{\lvert\vec{y^\prime}-\vec{y^{\prime\prime}}\rvert} \left\lbrack \gamma^{\alpha\beta} - \frac{(y^{\prime\alpha} - y^{\prime\prime\alpha})(y^{\prime\beta} - y^{\prime\prime\beta})}{\lvert\vec{y^\prime}-\vec{y^{\prime\prime}}\rvert^2}\right\rbrack,
\end{equation}
whose inner integrand is a delta distribution, which after evaluation yields a delta distribution as outer integrand. Making sure not to include diverging terms, as explained earlier, the evaluation yields
\begin{equation}
  U^{\alpha\beta} = \frac{4\pi\eta m^2}{r}\lbrack\gamma^{\alpha\beta} - n^\alpha n^\beta\rbrack.
\end{equation}

We finally put together both parts with the proper constants and the prefactor of $1/R$. The result is the lowest non-trivial order of the gravitational field that is radiated away into the far zone from a binary star in circular motion,
\begin{equation}\label{metric_radiation}
  G^2h^{\alpha\beta}_{(2)} = \frac{4\eta}{c^4 R} \frac{(Gm)^2}{r}\lbrack\lambda^\alpha\lambda^\beta - n^\alpha n^\beta\rbrack^\text{TT},
\end{equation}
parameterized by properties of the matter distribution (total mass $m$, reduced mass $\eta$, separation $r$), the speed of light $c$, and Newton's gravitational constant $G$. This is in exact accordance with the literature\cite{poisson2014gravity} and, of course, no surprise: Contemporary methods employ what is called post-Minkowskian and post-Newtonian theory\cite{poisson2014gravity}, which provides a framework for more complex calculations. However, the pedestrian approach presented here is derived from the same full theory and is thus equally valid.

The strength of this solution procedure is that it does not presuppose knowledge of the exact (i.e.~unperturbed) dynamics and is not restricted to metric theories. Both properties are important for the analysis of the binary star in area metric gravity, a non-metric theory of gravity for which there are no known exact dynamics. Even though we followed a top-down approach and derived the perturbative expansion of the Einstein field equations from its full form, it would have been possible to construct this expansion from the bottom up, order by order. In the following section, this will be the \emph{only} option.

Finally, it should be noted (see also the discussion in Ref.~\cite{poisson2014gravity}) that the radiation emitted by the binary star is indeed an effect of second order. The presence of the masses alone sources a gravitational field, which to first order is given by the Newtonian potential. Under the influence of this field, the masses are confined to circular orbits---a first-order effect. \emph{This} refined motion, together with the first order of the gravitational field\footnote{Via the contribution $e_{(2)}\lbrack h_{(1)}\rbrack$, where the second order of the field equations enters.}, is the source of the gravitational radiation produced in the second iteration. Knowledge of the first-order field equations is not sufficient in order to derive the result \eqref{metric_radiation}, contrary to the impression that the reading of derivations in the older literature might leave \cite{Misner_1973}. Whoever arrives at the conclusion \eqref{metric_radiation}, or its generalization for more general matter configurations called \emph{quadrupole formula}, using only the linearized gravitational field equations either did so out of pure luck, by silently slipping in knowledge about the second order, or by having inadvertently constructed this order during the process. If, for example, the derivation involves some basic assumptions about the theory, such as restrictions concerning derivative orders, and also diffeorphism invariance, it is no surprise that a correct formula may be obtained---after all, as discussed in Chapter \ref{chapter_construction_algorithm}, Einstein gravity is \emph{unique} if certain assumptions are met.

\subsection{Solution in area metric gravity}
In the area metric gravity scenario, the point masses are subject to the action
\begin{equation}\label{area_matter_action}
  S_\text{matter}\lbrack\gamma_{(1)},\gamma_{(2)},G) = \sum_{i=1,2} m_i c\int\mathrm d\lambda \mathcal P_\text{GLED}(L^{-1}(\dot\gamma_{(i)}(\lambda)))^{-\frac{1}{4}},
\end{equation}
which we already encountered when discussing the linearized Schwarzschild solution. This time, the masses are not at rest, such that the ``full'' linearized expression for generic worldlines is needed. It comes in very handy that the GLED polynomial is to first order equivalent to the quadratic polynomial (see Eq.~\eqref{gled_poly_first_order})
\begin{equation}
  P^{(\leq 1)}_\text{GLED}(k) = [1-\frac{1}{24}\epsilon(H)] \eta(k,k) + \frac{1}{2} H(k,k),
\end{equation}
which using the $3+1$ split introduced in Sect.~\ref{sect_three_plus_one} decomposes into
\begin{equation}\label{area_metric_linearized_poly}
    P^{(\leq 1)}_\text{GLED}(k) = \eta(k,k) + \lbrack -2A\rbrack (k_0)^2 + \lbrack -2b^\alpha\rbrack k_0 k_\alpha + \lbrack -\frac{1}{2} u^{\alpha\beta} - \frac{1}{2} \gamma_{\mu\nu} v^{\mu\nu} \gamma^{\alpha\beta} \rbrack k_\alpha k_\beta.
\end{equation}
Since the causality is effectively metric, the integrand in the point particle action \eqref{area_matter_action} is given by the inverse of this metric \cite{R_tzel_2011,Rivera_2012}. To linear order, the inverse is calculated as
\begin{equation}\label{area_matter_linearized_poly}
  \lbrack\eta + h\rbrack^{-1}_{\hphantom{-1}ab} = \eta_{ab} - \eta_{ap}\eta_{pq} h^{pq} + \mathcal O(h^2),
\end{equation}
such that we obtain the linearized action
\begin{equation}\label{area_linearized_matter_action}
  \begin{aligned}
    S_\text{matter}\lbrack\gamma_{(1)},\gamma_{(2)},N+H) = {} & \sum_{i=1,2} m_i c\int\mathrm d\lambda \Big\{ \eta_{ab}\dot\gamma^a_{(i)}\dot\gamma^b_{(i)} + 2A\dot\gamma^0_{(i)}\dot\gamma^0_{(i)} - 2b_\alpha\dot\gamma^0_{(i)}\dot\gamma^\alpha_{(i)} \\
    {} & \quad\quad + \left\lbrack\frac{1}{2}u_{\alpha\beta} + \frac{1}{2}\gamma^{\mu\nu}v_{\mu\nu}\gamma^{\alpha\beta}\right\rbrack \dot\gamma^\alpha_{(i)} \dot\gamma^\beta_{(i)} \Big\} + \mathcal O(H^2).
  \end{aligned}
\end{equation}

In addition to the linearized matter action \eqref{area_linearized_matter_action}, we also need the gravitational action expanded to third order in the area metric perturbation. Sect.~\ref{section_area_construction} was dedicated to the construction of third-order area metric Lagrangian densities $\mathcal L = L \mathrm d^4x$. The result of this construction procedure will be used here in the action
\begin{equation}
  S_\text{gravity}\lbrack N+H\rbrack = \frac{c^3}{16\pi G}\int \mathrm d^4x \mathcal L + \mathcal O(H^4).
\end{equation}

Like before, the zeroth iteration is already solved by construction---the flat instance $N$ of the area metric field solves the vacuum field equations.

Due to the slow-motion condition, the first-order field equations
\begin{equation}
  e_{(1)}\lbrack H_{(1)}\rbrack = - T_{(0)}\lbrack\gamma_{(1)},\gamma_{(2)}\rbrack,
\end{equation}
which where derived in Sect.~\ref{section_iterative_solution}, are only sourced by the variation of the matter action \eqref{area_linearized_matter_action} with respect to the lapse, via the equation
\begin{equation}
  \left(\frac{\delta S_\text{matter}}{\delta A}\right)_{(1)}\lbrack H_{(1)}\rbrack = c \rho(\vec x,t).
\end{equation}
This is similar to the stationary case considered in Sect.~\ref{sect_area_linear_eom}, with the difference that the delta distribution is not centred at the origin but given as
\begin{equation}
  \rho(\vec x,t) = \sum_{i=1,2} m_i \delta^{(3)}(\vec x - \vec\gamma_{(i)}(t)).
\end{equation}
Since the vector and transverse traceless tensor equations are not sourced at all, we solve these by setting the respective modes to zero. Again, it is possible to add background radiation to the solution, as long as it remains negligible. The scalar modes are solved by superposition of the linearized Schwarzschild solutions \eqref{linearized_schwarzschild_solution}, in the integral representation as
\begin{equation}\label{area_schwarzschild_integral}
  \begin{aligned}
    A_{(1)} = {} & -\frac{1}{c^2} \int\mathrm d^3\vec y\rho(\vec y)\left\lbrack\frac{\alpha}{\lvert\vec x-\vec y\rvert} + \frac{\beta\mathrm e^{-\mu\lvert\vec x-\vec y\rvert}}{\lvert\vec x-\vec y\rvert} \right\rbrack, \\
    \tilde V_{(1)} = {} & -\frac{1}{c^2} \int\mathrm d^3\vec y\rho(\vec y)\left\lbrack\frac{\gamma\mathrm e^{-\mu\lvert\vec x-\vec y\rvert}}{\lvert\vec x-\vec y\rvert} \right\rbrack, \\
    \tilde U_{(1)} = {} & 4 A_{(1)} - (3 + 8 \frac{\beta}{\gamma})\tilde V_{(1)},
  \end{aligned}
\end{equation}
or in the evaluated form
\begin{equation}
  \begin{aligned}
    A_{(1)} = {} & -\frac{1}{c^2} \sum_{i=1,2} m_i \left\lbrack\frac{\alpha}{\lvert\vec x-\vec\gamma_{(i)}(t)\rvert} + \frac{\beta\mathrm e^{-\mu\lvert\vec x-\vec\gamma_{(i)}(t)\rvert}}{\lvert\vec x-\vec\gamma_{(i)}(t)\rvert} \right\rbrack, \\
    \tilde V_{(1)} = {} & -\frac{1}{c^2} \sum_{i=1,2} m_i \left\lbrack\frac{\gamma\mathrm e^{-\mu\lvert\vec x-\vec\gamma_{(i)}(t)\rvert}}{\lvert\vec x-\vec\gamma_{(i)}(t)\rvert} \right\rbrack, \\
    \tilde U_{(1)} = {} & 4 A_{(1)} - (3 + 8 \frac{\beta}{\gamma})\tilde V_{(1)}.
  \end{aligned}
\end{equation}
The constants $\alpha$, $\beta$, $\gamma$, and $\mu$ are the four relevant first-order gravitational constants \eqref{schwarzschild_constants} for stationary or slowly moving matter configurations.

For the second part of the first iteration, the matter trajectories have to be fixed. We again exploit the fact that the matter action is effectively metric, because as a consequence of this circumstance, the stars are to this order subject to the same geodesic equation \eqref{newton_equation} as in Einstein gravity. The Christoffel symbols are derived from the effective metric \eqref{area_matter_linearized_poly} with $h^{00} = -2A$, such that
\begin{equation}\label{newton_equation_area}
  \ddot \gamma^{\alpha}_{(i)} = - c^2 G \partial^\alpha A_{(1)},
\end{equation}
where the integrals \eqref{area_schwarzschild_integral} have to be regularized by, effectively, dropping the divergent terms (see Sect.~\ref{section_einstein_waves}).

The equations of motion \eqref{newton_equation_area} constitute a refined Kepler problem. Instead of the Newtonian potential $\propto \frac{1}{r}$, the stars move in modified potentials with additional Yukawa terms $\propto \frac{1}{r} \mathrm e^{-\mu r}$. Still, this potential has a spherical symmetry and circular orbits remain solutions to the geodesic equations. This is seen by making the ansatz \eqref{circular_orbit} and solving for the angular frequency $\omega$, which yields the refined relation
\begin{equation}
  \omega^2 = \frac{(G\alpha) m}{r^3}\left\lbrack 1 + \frac{\beta}{\alpha}\mathrm e^{-\mu r}(1 + \mu r)\right\rbrack,
\end{equation}
i.e.~a modification of Kepler's third law.

Let us start solving the second iteration by considering the massless transverse traceless tensor mode $U^{\alpha\beta}_{(2)}$. The contribution $e_{(1)}\lbrack H_{(2)}\rbrack$ follows from the first of the linearized transverse traceless tensor field equations \eqref{tensor_equations_reduced}, which reads
\begin{equation}
  \bigg\lbrack\frac{\delta L}{\delta u^{\alpha\beta}}\bigg\rbrack^\text{TT} = \frac{1}{8\alpha} \Box U_{\alpha\beta}.
\end{equation}
As before, when solving the metric problem, the vectors tangent to the worldlines contribute a factor of $\sqrt{G}$ each, such that there is no contribution from $T_{(1)}\lbrack\gamma_{(1)},\gamma_{(2)},H_{(1)})$, but only from the lower order $T_{(0)}\lbrack\gamma_{(1)},\gamma_{(2)}\rbrack$. Evaluating this term by variation of the matter action \eqref{area_linearized_matter_action} with respect to the field $u^{\alpha\beta}$, we obtain the equation
\begin{equation}\label{area_massless_tt_equation}
  -\frac{c^3G}{16\pi} \frac{1}{8\alpha} \Box U^{\alpha\beta}_{(2)} = \left\{\frac{1}{4c}\sum_{i=1,2} m_i \delta^{(3)}(\vec x-\vec\gamma_{(i)}(t))\dot\gamma^\alpha_{(i)}\dot\gamma^\beta_{(i)} + \left(\frac{\delta S_\text{gravity}}{\delta u_{\alpha\beta}}\right)_{(2)}\lbrack GH_{(1)}\rbrack \right\}^\text{TT}.
\end{equation}

The contribution from the second-order field equations is again calculated using \texttt{cadabra}. The process is roughly as follows: First, the the third-order ansätze (see Appendix \ref{}) and the corresponding coefficient relations (see Appendix \ref{}) are loaded into the programme. Then, the Lagrangian is decomposed into observer quantities, shift, and lapse, according to the $3+1$ decomposition introduced in Sect.~\ref{sect_three_plus_one}. All fields, except for the field $u^{\alpha\beta}$ which will be varied, are replaced with the solution from the first iteration, using abbreviations
\begin{equation}
  \begin{aligned}
    X = &{} \int\mathrm d^3\vec y\rho(\vec y)\frac{1}{\lvert\vec x-\vec y\rvert}, \\
    Y = &{} \int\mathrm d^3\vec y\rho(\vec y)\frac{\mathrm e^{-\mu\lvert\vec x-\vec y\rvert}}{\lvert\vec x-\vec y\rvert}.
  \end{aligned}
\end{equation}
This simplifies the Lagrangian significantly, because it only depends on the scalar fields $X$ and $Y$, as well as the field $u^{\alpha\beta}$. Performing the variation with respect to the remaining tensorial field, projecting the result onto the transverse traceless tensor mode, and further simplifying finally yields
\begin{equation}
  \left(\frac{\delta S_\text{gravity}}{\delta u_{\alpha\beta}}\right)^\text{TT}_{(2)}\lbrack GH_{(1)}\rbrack = \frac{G}{16\pi c}\lbrack\alpha\partial^\alpha X\partial^\beta X + \beta\partial^\alpha Y\partial^\beta Y\rbrack^\text{TT}.
\end{equation}
The \texttt{cadabra} code can be found in Ref.~\cite{Alex_2020_area-metric-gravity} and in Appendix \ref{}.

Being also a massless wave equation, the differential equation \eqref{area_massless_tt_equation} is solved like before in Sect.~\ref{section_einstein_waves}, by convolution of the right-hand side with the retarded Green's function of the d'Alembert operator. Taking the same limit for the far zone, the solution is
\begin{equation}
  U^{\alpha\beta}_{(2)} = -\frac{\alpha}{c^4R}\left\lbrack \frac{8}{G} K^{\alpha\beta} + \frac{2\alpha}{\pi}\Phi^{\alpha\beta}_{(0)} + \frac{2\beta}{\pi}\Phi^{\alpha\beta}_{(\mu)} \right\rbrack^\text{TT},
\end{equation}
with a kinetic term
\begin{equation}
  K^{\alpha\beta} = \int\mathrm d^3\vec y\sum_{i=1,2}m_i\delta^{(3)}(\vec y-\vec\gamma_{(i)}(\tau))\dot\gamma^\alpha_{(i)}\dot\gamma^\beta_{(i)}
\end{equation}
and the potential terms
\begin{equation}
  \Phi_{(\mu)\alpha\beta} = \int\mathrm d^3\vec y \int\mathrm d^3\vec{y^\prime} \int\mathrm d^3\vec{y^{\prime\prime}} \rho(\vec{y^\prime}) \rho(\vec{y^{\prime\prime}}) \left(\partial_\alpha\frac{\mathrm e^{-\mu\lvert\vec z\rvert}}{\lvert\vec z\rvert}\right)_{\!\!\vec z=\vec y-\vec{y^\prime}} \left(\partial_\beta\frac{\mathrm e^{-\mu\lvert\vec z\rvert}}{\lvert\vec z\rvert}\right)_{\!\!\vec z=\vec y-\vec{y^{\prime\prime}}}.
\end{equation}

Working out the integrals results in a first prediction for the gravitational radiation produced by a binary star subject to area metric gravity. On the massless transverse traceless tensor mode, radiation is emitted into the far zone $R\gg r$ according to the formula
\begin{equation}\label{area_metric_radiation_massless}
  G^2 U^{\alpha\beta}_{(2)} = -\frac{8\eta}{c^4 R} \frac{(G\alpha m)^2}{r}\lbrack 1+f(r)\rbrack \lbrack\lambda^\alpha\lambda^\beta - n^\alpha n^\beta\rbrack^\text{TT},
\end{equation}
where the correction term $f(r)$ is given by
\begin{equation}
    f(r) = \frac{\beta}{\alpha}(1 + \mu r)\mathrm e^{-\mu r}.
\end{equation}

In order to point out the significance of Eq.~\eqref{area_metric_radiation_massless}, let us come back to the metric radiation formula \eqref{metric_radiation} for the modes $E^{\alpha\beta}$. The area metric result amounts to the metric result up to a correction proportional to $f(r)$, which---being of Yukawa type---falls off exponentially with the separation $r$. In formulae,
\begin{equation}
  G^2 U^{\alpha\beta}_{(2)} = 2 (\alpha G)^2 E^{\alpha\beta}_{(2)}\lbrack 1 + f(r)\rbrack.
\end{equation}
Considering that the area metric perturbation induced by the metric perturbation \eqref{metric_radiation} has the only non-vanishing modes $U_{(2)}^{\alpha\beta} = 2 E_{(2)}^{\alpha\beta}$, we arrive at the remarkable conclusion that---on the metrically inducible modes and in the far zone---the radiation emitted by a binary star in circular motion is qualitatively the same, but quantitatively \emph{refined}. Both Kepler's third law and the amplitude of the emitted waves pick up Yukawa corrections, which originate from the presence of mass terms in the scalar field equations for stationary and slowly moving sources. These corrections can become arbitrarily small---by restricting the parameter range or considering large enough radii $r$. In this sense, gravitational radiation as predicted in Einstein gravity is contained within the area metric result.

The remaining propagating modes of the area metric perturbation have all shown to be governed by massive wave equations (see Sect.~\ref{sect_area_linear_eom}). Let us first consider the traceless modes, i.e.~the massive transverse traceless tensors, the vectors, and the traceless scalars. Since the coefficients in the relevant wave equations \eqref{scalar_equations_reduced}--\eqref{tensor_equations_reduced} are the same for all of these modes, regardless of whether the equations are of scalar, vector, or tensor type, all propagating traceless modes can be considered on the same footing.

We define the tracefree auxiliary field $\tilde v^{\alpha\beta} = v^{\alpha\beta} - \frac{1}{3}\gamma^{\alpha\beta}\gamma_{\mu\nu}v^{\mu\nu}$. Taking appropriate linear combinations of the linearized field equations (see e.g.~\eqref{tensor_equations_decoupled}), the modes $\tilde v^{\alpha\beta}$ decouple from the modes $w^{\alpha\beta}$, such that the left-hand side $e_{(1)}\lbrack H_{(2)}\rbrack$ of the second iteration equations is given by $\Box\tilde v^{\alpha\beta}_{(2)} + \nu^2\tilde v^{\alpha\beta}_{(2)}$ and $\Box w^{\alpha\beta}_{(2)} + \nu^2 w^{\alpha\beta}_{(2)}$, respectively. Because the linearized matter action \eqref{area_linearized_matter_action} does only depend on the trace of $v^{\alpha\beta}$ and is entirely independent of $w^{\alpha\beta}$, there is no matter contribution to the second iteration. A calculation of the contribution $e_{(2)}\lbrack H_{(1)}\rbrack$, employing the previously outlined \texttt{cadabra}-based technique, yields the wave equations
\begin{equation}\label{tracefree_second_order}
  \begin{aligned}
    \Box\tilde v^{\alpha\beta}_{(2)} + \nu^2\tilde v^{\alpha\beta}_{(2)} = {} & \delta \lbrack\partial^\alpha X\partial^\beta Y\rbrack^\text{TF}, \\
    \Box w^{\alpha\beta}_{(2)} + \nu^2w^{\alpha\beta}_{(2)} = {} & \epsilon \lbrack\partial^\alpha X\partial^\beta Y\rbrack^\text{TF}.
  \end{aligned}
\end{equation}
The label $[\cdot]^\text{TF}$ denotes the idempotent projection
\begin{equation}
  \lbrack t^{\alpha\beta}\rbrack^\text{TF} = t^{(\alpha\beta)} - \frac{1}{3}\gamma_{\mu\nu}t^{\mu\nu}\gamma^{\alpha\beta}
\end{equation}
onto the tracefree symmetric part. Both $\delta$ and $\epsilon$ are combinations of gravitational constants that include genuine third-order constants, i.e.~coefficients for the third order in the area metric Lagrangian expansion that are not solely determined by second-order coefficients.\footnote{This has been the case earlier: For the massless mode, the second iteration equation---although derived from the third-order expansion---was completely determined by second-order coefficients of the Lagrangian expansion. The perturbative equivariance equations can, and in general will, in each iteration determine some higher-order coefficients by lower-order coefficients.}

Solving the wave equations \eqref{tracefree_second_order} is again a matter of convoluting the source terms with a retarded Green's function. This time, the differential equation is of the kind
\begin{equation}
  (\Box + m^2)\psi(x) = 4\pi\varphi(x)
\end{equation}
and thus solved by the massive propagator\cite{}
\begin{equation}
  G_\text{ret}(x,y) = \theta(x^0-y^0)\int\frac{\mathrm d^3\vec k}{(2\pi)^3}\frac{\operatorname{sin}\lbrack\omega_k(x^0-y^0)\rbrack}{\omega_k} \mathrm e^{\mathrm i\vec k\cdot(\vec x-\vec y)},
\end{equation}
where $\theta$ is the Heaviside step function and the massive dispersion relation
\begin{equation}
  \omega_k = \sqrt{\lvert\vec k\rvert^2 + m^2}
\end{equation}
holds. The convolution integrals work out differently this time, depending on the value of $\omega_0 \vcentcolon= 2\omega$.
\begin{enumerate}
  \item \underline{nonradiating solution}: $\omega_0 < c\nu$ \\ The gravitational fields decay exponentially with distance $R$ from the binary system, e.g.~in the orbit-adapted frame they are given as
    \begin{equation}\label{nonradiation_solution}
      G^2\tilde v^{\alpha\beta}_{(2)} = \frac{\delta\eta}{c^4R}\frac{(Gm)^2}{r} g(r) \left\lbrack 3\mathrm e^{-\tilde\nu R}\begin{pmatrix}\operatorname{cos}\omega_0t & \operatorname{sin}\omega_0t & 0 \\ \operatorname{sin}\omega_0t & -\operatorname{cos}\omega_0t \\ 0 & 0 & 0 \end{pmatrix} + \mathrm e^{-\nu R} \begin{pmatrix}\frac{1}{2}\hspace{2em}\\\hspace{0.5em}\frac{1}{2}\hspace{0.5em}\\\hspace{1.5em}-1\end{pmatrix}\right\rbrack^{\alpha\beta}
    \end{equation}
    with the abbreviations
    \begin{equation}
      \begin{aligned}
        g(r) = {} & \frac{1 - \lbrack 1 + \mu r + \frac{1}{3}(\mu r)^2\mathrm e^{-\mu r}\rbrack}{(\mu r)^2}, \\
        \tilde\nu = {} & \sqrt{\nu^2 - \left(\frac{\omega_0}{c}\right)^2}.
      \end{aligned}
    \end{equation}
    For the traceless modes $w^{\alpha\beta}$, we obtain the same solution, but with prefactor $\epsilon$ instead of $\delta$. Note that the oscillating part has a ``direct'' dependence on the coordinate time $t$. Characteristic behaviour of a radiating solution would be a dependence through the retarded time $\tau$.
  \item \underline{radiating solution}: $\omega_0 > c\nu$ \\ The nonoscillating part of the previous solution \eqref{nonradiation_solution} is not affected. Since it decreases exponentially with $R$ and we are interested in the far zone $R \gg r$, it will be dropped from now on---being shadowed by another contribution that is proportional to $\frac{1}{R}$. This term \emph{radiates} according to
    \begin{equation}
      G^2\tilde v^{\alpha\beta}_{(2)} = \frac{3\delta\eta}{c^4R}\frac{(Gm)^2}{r}g(r)\lbrack n^\alpha n^\beta - \lambda^\alpha\lambda^\beta\rbrack,
    \end{equation}
    where the phase of the orbit-adapted frame is now $\frac{\varphi}{2}$ with
    \begin{equation}\label{retardation}
      \varphi = \omega_0t-\sqrt{\omega_0^2 - (c\nu)^2} \frac{R}{c} =\vcentcolon \omega_0t - \tilde\omega\frac{R}{c}.
    \end{equation}
    Earlier, for the massless modes, we had $\varphi = \omega_0t$, or equivalently $\frac{\varphi}{2} = \omega t$. Again, the solution for $w^{\alpha\beta}$ is the same up to the prefactor of $\delta$, which has to be replaced with $\epsilon$. The dependence on coordinate time is only via a retardation term \eqref{retardation}.
\end{enumerate}

The fact that radiation is ``switched on'' only above a certain angular frequency threshold is an expected and welcome property. It is \emph{expected} because of the mass $\nu$ in the wave equations. An analogy would be a massive particle in relativistic quantum field theory, which requires a minimum energy---its mass---in order to be created. Earlier results \cite{} in area metric gravity discovered a similar behaviour for electromagnetically bound binaries, which has now been shown to extend to gravitationally bound systems, where radiation is an effect of gravitational self-coupling. The result is certainly encouraging for the viability of area metric gravity, as it once again keeps the theory very close to Einstein gravity and introduces only modest modifications, assuming that parameters are chosen appropriately. Without this property, it would be impossible to reconcile area metric gravity with the metric theory, ruling it out as a candidate for a modified theory of gravity. On the other hand, we observe a new \emph{quality}: propagation of massive gravitational waves on nonmetric\footnote{In the sense of not inducible by a metric tensor.} modes.

One propagating degree of freedom has not been considered so far. The scalar degree of freedom $\tilde V$ obeys a massive wave equation \eqref{trace_massive_wave}, which is of mass $\mu$, but comes with an additional complexity: it is a linear combination of scalar field equations, such that the second iteration equation takes the form
\begin{equation}\label{trace_scalar_equation}
  \Box\tilde V_{(2)} + \mu^2 \tilde V_{(2)} = -\gamma\left\lbrack\frac{1}{4}\rho_A - \left(1 + \frac{3}{4}\frac{\gamma}{\beta}\right)\rho_u + \frac{\gamma}{4\beta}\rho_v\right\rbrack.
\end{equation}
The three source terms $\rho_u$, $\rho_v$, and $\rho_A$ denote the right-hand sides of the second iteration equations \eqref{second_order} that originate from variations of the actions with respect to $u^{\alpha\beta}$, $v^{\alpha\beta}$, and $A$, respectively. Except for the $A$ variation, the trace has to be taken afterwards. For our previous results, we only picked up zeroth-order contributions from the variations of the matter action, such as
\begin{equation}
  \left(\frac{\delta S_\text{matter}}{\delta u^{\alpha\beta}}\right),
\end{equation}
because the fields would couple to the spatial components of the particle worldline tangents. These come with factors of $\sqrt G$, rendering the first-order contributions a higher order than considered for the second iteration. For the variation with respect to the lapse, this is \emph{not} the case, as an expansion of the GLED polynomial to second order shows. The lapse perturbation comes with terms proportional to $\lbrack\dot\gamma^0_{(i)}\rbrack^4=c^4=\mathcal O(G^0)$, which illustrates that we have to expect contributions $T_{(1)}\lbrack\gamma_{(1)},\gamma_{(2)},H_{(1)})$ in the second iteration equation \eqref{trace_scalar_equation}.

Unfortunately, the second-order expansion of the GLED principal polynomial is not effectively metric anymore. The straightforward way of applying the Legendre transform---lowering indices with the help of the covariant metric---is not available in this case. In fact, there is no closed expression for the Legendre transform corresponding to a quartic polynomial.\cite{Rivera_2012} While it is certainly possible to treat the problem perturbatively, it is considered out of scope for the purpose of this thesis. After all, we do not seek to derive a comprehensive solution, but rather wish to demonstrate the ramifications of novel matter dynamics on the gravitational phenomenology. The results for the massless transverse traceless tensor modes, as well as the massive tracefree modes, already allow to make nontrivial predictions concerning the binary star, such that more extensive knowledge of the remaining scalar modes is deemed dispensable.

\section{Phenomenology of area metric gravitational radiation}
The quintessence of Sect.~\ref{section_binary_star} is the prediction of gravitational waves emitted by a binary star subject to area metric gravity. While compatible with the behaviour of Einstein gravity in certain limits, the area metric result offers new features such as a modification of Kepler's third law, which determines the angular frequency, or radiation on massive modes. All of these effects, however, concern the \emph{gravitational field} and are thus inaccessible to direct observations. This is because the geometric fields only play an auxiliary role in the ensemble of physical fields. Observable effects of gravity involve the \emph{matter fields} whose dynamics is governed by the geometry in question.

In order to derive observable predictions from the previous results, we will first consider a distribution of test matter and study the effect of a passing gravitational wave. This yields the usual deformations known as geodesic deviation, amended by novel deformation patterns. A more direct effect of the radiation that is emitted by a binary star is its energy loss, which makes the binaries reduce their distance and spin faster as the system loses energy through radiation. This will serve as second prediction.

\subsection{Effect on test matter}
Let us probe the gravitational field using an arrangement of matter called a \emph{geodesic sphere}. It is composed of freely falling point masses that are, at least initially, distributed spherically on a spatial hypersurface. The masses are \emph{test masses}, which is to say that their gravitational field is negligible compared to the field wie like to probe, the incident gravitational wave. To first order, the dynamics of point masses in area metric gravity is effectively metric (see Eq.~\eqref{gled_poly_first_order}). The standard procedures from metric gravity for studying the motion of point masses are thus applicable, including the geodesic deviation equation\cite{Misner_1973}
\begin{equation}
  \frac{1}{c^2} \ddot X^\alpha = - R^\alpha_{\hphantom\alpha 0\beta 0} X^\beta
\end{equation}
for the spatial deviation vector $\vec X$. Applying the $3+1$ split of a metric tensor (see Eqns.~\eqref{metric_three_plus_one} and \eqref{metric_expansion}) to the effective metric, the Riemann tensor $R^\alpha_{\hphantom\alpha 0\beta 0}$ can be expanded to linear order, such that the deviation equation assumes the form
\begin{equation}
  \ddot X^\alpha = -\frac{1}{2}\lbrack\ddot\varphi^\alpha_{\hphantom\alpha\beta} + c(\partial_\beta\dot b^\alpha + \partial^\alpha\dot b_\beta) + 2c^2\partial^\alpha\partial_\beta A\rbrack X^\beta.
\end{equation}
For small perturbations, the deviation due to purely spatial fields $\varphi^{\alpha\beta}$ is integrated as
\begin{equation}
  X^\alpha(t) = X^\alpha(0) - \frac{1}{2}\varphi^\alpha_{\hphantom\alpha\beta}(t)X^\beta(0).
\end{equation}
This constitutes the starting point for the following predictions.

\textbf{rethink the calculation: maybe only tt contributions?? or tt + scalar??}

\subsection{Binar star spin-up}
While the gravitational radiation that causes geodesic deviation is produced as second-order effect, the deviation \emph{per se} has only been studied to first order in the previous section. Similar results would hold for incident waves that have their origin in linearized gravity, like the radiation that is emitted from nongravitationally bound systems \cite{}.

An effect that cannot be observed in a solely linearized setting is \emph{radiation reaction}: during the iterative solution procedure, the matter trajectories have been fixed to first order, which provides the background for the second-order gravitational field we were interested in. This is by no means necessary---it is possible to solve for higher orders $n$ before fixing the matter fields, which then enables the prediction of the gravitational field to order $n+1$. The modern treatment of post-Newtonian and post-Minkowskian general relativity proceeds in exactly this way \cite{poisson2014gravity}. Doing so, the matter trajectories accumulate corrections, which are \emph{backreactions} from the gravitational field sourced by the matter content itself. In the context of a gravitationally bound matter distribution which emits gravitational radiation to second order, these backreactions are often referred to as radiation reaction.

In general relativity, the aforementioned modern perturbative treatment yields detailed predictions for the deviation of a binary system from the kepler orbits. These calculation are quite intricate, taking into account not only higher perturbation orders but also the internal structure of the binaries. For a tentative qualitative result, however, we do not need to go there. Noether's second theorem (see Thm.~\ref{thm_second_noether}) provides us with a tool that resembles energy conservation equations.\footnote{The notion of \emph{energy} in general relativity and, more specifically, \emph{energy conservation} is subject to many debates. General relativity does not exhibit the kind of time-translation symmetry that is usually the justification for the definition of energy. In the setting considered here, where the matter content is localized to a specific region of spacetime and the geometry is asymptotically flat, such a notion can be recovered from symmetries that hold asymptotically. \cite{} For our purposes, we do not rely on the interpretation of certain quantities as energies or momenta. They are just derived quantities from the fundamental fields. Changes in these quantities are interesting insofar as they pertain to changes in the underlying fields.} Loosely speaking, gravitational waves radiate away energy from the system, decimating the radius of the kepler orbits, which results---via Kepler's third law---in the binary star \emph{spinning up}. The qualitative analysis from ``energy conservation'' yields a rate of change for the angular velocity but cannot predict how exactly the trajectories are affected, e.g.~how the phase shifts. Still, the prediction\footnote{Considering, of course, the excentric case with the appropriate parameters.} for the orbital period decrease $\mathrm dP/\mathrm dt$ of the Hulse-Taylor pulsar $\text{PSR} 1913+6$ is in very strong agreement with the measurement, as the ratio amounts to \cite{poisson2014gravity}
\begin{equation}
  \frac{(\mathrm dP/\mathrm dt)_\text{observed}}{(\mathrm dP/\mathrm dt)_\text{predicted}} = 0.997 \pm 0.002.
\end{equation}

Again, let us first illustrate the calculation for metric gravity before diving into area metric gravity. The total Lagrangian density of Maxwell electrodynamics and Einstein gravity is
\begin{equation}\label{metric_noether}
  L = L_\text{matter} + L_\text{gravity}.
\end{equation}
Following a normal coordinate argument\footnote{For the metric tensor bundle, there are always local coordinates such that $g^{ab}_{\hphantom{ab},p}=0$. In this coordinate chart, the divergence of the SEM tensor density vanishes. Being a tensor density, it follows that the components vanish in \emph{any} coordinate chart.} from Ref.~\cite{Gotay_1992}, the second Noether identity \eqref{second_noether} reduces to the vanishing of the divergence of the Gotay-Marsden stress-energy-momentum tensor density, i.e.~
\begin{equation}
  0 = \partial_n\left\lbrack\mathcal  T^n_m(g)\right\rbrack.
\end{equation}
If the section $g$ of the metric bundle satisfies the Euler-Lagrange equations, the integral equation
\begin{equation}
  \begin{aligned}
    0 = {} & \int_\Sigma \partial_0\left\lbrack \gmc{A}{B}{0}{0} g^B \frac{\delta L}{\delta g^A}\right\rbrack\mathrm d^3x \\
    = {} & \int_\Sigma \partial_0\left\lbrack \gmc{A}{B}{0}{0} g^B \frac{\delta L_\text{matter}}{\delta g^A}\right\rbrack\mathrm d^3x + \int_\Sigma \partial_0\left\lbrack \gmc{A}{B}{0}{0} g^B \frac{\delta L_\text{gravity}}{\delta g^A}\right\rbrack\mathrm d^3x
  \end{aligned}
\end{equation}
for a spatial slice $\Sigma$ vanishes. Renaming the first term and making use of the Noether identity \eqref{metric_noether} for the second term, this yields
\begin{equation}
  \begin{aligned}
    0 = {} & \partial_0 \underbrace{\int_\Sigma \gmc{A}{B}{0}{0} g^B \frac{\delta L_\text{matter}}{\delta g^A}\mathrm d^3x}_{=\vcentcolon \mathcal H_\text{matter}} + \int_\Sigma \partial_0\left\lbrack \gmc{A}{B}{0}{0} g^B \frac{\delta L_\text{gravity}}{\delta g^A}\right\rbrack\mathrm d^3x \\
    = {} & \frac{1}{c} \dot{\mathcal H}_\text{matter} - \int_\Sigma \partial_\alpha\left\lbrack \gmc{A}{B}{\alpha}{0} g^B \frac{\delta L_\text{gravity}}{\delta g^A}\right\rbrack\mathrm d^3x.
  \end{aligned}
\end{equation}
Finally, we apply the Gauß theorem to the second integral, such that
\begin{equation}\label{first_balance_equation}
  \dot{\mathcal H}_\text{matter} = c \int_{S_\infty} \gmc{A}{B}{\alpha}{0} g^B \frac{\delta L_\text{gravity}}{\delta g^A} \mathrm dS_\alpha,
\end{equation}
which should be understood as the limit of surface integrals over a family of appropriate closed surfaces that approach infinity.

The variations of both Lagrangian have already been worked out for Eq.~\eqref{einstein_equation}. For the matter part, we obtain to lowest non-trivial order
\begin{equation}
  \mathcal H_\text{matter} = 2\int_\Sigma g^{0a}\frac{\delta L_\text{matter}}{\delta g^{0a}}\mathrm d^3x \approx -\frac{1}{c} \lbrack mc^2 + \frac{1}{2}m\eta r^2\omega^2\rbrack = \frac{1}{c}\lbrack E_0 + E\rbrack
\end{equation}
with the constant energy $E_0 = -mc^2$ for the system at rest and the first-order energy
\begin{equation}
  E = -\frac{1}{2}m\eta r^2\omega^2 = -\frac{1}{2} \eta \frac{Gm^2}{r}.
\end{equation}
Thus, the left-hand side of the balance equation \eqref{first_balance_equation} is given as
\begin{equation}\label{balance_lhs}
  \dot{\mathcal H}_\text{matter} = \frac{1}{c}\dot E = \frac{1}{2c}\eta G m^2 \frac{\dot r}{r^2}.
\end{equation}
For the right hand side, we have the full densitized Einstein tensor $\mathcal G^{ab} = \frac{16\pi G}{c^3} \frac{\delta L_\text{gravity}}{\delta g_{ab}}$ at our disposal. To lowest order, the integral amounts to
\begin{equation}\label{metric_spindown_rhs}
  \begin{aligned}
    c\int_{S_\infty} \gmc{A}{B}{\alpha}{0} \frac{\delta L_\text{gravity}}{\delta g^A}g^B\mathrm dS_\alpha = {} & \frac{c^4}{8\pi G}\int_{S_\infty} g_{0a} \mathcal G^{a\alpha}\mathrm dS_\alpha \\
    \approx {} & \frac{c^4}{8\pi G} \int_{S_\infty} \mathcal G^{0\alpha} \mathrm dS_\alpha.
  \end{aligned}
\end{equation}
Since the integral is evaluated at infinity, only the radiation part is relevant. But this part is given by transverse traceless tensor perturbations, such that divergences and traces of the spatial perturbation $h^{\mu\nu}$ can readily be dropped when extracting the lowest non-vanishing order of the integral \eqref{metric_spindown_rhs}. Doing so, we arrive at
\begin{equation}
  \begin{aligned}
    \int_{S_\infty} \mathcal G^{0\alpha}\mathrm dS_\alpha = {} & -\frac{1}{4} \int_{S_\infty} \lbrack \partial^\alpha h_{\mu\nu}\partial_0 h^{\mu\nu} + 2\partial_0\partial^\alpha h_{\mu\nu} h^{\mu\nu}\rbrack \mathrm dS_\alpha \\
    = {} & \frac{1}{4c^2} \int_{S_\infty} \lbrack \dot h_{\mu\nu}\dot h^{\mu\nu} + 2\ddot h_{\mu\nu} h^{\mu\nu}\rbrack \mathrm dS \\
  = {} & -\frac{1}{4c^2} \int_{S_\infty} \dot h_{\mu\nu}\dot h^{\mu\nu} \mathrm dS + \frac{1}{4c^2}\int_{S_\infty} \lbrack h_{\mu\nu}h^{\mu\nu}\rbrack^{\scalebox{1.5}{$\cdot\cdot$}} \mathrm dS,
  \end{aligned}
\end{equation}
which we further simplified using the identity (letting $N^\alpha = \frac{1}{R}x^\alpha$)
\begin{equation}
  \partial_\alpha h^{\mu\nu} = -N^\alpha \dot h^{\mu\nu} + \mathcal O\left(\frac{1}{R^2}\right)
\end{equation}
for radiation terms $\propto\frac{1}{R}f(\omega(ct-R))$.

It is now time to take the concrete form of the metric perturbation into account. Earlier, we arrived at the result \eqref{metric_radiation}
\begin{equation}
  h^{\mu\nu} = \frac{4\eta}{c^4 R}\frac{(Gm)^2}{r}\lbrack n^\alpha n^\beta-\lambda^\alpha\lambda^\beta\rbrack^{\scalebox{1.5}{$\cdot$}} (\mathrm{TT})^{\mu\nu}_{\hphantom{\mu\nu}\alpha\beta},
\end{equation}
where this time the projection onto the transverse traceless tensor mode is made explicit using the projector (see \cite{poisson2014gravity})
\begin{equation}
  \begin{aligned}
    (\mathrm{TT})^{\mu\nu}_{\hphantom{\mu\nu}\alpha\beta} = {} & P^{(\mu}_{\hphantom{(\mu}\alpha} P^{\nu)}_{\hphantom{\nu}\beta} - \frac{1}{2} P^{\mu\nu} P_{\alpha\beta}, \\
    P^{\alpha}_{\hphantom{\alpha}\beta} = {} & \delta^\alpha_\beta - N^\alpha N_\beta.
  \end{aligned}
\end{equation}
Under these circumstances, the contraction $h_{\mu\nu} h^{\mu\nu}$ is constant with respect to coordinate time and thus does not contribute to the integral. The vectors $n$ and $\lambda$ only depend on the radius $R$ and coordinate time $t$, such that the angular dependence is completely contained within the TT projector. This reduces the integral to
\begin{equation}
  \int_{S_\infty}\mathcal G^{0\alpha}\mathrm dS_\alpha = -\frac{256\pi\eta^2(Gm)^5}{c^{10}r^5} n^\alpha n_\mu \lambda^\beta \lambda_\mu \langle(\mathrm{TT})^{\mu\nu}_{\hphantom{\mu\nu}\alpha\beta}\rangle,
\end{equation}
where only the spherical average
\begin{equation}
  \langle X\rangle \vcentcolon= \frac{1}{4\pi}\int_S X \mathrm d\Omega
\end{equation}
of the projector remains to be calculated. Referring to Ref.~\cite{poisson2014gravity} for the details, we just make use of the result
\begin{equation}
  \langle(\mathrm{TT})^{\mu\nu}_{\hphantom{\mu\nu}\alpha\beta}\rangle = \frac{2}{5} \delta^{(\mu}_\alpha\delta^{\nu)}_\beta
\end{equation}
and arrive at
\begin{equation}
  \int_{S_\infty}\mathcal G^{0\alpha}\mathrm dS_\alpha = -\frac{256\pi}{5}\eta^2\left(\frac{Gm}{c^2r}\right)^5.
\end{equation}
Together with the left-hand side \eqref{balance_lhs} of the balance equation \eqref{first_balance_equation}, this yields a first approximation for the spin-up of a binary star due to radiation loss. The separation $r$ of the stars decreases with the rate
\begin{equation}
  \dot r = -\frac{64}{5}\eta c\left(\frac{Gm}{c^2r}\right)^3,
\end{equation}
which translates into an increase of the angular velocity $\omega$, according to Kepler's third law.

This result for the lowest-order radiation loss approximation is in agreement with the literature \cite{poisson2014gravity}. While we had prior knowledge of the full Lagrangian and the corresponding Einstein equations, the approach is not restricted to such theories. A perturbatively constructed third-order Lagrangian can be used just as well and will yield a comparable prediction of binary star spin-up in area metric gravity.

The area metric calculation starts out similarly. Even though the right-hand side of the Noether identity \eqref{second_noether} does not vanish this time, it can be neglected because it is always of one order higher than the lowest order of the left-hand side, due to the appearance of $G^A_{,m} = H^A_{,m}$. Let us also consider only radiation on the massless TT mode $U^{\alpha\beta}$. This is sufficient in order to derive a non-trivial effect and it can be interpreted as the phase of binary spin-up during which the angular velocity is not yet high enough in order for the system to produce massive waves on the nonmetric modes.

As we did before when deriving Eq.~\eqref{second_noether_area_linear}, we start by inverting the relation \eqref{area_metric_perturbation} between the spacetime area metric perturbation and the perturbed observer quantities. For the matter contribution on the left-hand side of the balance equation, this allows us to calculate the variation by only varying with respect to $A$, as
\begin{equation}
  \gmc{A}{B}{0}{0}G^B\frac{\delta L_\text{matter}}{\delta G^A} = -4 \frac{\delta L_\text{matter}}{\delta A} + \mathcal O(H^2).
\end{equation}
The second variation in question behaves similarly, reducing the relevant variation of the gravity Lagrangian to
\begin{equation}
  \gmc{A}{B}{\alpha}{0}G^B\frac{\delta L_\text{gravity}}{\delta G^A} = 4 \frac{\delta L_\text{gravity}}{\delta b_\alpha} + \mathcal O(H^2).
\end{equation}
Roughly the same result as for metric gravity is obtained for the matter part,
\begin{equation}
  \mathcal H_\text{matter} = \frac{4}{c}\lbrack E_0 + E\rbrack
\end{equation}
where $E_0 = -mc^2$ and
\begin{equation}
E = -\frac{1}{2} m \eta r^2\omega^2 = -\frac{1}{2} m\eta \frac{\alpha Gm}{r}\lbrack 1+f(r)\rbrack.
\end{equation}
For the gravity part, we again use \texttt{cadabra} in order to derive the contributions from the transverse traceless $U^{\alpha\beta}$ modes \eqref{area_metric_radiation_massless} to $\frac{\delta L_\text{gravity}}{\delta b^\alpha}$. This yields via a similar calculation as before the right-hand side
\begin{equation}
  \begin{aligned}
    4 c \int_{S_\infty} \frac{\delta L_\text{gravity}}{\delta b_\alpha} \mathrm dS_\alpha = {} & \frac{c^2}{4\pi G} \int_{S_\infty} \left\lbrack (-\frac{1}{8\alpha})\dot U_{\alpha\beta}\dot U^{\alpha\beta} + (\frac{1}{4\alpha}-4k_{12})\left(U_{\alpha\beta}\dot U^{\alpha\beta}\right)^{\scalebox{1.5}{$\cdot$}}\right\rbrack \mathrm dS \\
    = {} & -\frac{c^2}{32\pi\alpha G} \int_{S_\infty}\dot U_{\alpha\beta}\dot U^{\alpha\beta}\mathrm dS \\
    = {} & -\frac{128}{5} m \eta^2 \frac{(1+f(r))^3}{r^5}\frac{(\alpha Gm)^4}{c^6}
  \end{aligned}
\end{equation}
of the balance equations. Putting both sides together yields the rate
\begin{equation}\label{area_metric_balance}
  \left(\frac{1+f(r)}{r}\right)^{\!\!\scalebox{1.5}{$\cdot$}} = -\frac{\dot r}{r^2}\lbrack 1+f(r) - r f^\prime(r)\rbrack = \frac{64}{5} \eta c \frac{1}{r^2} \left(\frac{\alpha Gm}{c^2r}\lbrack 1+f(r)\rbrack\right)^3.
\end{equation}

The limit $f(r) \to 0$ of Eq.~\eqref{area_metric_balance} reduces reproduces the metric result
\begin{equation}
  \dot r = -\frac{64}{5} \eta c \left(\frac{\alpha Gm}{c^2r}\right)^3,
\end{equation}
which again shows the correspondence of both theories for a suitable parameter range. However, when the correction $f(r)$ is not negligible, area metric gravity introduces an interesting deviation from the binary star spin-up behaviour in metric gravity.

\textbf{to do: plots of spin-up}

\textbf{take-home message: application to a refined matter theory yields interesting phenomenology}
