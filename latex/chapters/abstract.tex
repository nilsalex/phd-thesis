\begin{abstract}
  Matter theories are not predictive if they couple to geometry with unknown dynamics: it is not possible to anticipate how matter behaves in the future without knowing how the geometry evolves. This thesis studies the completion of such matter theories to predictive theories of matter and gravity. In essence, Einstein solved this problem for Maxwell's electrodynamics by making the metric tensor dynamic and, most importantly, providing the dynamics in the form of the Einstein equations. Indeed, a recurring theme of the work presented here is that general relativity is recovered as far as metric theories are concerned.

  For novel matter theories that may couple to nonmetric geometries, two axioms guide the completion to predictive theories: general covariance of gravity and causal compatibility of gravity with the matter theory. Both notions have a strong mathematical basis, which is presented first, before formulating precise definitions of the axioms. The implementation of the axioms for specific geometries follows a pattern that is to a certain degree independent of the geometry. This pattern is stated in the condensed form of an algorithm. Since this algorithm provides a construction procedure for gravitational theories on the basis of general covariance, it is called \emph{covariant constructive gravity}---highlighting both the similarities with and differences to (canonical) constructive gravity.

  The execution of this algorithm is, for most geometries, not feasible as of today. However, a perturbative treatment 
\end{abstract}

\newpage
\begin{otherlanguage}{ngerman}
\renewcommand{\abstractname}{Kurzzusammenfassung}
\begin{abstract}
  Materietheorien, denen Geometrie mit unbekannter Dynamik zugrunde liegen, sind nicht prädiktiv. Es fehlt ihnen die Möglichkeit, das Verhalten von Materie in der Zukunft vorherzusagen, denn die Entwicklung der Geometrie ist nicht bekannt. In dieser Dissertation soll die Vervollständigung von solchen Materietheorien zu prädiktiven Theorien von Materie \emph{und} Gravitation untersucht werden.
\end{abstract}
\end{otherlanguage}

