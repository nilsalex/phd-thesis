\documentclass{beamer}
\usepackage{fontspec}
\usepackage{polyglossia}
\usepackage{amsmath}
\usepackage{mathtools}
\usepackage{mathrsfs}
\usepackage{xcolor}
\usepackage{tikz}
\usetikzlibrary{calc,tikzmark}

\usetheme{metropolis}
\setmainlanguage[babelshorthands=true]{german}

\title{Covariant Constructive Gravity}
\date{14.\ Juni 2022}
\author{Nils Alex}
\institute{FAU Erlangen-Nürnberg}

\begin{document}
    \maketitle

    \begin{frame}{Gliederung}
        \tableofcontents[pausesections]
    \end{frame}


    \section{Gravitation: Zusammenspiel von Materie und Geometrie}\label{sec:materie-und-geometrie}

    \begin{frame}{}
        \Large
        \textit{„Space tells matter how to move \\
        Matter tells space how to curve“}

        \normalsize
        ---John Archibald Wheeler, Gravitation (1973)
    \end{frame}

    \begin{frame}{Space tells matter how to move \ldots}
        \begin{itemize}
            \item {\setlength{\belowdisplayskip}{-5pt}Punktmassen maximieren die {\color{blue}Länge} ihrer Weltlinie
                \[ S = m \int \sqrt{{\color{blue}g(}\dot\gamma(\lambda),\dot\gamma(\lambda){\color{blue})}} d\lambda \]} \pause
            \item Die Konstanten der Maxwell-Gleichungen
            \begin{align*}
                \mathop{\color{blue}\operatorname{div}} E &= 4\pi \rho & \operatorname{div} B &= 0 \\
                \mathop{\color{blue}+} \partial_t E \mathop{\color{blue}-} \mathop{\color{blue}\operatorname{rot}} B &= \mathop{\color{blue}-}4\pi j & \partial_t B + \operatorname{rot} E &= 0,
            \end{align*}
            sind Geometrie, erkennbar an der kovarianten Wirkung
            \[
                S = \int \sqrt{-\color{blue}g}{\color{blue}g^{ac}g^{bd}} F_{ab} F_{cd} d^4 x
            \]
        \end{itemize}
    \end{frame}

    \begin{frame}{Space tells matter how to move \ldots}
        \begin{itemize}
            \item Das gesamte Standardmodell der Teilchenphysik (SMTP) ist von metrischer Geometrie bestimmt. \pause
            \item \ldots wer bestimmt die Geometrie?
        \end{itemize}
    \end{frame}

    \begin{frame}{... Matter tells space how to curve}
        \begin{itemize}
            \item Gravitation ist die dynamische Theorie der Raumzeitmetrik \pause
            \item Einstein-Hilbert:
            \begin{gather*}
                S = \int \sqrt{-g} R d^4 x \\
                \Rightarrow G_{ab} = T_{ab}
            \end{gather*}
        \end{itemize}
    \end{frame}


    \section{Konstruktive Gravitation}\label{sec:constructive-gravity}

    \begin{frame}{Eindeutigkeit der Allgemeinen Relativitätstheorie}
        Einsteins Allgemeine Relativitätstheorie (ART) ist die einzig mögliche dynamische Theorie
        der Raumzeitmetrik \pause
        \begin{itemize}
            \setlength{\belowdisplayskip}{-10pt}
            \item \textbf{Hochman, Kuchař, Teitelboim (1976):} kanonische ART als eindeutige Darstellung der Dirac-Algebra
            \begin{align*}
                \{ D(\vec N), D(\vec M) \} &= -D(\mathcal L_{\vec M} \vec N) \\
                \{ H(N), D(\vec M) \} &= -H(\mathcal L_{\vec M} N) \\
                \{ H(N), H(M) \} &= H(g^{\alpha\beta}(M\partial_\alpha N - N\partial_\alpha M))
            \end{align*} \pause
            \item \textbf{Lovelock (1969):} Einstein-Gleichungen folgen aus allgemeiner Kovarianz \pause
            \item \textbf{Deser (1970):} Einstein-Hilbert-Wirkung folgt aus linearisierter ART und allgemein kovarianten
            Selbstwechselwirkungstermen bis Ordnung $\infty$
        \end{itemize}
    \end{frame}

    \begin{frame}{Modifizierte Gravitation}
        \begin{itemize}
            \item SMTP + ART haben nur begrenzte Gültigkeit \pause
            \begin{itemize}
                \item Rotationskurven von Galaxien \pause
                \item Singularitäten (Schwarze Löcher, Urknall) \pause
                \item Teilchenphysik ?? \pause
                \item \ldots \pause
            \end{itemize}
            \item Modifikationen oder Erweiterungen unterliegen den Eindeutigkeitsresultaten! \pause
            \begin{itemize}
                \item Materietheorie bleibt rein metrisch, oder \pause
                \item \only<8-9>{Materietheorie benutzt andere Geometrie, oder}\only<10>{\alert{\tikzmarknode{a}{Materietheorie benutzt andere Geometrie}}, oder} \pause
                \item Grundannahmen müssen verworfen werden \pause
            \end{itemize}
        \end{itemize}
        \begin{tikzpicture}[overlay,remember picture]
            \draw<10>[->,mLightBrown,thick,smooth,out=30,in=270] ([yshift=-1.2cm]a.south) to ([xshift=.77cm,yshift=-3pt]a.south);
            \node<10>[mLightBrown] at ([xshift=-.45cm,yshift=-1.45cm]a.south) {Constructive Gravity};
        \end{tikzpicture}
    \end{frame}

    \begin{frame}{Konstruktive Gravitation}
        \begin{itemize}
            \item \textbf{Grundidee der Konstruktiven Gravitation:} \\
            Modifizierte Gravitation ergibt sich aus der Betrachtung neuartiger Materietheorien. \pause
            \item \textbf{Konstruktionsprinzip:} \\
            Reproduktion der Eindeutigkeitstheoreme für Maxwell-Einstein \pause
            \item Unterteilung je nach Art des Eindeutigkeitstheorems
        \end{itemize}
    \end{frame}

    \begin{frame}{Kanonische Konstruktive Gravitation}
        \begin{itemize}
            \item Resultat von HKT lässt sich verallgemeinern für beliebige tensorielle Geometrien \pause
            \item Prinzip: Constraint-Algebra der Gravitation muss die
            Hyperflächendeformationsalgebra implementieren \pause
            \item Entspricht einem unendlichen System von PDEs für die Lagrangedichte \pause
            \item Störungstheorie oder Symmetriereduktion möglich \pause
            \item Kausalität ?? \pause
            \item Giesel et al.\ (2012), ?? \ldots
        \end{itemize}
    \end{frame}

    \begin{frame}{Kovariante Konstrutive Gravitation}
        \begin{itemize}
            \item Komplementärer Ansatz: Postulat der allgemeinen Kovarianz direkt für die Lagrangedichte \pause
            \item Verwandt mit Lovelock und Deser \pause
            \item „Kovariant“, weil kein 3+1 Split die manifeste Diffeomorphismeninvarianz bricht \pause
            \item Prinzipien, mathematische Implementierung, Beispiele im Rest dieses Vortrages
        \end{itemize}
    \end{frame}


    \section{Kovariante Konstruktive Gravitation}\label{sec:covariant-constructive-gravity}

    \begin{frame}{Lagrange-Formalismus}
        \begin{itemize}
            \item Geometrie ist ein Tensorbündel $E \xrightarrow{\pi} M$ über einer differenzierbaren
            Raumzeitmannigfaltigkeit\footnote{$\operatorname{dim} M = n, \operatorname{dim} E = n + m$}, z.B.\ \pause
            \begin{itemize}
                \item Metrisches Bündel $\operatorname{Sym}(T^0_2 M)$ \pause
                \item Bi-Metrisches Bündel $\operatorname{Sym}(T^0_2 M) \oplus_M \operatorname{Sym}(T^0_2 M)$ \pause
                \item Area-Metrisches Bündel als Sub-Bündel von $T^4_0 M$ \pause
            \end{itemize}
            \item Lagrangian $\mathscr L \in \bigwedge^n_0 \pi_2$ ist eine horizontale $n$-Form auf dem
            2.\ Jet-Bündel $J^2 E \xrightarrow{\pi_2} M$ \pause
            \item In Koordinaten
            \[ \mathscr{L} = L(x^i, u^A, u^A_i, u^A_{ij}) dx^1 \wedge \dots \wedge dx^n \]
        \end{itemize}
    \end{frame}


    \section{Störungstheorie}\label{sec:stoerungstheorie}


    \section{Anwendung: Gravitationswellen von Doppelbrechender Materie}\label{sec:anwendung}


    \section{Ausblick}\label{sec:ausblick}

\end{document}
