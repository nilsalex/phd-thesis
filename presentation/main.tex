\documentclass{beamer}
\usepackage{amsmath}
\usetheme{metropolis}
\usepackage{xcolor}

\title{Covariant Constructive Gravity}
\date{14 June 2022}
\author{Nils Alex}
\institute{FAU Erlangen-Nürnberg}

\begin{document}
    \maketitle

    \begin{frame}{Outline}
        \tableofcontents
    \end{frame}


    \section{Matter and Gravity}\label{sec:matter-and-gravity}

    \begin{frame}{Matter and Gravity}
        \Large
        \textit{``Space tells matter how to move \\
        Matter tells space how to curve''}

        \normalsize
        ---John Archibald Wheeler, Gravitation (1973)
    \end{frame}

    \begin{frame}{Matter and Gravity}
        Maxwell's equations
        \begin{align*}
            \operatorname{div} E &= 4\pi \rho & \operatorname{div} B &= 0 \\
            \operatorname{rot} E &= - \partial_t B & \operatorname{rot} B &= 4\pi j + \partial_t E
        \end{align*} \pause
        derived from the covariant action
        \[
            S_\text{Maxwell} = \int \only<2>{\eta^{ac} \eta^{bd} F_{ab} F_{cd}} \only<3>{{\color{red}\eta^{ac} \eta^{bd}} {\color{blue}F_{ab} F_{cd}}} d^4 x
        \]
    \end{frame}

    \section{Covariant Constructive Gravity}\label{sec:covariant-constructive-gravity}

    \section{Perturbation Theory}\label{sec:perturbation-theory}

    \section{Application: Gravitational Radiation from Birefringent Theories}\label{sec:application}

    \section{Conclusions}\label{sec:conclusions}

\end{document}
