\chapter{Perturbative construction of gravitational theories}
\label{chapter_perturbation}

\textit{Following the presentation of the axioms of covariant constructive gravity and the construction algorithm, we now develop a perturbative approach for the implementation of both axioms. The equivariance equations turn out to lend themselves to an iterative solution strategy where the expansion coefficients of a power series ansatz are determined iteratively, power by power. A first approximation of the gravitational theory valid for weak fields is obtained already after the second iteration, which yields a quadratic Lagrangian with linear field equations. In a sense, this is the free theory without self-coupling. In order to investigate the lowest-order effects of self-coupling, which we will dare in the subsequent chapter, the next order is indispensable. Therefore, after establishing the general principle, we focus on the perturbation theory up to third order in the Lagrangian.}

\textit{The development in this chapter follows closely the presentation in Ref.~\cite{ccg_paper}.}

\textbf{take-home message: approach ideally suits itself for construction of theories which govern weak gravitational fields}

\section{Perturbative implementation of axiom 1}

Let us state again, for reference, the equivariance equations \eqref{equivariance_eqn_2}--\eqref{equivariance_eqn_4} we are going to solve perturbatively
\begin{equation*}
    \begin{aligned}
      0 &{} = L_{:A} \gmc{A}{B}{n}{m}u^B + L_{:A}^{\hphantom{:A}p}\left\lbrack\gmc{A}{B}{n}{m} \delta^q_p - \delta^A_B\delta^q_m\delta^n_p\right\rbrack u^{B}_{\hphantom Bq} \nonumber \\
        &{} \hphantom{=} + L_{:A}^{\hphantom{:A}I} \left\lbrack\gmc{A}{B}{n}{m} \delta^J_I - 2 \delta^A_B J^{pn}_{I} I^J_{pm}\right\rbrack u^B_{\hphantom BJ} + L\delta^n_m \\
      0 &{} = L_{:A}^{\hphantom{:A}(p\mid} \gmc{A}{B}{\mid n)}{m} u^B + L_{:A}^{\hphantom{:A}I}\left\lbrack\gmc{A}{B}{(n}{m} 2J^{p)q}_I - \delta^A_B J^{pn}_I \delta^q_m\right\rbrack u^B_{\hphantom Bq} \\
      0 &{} = L_{:A}^{\hphantom{:A}I} \gmc{A}{B}{(n}{m} J^{pq)}_I u^B.
    \end{aligned}
\end{equation*}
The first equivariance equation $0 = L_{,m}$ has been omitted, because we will consider it solved from now on by restricting the problem to Lagrangian densities $L$ which depend only on the fiber coordinates.

Perturbation theory starts with choosing an expansion point $p\in J^2E$. Let $p$ have fibre coordinates $(N^A,N^A_{\hphantom Ap},N^A_{\hphantom AI})$. The deviation of any point $q\in J^2E$ with fibre coordinates $(G^A,G^A_{\hphantom Ap},G^A_{\hphantom AI})$ is then defined as the difference
\begin{equation}
  (H^A, H^A_{\hphantom Ap}, H^A_{\hphantom AI}) \vcentcolon= (G^A - N^A, G^A_{\hphantom Ap} - N^A_{\hphantom Ap}, G^A_{\hphantom AI} - H^A_{\hphantom AI}).
\end{equation}
Around $p$, the formal power series ansatz is
\begin{equation}\label{power_series_ansatz}
  \begin{aligned}
    L &{} = a + a_A H^A + a_A^{\hphantom Ap} H^A_{\hphantom Ap} + a_A^{\hphantom AI} H^A_{\hphantom AI} \\
      &{} \hphantom{=} + a_{AB} H^A H^B + a_{AB}^{\hphantom{AB}p} H^A H^B_{\hphantom Bp} + a_{AB}^{\hphantom{AB}I} H^A H^B_{\hphantom BI} \\
      &{} \hphantom{=} + a_{A\hphantom{p}B}^{\hphantom Ap\hphantom{B}q} H^A_{\hphantom Ap} H^B_{\hphantom Bq} + a_{A\hphantom{p}B}^{\hphantom Ap\hphantom{B}I} H^A_{\hphantom Ap} H^B_{\hphantom BI} + a_{A\hphantom{I}B}^{\hphantom AI\hphantom{B}J} H^A_{\hphantom AI} H^B_{\hphantom BJ} \\
      &{} \hphantom{=} + a_{ABC} H^A H^B H^C + \dots
  \end{aligned}
\end{equation}
and is called \emph{formal} because at this point there is no assumption about the convergence of the power series. We do, however, make two assumptions about admissible expansion points, in order to justify the interpretation of perturbatively constructed theories as valid theories for \emph{weak} gravitational fields.
\begin{enumerate}
  \item{The expansion point represents a \emph{flat} instance of the gravitational field, i.e., both $N^A_{\hphantom Ap}$ and $N^A_{\hphantom AI}$ vanish.}
  \item{At the expansion point, the matter theory that is used to bootstrap the construction procedure reduces to a theory that is equivalent to a matter theory on Minkowski spacetime.}
\end{enumerate}
Both restrictions for $p$ ensure that the limit of weak gravitational fields can match our observations for situations with weak gravity: Matter fields couple to flat geometry in the sense that there are coordinate charts where the geometric coefficients are constant and this geometry is determined by the Minkowski metric. Curvature effects of non-Lorentzian geometry are expected to arised as \emph{deviations} from this ground state. After all, we have formulated another consequence of the correspondence principle for modified gravity.

The first assumption is easily implemented: In the chosen coordinate chart, $p$ takes the form $(N^A,0,0)$. The best way to make sense of the second assumption is by considering a few examples:
\begin{example}[flat Lorentzian expansion points]
  Let us choose appropriate expansion points $(N^A,0,0)$ for GLED and bimetric theories introduced in Sections \ref{section_gled} and \ref{section_bimetric}, respectively. In order to satisfy the second assumption made for expansion points, we construct $N^A$ from the Minkowski metric $\eta^{ab} = \operatorname{diag}(1,-1,-1,-1)^{ab}$ in the following ways.
  \begin{enumerate}
    \item{For bimetric theories, a suitable expansion point is given by $N^{\bar A} = J_{ab}^{\bar A} \eta^{ab}$ and $N^{\bar{\bar A}} = J_{ab}^{\bar{\bar A}} \eta^{ab}$, i.e.~setting both metrics $g$ and $h$ equal to $\eta$. Where a scalar density is needed, we choose $\omega = \sqrt{-\operatorname{det}\eta} = 1$. This choice reduceds the bimetric Klein-Gordon theory to the standard Klein-Gordon theory for two scalar fields on Minkowski spacetime
      \begin{equation}
        L_\text{2KG}(N) = \eta^{ab} \phi_{,a} \phi_{,b} - m_\phi^2\phi^2 + \eta^{ab} \psi_{,a} \psi_{,b} - m_\psi^2\psi^2.
      \end{equation}
    Similarly, the refined Proca theory reduces to the standard Proca theory
      \begin{equation}
        L_\text{bi-Proca}(N) = -\eta^{ac}\eta^{bd}F_{ab}F_{cd} + m^2\eta^{ab}A_aA_b.
      \end{equation}}
    \item{For GLED, we choose the expansion point $N^A = J_{abcd}^A (\eta^{ac} \eta^{bd} - \eta^{ad} \eta^{bc} + \epsilon^{abcd})$. Using the density $\omega_G = (\frac{1}{24}\epsilon_{abcd}G^{abcd})^{-1}$, which at $G^A=N^A$ results in $\omega_N = 1$, the Lagrangian density for GLED reduces to
        \begin{equation}\label{GLED_reduction}
        L_\text{GLED} = 2 \eta^{ac}\eta^{bd} F_{ab}F_{cd},
      \end{equation}
    i.e.~Maxwell electrodynamics on Minkowski spacetime.\footnote{The term $\epsilon^{abcd}F_{ab}F_{cd}$ is a surface term and thus does not contribute to the field equations. As such, it has been omitted in Eq.~\ref{GLED_reduction}}}
  \end{enumerate}
\end{example}
Both choices of expansion points ensure that the perturbatively constructed gravitational theories provide to zeroth order in the deviation a background on which known physics is reproduced. Novel physics---the coupling of matter fields to non-metric geometries and the self-coupling of such geometries---should emerge as effect of first and higher orders in the deviation from the Minkowski background.

Having defined an expansion point, the equivariance equations can---in principle---be solved perturbatively by repeating the following process: All equations in the system contain derivatives of first order, so the expansion coefficient $a$ of zeroth order remain undetermined. In order to determine the expansion coefficients $a_A$, $a_A^{\hphantom Ap}$, and $a_A^{\hphantom aI}$, substitute the formal power series ansatz \eqref{power_series_ansatz} for the Lagrangian density $L$ in the equivariance equations, evaluate the result at $N$ (i.e.~set the deviation $H$ to zero) and solve the resulting linear equations for the first-order coefficients. Next, differentiate each PDE once with respect to every independent variable, substitute again the power series ansatz, evaluate at $N$ and solve the linear system for the expansion coefficients of second order. Now repeat this process of differentiation, substitution, evaluation, and solving of linear equations \emph{ad infinitum}---or up to the desired perturbation order.

However, for this solution process to play out as desired, the PDE system must observe an important property: We need to be sure that each step really determines all expansion coefficients for the corresponding order to the extend that this is possible\footnote{Of course, the equivariance equations, in general, will not determine \emph{all} expansion coefficients. Rather, the solutions will be parameterized exactly by the coefficients that cannot be determined.}. This is not always guaranteed, as a simple example demonstrates.
\begin{example}[integrability conditions\cite{seiler_habil}]
  Consider the linear, first-order PDE system
  \begin{equation}\label{example_integrability_condition}
    \begin{aligned}
      u_{,z} + y u_{,x} &{} = 0, \\
      u_{,y} &{} = 0
    \end{aligned}
  \end{equation}
  for one function $u$ which depends on three independent variables $x,y,z$. Making a formal power series ansatz, inserting this ansatz into the system \label{example_integrability_condition}, and evaluating at the expansion point yields a linear system of rank two for the three expansion coefficients of first order.

  There are, however, hidden equations governing the first order, which emerge after differentiating the first equation with respect to $y$ and the second equation with respect to $z$ and $x$. This gives new PDEs
  \begin{equation}
    \begin{aligned}
      u_{,yz} + y u_{,xy} + u_{,x} &{} = 0, \\
      u_{,xy} &{} = 0, \\
      u_{,yz} &{} = 0.
    \end{aligned}
  \end{equation}
  The second derivatives in the first equation can be cancelled using the second and third equation, yielding the first-order PDE $u_{,x} = 0$. This new equation is independent of the original PDEs \eqref{example_integrability_condition}. Including it in the first-order system and simplifying a bit, we get
  \begin{equation}
    \begin{aligned}
      u_{,x} &{} = 0, \\
      u_{,y} &{} = 0, \\
      u_{,z} &{} = 0.
    \end{aligned}
  \end{equation}
  Only after performing this procedure of making explicit the hidden first-order PDEs
\end{example}

\begin{itemize}
\item power series ansatz
\item involution theory
\item proof: equivariance equations are involutive
\end{itemize}

\section{Lorentz invariant ansätze}

\section{Perturbative implementation of axiom 2}
\begin{itemize}
\item expansion of the principal polynomial
\end{itemize}

