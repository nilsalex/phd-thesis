\documentclass{beamer}
\usepackage{fontspec}
\usepackage{polyglossia}
\usepackage{amsmath}
\usepackage{mathtools}
\usepackage{mathrsfs}
\usepackage{xcolor}
\usepackage{tikz}
\usetikzlibrary{calc,tikzmark}

\usetheme{metropolis}
\setmainlanguage[babelshorthands=true]{german}
\setbeamercovered{transparent}

\newcommand{\gmc}[4]{\ensuremath{C^{#1\hphantom{#2}#3}_{\hphantom{#1}{#2}\hphantom{#3}#4}}}
\newcommand{\gmcd}[4]{\ensuremath{C_{#1\hphantom{#2#3}#4}^{\hphantom{#1}{#2}#3}}}

\title{Covariant Constructive Gravity}
\date{14.\ Juni 2022}
\author{Nils Alex}
\institute{FAU Erlangen-Nürnberg}

\begin{document}
    \maketitle

    \begin{frame}{Gliederung}
        \tableofcontents[pausesections]
    \end{frame}


    \section{Gravitation: Zusammenspiel von Materie und Geometrie}\label{sec:materie-und-geometrie}

    \begin{frame}{}
        \Large
        \textit{„Space tells matter how to move \\
        Matter tells space how to curve“}

        \normalsize
        ---John Archibald Wheeler, Gravitation (1973)
    \end{frame}

    \begin{frame}{Space tells matter how to move \ldots}
        \begin{itemize}
            \item {\setlength{\belowdisplayskip}{-5pt}Punktmassen maximieren die {\color{blue}Länge} ihrer Weltlinie
                \[ S = m \int \sqrt{{\color{blue}g(}\dot\gamma(\lambda),\dot\gamma(\lambda){\color{blue})}} d\lambda \]} \pause
            \item Die Konstanten der Maxwell-Gleichungen
            \begin{align*}
                \mathop{\color{blue}\operatorname{div}} E &= 4\pi \rho & \operatorname{div} B &= 0 \\
                \mathop{\color{blue}+} \partial_t E \mathop{\color{blue}-} \mathop{\color{blue}\operatorname{rot}} B &= \mathop{\color{blue}-}4\pi j & \partial_t B + \operatorname{rot} E &= 0,
            \end{align*}
            sind Geometrie, erkennbar an der kovarianten Wirkung
            \[
                S = \int {\color{blue}\sqrt{-g}g^{ac}g^{bd}} F_{ab} F_{cd} d^4 x
            \]
        \end{itemize}
    \end{frame}

    \begin{frame}{Space tells matter how to move \ldots}
        \begin{itemize}
            \item Das gesamte Standardmodell der Teilchenphysik (SMTP) ist von metrischer Geometrie bestimmt. \pause
            \item \ldots wer bestimmt die Geometrie?
        \end{itemize}
    \end{frame}

    \begin{frame}{... Matter tells space how to curve}
        \begin{itemize}
            \item Gravitation ist die dynamische Theorie der Raumzeitmetrik \pause
            \item Einstein-Hilbert:
            \begin{gather*}
                S = \int \sqrt{-g} R d^4 x \\
                \Rightarrow G_{ab} = T_{ab}
            \end{gather*}
        \end{itemize}
    \end{frame}


    \section{Konstruktive Gravitation}\label{sec:constructive-gravity}

    \begin{frame}{Eindeutigkeit der Allgemeinen Relativitätstheorie}
        Einsteins Allgemeine Relativitätstheorie (ART) ist die einzig mögliche dynamische Theorie
        der Raumzeitmetrik \pause
        \begin{itemize}
            \setlength{\belowdisplayskip}{-10pt}
            \item \textbf{Hochman, Kuchař, Teitelboim (1976):} kanonische ART
            als eindeutige Darstellung der Dirac-Algebra
            \begin{align*}
                \{ D(\vec N), D(\vec M) \} &= -D(\mathcal L_{\vec M} \vec N) \\
                \{ H(N), D(\vec M) \} &= -H(\mathcal L_{\vec M} N) \\
                \{ H(N), H(M) \} &= H(g^{\alpha\beta}(M\partial_\alpha N - N\partial_\alpha M))
            \end{align*} \pause
            \item \textbf{Lovelock (1969):} Einstein-Gleichungen folgen aus allgemeiner Kovarianz \pause
            \item \textbf{Deser (1970):} Einstein-Hilbert-Wirkung folgt aus linearisierter ART und allgemein kovarianten
            Selbstwechselwirkungstermen bis Ordnung $\infty$
        \end{itemize}
    \end{frame}

    \begin{frame}{Modifizierte Gravitation}
        \begin{itemize}
            \item<1-> SMTP + ART haben nur begrenzte Gültigkeit
            \begin{itemize}
                \item<2-> Rotationskurven von Galaxien
                \item<3-> Singularitäten (Schwarze Löcher, Urknall)
                \item<4-> Teilchenphysik ??
                \item<5-> \ldots
            \end{itemize}
            \item<6-> Modifikationen oder Erweiterungen unterliegen den Eindeutigkeitsresultaten!
            \begin{itemize}
                \item<7-> Materietheorie bleibt rein metrisch, oder
                \item<8-> \alert{\tikzmarknode{a}{Materietheorie benutzt andere Geometrie, oder}}
                \item<9-> Grundannahmen müssen verworfen werden
            \end{itemize}
        \end{itemize}
        \onslide<10>
        \begin{tikzpicture}[overlay,remember picture]
            \draw<10>[->,mLightBrown,thick,smooth,out=30,in=270] ([xshift=-.4cm,yshift=-1.2cm]a.south) to ([xshift=.3cm,yshift=-3pt]a.south);
            \node<10>[mLightBrown] at ([xshift=-.85cm,yshift=-1.45cm]a.south) {Constructive Gravity};
        \end{tikzpicture}
    \end{frame}

    \begin{frame}{Konstruktive Gravitation}
        \begin{itemize}
            \item \textbf{Grundidee der Konstruktiven Gravitation:} \\
            Modifizierte Gravitation ergibt sich aus der Betrachtung neuartiger Materietheorien. \pause
            \item \textbf{Konstruktionsprinzip:} \\
            Reproduktion der Eindeutigkeitstheoreme für Maxwell-Einstein \pause
            \item Unterteilung je nach Art des Eindeutigkeitstheorems
        \end{itemize}
    \end{frame}

    \begin{frame}{Kanonische Konstruktive Gravitation}
        \begin{itemize}
            \item Resultat von HKT lässt sich verallgemeinern für beliebige tensorielle Geometrien \pause
            \item Prinzip: Constraint-Algebra der Gravitation muss die
            Hyperflächendeformationsalgebra implementieren \pause
            \item Entspricht einem unendlichen System von PDEs für die Lagrangedichte \pause
            \item Störungstheorie oder Symmetriereduktion möglich \pause
            \item Kausalität ?? \pause
            \item Giesel et al.\ (2012), ?? \ldots
        \end{itemize}
    \end{frame}

    \begin{frame}{Kovariante Konstrutive Gravitation}
        \begin{itemize}
            \item Postulat der allgemeinen Kovarianz direkt für Lagrangian \pause
            \item Verwandt mit Lovelock und Deser \pause
            \item „Kovariant“, weil kein 3+1 Split die manifeste Diffeomorphismeninvarianz bricht \pause
            \item Prinzipien, mathematische Implementierung, Beispiele im Rest dieses Vortrages \pause
            \item N.A., Reinhart (2020); N.A. (2020)
        \end{itemize}
    \end{frame}


    \section{Kovariante Konstruktive Gravitation}\label{sec:covariant-constructive-gravity}

    \begin{frame}{Lagrange-Formalismus}
        \begin{itemize}
            \item Geometrie ist ein Tensorbündel $E \xrightarrow{\pi} M$ über einer differenzierbaren
            Raumzeitmannigfaltigkeit, z.B.\ \pause
            \begin{itemize}
                \item Metrisches Bündel $\operatorname{Sym}(T^0_2 M)$ \pause
                \item Bi-Metrisches Bündel $\operatorname{Sym}(T^0_2 M) \oplus_M \operatorname{Sym}(T^0_2 M)$ \pause
                \item Area-Metrisches Bündel als Sub-Bündel von $T^4_0 M$ \pause
            \end{itemize}
            \item Lagrangian $\mathscr L \in \bigwedge^n_0 \pi_2$ ist eine horizontale $n$-Form auf dem
            2.\ Jet-Bündel $J^2 E \xrightarrow{\pi_2} M$ \pause
            \item {\setlength{\belowdisplayskip}{-10pt}In Koordinaten
                \[ \mathscr{L} = L(x^i, u^A, u^A_i, u^A_{ij}) dx^1 \wedge \dots \wedge dx^n \]} \pause
            \item Euler-Lagrange-Gleichungen: Notwendige Bedingung für physikalische Felder $\sigma\in\Gamma(\pi)$
            \[ (j^2\sigma)^\ast \left( \frac{\partial L}{\partial u^A} - D_i \frac{\partial L}{\partial u^A_i} + D_{ij} \frac{\partial L}{\partial u^A_{ij}}\right) = 0 \]
        \end{itemize}
    \end{frame}

    \begin{frame}{Allgemeine Kovarianz}
        \begin{itemize}
            \item Raumzeit-Diffeomorphismen $\varphi\in\operatorname{Diff}(M)$ induzieren Vektorbündel-Automorphismen
            $\varphi_E\in\operatorname{Aut}(E)$ (Pushforward, Pullback) \pause
            \item Diese wiederum induzieren Jetbündel-Morphismen $j^2(\varphi_E)$ \pause
            \item $\mathscr L$ ist invariant unter Diffeo's \pause $\Leftrightarrow$ \alert{$j^2(\varphi_E)^\ast\mathscr{L} = \mathscr{L}$} \pause
            \item Lokale Äquivarianz: $\mathscr L = L d^n x$ \pause $\Rightarrow$ \alert{$L\circ j^2(\varphi_E) = \lvert d\varphi\rvert^{-1} L$}
        \end{itemize}
    \end{frame}

    \begin{frame}{Äquivarianzgleichungen}
        \begin{itemize}
            \item<1-> Vektorfelder $\xi\in\Gamma(TM)$ sind infinitesimale Diffeo's \[ x^i \mapsto x^i + \xi^i \]
            \item<2-> Lift auf das Tangentialbündel von $E$ \[ \xi \mapsto \xi_E \vcentcolon= \xi^m\partial_m + \onslide<3-> \underbrace{\onslide<2-> \gmc{A}{B}{m}{n} \onslide<3-> }_{\mathclap{\text{\alert{Gotay-Marsden-Koeffizienten}}}} \onslide<2-> u^B \xi^m_{,n} \partial_A \]
            \item<4-> \makebox[2cm]{Kovektor:\hfill} $\xi_E  =\xi^m \partial_m \underbrace{- \delta^a_m \delta^n_b}_{\gmcd{a}{b}{m}{n}} \omega_a \xi^m_{,n} \frac{\partial}{\partial \omega^b} $
            \item<5-> \makebox[2cm]{Metrik:\hfill} $\xi_E  =\xi^m \partial_m + \underbrace{2 \delta^a_m \delta^b_c \delta^n_d}_{\gmc{ab}{cd}{m}{n}} g^{cd} \xi^m_{,n} \frac{\partial}{\partial g^{ab}} $
        \end{itemize}
    \end{frame}

    \begin{frame}{Äquivarianzgleichungen}
        \setlength{\belowdisplayskip}{-10pt}
        Äquivarianzgleichungen für infinitesimale Diffeo's:
        \begin{align*}
            0 =& {\color{blue}L_{,m}} \\
            0 =& {\color{blue}L_{:A}} {\color{mLightBrown}\gmc{A}{B}{n}{m}} {\color{purple}u^B} + {\color{blue}L_{:A}} \left\lbrack {\color{mLightBrown}\gmc{A}{B}{n}{m}} \delta^q_p - \delta^A_B \delta^q_m \delta^n_p \right\rbrack {\color{purple}u^B_q} \\
            & + {\color{blue}L_{:A}^{\hphantom{:A}I}} \left\lbrack {\color{mLightBrown}\gmc{A}{B}{n}{m}} \delta^J_I - 2\delta^A_B J^{pn}_I I^J_{pm}\right\rbrack {\color{purple}u^B_J} + {\color{blue}L} \delta^n_m \\
            0 =& {\color{blue}L_{:A}^{\hphantom{:A}(p\mid}} {\color{mLightBrown}\gmc{A}{B}{\mid n)}{m}} {\color{purple}u^B} + {\color{blue}L_{:A}^{\hphantom{:A}I}}\left\lbrack {\color{mLightBrown}\gmc{A}{B}{(n}{m}} 2 J^{p)q}_I - \delta^A_B J^{pn}_I \delta^q_m\right\rbrack {\color{purple}u^B_q} \\
            0 =& {\color{blue}L_{:A}^{\hphantom{:A}I}} {\color{mLightBrown}\gmc{A}{B}{(n}{m}} J^{pq)}_I {\color{purple}u^B}
        \end{align*} \pause
        \begin{itemize}
            \item System aus 140 linearen PDE erster Ordnung mit konstanten und linearen Koeffizienten \pause
            \item Geometrieabhängigkeit nur in {\color{mLightBrown}Gotay-Marsden-Koeff.} \pause
            \item Lösungen des Systems sind Kandidaten für Gravitation!
        \end{itemize}
    \end{frame}

    \begin{frame}{Eigenschaften der Äquivarianzgleichungen}
        \begin{itemize}
            \item \textbf{Theorem:} Die Äquivarianzgleichungen sind
            \alert{involutiv} und folglich \alert{formal integrierbar}. \pause
            \item Folgt aus der Involutionstheorie für PDEs und einem Beweis für das spezielle System \pause
            \item Ergebnis ist von fundamentaler Bedeuting für den konstruktiven Ansatz \pause
            \begin{itemize}
                \item Aussagen über den Lösungsraum $\Rightarrow$ exakte Lösung\pause
                \item Formaler Potenzreihenansatz $\Rightarrow$ perturbative Lösung \pause
            \end{itemize}
            \item Dieses Resultat existiert nicht im kanonischen Bild!
        \end{itemize}
    \end{frame}

    \begin{frame}{Exakte Lösung: Einstein}
        \begin{itemize}
            \item Zweites Jet-Bündel über dem metrischen Tensorbündel hat
            Dimension $4+10+40+100 = 154$ ($\operatorname{dim}M = 4$) \pause
            \item Äquivarianzgleichungen sind involutiv und haben Rang 140 \pause
            \item $\Rightarrow$ allgemeine Lösung ist von der Form
            \[ L(x,g,\partial g,\partial\partial g) = \sqrt{-g} \cdot F(\psi_1,\ldots,\psi_{14}) \]
            mit beliebiger Funktion $F$ und 14 Krümmungsskalaren $\Psi_i(g,\partial g,\partial\partial g)$ \pause
            \item Beschränkung auf 2.\ Ableitungsordnung $\Rightarrow$ \alert{Einstein-Hilbert}
            \[ L = \frac{1}{2\kappa}\sqrt{-g}(\underbrace{R}_{\psi_1} - 2\Lambda)\]
        \end{itemize}
    \end{frame}

    \begin{frame}{Neuartige Materie}
        \begin{itemize}
            \item \textbf{Beispiel:} Zwei Skalarfelder gekoppelt an zwei verschiedene Metriken
            \[ L = \sqrt{-g} g^{ab} \phi_{,a}\phi_{,b} - m_\phi^2 \phi^2 + \sqrt{-h} h^{ab} \psi_{,a}\psi_{,b} - m_\psi^2 \psi^2 \] \pause
            \item Geometriebündel ist die Summe von zwei Metrikbündeln \pause
            \item Gotay-Marsden-Koeffizienten $\Rightarrow$ Äquivarianzgleichungen \pause
            \item Lichtkegel
            \[ P(k) = g(k,k) h(k,k) = 0 \]
            gibt zusätzliche Bedingungen für relevante Lösungen im Sinne von Kovarianter
            Konstruktiver Gravitation (kausale Kompatibilität)
        \end{itemize}
    \end{frame}


    \section{Störungstheorie}\label{sec:stoerungstheorie}

    \begin{frame}{Potenzreihenansatz}
        \begin{itemize}
            \item Involutivität der Äquivarianzgleichungen impliziert formale Integrierbarkeit
            \begin{itemize}
                \item Formaler Potenzreihenansatz \[ L(x) = \sum_{k=0}^\infty a_k (x-x_0)^k \] ist iterativ lösbar
                \item $n$-te Prolongation der PDE löst Ordnung $n$ des Ansatzes
            \end{itemize}
            \item \textbf{Perturbative Kovariante Konstruktive Gravitation:}
            \begin{itemize}
                \item \makebox[2cm]{\hfill$n=1$:} Theorie in der \alert{Minkowski}-Raumzeit
                \item \makebox[2cm]{\hfill$1<n<\infty$:} Theorie für schwache Gravitationsfelder
            \end{itemize}
        \end{itemize}
    \end{frame}

    \begin{frame}{Lösungstechniken}
    \end{frame}


    \section{Anwendung: Gravitationswellen von Doppelbrechender Materie}\label{sec:anwendung}


    \section{Ausblick}\label{sec:ausblick}

\end{document}
