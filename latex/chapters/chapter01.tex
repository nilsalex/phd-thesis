\chapter{Introduction to the constructive gravity programme}

\section{The r\^ole of gravity in physics}
As the title suggests, this thesis is primarily concerned with \emph{gravity}. In the ensemble of physical theories, gravity plays a special r\^ole. It serves a different purpose than the theories we will call \emph{matter theories}. The latter are subject to direct observations: photons---quanta of the electomagnetic field---hit the observers retina, allowing her to make inferences about the source of the particles. Charged fermions---again quanta of a corresponding matter field---induce signals in a semiconductor detector. Specific signatures in the signals may be associated with certain events that contributed to the production of the incident fermions, such that the statistics of these observations is able to falsify hypotheses about the underlying mechanisms.

How does gravity fit into this picture? The revolution of a binary star about its center of mass, commonly known to be caused by gravity, is not observed directly. Neither are its gravitational spin-up and eventual merger. Rather, the stars emit photons that are picked up by the astronomer, who concludes details about the trajectories. When the LIGO and Virgo Collaborations announced the first observation of gravitational waves \cite{ligo}, the ground-breaking detection was earth-bound, but in a certain sense not \emph{direct}: it amounts to the analysis of interference patterns from photons that bounced off of mirrors at the end of the detector arms. General relativity predicts that these arms should expand and contract under the influence of incident gravitational waves. Eventually, the signature in the interference pattern was found to match the predictions for a binary black hole merger.

From this point of view, gravity sets the stage for the propagation of matter fields. This is witnessed by the dynamics for matter fields, e.g.~the electromagnetic potential $A$. Maxwell's field equations are derived from the action functional
\begin{equation*}
  S_\text{Maxwell}\lbrack A\rbrack = \int\mathrm d^4x \sqrt{-g} g^{ac} g^{bc} F_{ab} F_{cd},
\end{equation*}
which depends on the potential $A$ via the field strength tensor $F = \mathrm dF$.

\begin{itemize}
\item general relativity as closure of Maxwell electrodynamics
\item uniqueness (different derivations and proofs)
\item modified gravity, how this is compatible to uniqueness
\end{itemize}

\section{Modified gravity from refined matter theories}

\begin{itemize}
\item motto: matter tells gravity how to curve, gravity tells matter how to move -> lift to theory level
\item 2d plot of theories?
\item what could hint at the existence of refined matter theories?
\item dark matter
\end{itemize}

\section{Canonical and covariant approaches to constructive gravity}

\begin{itemize}
\item canonical approach: implementing the constraint algebra
\item a success story, but does it keep its promises?
\item covariant view: equivalent and complementary
\item simpler?
\end{itemize}

Two kinds of inconsistencies: Within the SMs -> quantum gravity inevitable? But also pheno that does not match SM -> closure?

We only ever so slightly deviate from the established models about matter and gravity. For example, where the standard model is restricted to field equations of second derivative, we keep this restriction. This is not because other efforts are not deemed worthwile---they certainly are, but should be explored \emph{ceteris paribus}, one at a time. Our focus lies on novel matter theories and their gravitational implications \emph{within the existing meta-theory of classical physics}.

\textbf{game plan: structure of the thesis}
