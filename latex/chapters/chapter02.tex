\chapter{Covariant constructive gravity}

\textbf{take-home message: two conditions on gravitational dynamics determined by coupled matter theory}

This chapter is concerned with the two axioms of covariant constructive gravity. Their informal formulation is as follows:
\begin{axiom}[Diffeomorphism invariance]
  The dynamical laws that govern gravity are invariant under spacetime diffeomorphisms. \cite{cgg paper}
\end{axiom}
\begin{axiom}[Causal compatibility]
  Provided that spacetime is additionally inhabited by matter fields, their dynamics is causally compatible with the gravitational dynamics. \cite{cgg paper}
\end{axiom}

For a more precise formulation of the axioms, which will enable us to derive their consequences for gravitational theories coupled to novel matter, an introduction of the basic concepts of Lagrangian field theory is in order.

\section{Lagrangian field theory}
For the purpose of the present work, a \emph{Lagrangian field theory} will be a geometric formulation of certain conditions on sections $\sigma\in\Gamma(\pi)$---called \emph{fields}---of some bundle $E\overset{\pi}{\longrightarrow} M$. These conditions select the physical realizations of fields admissible by the theory and constitute the \emph{dynamical laws}. The bundle $\pi$ shall be constructed from a tensor bundle, i.e.~be a sub-bundle of some tensor bundle $T^r_sM$. It is possible to extend the framework to include other bundles, with the caveat that a lift of the action of the diffeomorphism group on $M$ to $E$ may have to be specified manually. Although not relevant for much of the development of the theory, the base manifolds to be considered later for concrete examples will be spacetime manifolds of dimension 4.

\begin{example}
  Two examples for Lagrangian field theories are
  \begin{itemize}
    \item Einstein gravity on the symmetric sub-bundle of $T^2_0M$ of inverse metric tensors\footnote{The equivalent formulation as a theory for a metric tensor fields takes place on the bundle $T^0_2M$.} with dynamical laws given by the Einstein equations action, and
    \item Maxwell electrodynamics on the bundle $T^\ast M$ of potential one-forms with dynamical laws given by the Maxwell equations.
  \end{itemize}
\end{example}

Both theories are \emph{Lagrangian} because they derive their dynamical laws in a certain geometric manner. The mechanism will be explained in the following, but first, let us fix some of the notation involved.

A bundle is denoted as $E\overset{\pi}{\longrightarrow}M$, where $E$ is the total space, $M$ is the base manifold $M$, $\pi$ is the submersion. As a shorthand, it is common to write just $\pi$---it is then understood that total space and base manifold are domain and co-domain of $\pi$, respectively. The dimension of $M$ is written as $n$, the dimension of a typical fibre $F$ of $\pi$ as $m$. Coordinate functions on $E$ are denoted by $(x^i,u^A)$. Such coordinates extend to the $k$th jet bundle $J^kE \overset{\pi_k}{\longrightarrow}M$ over $\pi$ as $(x^i,\allowbreak u^A,\allowbreak u^A_{i_1},\allowbreak u^A_{i_1i_2},\allowbreak \dots,\allowbreak u^A_{i_1\dots i_k})$. The literature on jet bundles mostly employs multi-indices for higher-order jet bundles (see e.g.~Ref.~\cite{saunders}), which is certainly the right approach for studying the properties of jet bundles, but for practical calculations on the second-order jet bundle performed below a different technique will be used which also takes care of ambiguities regarding symmetric indices. Prolongations of sections $\sigma$ are denoted with $j^k(\sigma)$, projections between jet bundles of different order with $\pi_{k,k^\prime}(\sigma)$. The latter are submersions in their own right and thus also define bundle projections. Throughout, the Einstein summation convention is used.

\begin{definition}{(Lagrangian)}
  Let $M$ be a smooth manifold of dimension $n$ and $E \overset{\pi}{\longrightarrow} M$ a smooth fibre bundle over $M$ with typical fibre $F$. A \textbf{Lagrangian} $\mathcal L$ of order $k$ is an element
  \begin{equation}
    \mathcal L \in \textstyle\bigwedge^n_0\pi_k.
  \end{equation}
In other words, $\mathcal L$ is a horizontal $n$-form on the $k$-th-order jet bundle $\pi_k$ of $\pi$.

Assuming $M$ to be orientable with volume form $\Omega\in\Lambda^nM$, a Lagrangian $\mathcal L$ is equivalently characterized by its \textbf{Lagrangian density} $L\in C^\infty(J^k\pi)$,
  \begin{equation}\label{lagrangian_density}
  \mathcal L = L\pi_k^\ast\Omega.
\end{equation}
\end{definition}

The claim of (\ref{lagrangian_density}) becomes apparent in local coordinates, where a horizontal $n$-form on $\pi_k$ appears as
\begin{equation}\label{lagrangian_local}
  \textstyle\bigwedge^n_0\pi_k\ni \mathcal L = \mathcal L(x^i,u^A,u^A_i,\dots) dx^1\wedge \dots \wedge dx^n.
\end{equation}
From (\ref{lagrangian_local}) and (\ref{lagrangian_density}), it is also clear how the notion of a Lagrangian as a horizontal $n$-form captures in a geometric way the notion used elsewhere as a bundle map $J^kE \rightarrow \Lambda^nM$ (see Refs.~\cite{saunders,marsden,ccg}).


From now on, we will consider smooth, orientable base manifolds $M$ and smooth bundles over the base manifold. Depending on the context, the symbol $\Omega$ will denote either the form on $M$ or the pullback to various bundles over $M$.

\begin{definition}{(local action functional)}
  Given a Lagrangian $\mathcal L \in \bigwedge^n_0\pi_k$ and a compact $n$-dimensional submanifold $C \subset M$, the \textbf{local action functional} is defined as the map
  \begin{equation}
    \sigma \mapsto S\lbrack\sigma\rbrack = \int_C (j^k\sigma)^\ast\mathcal L
  \end{equation}
for all local sections $\sigma$ of $\pi$ with support on $C$.
\end{definition}

Lagrangian field theory now stipulates that sections $\sigma\in\Gamma(\pi)$ are physical if they are extremals of the action functional \cite{}. The well-known \emph{Euler-Lagrange} equations from the calculus of variations provide a neccessary condition in local coordinates which such sections extremals must satisfy.
\begin{proposition}
  Let $L\in C^\infty(J^k\pi)$ be a Lagrangian density. Let $C$ be a compact submanifold of $M$ and $\sigma$ be a local section of $\pi$ such that the local action functional $S\lbrack\sigma\rbrack$ is defined. If $\sigma$ is an extremal of $S$, it satisfies the \textbf{Euler-Lagrange equations}
  \begin{equation}\label{euler_lagrange_local}
    (j^{2k}\sigma)^\ast\left( \sum_{l=0}^{k} (-1)^l D_{i_1} \cdots D_{i_l} \frac{\partial L}{\partial u^A_{i_1\cdots i_l}}\right) = 0.
  \end{equation}
\end{proposition}
\begin{proof}
  See Saunders 1989 or older, original work
\end{proof}

The intrinsic equivalent to the Euler-Lagrange equations in local coordinates introduces a new object, the \emph{Cartan form}, which plays a central role in the geometrization of Lagrangian field theory.
\begin{proposition}
  Given a Lagrangian $\mathcal L = L\Omega \in \bigwedge^n_0\pi_k$, there exists an $n+1$ form $\Theta_L\in\bigwedge^{n}_0\pi_{2k-1,k-1}\cap\bigwedge^{n}_{1}\pi_{2k-1}$, such that, globally, the variation of the Lagrangian is given by
  \begin{equation}\label{global_lagrange_form}
    \delta L = \pi^\ast_{2k,k} \left( dL\wedge\Omega\right) + d_h \Theta_L
  \end{equation}
  and extremals of $\mathcal L$ are extremals of $\Theta_L$ in the sense that
  \begin{equation}
    \pi_{2k-1,k}^\ast (j^k\sigma)^\ast\mathcal L = (j^{2k-1}\sigma)^\ast\Theta_L.
  \end{equation}
  Such a form $\Theta_L$ is called a \textbf{Cartan form}.
\end{proposition}
\begin{proof}
  See Saunders 1989 or older, original work
\end{proof}

This definition generalizes the local derivation of (\ref{euler_lagrange_local}): The variation $\delta L$ is obtained by lifting $d\mathcal L$ to $\pi_{2k}$ and cancelling non-horizontal terms (over $E$) by adding a derivative, which corresponds to repeated integrations by parts.

A possible\footnote{Generally, $\Theta_L$ is not uniquely defined.} coordinate expression for $\Theta_L$ is \ref{}
\begin{equation}\label{local_cartan_form}
  \Theta_L = L\Omega + \sum_{s=0}^{k-1} \sum_{l=0}^{k-s-1} (-1)^l D_{i_1} \cdots D_{i_l} \left(\frac{\partial L}{\partial u^A_{j i_1 \dots i_{l} p_1 \dots p_s}}\right) \psi^A_{p_1\dots p_s} \wedge \left(i_{\partial_j}\Omega\right),
\end{equation}
where the forms $\psi^A_{p_1\dots p_s} = du^A_{p_1\dots p_s} - u^A_{p_1\dots p_sq} dx^q$ span the contact system of $\pi_k$ (see \ref{}).

Straight-forward application of (\ref{global_lagrange_form}) to (\ref{local_cartan_form}) yields the well-known coordinate expression
\begin{equation}
  \delta L = \left( \sum_{l=0}^{k} (-1)^l D_{i_1} \cdots D_{i_l} \frac{\partial L}{\partial u^A_{i_1\cdots i_l}}\right) u^A \wedge \Omega
\end{equation}
for the Euler-Lagrange form, reconciling the intrinsic formulation using the Cartan form with the explicitly coordinate-dependent\footnote{Which is not to say \emph{ill-defined}.} formulation using the Euler-Lagrange equations.

In later sections, we will restrict our attention to Lagrangians of second derivative order. As it turns out, the Cartan form for such a theory is \emph{unique}.
\begin{proposition}
  The Cartan form is unique for second-order Lagrangians.
\end{proposition}
\begin{proof}
  See Saunders 1989 or older, original work
\end{proof}

\section{Axiom I: diffeomorphism invariance}

In the language of jet bundles, the first axiom can be formalized as equivariance condition on the Lagrangian. We will restrict our attention here to tensorial field theories of second derivative order.

\begin{definition}
  A Lagrangian on the second jet bundle over a sub-bundle of some tensor bundle is called a \textbf{tensorial field theory of second (derivative) order}.
\end{definition}

By virtue of the pushforward-pullback construction, sub-bundles of tensor bundles carry a canonical action of the diffeomorphism group $\mathrm{Diff}(M)$ as bundle automorphisms, denoted as $\varphi_E\in \mathrm{Aut}(E)$ for every $\varphi\in\mathrm{Diff}(M)$. We call this the \emph{lift} of the diffeomorphism $\varphi$. This action, in turn, lifts naturally to the jet bundles over $E$.

\begin{definition}[prolongation of morphisms]
  Let $E \overset{\pi_E}{\longrightarrow} M$ and $H \overset{\pi_H}{\longrightarrow} N$ be two bundles. The $k$th-order jet bundle lift of a bundle morphism $(F,f)$ from $\pi_E$ to $\pi_H$ is the unique bundle morphism $(j^k(F),f)$ from $J^k\pi_E$ to $J^k\pi_H$ that lets the diagram in Fig.~(??) commute.
\end{definition}
A proof for the existence and uniqueness of this construction can be found in Ref.~\cite{saunders}. With the notion of the \emph{lift} of a bundle automorphism at hand, we now give the first axiom a precise meaning
\begin{definition}[diffeomorphism invariant theory]
  A Lagrangian field theory is called \textbf{diffeomorphism invariant} if its Lagrangian $\mathcal L\in\textstyle\bigwedge_0^n\pi_k$ is invariant with respect to the lifted action of $\mathrm{Diff}(M)$ on $J^kE$, i.e.~ if for all $\varphi\in\mathrm{Diff}(M)$
  \begin{equation}\label{lagrangian_diffeo_invariance}
    j^k(\varphi_E)^\ast \mathcal L = \mathcal L.
  \end{equation}
\end{definition}
This definition applies to other bundles and groups as well, as long as the action on the total space is defined.

Using a local representation $\mathcal L = L d^nx$, the diffeomorphism invariance condition (\ref{lagrangian_diffeo_invariance}) translates into an equivariance condition on the Lagrangian density
\begin{equation}\label{lagrangian_density_diffeo_equivariance}
  L \circ j^k(\varphi_E) = \lvert D\varphi\rvert^{-1} L,
\end{equation}
where $\lvert D\varphi\rvert$ denotes the determinant of the Jacobian $D\varphi$. Covariant constructive gravity derives its calculational power from the observation that (\ref{lagrangian_density_diffeo_equivariance}) is locally equivalent to a system of linear partial differential equations (PDEs) for the Lagrangian density $L$. For this theorem and the remainder of the chapter, we will work on the second jet bundle, as we are ultimately interested in investigating field theories of second derivative order.
\begin{theorem}
  Let $L$ be the local Lagrangian density of a second derivative order Lagrangian $\mathcal L = Ld^4x \in \textstyle\bigwedge_0^n\pi_2$. If $\mathcal L$ is diffeomorphism invariant, i.e.~satisfies the invariance condition (\ref{lagrangian_diffeo_invariance}) of the first axiom of covariant constructive gravity, its local representation $L$ satisfies a system of first-order linear partial differential equations given by
  \begin{equation}
    \begin{aligned}
      0 &{} = L_{\colon m} \\
      0 &{} = L_{\colon m} \\
    \end{aligned}
  \end{equation}
\end{theorem}

\begin{itemize}
\item justification, history, general relativity
\item geometric formulation
\item equivariance equations
\item noether theorem
\item energy momentum tensor
\end{itemize}

\begin{proposition}{(First Noether theorem)}
  \begin{equation}
    0 = d \left((j^{2k-1}\sigma)^\ast J^\mathcal L(\xi_E)\right)
  \end{equation}
\end{proposition}

\section{Axiom II: causal compatibility}

\begin{itemize}
\item derivation (consistent co-evolution)
\item principal symbol, principal polynomial
\item geometric formulation
\end{itemize}

\section{Relation to canonical constructive gravity}
\begin{itemize}
\item relation between diffeo invariance and constraint algebra
\item the role of causality in the constraint algebra
\end{itemize}

\section{Example: Einstein gravity}
\begin{itemize}
\item Gotay-Marsden coefficients, equivariance equations
\item GR as solution
\item causality is already solved
\end{itemize}

\section{Example: area metric gravity}
\begin{itemize}
\item introduction to area metric gravity
\item principal polynomial
\item Gotay-Marsden coefficients, equivariance equations
\end{itemize}

\section{Example: bimetric gravity}
\begin{itemize}
\item what is bimetric gravity?
\item principal polynomial(s)
\item Gotay-Marsden coefficients, equivariance equations
\end{itemize}

