\chapter{Outlook: symmetry-reduced constructive gravity}\label{chapter_cosmo}

\textit{In the previous chapters, we reduced the complexity of the covariant construction procedure by considering a perturbative equivalent. Consequently, the result was an approximation of the exact gravitational theory, valid for sufficiently weak gravitational fields. A second approach towards reducing the complexity of the equivariance equations is symmetry reduction, which assumes that the gravitational field exhibits certain symmetries. Ideally, these symmetries bring the construction equations into a much simpler form. The solutions of the reduced equations should be theories of gravity valid in this symmetry-reduced sector, comparable to the Friedmann equations for spatially homogeneous and isotropic Einstein gravity. In this chapter, we explore a possible approach towards symmetry-reduced covariant constructive gravity by reducing the bundle on which the procedure operates. Our main result will be that the FLRW model can be recovered without the need to know the full Einstein-Hilbert Lagrangian beforehand. The corresponding area metric equivalent will not be solved, only the construction equations are derived. For a more in-depth study of symmetry reduction in the context of canonical constructive gravity, see Ref.~\cite{Duell_2020}.}

\section{The cosmological bundle}
As the symmetry in question, let us consider the cosmological symmetry, where spacetime is assumed to be spatially homogeneous and isotropic (see e.g.~\cite{Wald_1984}). It is well known what this entails for the metric field: implicitly, this symmetry comes with the assumption of a sliced spacetime. The spatial degrees of freedom in the metric tensor reduce to a single \emph{scale factor}.

More formally, we have a slicing $\phi\colon\mathbb R\times\Sigma\rightarrow M$, which induces embeddings
\begin{equation}
  \begin{aligned}
    \phi_\lambda\colon\Sigma\rightarrow {} & M \\
    p \mapsto {} & \phi_\lambda(p) \vcentcolon= \phi(\lambda,p)
  \end{aligned}
\end{equation}
of the spatial hypersurface $\Sigma$ into the spacetime manifold $M$.

\begin{itemize}
\item idea: symmetric field parameterized by vector fields and scalar (densities)
\item examples: metric theory, area metric theory
\end{itemize}

\section{Recovering the FLRW model}
\begin{itemize}
\item equivariance equations on cosmological bundle for symmetric metric are solved by Friedmann equations
\end{itemize}

\section{Construction equations for area metric cosmology}
\begin{itemize}
\item set up equivariance equations for area metric cosmological bundle
\item proof: matter causality is metric
\end{itemize}

