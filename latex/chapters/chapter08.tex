\chapter{Conclusions}

In this thesis, the concept of covariant constructive gravity has been put on a solid mathematical footing. Lagrangian field theory on jet bundles turned out to be ideally suited for the definition of the general covariance axiom. The equivariance equations that follow from this axiom transform the implementation into a computational task, opening up the \emph{constructive} pathway towards modified theories of gravity. Using the Cartan form that corresponds to a diffeomorphism invariant Lagrangian density, we have seen how general covariance implies a version of the first and second Noether theorem---important results that have proven very useful further down the line. Within this framework, the axiom of causal compatibility has been formulated as additional algebraic conditions on the gravitational field equations.

From the equivariance equations and causality conditions, we derived a concise algorithm which guides the construction of novel gravitational theories that implement general covariance and are causally compatible to a given matter theory. Because this can be seen as generalization of Lovelock's uniqueness theorems for Einstein gravity \cite{Lovelock_1969,Lovelock_1971,Lovelock_1972}, we could show that this construction procedure applied to Maxwell's electrodynamics indeed reproduces metric gravity as derived by Einstein.

Of course, covariant constructive gravity would not be that interesting if it were just another tool that reproduces Einstein gravity. Its \emph{raison d'\^etre} is the derivation of \emph{modified} theories of gravity that complete novel matter theories to predictive models of the universe. The remainder of the thesis was dedicated towards achieving this goal. First, we have discussed three examples of novel matter theories: birefringent electrodynamics and two bimetric theories. While it is straightforward to set up the construction procedure and derive general results concerning the solution space, \emph{finding} these solutions in practice is notoriously hard and turned out not to be feasible for the examples in question.

Therefore, we investigated possibilities to arrive at results that are valid in certain specific settings, without the need to know the ``full'' solutions. Our main strategy was perturbation theory, which seeks to find solutions that are valid for small deviations of the gravitational field from a Lorentz invariant background. With a corresponding perturbation ansatz, the equivariance equations transform into a system of linear equations for the expansion coefficients. As a consequence of the Lorentz invariance of the background geometry, the expansion coefficients themselves exhibit Lorentz invariance, which reduces their dimensionality a lot---before solving any equivariance equation.

Many of the computations that are necessary in order to construct Lorentz invariant ansätze and solve the perturbative equivariance equations have been delegated to the computer. For this purpose, two Haskell packages have been developed and presented in this thesis---with a focus on the package \texttt{sparse-tensor} which composes and solves the equivariance equations. Methods from functional programming lend themselves for an efficient and safe implementation of tensor algebra, enabling us to repeat and modify calculations whenever required, without having to redo them by hand.

Chapter \ref{chapter_weak_area} was the culmination of this thesis, where we put all pieces together and derived perturbative area metric gravity up to third perturbation order in the Lagrangian density. From this Lagrangian, the linearized gravitational field equations and their second perturbation order follows. Quite remarkably, the linearized field equations coincide with the equations derived in the canonical framework \cite{Schneider_2017,Alex_2019}---with an important caveat: the field equations in the canonical picture as obtained by solving the canonical closure equations \cite{D_ll_2018,Schneider_2017} are \emph{not} causally compatible with the matter theory, i.e.\ their principal polynomial does not reduce to the Minkowski metric.

In order to cure the causality, one of the \emph{eleven} gravitational constants had to be fixed \cite{Alex_2019}, reducing their number to \emph{ten}, which then equals the number of constants obtained in the covariant framework. The reason for this mismatch is believed to lie in the so-called ansatz equations, which enforce Lorentz invariance of the perturbation ansatz. In perturbative covariant constructive gravity, these are already solved by considering only Lorentz invariant ansätze to begin with. Canonical constructive gravity, on the other hand, makes ansätze \emph{after} performing the 3+1 split---effectively implementing a spatial $\mathrm{SO}(3)$ symmetry. But this is a \emph{weaker} requirement than the $\mathrm{SO}(1,3)$ symmetry that follows from the equivariance equations. In the case of the linearized field equations, not all of the ansatz equations seem to be solved---one condition on the gravitational constants must be missing. To remedy this, one has to find the equivalent of the ansatz equations in the canonical picture by prolonging and projecting the PDE or otherwise ensure Lorentz invariance of the ansätze.

Either way, the comparison demonstrates that the appearance of matter causality in the canonical constraint algebra is not responsible for the causality of the gravitational theory. In the case of linearized field equations\footnote{But also for the second-order equations, as shown in Sect.~\ref{section_area_construction}.}, diffeomorphism invariance is what constrains the causality of the gravitational theory. Whether diffeomorphism invariance is enforced by imposing it directly on the Lagrangian density or by requiring the canonical constraint algebra to implement the hypersurface deformation algebra (in whichever frame!) is secondary.

The fact that we can reason about the canonical closure programme using results from covariant constructive gravity

\begin{itemize}
\item complements canonical approach
\item impulses for canonical constructive gravity: causality, ansätze
\item from new concept to prediction of novel effects
\item conceptual, more work needed for prediction of outcome of high-precision experiments
\item relevance is given regardless of the status of quantum gravity
\end{itemize}

coming back to the quote
