\chapter{Covariant constructive gravity}

\textbf{take-home message: two conditions on gravitational dynamics determined by coupled matter theory}

\section{Lagrangian field theory}

\begin{definition}{(Lagrangian)}
  Let $M$ be a smooth manifold of dimension $n$ and $E \overset{\pi}{\longrightarrow} M$ a smooth fibre bundle over $M$ with typical fibre $F$. A \textbf{Lagrangian} $\mathcal L$ of order $k$ is an element
  \begin{equation}
    \mathcal L \in \Lambda^n_0(J^k\pi),
  \end{equation}
the space of horizontal $n$-forms on the $k$-th-order jet bundle of $\pi$.
\end{definition}

In local coordinates, a Lagrangian has the representation
\begin{equation}
  \mathcal L = L(x^i, u^A, u^A_i, \dots) \, \omega
\end{equation}
where $\omega = dx^1 \wedge \cdots \wedge dx^n$.

The \emph{action} $S\lbrack\phi\rbrack$ corresponding to a Lagrangian $\mathcal L$ is the integral of $\mathcal L$ composed with a section $\phi\in\Gamma(\pi)$
\begin{equation}
  S\lbrack\phi\rbrack = \int (j^k\phi)^\ast \mathcal L = \int L(x^i, \phi^A, \partial_i\phi^A, \dots) d^4x.
\end{equation}

Lagrangian field theory now stipulates that sections $\phi\in\Gamma(\pi)$ are physical if they are stationary points of the action functional.

\begin{itemize}
\item Lagrangian as differential form on a jet bundle
\item variational problem
\item cartan form
\item noether theorem
\item energy momentum tensor
\end{itemize}

\section{Axiom I: diffeomorphism invariance}
\begin{itemize}
\item justification, history, general relativity
\item geometric formulation
\item equivariance equations
\end{itemize}

\section{Axiom II: causal compatibility}
\begin{itemize}
\item derivation (consistent co-evolution)
\item principal symbol, principal polynomial
\item geometric formulation
\end{itemize}

\section{Relation to canonical constructive gravity}
\begin{itemize}
\item relation between diffeo invariance and constraint algebra
\item the role of causality in the constraint algebra
\end{itemize}

\section{Example: Einstein gravity}
\begin{itemize}
\item Gotay-Marsden coefficients, equivariance equations
\item GR as solution
\item causality is already solved
\end{itemize}

\section{Example: area metric gravity}
\begin{itemize}
\item introduction to area metric gravity
\item principal polynomial
\item Gotay-Marsden coefficients, equivariance equations
\end{itemize}

\section{Example: bimetric gravity}
\begin{itemize}
\item what is bimetric gravity?
\item principal polynomial(s)
\item Gotay-Marsden coefficients, equivariance equations
\end{itemize}

