\chapter{Application: gravitational radiation from birefringent matter dynamics}
\label{chapter_weak_area}

\textit{So far, we have developed the general framework of covariant constructive gravity and derived a perturbative equivalent. A few examples illustrated the constructions, but the presentation focused on broad applicability to various geometries, without any specific bundle or matter theory in mind. In this chapter, we shift our focus and consider in depth the application of the framework to generalized linear electrodynamics, a birefringent generalization of Maxwell electrodynamics introduced in Chapter \ref{chapter_construction_algorithm}. Applying the perturbative construction procedure to third order yields gravitational field equations to second order. We will carefully analyze a $3+1$ split for the linear part of this theory and restrict to a certain sector with, in a very specific sense, physically sane phenomenology. Afterwards, we solve the two-body problem to first order and obtain the orbits of a binary system in area metric gravity. Building up on this solution, the second order of the field equations is used to derive the emission of gravitational radiation from the binary system and the radiative loss, which causes spin-up of the system. The binary star subject to area metric gravity turns out to exhibit qualitatively new behaviour as compared to Einstein gravity, e.g.~additional massive modes of gravitational radiation and a modification of Kepler's third law.}

\textit{To a large extent, the work presented in this chapter has been published as Ref.~\cite{grav_rad_paper}. The results on radiation loss are not part of this publication.}

\section{Area metric gravity}
The matter theory in question is generalized linear electrodynamics (GLED) as defined in Def.~\ref{def_gled} with the Lagrangian density
\begin{equation}\label{lagrangian_gled}
  L_\text{GLED} = \omega_G G^{abcd} F_{ab} F_{cd},
\end{equation}
where we choose without loss of generality the scalar density
\begin{equation}\label{area_density}
  \omega_G = \left(\frac{1}{24}\epsilon_{abcd}G^{abcd}\right)^{-1}.
\end{equation}
The principal polynomial of GLED is quartic and takes the form
\begin{equation}
  \mathcal P_\text{GLED}(k) = -\frac{1}{24} \omega_G^2 \epsilon_{mnpq} \epsilon_{rstu} G^{mnra} G^{bpsc} G^{dqtu} k_a k_b k_c k_d.
\end{equation}
As appropriate Lorentz invariant expansion point constructed from the Minkowski metric $\eta$, we already determined in Example \ref{example_expansion_points}
\begin{equation}\label{area_expansion_point}
  N^A = J_{abcd}^A (\eta^{ac} \eta^{bd} - \eta^{ad} \eta^{bc} + \epsilon^{abcd}).
\end{equation}
Before solving the system of equivariance equations perturbatively around $N$, let us reconsider the reduced power series ansatz \eqref{ansatz_reduced}. In addition to dropping terms with a total number of derivatives that is odd or greater than 2, and dropping non-Lorentz invariant expansion coefficients, we can also discard the linear term $a_A H^A$. This term would yield a constant in the Euler-Lagrange equations, causing the flat expansion point $N$ to no longer constitute a solution to the vacuum field equations. However, the perturbation ansatz stipulates that we perturb around a solution of the field equations. Since it is obvious that Equation \eqref{prolong_0} implies from vanishing coefficients $a_A$ that also the coefficient $a$ vanishes, we readily drop both and make the further reduced ansatz
\begin{equation}\label{ansatz_area_gravity}
  \begin{aligned}
    L &{} = a_A^{\hphantom AI} H^A_{\hphantom AI} \\
      &{} \hphantom{=} + a_{AB} H^A H^B + a_{A\hphantom pB}^{\hphantom Ap\hphantom Bq} H^A_{\hphantom Ap} H^B_{\hphantom Bq} + a_{AB}^{\hphantom{AB}I} H^A H^B_{\hphantom BI} \\
      &{} \hphantom{=} + a_{ABC} H^A H^B H^C + a_{AB\hphantom pC}^{\hphantom{AB}p\hphantom Cq} H^A H^B_{\hphantom Bp} H^C_{\hphantom Cq} + a_{ABC}^{\hphantom{ABC}I} H^A H^B H^C_{\hphantom CI} \\
      &{} \hphantom{=} + \mathcal O(H^4).
  \end{aligned}
\end{equation}

\section{Solving axiom 1}\label{sect_area_lagrangian}
Step 1 of the perturbative construction algorithm \ref{perturbative_algorithm} consists in computing the Gotay-Marsden coefficients for the gravitational bundle. For area metric gravity, we found in Sect.~\ref{section_gled}
\begin{equation}
  \gmc{A}{B}{n}{m} = 4 I^{pqrn}_B J^A_{pqrm},
\end{equation}
which followed from the general result \eqref{gmc_contra} for purely contravariant tensor bundles.

Proceeding with step 2, we need to construct a basis for the Lorentz invariant expansion coefficients
\begin{equation}
  (a_A^{\hphantom AI}, a_{AB}, a_{A\hphantom pB}^{\hphantom Ap\hphantom Bq}, a_{AB}^{\hphantom{AB}I}, a_{ABC}, a_{AB\hphantom pC}^{\hphantom{AB}p\hphantom Cq}, a_{ABC}^{\hphantom{ABC}I})
\end{equation}
in the ansatz \eqref{ansatz_area_gravity}. This task is solved using the Haskell library \texttt{sparse-tensor} \cite{sparse-tensor} discussed in Chapter \ref{}. The result is a basis of dimension 237, enumerated in full in Appendix \ref{} and summarized in Table \ref{}. It should be emphasized that the requirement of Lorentz invariance, which is not a direct stipulation but follows via the equivariance equations from a physically motivated assumption about the expansion point, drastically reduces the dimensionality of the ansatz from
\begin{equation}
  210 + \frac{21\cdot 22}{2} + 21\cdot 210 + \frac{84\cdot 85}{2} + \frac{21\cdot 22\cdot 23}{6} + \frac{21\cdot 22}{2} \cdot  210 + 21\cdot\frac{84\cdot 85}{2} = 133672
\end{equation}
to only 237. In principle, the correctness of the ansatz can be verified by showing that it is the most generic solution to the ansatz equations \ref{}. All we have to show is that the dimensionality of the ansatz equals the corank of the linear system of ansatz equations. For the ansatz including third-order coefficients, the system is quite large---considering that the coefficient space is already of dimension 133672---such that, on standard hardware, the rank cannot be computed naively by storing the matrix in memory and using methods like singular value decomposition or fraction-free gaussian elimination. It is rather easy, however, to use the aforementioned methods and work out the corank of the linear system determining the Lorentz invariant ansatz coefficients to second order, as the dimension of this ansatz space is only $210 + \frac{21\cdot 22}{2} + 21 \cdot 210 + \frac{84\cdot 85}{2} = 8421$. Confirming the number of obtained basis ansätze up to second order, the corank of the corresponding system is indeed 40. The computer code for this computation is published as Ref.~\cite{second-order-area-repo}.
\begin{table}
  \centering
  \begin{tabular}{l r r}
    \toprule
    coefficient & dimension & gravitational constants \\
    \midrule
    $a_A^{\hphantom AI}$ & 3 & $(e_{38},\dots ,e_{40})$ \\ \addlinespace[2pt]
    $a_{AB}$ & 6 & $(e_{1},\dots ,e_{6})$ \\ \addlinespace[2pt]
    $a_{A\hphantom pB}^{\hphantom Ap\hphantom Bq}$ & 15 & $(e_{7},\dots ,e_{21})$ \\ \addlinespace[2pt]
    $a_{AB}^{\hphantom{AB}I}$ & 16 & $(e_{22},\dots ,e_{37})$ \\ \addlinespace[2pt]
    $a_{ABC}$ & 15 & $(e_{41},\dots ,e_{55})$ \\ \addlinespace[2pt]
    $a_{AB\hphantom pC}^{\hphantom{AB}p\hphantom Cq}$ & 110 & $(e_{56},\dots ,e_{165})$ \\ \addlinespace[2pt]
    $a_{ABC}^{\hphantom{ABC}I}$ & 72 & $(e_{166},\dots ,e_{237})$ \\ \addlinespace[2pt]
    \bottomrule 
  \end{tabular}
  \caption{Summary of the Lorentz invariant expansion coefficients for the area metric gravity ansatz \eqref{ansatz_area_gravity} obtained from the Haskell library \texttt{sparse-tensor} \cite{}. The dimension is the number of linearly independent basis tensors returned from the computer program. Assigning labels from $1$ to $237$ to all basis tensors, an ansatz is represented by real numbers $e_1\dots e_{237}$ using its unique basis decomposition. These numbers parameterize the gravitational theory and are thus referred to as \emph{gravitational constants}. For a complete picture of the decomposition of ansätze using basis tensors, refer to Appendix \ref{} or the computer code in Ref.~\cite{second-order-area-repo}.}
\end{table}

With the 237 ansatz coefficients at hand, solving the equivariance equations as required for step 5 is only a matter of inserting the ansatz in the system and its first two prolongations as displayed in Eqns.~\ref{prolong_0}--\ref{prolong_2}, extracting a system of linear equations for the gravitational constants, and solving this system. This task is again performed using efficient computer algebra, implemented in the Haskell library \texttt{safe-tensor}, which is introduced in Chapter \ref{}. The procedure is roughly as follows: A compatibility layer with \texttt{sparse-tensor} is used in order to construct the ansatz tensors and make them available as \texttt{Tensor} types with generalized rank (see Sect.~\ref{}). Together with predefined tensors like Kronecker deltas, intertwiners, Gotay-Marsden coefficients, or the Minkowski metric, the ansatz tensors are used in order to construct the (prolonged) equivariance equations evaluated at $N$ (Eqns.~\ref{prolong_0}--\ref{prolong_2}). Each tensorial equation is a value of type \texttt{Tensor} and, as such, can be evaluated into a list of its components. Every component is a linear equation for the 237 gravitational constants. Collecting all components for all tensorial equations, we obtain a matrix representing the linear system for the constants $e_1\dots e_{237}$. The system is small enough to be brought into reduced row echelon form applying fraction-free gaussian elimination and backward substitution using 64-bit integers\footnote{Exploiting the observation we made earlier that, using intertwiners with purely rational components, all coefficients in the system remain rational.}, which yields a solution that parameterizes the constants with a few remaining indeterminate gravitational constants. As an example for the process, let us walk through the solution for the linear expansion coefficient $a_A^{\hphantom AI}$.

\begin{example}[solution of the equivariance equations to first order]
  Having set $a_A = 0$, the remaining expansion coefficient for the linear order is $a_A^{\hphantom AI}$, which is determined in part by the second unprolonged equation \eqref{prolong_0}. A suitable basis for this coefficient is
  \begin{equation}
    a_A^{\hphantom AI} = J_A^{abcd} J_{pq}^I\lbrack e_1 \cdot \eta_{ac} \eta_{bd} \eta^{pq} + e_2 \cdot \eta_{ac} \delta_b^p \delta_d^q + e_3 \cdot \epsilon_{abcd} \eta^{pq} \rbrack
  \end{equation}
  with three gravitational constants $e_1,e_2,e_3$. Inserting this ansatz into the unprolonged equation
  \begin{equation}
    0 = a_A^{\hphantom AI} \gmc{A}{B}{(n}{m} J^{pq)}_I N^B =\vcentcolon T^{npq}_m
  \end{equation}
  yields a tensorial equation $0 = T^{npq}_m$ with 256 components. Each component is of the form
  \begin{equation}
    0 = c_1\cdot e_1  + c_2\cdot e_2 + c_3\cdot e_3.
  \end{equation}
  The collection of all components is a system of 256 linear equations for three variables. A lot of these equations are redundant, because they are trivial or linearly dependent. A naive reduction by eliminating trivial equations and choosing only one representative for equations that are multiples of each other already reduces the system to the single equation
  \begin{equation}
    0 = 2 e_1 + e_2 + 4 e_3.
  \end{equation}
  Setting e.g.~$e_2 = -2 e_1 -4 e_3$ solves the equivariance equation for the coefficient $a_A^{\hphantom AI}$, leaving it parameterized by two gravitational constants $e_1$ and $e_3$.
\end{example}

Applied to the whole system of equivariance equations, we obtain a parameterization of the solution (displayed in full in Appendix \ref{}) by 50 independent gravitational constants. A subset of 16 constants governs \emph{linearized} area metric gravity via the quadratic Lagrangian density, from which---as we will encounter later---only 11 independent linear combinations play a role for the Euler-Lagrange equations. The procedure outlined here is implemented in Haskell using the library \texttt{sparse-tensor} for ansatz generation as well as the library \texttt{safe-tensor} for constructing and solving the equivariance equations, with the source code and results published as Ref.~\cite{second-order-area-repo}.

\section{Solving axiom 2}
The pedestrian approach towards implementing causal compatibility of the just constructed gravitational theory with GLED is to carefully execute steps 6--12 of the perturbative construction algorithm. This way, we obtain an approximation of the area metric gravity principal polynomial and have to match the causal structure with a first-order expansion of the GLED principal polynomial. While entirely feasible, this approach is less illustrative than the \emph{constructive} approach we employ instead. The underlying realization behind this technique is that the diffeomorphism invariance of the gravitational theory dramatically restricts the possible principal polynomials. In fact, we will see that for third-order area metric Lagrangians, the admissible principal polynomials are already causally compatible with the corresponding expansion of the GLED polynomial. \emph{There is no causality mismatch left to be fixed.}

To this end, recall the GLED polynomial \eqref{gled_polynomial}, which using the scalar density \eqref{area_density} assumes the form
\begin{equation}
  \mathcal P_\text{GLED}(k) = -\frac{1}{\frac{1}{24}(\epsilon_{abcd}G^{abcd})^2} \epsilon_{mnpq} \epsilon_{rstu} G^{mnra} G^{bpsc} G^{dqtu} k_a k_b k_c k_d.
\end{equation}
Expanding this expression to linear order in the perturbation yields
\begin{equation}\label{gled_poly_first_order}
  \begin{aligned}
    \mathcal P_\text{GLED}(k) &{} = \left\{ \left\lbrack 1 - \frac{1}{24} \epsilon(H) \right\rbrack \eta(k,k) + \frac{1}{2} H(k,k) \right\}^2 + \mathcal O(H^2) \\
                              &{} = \lbrack P^{(\leq 1)}_\text{GLED} \rbrack^2 + \mathcal O(H^2),
  \end{aligned}
\end{equation}
where the abbreviations
\begin{equation}
  \epsilon(H) = \epsilon_{abcd} H^{abcd}\quad\text{and}\quad H(k,k) = \eta_{ac} H^{abcd} k_b k_d
\end{equation}
have been introduced. In the following, we will also make use of the contraction
\begin{equation}
  \eta(H) = \eta_{ac} \eta_{bd} H^{abcd}.
\end{equation}
Up to first order, we find that the GLED polynomial factors into the square of a metric polynomial $P^{(\leq 1)}_\text{GLED}$. This has a remarkable consequence: For weak gravitational fields, where the approxmiation to first order is sufficiently good, the physics of point particles adhering to GLED dynamics is indistinguishable from the Maxwellian setting with a metric perturbation $h$ by virtue of the identification
\begin{equation}
  h^{ab} = \left\lbrack 1 - \frac{1}{24} \epsilon(H) \right\rbrack \eta^{ab} + \frac{1}{2} \eta_{cd}H^{acbd} = (P^{(\leq 1)}_\text{GLED})^{ab}.
\end{equation}
This effect only holds in the limit of geometric optics---the GLED field equations do \emph{not} reduce to Maxwell equations with a metric perturbation. Consequently, even to first order in the area metric perturbation, nonmetric effects can be observed. An in-depth study of classical and quantum electrodynamics on weakly birefringent backgrounds based on exactly this realization has been conducted in Ref.~\cite{quantum_gled}.

We will now proceed to show that the possible principal polynomials arising from third-order area metric gravity Lagrangians as constructed in the previous section are only mildly more general than the effectively quadratic first-order GLED polynomial \eqref{gled_poly_first_order}. This issue is approached by first considering the corresponding Euler-Lagrange equations.

\begin{proposition}\label{prop_euler_tensor}
  Let $E\overset{\pi}{\longrightarrow}M$ be a sub-bundle of some tensor bundle over $M$. Consider a Lagrangian field theory on $J^2\pi$ that is degenerate in the sense that the Euler-Lagrange equations are of second derivative order, i.e.~are also defined on $J^2\pi$. If the Lagrangian field theory is diffeomorphism invariant with respect to the diffeomorphism action on the second jet bundle, it follows that the Euler-Lagrange equations are diffeomorphism equivariant. In particular, a local representation of the Euler-Lagrange equations
  \begin{equation}\label{euler_lagrange_local_repeat}
  E_A = L_{:A} - D_p L_{:A}^{\hphantom{:A}p} + I_I^{pq} D_p D_q L_{:A}^{\hphantom{:A}I}
  \end{equation}
  exhibits the transformation behaviour
  \begin{equation}\label{trafo_euler_lagrange}
    \delta_\xi E_A = -E_A \xi^m_{,m} - E_B \gmc{B}{A}{n}{m} \xi^m_{,n},
  \end{equation}
  where $\gmc{B}{A}{n}{m}$ are the Gotay-Marsden coefficients corresponding to the field bundle. In other words, the Euler-Lagrange equations transform as tensor density of weight 1.
\end{proposition}
\begin{proof}
  The claim follows from expanding the left hand side of Eq.~\ref{trafo_euler_lagrange} as
  \begin{equation}
    \delta_\xi E_A = E_{A:B} \delta_\xi u^B + E_{A:B}^{\hphantom{A:B}p} \delta_\xi u^B_{\hphantom Bp} + E_{A:B}^{\hphantom{A:B}I} \delta_\xi u^B_{\hphantom BI},
  \end{equation}
  then replacing $E_A$ with its definition \eqref{euler_lagrange_local_repeat} and simplifying the result using the equivariance of the Lagrangian density $L$. Rather than performing this tedious calculation, we can alternatively consider the geometric definition \ref{global_lagrange_form} of the Euler-Lagrange form and deduce that it must transform covariantly (for a contravariant tensor bundle) with density weight of one, i.e.~according to the local expression \eqref{trafo_euler_lagrange}.
\end{proof}
This transformation behaviour carries over to the principal symbol of the Euler-Lagrange equations, which is also a tensor density of weight 1.
\begin{proposition}
  Consider the same Lagrangian field theory as in Prop.~\ref{prop_euler_tensor}. The principal symbol
  \begin{equation}\label{trafo_symbol}
    T_{AB}(k) = E_{A:B}^{\hphantom{A:B}I} J_I^{pq} k_p k_q
  \end{equation}
  of the corresponding Euler-Lagrange equations $E_A$, where $k \in T^\ast M$ denotes a covector, transforms as a tensor density of weight one, i.e.~an infinitesimal diffeomorphism acts as
  \begin{equation}
    \delta_\xi T_{AB}(k) = - T_{AB}(k) \xi^m_{,m} - T_{CB}(k) \gmc{C}{A}{n}{m} \xi^m_{,n} - T_{AC}(k) \gmc{C}{B}{n}{m} \xi^m_{,n}.
  \end{equation}
\end{proposition}
\begin{proof}
  The idea of the proof is as before: We insert the just proven transformation behaviour of the Euler-Lagrange equations $E_A$ and of covectors $k$, which is
  \begin{equation}
    \delta_\xi k_a = - k_m \xi^m_{,a},
  \end{equation}
  into the transformation
  \begin{equation}
    \begin{aligned}
      \delta_\xi T_{AB}(k) &{} = (T_{AB}(k))_{:C} \delta_\xi u^C + (T_{AB}(k))_{:C}^{\hphantom{:C}p} \delta_\xi u^C_{\hphantom Cp} + (T_{AB}(k))_{:C}^{\hphantom{:C}I} \delta_\xi u^C_{\hphantom CI} \\
                           &{} \hphantom{=} + \frac{\partial T_{AB}}{\partial k_a}(k) \delta_\xi k_a.
    \end{aligned}
  \end{equation}
  This time, the calculation is rather trivial and the claim \eqref{trafo_symbol} follows almost immediately.
\end{proof}
We are now in a position to prove the first part of the central result, which is that the principal polynomial of area metric gravity is a scalar density. Note that we restrict our considerations to the case of a principal symbol that is independent from the derivatives of the derivatives of the gravitational field, as otherwise the causality could not be matched anyway (see Sect.~\ref{section_axiom2_perturb}).
\begin{theorem}\label{thm_poly_trafo}
  Let $\pi$ be the area metric bundle. Conside a degenerate Lagrangian field theory with a principal symbol that is independent from the derivatives of the area metric field. The principal polynomial $\mathcal P(k)$ corresponding to the symbol, as defined in Def.~\ref{def_principal_polynomial} is a scalar density of weight 23, i.e.~transforms locally under infinitesimal spacetime diffeomorphisms as\footnote{Here, we correct a numerical---but inconsequential---error in the calculation presented in Ref.~\cite{paper}, where the factor was calculated to be 57 instead of 23.}
  \begin{equation}
    \delta_\xi \mathcal P(k) = -23 \cdot \mathcal P(k) \xi^m_{,m}.
  \end{equation}
\end{theorem}
\begin{proof}
  From the transformation behaviour of the area metric field and covectors, it follows that an infinitesimal diffeomorphism acts on generators $\chi^A_{(i)}(k) = \gmc{A}{B}{n}{i} u^B k_n$ of gauge transforms as
  \begin{equation}
    \delta_\xi \chi^A_{(i)}(k) = \gmc{A}{B}{n}{m} \chi^B_{(i)}(k) \xi^m_{,n} - \chi^A_{(m)}(k)\xi^m_{,i}.
  \end{equation}
  Now calculating the transformation behaviour of the principal polynomial numerator $Q^{(A_1\dots A_4)(B_1\dots B_4)}$ (dropping the covector $k$ from the notation) we obtain
  \begin{equation}\label{Q_trafo}
    \begin{aligned}
      \delta_\xi Q^{(A_1\dots A_4)(B_1\dots B_4)} &{} = \delta_\xi \frac{\partial^4 \operatorname{det} T}{\partial T_{A_1B_1}\dots\partial T_{A_4B_4}} \\
                                                  &{} = \delta_\xi \left\lbrack \frac{4}{21!} \epsilon^{A_1\dots A_{21}} \epsilon^{B_1\dots B_{21}} T_{A_5B_5} \dots T_{A_{21}B_{21}}\right\rbrack \\
                                                  &{} = \frac{4\cdot 17}{21!} \epsilon^{A_1\dots A_{21}} \epsilon^{B_1\dots B_{21}} \lbrack\delta_\xi T_{A_5B_5}\rbrack T_{A_6B_6}\dots T_{A_{21}B_{21}} \\
                                                  &{} = -17\cdot \delta_\xi Q^{(A_1\dots A_4)(B_1\dots B_4)} \xi^m_{,m} \\
                                                  &{} \hphantom{=} -\frac{4\cdot 17}{21!} \epsilon^{A_1\dots A_{21}} \gmc{A}{A_5}{n}{m} \epsilon^{B_1\dots B_{21}} T_{AB_5} \dots T_{A_{21}B_{21}} \xi^m_{,n} \\
                                                  &{} \hphantom{=} -\frac{4\cdot 17}{21!} \epsilon^{A_1\dots A_{21}} \epsilon^{B_1\dots B_{21}} \gmc{B}{B_5}{n}{m} T_{A_5B} \dots T_{A_{21}B_{21}} \xi^m_{,n}.
    \end{aligned}
  \end{equation}
  This is further simplified using the identity $0 = \epsilon^{\lbrack A_1\dots A_{21}} X^{A\rbrack\dots}$, from which we derive after a few index relabellings
  \begin{equation}
    \begin{aligned}
      0 &{} = 22 \cdot \epsilon^{\lbrack A_1\dots A_{21}} \gmc{A\rbrack}{A_5}{n}{m} \epsilon^{B_1\dots B_21} T_{AB_5}T_{A_6B_6}\dots T_{A_{21}B_{21}} \xi^m_{,n} \\
        &{} = 17 \cdot \epsilon^{A_1\dots A_{21}} \gmc{A}{A_5}{n}{m} \epsilon^{B_1\dots B_{21}} T_{AB_5}T_{A_6B_6}\dots T_{A_{21}B_{21}} \xi^m_{,n} \\
        &{} \hphantom{=} - \gmc{A}{A}{n}{m} \epsilon^{A_1\dots A_{21}}\epsilon^{B_1\dots B_{21}} T_{A_5B_5}\dots T_{A_{21}B_{21}} \xi^m_{,n} \\
        &{} \hphantom{=} + \epsilon^{AA_2A_3A_4\dots A_{21}} \gmc{A_1}{A}{n}{m} \epsilon^{B_1\dots B_{21}} T_{A_5B_5}T_{A_6B_6}\dots T_{A_{21}B_{21}} \xi^m_{,n} \\
        &{} \hphantom{=} + \epsilon^{A_1AA_3A_4\dots A_{21}} \gmc{A_2}{A}{n}{m} \epsilon^{B_1\dots B_{21}} T_{A_5B_5}T_{A_6B_6}\dots T_{A_{21}B_{21}} \xi^m_{,n} \\
        &{} \hphantom{=} + \epsilon^{A_1A_2AA_4\dots A_{21}} \gmc{A_3}{A}{n}{m} \epsilon^{B_1\dots B_{21}} T_{A_5B_5}T_{A_6B_6}\dots T_{A_{21}B_{21}} \xi^m_{,n} \\
        &{} \hphantom{=} + \epsilon^{A_1A_2A_3A\dots A_{21}} \gmc{A_4}{A}{n}{m} \epsilon^{B_1\dots B_{21}} T_{A_5B_5}T_{A_6B_6}\dots T_{A_{21}B_{21}} \xi^m_{,n}. \\
    \end{aligned}
  \end{equation}
  Applying the same technique to the index set $[B_1\dots B_{21} B]$ and carrying out the contraction $\gmc{A}{A}{n}{m} = 4\delta^{n}_{m}$, the identity can be applied to the second and third terms in Eq.~\ref{Q_trafo}, such that we finally obtain
  \begin{equation}
    \begin{aligned}
      \delta_\xi Q^{(A_1\dots A_4)(B_1\dots B_4)} &{} = -25 \cdot Q^{(A_1\dots A_4)(B_1\dots B_4)} \xi^m_{,m} \\
                                                  &{} \hphantom{=} + \gmc{A_1}{A}{n}{m} Q^{(AA_2A_3A_4)(B_1\dots B_4)} \xi^m_{,n} + \gmc{A_2}{A}{n}{m} Q^{(A_1AA_3A_4)(B_1\dots B_4)} \xi^m_{,n} \\
                                                  &{} \hphantom{=} + \gmc{A_3}{A}{n}{m} Q^{(A_1A_2AA_4)(B_1\dots B_4)} \xi^m_{,n} + \gmc{A_4}{A}{n}{m} Q^{(A_1A_2A_3A)(B_1\dots B_4)} \xi^m_{,n} \\
                                                  &{} \hphantom{=} + \gmc{B_1}{B}{n}{m} Q^{(A_1\dots A_4)(BB_2B_3B_4)} \xi^m_{,n} + \gmc{B_2}{B}{n}{m} Q^{(A_1\dots A_4)(B_1BB_3B_4)} \xi^m_{,n} \\
                                                  &{} \hphantom{=} + \gmc{B_3}{B}{n}{m} Q^{(A_1\dots A_4)(B_1B_2BB_4)} \xi^m_{,n} + \gmc{B_4}{B}{n}{m} Q^{(A_1\dots A_4)(B_1B_2B_3B)} \xi^m_{,n}.
    \end{aligned}
  \end{equation}
  A similar calculation, this time using the identity $0 = \epsilon^{\lbrack a_1a_2a_3a_4} X^{a\rbrack\dots}$, yields the transformation of the denominator $f^{(A_1\dots A_4)(B_1\dots B_4)}$,
  \begin{equation}
    \begin{aligned}
      \delta_\xi f^{(A_1\dots A_4)(B_1\dots B_4)} &{} = \delta_\xi\left\lbrack \epsilon^{a_1\dots a_4} \epsilon^{b_1\dots b_4} \prod_{i=1}^4 \chi^{A_i}_{(a_i)} \chi^{B_i}_{(b_i)} \right\rbrack \\
                                                  &{} = -2 \cdot f^{(A_1\dots A_4)(B_1\dots B_4)} \xi^m_{,m} \\
                                                  &{} \hphantom{=} + \gmc{A_1}{A}{n}{m} f^{(AA_2A_3A_4)(B_1\dots B_4)} \xi^m_{,n} + \gmc{A_2}{A}{n}{m} f^{(A_1AA_3A_4)(B_1\dots B_4)} \xi^m_{,n} \\
                                                  &{} \hphantom{=} + \gmc{A_3}{A}{n}{m} f^{(A_1A_2AA_4)(B_1\dots B_4)} \xi^m_{,n} + \gmc{A_4}{A}{n}{m} f^{(A_1A_2A_3A)(B_1\dots B_4)} \xi^m_{,n} \\
                                                  &{} \hphantom{=} + \gmc{B_1}{B}{n}{m} f^{(A_1\dots A_4)(BB_2B_3B_4)} \xi^m_{,n} + \gmc{B_2}{B}{n}{m} f^{(A_1\dots A_4)(B_1BB_3B_4)} \xi^m_{,n} \\
                                                  &{} \hphantom{=} + \gmc{B_3}{B}{n}{m} f^{(A_1\dots A_4)(B_1B_2BB_4)} \xi^m_{,n} + \gmc{B_4}{B}{n}{m} f^{(A_1\dots A_4)(B_1B_2B_3B)} \xi^m_{,n}.
    \end{aligned}
  \end{equation}
  Putting both numerator and denominator together proves the claim
  \begin{equation}
    \delta_\xi \mathcal P(k) = -23 \cdot \mathcal P(k) \xi^m_{,m}.
  \end{equation}
\end{proof}
An equivalent formulation of the fact that $\mathcal P(k)$ is a density of weight 23 is that the symmetric coefficients\footnote{Recall that the principal polynomial for area metric gravity is homogeneous and of degree 26.} $P^{a_1\dots a_{26}}$ constitute a tensor density of the same weight, i.e.~live of the bundle of symmetric tensor densities of contravariant rank 26 with weight 23. For this geometry, the equivariance equations on the ``zeroth jet bundle'' (since the polynomial must not depend on derivatives of the geometry) are
\begin{equation}\label{polynomial_equivariance_eqns}
  \begin{aligned}
    P^{a_1\dots a_{26}}_{\hphantom{a_1\dots a_{26}},m} &{} = 0, \\
    P^{a_1\dots a_{26}}_{\hphantom{a_1\dots a_{26}}:A} \gmc{A}{B}{n}{m} u^B &{} = - 23\cdot P^{a_1\dots a_{26}} \delta^n_m + 26\cdot P^{n(a_1\dots a_{25}} \delta^{a_{26})}_m.
  \end{aligned}
\end{equation}

The second part of the central result follows from these equations. All we have to do is construct the perturbative solution to first order and see that it is impossible \emph{not} to have the causality match GLED causality to the same order.

\begin{theorem}
  Let $\mathcal P_\text{area}$ be the principal polynomial of area metric gravity as considered in Thm.~\ref{thm_poly_trafo}. To first order in the expansion $G = N + H$ of the area metric field, where $N$ is the Lorentz invariant expansion point \eqref{area_expansion_point}, $\mathcal P_\text{area}$ is equivalent to the GLED principal polynomial $\mathcal P_\text{GLED}$ in the sense that
  \begin{equation}\label{eq_thm_poly}
    \mathcal P_\text{area} = \lbrack \omega P_\text{GLED}^{(\leq 1)}\rbrack^{13} + \mathcal O(H^2),
  \end{equation}
  where $\omega$ denotes a density of weight $\frac{23}{13}$ on the area metric bundle and $P_\text{GLED}^{(\leq 1)}$ is the expansion of the GLED polynomial to first order. In particular, to first order in the perturbation, both principal polynomials describe the same null surfaces and hyperbolicity cones.
\end{theorem}
\begin{proof}
  Knowing that the principal polynomial of area metric gravity transforms as a density of weight 23, we can construct possible candidates by solving the equivariance equations \eqref{polynomial_equivariance_eqns}. To this end, we make the ansatz
  \begin{equation}\label{poly_ansatz}
    \begin{aligned}
      \mathcal P_\text{area}(k) &{} = \eta(k,k)^{13} \\
                                &{} \hphantom{=} + A\cdot \epsilon(H) \eta(k,k)^{13} + B\cdot \eta(H) \eta(k,k)^{13} + C\cdot H(k,k)\eta(k,k)^{12} \\
                                &{} \hphantom{=} + \mathcal O(H^2).
    \end{aligned}
  \end{equation}
An overall factor would be irrelevant, so it has already been dropped when setting the coefficient of the constant term to 1. The generality of the ansatz can, as always, be verified by calculating the corank of the ansatz equations, which will yield 4---the number of ansatz tensors in Eq.~\eqref{poly_ansatz}. Evaluating the equivariance equation at the ansatz and contracting the 26 symmetric indices with covector components, for the sake of a cleaner presentation, yields an equation where we can cancel a common factor of $\eta(k,k)^{(12)}$. The remaining equation has a covariant and a contravariant spacetime index, such that a decomposition into the trace
\begin{equation}
  0 = \lbrack 24 A + 12 B + 3 C + 23 - \frac{13}{2}\rbrack \delta^n_m
\end{equation}
and the tracefree part
\begin{equation}
  0 = \lbrack 4 C - 26\rbrack [\eta^{na}\delta^b_m k_a k_b - \frac{1}{4} \delta^n_m\eta(k,k)]
\end{equation}
lends itself for a first attempt in order to retrieve scalar equations for the system. As it turns out, these two equations are already maximal. Parameterizing the solution with $B$ yields
\begin{equation}\label{general_area_metric_gravity_poly}
  \begin{aligned}
    \mathcal P_\text{area}(k) &{} = \eta(k,k)^{13} \\
                              &{} \hphantom{=} - \frac{3}{2} \epsilon(H)\eta(k,k)^{13} + B(\eta(H)-\frac{1}{2}\epsilon(H))\eta(k)^{13} + \frac{13}{2} H(k,k)\eta(k,k)^{12} \\ 
                              &{} \hphantom{=} + \mathcal O(H^2) \\
                              &{} = \left\{ \left\lbrack 1 - \frac{3}{2\cdot 13} \epsilon(H) + \frac{B}{13}\left(\eta(H) - \frac{1}{2}\epsilon(H) \right)\right\rbrack \eta(k,k) + \frac{1}{2} H(k,k) \right\}^{13} \\
                              &{} \hphantom{=} + \mathcal O(H^2),
  \end{aligned}
\end{equation}
where for the last equality we completed the thirteenth power as
\begin{equation}
  1 + \epsilon = \left( 1 + \frac{1}{13}\epsilon\right)^{13} + \mathcal O(\epsilon^2).
\end{equation}

In order to relate the quadratic polynomial that determines the first order of $\mathcal P_\text{area}(k)$ to $\mathcal P_\text{GLED}^{(\leq 1)}$ via a scalar density, as claimed in Eq.~\eqref{eq_thm_poly}, we consider the equivariance equations
\begin{equation}
  \begin{aligned}
    \omega_{,m} &{} = 0, \\
    \omega_{:A} \gmc{A}{B}{n}{m} u^B &{} = -\frac{23}{13}\,\omega \delta^n_m
  \end{aligned}
\end{equation}
for such a density $\omega$ of weight $\frac{23}{13}$. This time, the Lorentz invariant ansatz is just
\begin{equation}
  \omega = 1 + A\cdot \epsilon(H) + B\cdot \eta(H) + \mathcal O(H^2)
\end{equation}
and reduces the equivariance equations to the single condition
\begin{equation}
  24A + 12B = -\frac{23}{13},
\end{equation}
such that the most general scalar density of weight $\frac{23}{13}$ is to first order given by
\begin{equation}
  \omega = 1 - \frac{23}{13\cdot 24} \epsilon(H) + B \lbrack \eta(H) - \frac{1}{2} \epsilon(H)\rbrack + \mathcal O(H^2).
\end{equation}
The result now follows from multiplication of $\mathcal P_\text{area}^{(\leq 1)}$ with $\omega$, which yields exactly the area metric gravity polynomial \eqref{general_area_metric_gravity_poly}. To first order, the principal polynomial of area metric gravity is determined by a quadratic polynomial which reduces to the quadratic first-order GLED polynomial \emph{up to a factor}. Because such an overall factor is irrelevant for vanishing sets and hyperbolicity cones, the polynomials must be considered identical for the purpose of comparing their causal structure.
\end{proof}

Having fixed the causality of third-order perturbative area metric gravity---by proof, rather than by explicit calculation---the construction procedure up to this order is completed. Third-order area metric gravity\footnote{With second-order field equations and, therefore, a principal polynomial of first order.} is determined by the ansatz \eqref{ansatz_area_gravity} which is constructed from the Lorentz-invariant basis tensors \eqref{???}. From the 237 gravitational constants---the coefficients in the basis expansion---50 constants turn out to be independent, 11 of which govern the linearized field equations. The relations between gravitational constants are collected in Appendix \ref{}. In the following, we will examine the linear theory, which forms the basis for predicting first-order and, later on, second-order effects of area metric gravity.

\section{3+1 decomposition}
As remarked in Sect.~\ref{section_gled}, the expansion point should be an area metric of a certain subclass in order to guarantee hyperbolicity of the GLED principal polynomial---which encompasses, by the previously proven result, hyperbolicity of third-order area metric gravity. Indeed, $N$ is of subclass I according to the classification in Ref.~\ref{sergio}. Thus, we can turn to a $3+1$ formulation, starting with the definition of a slicing.
\begin{definition}[slicing]
  Consider a spacetime manifold $M$ of dimension four. Any diffeomorphism
  \begin{equation}
    \phi\colon \Sigma\times\mathbb R \rightarrow M
  \end{equation}
  from a three-dimensional spatial manifold $\Sigma$ and the reals to $M$ is called a slicing of $M$.
\end{definition}
Such a slicing always exists, as we only consider matter theories that have a well-defined initial value problem. It is, however, not unique: Any diffeomorphism $\psi\colon M\rightarrow M$ yields a new slicing $\tilde\phi = \psi\circ\phi$. Since the spatial manifold is of dimension three and not four, working with slicings comes with new indices running from one to three. These will be denoted with lowercase greek letters, while lowercase latin letters represent spacetime indices running from zero to three.

Every tangent space $T_{\phi(s,\lambda)}M$ has a holonomic basis
\begin{equation}
  \frac{\partial}{\partial x^a} = \left( \frac{\partial}{\partial t},\frac{\partial}{\partial x^\alpha}\right),
\end{equation}
where the vectors on the right are understood as pushforwards of holonomic basis vectors on $T_s\Sigma$ and $T_\lambda\mathbb R$. The same construction yields a holonomic basis
\begin{equation}
  \mathrm dx^a = (\mathrm dt, \mathrm dx^\alpha)
\end{equation}
for the cotangent spaces $T_{\phi(s,\lambda)}^\ast M$. The bundle $\pi_\text{area}$, constructed as subbundle of $T^4_0M$, inherits a $3+1$ split from the decomposition of tangent and cotangent spaces, and so does the second jet bundle of $\pi_\text{area}$.

Based on a slicing, we now introduce an observer definition\footnote{see also Ref.~\ref{observer (giesel_schuller_witte_wohlfarth)}} for arbitrary tensor theories. Only the principal polynomial is needed for this notion.
\begin{definition}[observer frame, lapse and shift]
  Let $P$ be the principal polynomial of a field theory on a tensor bundle. An observer frame consists of a nonholonomic frame
  \begin{equation}
    (T, e_\alpha = \frac{\partial}{\partial x^\alpha})
  \end{equation}
  and a dual coframe
  \begin{equation}
    (n = \lambda\cdot \mathrm dt, \epsilon^{\alpha}),
  \end{equation}
  where the temporal direction and codirection must satisfy\footnote{$DP$ denotes the formal derivative of $P$ as a polynomial.}
  \begin{equation}\label{frame_conditions}
    P(n) = 1\quad\text{and}\quad T=\frac{1}{\operatorname{deg}P}\frac{DP(n)}{P(n)}.
  \end{equation}
  In the following, we assume $P(n) = 1$ to be solved by choosing an appropriate basis on $T\mathbb R$ and setting $\lambda = 1$.

  The holonomic time direction $\frac{\partial}{\partial t}$ decomposes in the observer frame as
  \begin{equation}
    \frac{\partial}{\partial t} = NT + N^\alpha \frac{\partial}{\partial x^\alpha}
  \end{equation}
  with the lapse $N$ and shift $N^\alpha$.
\end{definition}

Essential for the $3+1$ split is the parameterization of the geometry with quantities an observer can measure in her frame, as well as lapse and shift. For example, using the completeness relation
\begin{equation}
  \mathrm{id} = T\otimes n + e_\alpha \otimes \epsilon^\alpha = \frac{1}{N} \frac{\partial}{\partial t} \otimes n - \frac{1}{N} N^\alpha e_\alpha \otimes n + e_\alpha \otimes \epsilon^\alpha,
\end{equation}
a vector field $v$ decomposes as
\begin{equation}
  v = v\circ \mathrm{id} = v(n)\, T + v(\epsilon^\alpha)\, e_\alpha.
\end{equation}
The holonomic components are thus determined by lapse $N$, shift $N^\alpha$, and the observer quantities $v(n)$ and $v(\epsilon^\alpha)$ as
\begin{equation}
  v(\mathrm dt) = v(n)\quad\text{and}\quad v(\mathrm dx^\alpha) = -\frac{1}{N}N^\alpha v(n) + v(\epsilon^\alpha).
\end{equation}
Obviously, the information contained in $N$, $N^\alpha$, $v(n)$, and $v(\epsilon^\alpha)$ is redundant---four holonomic components are represented using 8 observer quantities. This is where the frame conditions \eqref{frame_conditions} come into play: Consider the decomposition of the area metric field into \cite{giesel_schuller_witte_wohlfarth}
\begin{gather}
  G(\mathrm dt,\mathrm dx^\alpha,\mathrm dt,\mathrm dx^\beta) = \frac{1}{N^2} G(n,\epsilon^\alpha,n,\epsilon^\beta), \\
  G(\mathrm dt,\mathrm dx^\alpha, \mathrm dx^\beta, \mathrm dx^\gamma) = -\frac{2}{N^2} G(n,\epsilon^\alpha,n,\epsilon^{\lbrack\gamma})N^{\beta\rbrack} + \frac{1}{N} G(n,\epsilon^\alpha,\epsilon^\beta,\epsilon^\gamma), \\
  \begin{aligned}
    G(\mathrm dx^\alpha, \mathrm dx^\beta, \mathrm dx^\gamma, \mathrm dx^\delta) &{} = \frac{4}{N^2} N^{\lbrack\alpha}G(n,\epsilon^{\beta\rbrack},n,\epsilon^{\lbrack\delta})N^{\gamma\rbrack} + \frac{2}{N} N^{\lbrack\alpha}G(n,\epsilon^{\beta\rbrack},\epsilon^\gamma,\epsilon^\delta) \\
                                                                                 &{} \hphantom{=} + \frac{2}{N} N^{\lbrack\gamma}G(n,\epsilon^{\delta\rbrack},\epsilon^\alpha,\epsilon^\beta) + G(\epsilon^\alpha,\epsilon^\beta,\epsilon^\gamma,\epsilon^\delta).
  \end{aligned}
\end{gather}
So far, the situation seems to be similar---21 area metric components are determined by 21 observer quantities plus lapse and shift. The difference to the decomposition of a vector is that the frame conditions \eqref{frame_conditions} depend---via the principal polynomial---on the area metric, which introduces dependences among area metric, lapse, and shift. To formulate these conditions, it is more convenient to redefine the observer quantities as \cite{giesel_schuller_witte_wohlfarth}
\begin{equation}
  \begin{aligned}
    \hat{G}^{\alpha\beta}                  &{} = -G(n,\epsilon^\alpha,n,\epsilon^\beta), \\
    \hat{G}^\alpha_{\hphantom\alpha\beta}  &{} = \frac{1}{2} (\omega_{\hat G})^{-1} \epsilon_{\beta\mu\nu} G(n, \epsilon^\alpha, \epsilon^\mu, \epsilon^\nu) - \delta^\alpha_{\hphantom\alpha\beta}, \\
    \hat{G}_{\alpha\beta}                  &{} = \frac{1}{4} (\omega_{\hat G})^{-2} \epsilon_{\alpha\mu\nu} \epsilon_{\beta\rho\sigma} G(\epsilon^\mu, \epsilon^\nu, \epsilon^\rho, \epsilon^\sigma),
  \end{aligned}
\end{equation}
with the spatial density
\begin{equation}
  \omega_{\hat G} = \sqrt{\operatorname{det}\hat G^{\cdot\cdot}}.
\end{equation}
By definition, $\hat G^{\alpha\beta}$ and $\hat G_{\alpha\beta}$ are symmetric. The frame conditions \eqref{frame_conditions} translate into the two additional properties\cite{giesel_schuller_witte_wohlfarth}
\begin{equation}
  0 = \hat G^\alpha_{\hphantom\alpha\alpha}\quad\text{and}\quad0 = \hat G^{\mu\lbrack\alpha} \hat G^{\beta\rbrack}_{\hphantom\beta\mu},
\end{equation}
i.e.~$\hat G^\alpha_{\hphantom\alpha\beta}$ is trace-free and symmetric with respect to $\hat G^{\alpha\beta}$. In total, lapse and shift and the observer quantities $\hat G^{\alpha\beta}$, $\hat G^{\alpha}_{\hphantom\alpha\beta}$, $\hat G_{\alpha\beta}$ have $1+3+6+5+6 = 21$ degrees of freedom, such that they are in one-to-one correspondence with the area metric field $G^{abcd}$. Note the similarity to the $3+1$ decomposition of the metric tensor $g^{ab}$ into shift $N^\alpha$, lapse $N$, and spatial metric $\hat g^{\alpha\beta}$---the purely temporal and spatio-temporal components of the metric are parameterized only by shift and lapse, due to the frame conditions \eqref{frame_conditions}.

Around the perturbation point $N$, the area metric observer quantities expand as
\begin{equation}
  \begin{aligned}
    N &{} = 1 + A, \\
    N^\alpha &{} = b^\alpha, \\
    \hat G^{\alpha\beta} &{} = \gamma^{\alpha\beta} + h^{\alpha\beta}, \\
    \hat G^\alpha_{\hphantom\alpha\beta} &{} = k^\alpha_{\hphantom\alpha\beta}, \\
    \hat G_{\alpha\beta} &{} = \gamma_{\alpha\beta} + l_{\alpha\beta}.
  \end{aligned}
\end{equation}
With $\gamma$ we denote the positive-definite spatial part of the Minkowski metric, i.e.~$\eta^{\alpha\beta} = - \gamma^{\alpha\beta}$. From now on, spatial indices are raised and lowered at will using $\gamma$ and its inverse. The perturbations $A$, $b$, $h$, $k$, and $l$ are again in one-to-one correspondence with the 21 perturbations $H$, by virtue of
\begin{equation}\label{area_metric_perturbation}
  \begin{aligned}
    H^{0\alpha0\beta} &{} = 2 A \gamma^{\alpha\beta} - h^{\alpha\beta}, \\
    H^{0\alpha\beta\gamma} &{} = -A \epsilon^{\alpha\beta\gamma} + 2 b^{\lbrack\beta} \gamma^{\gamma\rbrack\alpha} + \frac{1}{2} \epsilon^{\alpha\beta\gamma} \gamma_{\mu\nu} h^{\mu\nu} + \epsilon^{\mu\beta\gamma} k^\alpha_{\hphantom\alpha\mu}, \\
    H^{\alpha\beta\gamma\delta} &{} = 2\gamma^{\alpha\lbrack\gamma}\gamma^{\delta\rbrack\beta} \gamma_{\mu\nu} h^{\mu\nu} + \epsilon^{\mu\alpha\beta} \epsilon^{\nu\gamma\delta} l_{\mu\nu}.
  \end{aligned}
\end{equation}
A set of perturbations that is more convenient to work with is given by the linear combinations
\begin{equation}
  u^{\alpha\beta} = h^{\alpha\beta} - l^{\alpha\beta},\quad v^{\alpha\beta} = h^{\alpha\beta} + l^{\alpha\beta},\quad w^{\alpha\beta} = 2 k^{\alpha\beta}.
\end{equation}
Using these fields rather then the original ones, the field equations assume a particularly simple form. In fact, we find in Sect.~\ref{sect_area_linear_eom} that this choice yields \emph{decoupled} equations for the individual fields.

Area metric gravity as constructed in the framework of covariant constructive gravity is---by the first axiom---diffeomorphism invariant. For the linear theory, this invariance manifests itself in the presence of a gauge symmetry
\begin{equation}\label{area_gauge_transform}
  H^{\prime A} = H^A + \gmc{A}{B}{n}{m} N^B \xi^m_{,n}
\end{equation}
generated by vector fields $\xi \in \Gamma(TM)$. As a result, the Euler-Lagrange equations are underdetermined, as solutions can only be obtained up to a gauge transform.

In order to have a determined system for our following analysis, we fix the gauge by reducing the number of perturbation fields in a way that can always be reproduced using appropriate gauge transforms. The tool that makes the gauge fixing quite straightforward is Helmholtz' theorem\footnote{The Helmholtz theorem is only valid for certain classes of functions. Applicability to linearized area metric gravitay, i.e.~sufficiently well-behaved perturbations, is assumed.}, which allows us to decompose the spatial vector field $b$ into a so-called longitudinal scalar $B$ and a divergence-free transverse vector $B^\alpha$ satisfying $\partial_\alpha B^\alpha = 0$ as
\begin{equation}
  b^\alpha = \partial^\alpha B + B^\alpha.
\end{equation}
Applied to a tensor of rank 2, the Helmholtz theorem yields a decomposition
\begin{equation}
  u^{\alpha\beta} = U^{\alpha\beta} + 2\partial^{(\alpha}U^{\beta)} + \gamma^{\alpha\beta} \tilde U + \Delta^{\alpha\beta} U.
\end{equation}
In this decomposition, $U^{\alpha\beta}$ is the transverse traceless (TT) tensor satisfying $\partial_\alpha U^{\alpha\beta} = 0$ and $\gamma_{\alpha\beta}U^{\alpha\beta}=0$. The vector $U^\alpha$ is again a transverse vector, $U$ and $\tilde U$ are scalars, and $\Delta_{\alpha\beta} = \partial_\alpha\partial_\beta - \frac{1}{3}\gamma_{\alpha\beta}\Delta$, with the Laplacian $\Delta$, denotes the traceless Hessian. The same decomposition
\begin{equation}
  v^{\alpha\beta} = V^{\alpha\beta} + 2\partial^{(\alpha}V^{\beta)} + \gamma^{\alpha\beta} \tilde V + \Delta^{\alpha\beta} V.
\end{equation}
applies to $v^{\alpha\beta}$. Being traceless, the field $w^{\alpha\beta}$ is missing the trace scalar $\tilde W$, but otherwise admits a similar deconstruction into transverse traceless tensor $W^{\alpha\beta}$, transverse vector $W^\alpha$, and longitudinal scalar $W$. At last, we have the lapse perturbation $A$, which is already a scalar.

Explicitly carrying out the gauge transform \eqref{area_gauge_transform} and carefully inspecting the components of $H^{\prime A}$, we find that the vector field $\xi$ can always be chosen such that the four gauge conditions
\begin{equation}
  0 = B,\quad 0 = U^\alpha - V^\alpha,\quad 0 = U + V
\end{equation}
are satisfied (see \cite{perturbation_paper}). This choice reduces the degrees of freedom to 17, wich are summarized in Table \ref{table_area_dof}.
\begin{table}
  \centering
  \begin{tabular}{l c c c}
    \toprule
    perturbation kind & dof per field & fields & total dof \\
    \midrule
    scalar & 1 & $A,\tilde U,\tilde V,V,W$ & 5 \\
    transverse vector & 2 & $B^\alpha,U^\alpha,W^\alpha$ & 6 \\
    transverse traceless tensor & 2 & $U^{\alpha\beta},V^{\alpha\beta},W^{\alpha\beta}$ & 6 \\
    \bottomrule 
  \end{tabular}
  \caption{The 17 gauge-fixed degrees of freedom (dof) in linearized area metric gravity. Transverse vectors are divergence free, i.e.~satisfy $0 = \partial_\alpha U^\alpha$. Transverse traceless vectors are symmetric, tracefree, and divergence free, i.e.~$0 = U^{\lbrack\alpha\beta\rbrack}$, $0 = \gamma_{\alpha\beta} U^{\alpha\beta}$, and $0 = \partial_\alpha U^{\alpha\beta}$. Together with the four gauge-fixed fields $B=0$, $V^\alpha = U^\alpha$, and $U=-V$, the area metric perturbation in this particular gauge is reproduced using Eq.~\eqref{area_metric_perturbation}.}
\end{table}

Let us briefly collect the results of a similar decomposition and gauge fixing for metric gravity perturbed around the Minkowski metric. This will be of use later when we compare area metric gravity with metric gravity and highlight the differences. The metric tensor has 10 degrees of freedom and, as already remarked, decomposes into shift $N^\alpha$, lapse $N$, and spatial metric $\hat g^{\alpha\beta}$ by virtue of the relations
\begin{equation}\label{metric_three_plus_one}
  \begin{aligned}
    g(\mathrm dt,\mathrm dt) &{} = \frac{1}{N^2}, \\
    g(\mathrm dt,\mathrm dx^\alpha) &{} = -\frac{N^\alpha}{N^2}, \\
    g(\mathrm dx^\alpha,\mathrm dx^\beta) &{} = \frac{N^\alpha N^\beta}{N^2} - \hat g^{\alpha\beta}.
  \end{aligned}
\end{equation}
Around $\eta$, the observer quantities expand as
\begin{equation}\label{metric_expansion}
  \begin{aligned}
    N &{} = 1 + A, \\
    N^\alpha &{} = b^\alpha, \\
    \hat g^{\alpha\beta} &{} = \gamma^{\alpha\beta} + \varphi^{\alpha\beta}.
  \end{aligned}
\end{equation}
Like before, we use the Helmholtz theorem to write
\begin{equation}
  b^\alpha = \partial^\alpha B + B^\alpha
\end{equation}
and
\begin{equation}
  \varphi^{\alpha\beta} = E^{\alpha\beta} + 2\partial^{(\alpha}V^{\beta)} + C \gamma^{\alpha\beta} + \Delta^{\alpha\beta} D.
\end{equation}
A possible choice of gauge conditions is to set $B$, $D$, and $V^\alpha$ to zero, leaving us with 6 degrees of freedom in the fields $A$, $B^\alpha$, $C$, and $E^{\alpha\beta}$.

\section{Linearized field equations}\label{sect_area_linear_eom}
Applying the $3+1$ decomposition of the area metric field to the Lagrangian density constructed in Sect.~\ref{sect_area_lagrangian} yields an expression that is determined only by lapse, shift, and observer quantities. The corresponding field equations are obtained by the variations
\begin{equation}
  \frac{\delta L}{\delta N},\ \frac{\delta L}{\delta N^\alpha},\ \frac{\delta L}{\delta G^{\alpha\beta}},\ \frac{\delta L}{\delta G^\alpha_{\hphantom\alpha\beta}},\ \frac{\delta L}{\delta G_{\alpha\beta}}
\end{equation}
with respect to all of these fields---as opposed to the ``single'' variation
\begin{equation}
  \frac{\delta L}{\delta G^{abcd}}
\end{equation}
with respect to the area metric in the spacetime picture.

For the linearized field equations, we automatically obtain a Helmholtz decomposition of the Euler-Lagrange equations: The variation with respect to the lapse $N$ is a scalar and contains only contributions from scalars, the variation with respect to the shift $N^\alpha$ is a vector and contains only contributions from vectors. The same holds for the scalar and vector constituents of the observer quantities $\hat G$. Also the variations with respect to transverse traceless tensors are again tensors and only comprised of tensors. As a result, the field equations already decouple to a large extent. For this reason, the terminology of the individual scalar, vector, and tensor fields as \emph{modes} of the gravitational field is justified.

Performing the $3+1$ split is a computationally heavy task. Essentially, the perturbation \eqref{area_metric_perturbation} has to be inserted into the Ansätze (see Appendix \ref{appendix}), the resulting expression must be simplified, then varied with respect to the different modes, and simplified again. In order to gain confidence in the result, speed up the computation, and---very importantly---have a calculation that can be reproduced and amended, the task has been offloaded to the computer algebra system \texttt{cadabra}\cite{cadabra}. The code is available at Ref.~\cite{area_metric_gravity_code} and, in parts, in Appendix \ref{}.

The result of this computation finally yields the field equations of perturbative area metric gravity in a gauge-fixed $3+1$ setting. Of the 16 undetermined gravitational constants $k_i$ that determine the expansion coefficients $e_i$ (see Appendix \ref{appendix_ansaetze}), ten independent linear combinations $s_i$ (listed in Appendix \ref{appendix_reduction}) make up the scalar equations
\begingroup\allowdisplaybreaks
\begin{align}\footnotesize
  \bigg\lbrack\frac{\delta L}{\delta u^{\alpha\beta}}\bigg\rbrack^\text{S-TF} = {} & \Delta_{\alpha\beta} \bigg\lbrack s_1 A -\frac{s_1}{4} \tilde U + s_3 \tilde V + s_4 \ddot V - \frac{s_4}{3} \Delta V + s_6 \ddot W - \frac{s_6}{3} \Delta W\bigg\rbrack, \nonumber \\
  \bigg\lbrack\frac{\delta L}{\delta v^{\alpha\beta}}\bigg\rbrack^\text{S-TF} = {} & \Delta_{\alpha\beta} \bigg\lbrack (s_1 + 4s_4) A + (\frac{s_1}{4} + s_4) \tilde U + (\frac{3s_1}{4} + 3 s_4) \tilde V \nonumber \\ & + s_{11} \ddot V - (\frac{s_1}{3} + \frac{4s_4}{3} + s_{11}) \Delta V + s_{13} V + s_{14} \Box W + s_{16} W \bigg\rbrack, \nonumber \\
  \bigg\lbrack\frac{\delta L}{\delta w^{\alpha\beta}}\bigg\rbrack^\text{S-TF} = {} & \Delta_{\alpha\beta} \bigg\lbrack 4s_6 A + s_6 \tilde U + 3 s_6 \tilde V \nonumber \\ & + (-s_6 + s_{14}) \ddot V - (\frac{s_6}{3} + s_{14}) \Delta V + s_{16} V - (\frac{s_1}{4} + s_4 + s_{11}) \Box W - s_{13} W \bigg\rbrack, \nonumber \\
  \bigg\lbrack\frac{\delta L}{\delta u^{\alpha\beta}}\bigg\rbrack^\text{S-TR} = {} & \gamma_{\alpha\beta} \bigg\lbrack -\frac{2s_1}{3} \Delta A -\frac{s_1}{2} \ddot{\tilde U} + \frac{s_1}{6} \Delta\tilde U + (-\frac{3s_1}{4} + s_3) \ddot{\tilde V} -\frac{2s_3}{3} \Delta\tilde V \nonumber \\ & + \frac{s_1}{3} \Delta\ddot V + \frac{2s_4}{9} \Delta\Delta V + \frac{2s_6}{9} \Delta\Delta W\bigg\rbrack, \nonumber \\
  \bigg\lbrack\frac{\delta L}{\delta v^{\alpha\beta}}\bigg\rbrack^\text{S-TR} = {} & \gamma_{\alpha\beta} \bigg\lbrack (-s_1 + \frac{4s_3}{3}) \Delta A + (-\frac{3s_1}{4} + s_3) \ddot{\tilde U} - \frac{2s_3}{3} \Delta\tilde U \nonumber \\ & + s_{37} \ddot{\tilde V} - (\frac{3s_1}{2} - 2s_3 + s_{37}) \Delta\tilde V + s_{39} \tilde V \nonumber \\ & + (\frac{s_1}{2} - \frac{2s_3}{3}) \Delta\ddot V + (\frac{s_1}{6} + \frac{2s_3}{9} + \frac{2s_4}{3}) \Delta\Delta V + \frac{2s_6}{3} \Delta\Delta W\bigg\rbrack, \nonumber \\
  \bigg\lbrack\frac{\delta L}{\delta b^{\alpha}}\bigg\rbrack^\text{S} = {} & \partial_\alpha \partial_t \bigg\lbrack -2s_1\tilde U + (-3s_1 + 4s_3) \tilde V + (\frac{4s_1}{3} + \frac{8s_4}{3}) \Delta V + \frac{8s_6}{3} \Delta W\bigg\rbrack, \nonumber \\
  \frac{\delta L}{\delta A} = {} & -2s_1 \Delta\tilde U + (-3s_1 + 4s_3) \Delta\tilde V + (\frac{4s_1}{3} + \frac{8s_4}{3}) \Delta\Delta V + \frac{8s_6}{3} \Delta\Delta W. \label{scalar_equations}
\end{align}%
\endgroup
The label $(\text{S-TF})$ denotes the projection of a tensor onto the tracefree scalar, $(\text{S-TR})$ means the projection onto the trace.

A subset of seven constants out of the ten constants $s_i$ parameterizes the vector field equations
\begingroup\allowdisplaybreaks
\begin{align}\footnotesize
 \bigg\lbrack\frac{\delta L}{\delta u^{\alpha\beta}}\bigg\rbrack^\text{V} = {} & \partial_t \partial_{(\alpha} \bigg\lbrack s_1 B_{\beta)} - 2s_4 \dot U_{\beta)} - 2s_6 \epsilon_{\beta)}^{\hphantom{\beta)}\mu\nu} U_{\mu,\nu} + 2s_6 \dot W_{\beta)} + (-\frac{s_1}{2} - 2s_4) \epsilon_{\beta)}^{\hphantom{\beta)}\mu\nu} W_{\mu,\nu} \bigg\rbrack, \nonumber \\
 \bigg\lbrack\frac{\delta L}{\delta v^{\alpha\beta}}\bigg\rbrack^\text{V} = {} & \partial_{(\alpha} \bigg\lbrack (-s_1 - 4s_4) \dot B_{\beta)} + 4s_6 \epsilon_{\beta)}^{\hphantom{\beta)}\mu\nu} B_{\mu,\nu} \nonumber \\ & + (s_1 + 4s_4 + 2s_{11}) \ddot U_{\beta)} + (-\frac{3s_1}{2} - 6s_4 - 2s_{11}) \Delta U_{\beta)} + 2s_6 \epsilon_{\beta)}^{\hphantom{\beta)}\mu\nu} \dot U_{\mu,\nu} + 2s_{13} U_{\beta)} \nonumber \\ & + 2 s_{14} \Box W_{\beta)} + 2 s_{16} W_{\beta)} \bigg\rbrack, \nonumber \\
 \bigg\lbrack\frac{\delta L}{\delta w^{\alpha\beta}}\bigg\rbrack^\text{V} = {} & \partial_{(\alpha} \bigg\lbrack 4s_6 \dot B_{\beta)} + (s_1 + 4s_4) \epsilon_{\beta)}^{\hphantom{\beta)}\mu\nu} B_{\mu,\nu} \nonumber \\ & + (2s_6 + 2 s_{14}) \ddot U_{\beta)} - 2 s_{14} \Delta U_{\beta)} + (\frac{s_1}{2} + 2 s_4) \epsilon_{\beta)}^{\hphantom{\beta)}\mu\nu} \dot U_{\mu,\nu} + 2 s_{16} U_{\beta)} \nonumber \\ & + (-\frac{3s_1}{2} - 6s_4 - 2s_{11}) \Box W_{\beta)} - 2s_{13} W_{\beta)} \bigg\rbrack, \nonumber \\
 \bigg\lbrack\frac{\delta L}{\delta b^\alpha}\bigg\rbrack^\text{V} = {} & \Delta\bigg\lbrack 2s_1 B_{\alpha} - 4s_4 \dot U_{\alpha} - 4s_6 \epsilon_{\alpha}^{\hphantom{\alpha}\mu\nu} U_{\mu,\nu} + 4s_6 \dot W_{\alpha} + (-s_1 - 4s_4) \epsilon_{\alpha}^{\hphantom{\alpha}\mu\nu} W_{\mu,\nu} \bigg\rbrack, \label{vector_equations}
\end{align}%
\endgroup
as well as the transverse traceless tensor field equations
\begingroup\allowdisplaybreaks
\begin{align}\footnotesize
 \bigg\lbrack\frac{\delta L}{\delta u^{\alpha\beta}}\bigg\rbrack^\text{TT} = {} & \frac{s_1}{4} \Box U_{\alpha\beta} \nonumber \\ &  + (\frac{s_1}{4} + s_4) \ddot V_{\alpha\beta} + (\frac{s_1}{4} + s_4) \Delta V_{\alpha\beta} - 2s_6 \epsilon_{(\alpha}^{\hphantom{(\alpha}\mu\nu} \dot V_{\beta)\mu,\nu} \nonumber \\ & + s_6 \ddot W_{\alpha\beta} + s_6 \Delta W_{\alpha\beta} + (\frac{s_1}{2} + 2 s_4) \epsilon_{(\alpha}^{\hphantom{(\alpha}\mu\nu} \dot W_{\beta)\mu,\nu}, \nonumber \\
 \bigg\lbrack\frac{\delta L}{\delta v^{\alpha\beta}}\bigg\rbrack^\text{TT} = {} & (\frac{s_1}{4} + s_4) \ddot U_{\alpha\beta} + (\frac{s_1}{4} + s_4) \Delta U_{\alpha\beta} + 2s_6 \epsilon_{(\alpha}^{\hphantom{(\alpha}\mu\nu} \dot U_{\beta)\mu,\nu} \nonumber \\ &  + (\frac{s_1}{4} + s_4 + s_{11}) \Box V_{\alpha\beta} + s_{13} V_{\alpha\beta} + s_{14} \Box W_{\alpha\beta} + s_{16} W_{\alpha\beta}, \nonumber \\
 \bigg\lbrack\frac{\delta L}{\delta w^{\alpha\beta}}\bigg\rbrack^\text{TT} = {} & s_6 \ddot U_{\alpha\beta} + s_6 \Delta U_{\alpha\beta} - (\frac{s_1}{2} + 2 s_4) \epsilon_{(\alpha}^{\hphantom{(\alpha}\mu\nu} \dot U_{\beta)\mu,\nu}\nonumber \\ &  + s_{14} \Box V_{\alpha\beta} + s_{16} V_{\alpha\beta} - (\frac{s_1}{4} + s_4 + s_{11}) \Box W_{\alpha\beta} - s_{13} W_{\alpha\beta}. \label{tensor_equations}
\end{align}

The second Noether theorem \eqref{}
\begin{equation}
  0 = D_n \mathcal T^n_m = - D_n \big\lbrack\frac{\delta L}{\delta u^A} \gmc{A}{B}{n}{m} u^B\big\rbrack
\end{equation}
provides a sanity check for the constructed field equations by virtue of its expansion around $N$
\begin{equation}
  0 = -\big\lbrack D_n \frac{\delta L}{\delta u^A}\big\rbrack_{N+H} \gmc{A}{B}{n}{m} N^B + \mathcal O(H^2).
\end{equation}
Inverting the relation \eqref{area_metric_perturbation} between spacetime area metric and observer fields, we can make use of the chain rule in order to express the variations with respect to the area metric in terms of variations with respect to the observer quantities. This renders the perturbative expansion of the Noether theorem in the particularly simple form
\begin{equation}
  0 = \partial_t \frac{\delta L}{\delta A} - \partial_\alpha \frac{\delta L}{\delta b_\alpha}\quad\text{and}\quad 0 = \partial_t\frac{\delta L}{\delta b^\alpha} - 4\partial_\beta\frac{\delta L}{\delta u_{\alpha\beta}},
\end{equation}
which is indeed satisfied by the system \eqref{scalar_equations}--\eqref{tensor_equations}. As a consequence of the diffeomorphism invariance of the theory, the field equations have four dependences among themselves. This is, of course, expected---not only from the Noether theorem, but also from the fact that gauge-fixing the observer quantities by constraining four fields reduces the 21 unknowns by four. In order for the system of 21 field equations not to be overdetermined, it must express additional dependences. These considerations are reminiscent of the rich field of constraint analysis\footnote{See e.g.~\cite{}}, which is predominantly studied in the Hamiltonian picture and also plays a role in canonical constructive gravity. For some results in the context of covariant constructive gravity, limited to first-derivative-order theories, see Ref.~\cite{}.

While the Noether identities are expected and, in fact, indispensable, a thorough analysis of the linearized field equations reveals further properties that are impossible to reconcile with our premises. After all, the axioms of covariant constructive gravity are only \emph{necessary} conditions for a theory to be viable. Any such constructed theory needs to be further specified by finding appropriate values for the gravitational constants. This also applies to Einstein gravity---the Newtonian and cosmological constants only match observations for specific ranges, where some possibilities like a negative Newtonian constant can be dismissed outright.

The first restriction of the area metric gravity parameter range we will make is to match the weak gravitational field sourced by a point mass with a modest generalization of the Einstein equivalent. More specifically, we consider the gravitational field sourced by a point mass $M$ which is at rest at the coordinate origin and thus describes the world line
\begin{equation}\label{stationary_worldline}
  \gamma^a(\lambda) = \lambda \delta^a_0.
\end{equation}
If the point particle $M$ is an idealization of a matter field that obeys GLED dynamics, its action is given by \cite{sergio}
\begin{equation}
  S_\text{matter}\lbrack\gamma\rbrack = -M \int \mathrm d\lambda \mathcal P_\text{GLED}(\mathcal L^{-1}(\dot\gamma(\lambda)))^{-\frac{1}{4}},
\end{equation}
where $\mathcal L^{-1}$ is the inverse of the Legendre map associated with the principal polynomial. In the Einstein equivalent, this action coincides with the common notion of the length of the particle worldline as measured using the covariant metric tensor. The full expansion for arbitrary curves $\gamma$ is employed in the following section, it suffices here to consider the special case \eqref{stationary_worldline} and find the only non-vanishing contribution
\begin{equation}\label{scalar_source}
  \frac{\delta S_\text{matter}}{\delta A(x)} = -M\delta^{(3)}(x).
\end{equation}
With the matter distribution being stationary, we consider a stationary ansatz for the solution to the field equations by assuming that the time derivatives of the gravitational field vanish. Using the source \eqref{scalar_source} as the right-hand side of the linearized field equations \eqref{scalar_equations}--\eqref{tensor_equations} yields vector and tensor equations that are trivially sourced by zero and as such only admit the trivial solution $B^\alpha = U^\alpha = W^\alpha = 0$ and $U^{\alpha\beta} = V^{\alpha\beta} = W^{\alpha\beta} = 0$.\footnote{\textbf{calculate?}} The scalar equations take the form
\begin{equation}
  E_i^{(\text{scalar})} = M \delta^{(3)}(x) \delta^0_i + \sum_j\lbrack a_{ij} S_j + b_{ij} \Delta S_j + c_{ij} \Delta\Delta S_j\rbrack
\end{equation}
for constant coefficients $a_{ij}, b_{ij}, c_{ij}$ and scalar fields $S^{(i)}$.

As solution to the scalar equations we obtain\footnote{\textbf{calculate?}} certain combinations of long-ranging Coulomb solutions $\propto \frac{1}{r}$ and short-ranging Yukawa solutions $\propto \frac{1}{r} \mathrm e^{-\mu r}$. While the coefficients of these combinations depend in an intricate way on the gravitational constants and are impossible to present in general, it \emph{is} feasible to make a generic argument concerning the phenomenology of the linearized result: \emph{The solution to the scalar field equations corresponds to the linearized Schwarzschild solution of general relativity for a central mass $M$ corrected by short-ranging Yukawa potentials if and only if two linear conditions on the gravitational constants $s_i$ hold.}

This statement concerns the metric limit of area metric gravity, which is reached using the metrically induced area metric \eqref{metric_induced_area}. Inserting the metric $3+1$ decomposition \eqref{metric_three_plus_one} and its perturbative expansion \eqref{metric_expansion} in the expression for the induced area metric yields
\begin{equation}
  \begin{aligned}
    \hat G^{\alpha\beta} &{} = \hat g^{\alpha\beta} = \gamma^{\alpha\beta} + \varphi^{\alpha\beta}, \\
    \hat G^\alpha_{\hphantom\alpha\beta} &{} = 0, \\
    \hat G_{\alpha\beta} &{} = (\hat g^{-1})_{\alpha\beta} \approx \gamma_{\alpha\beta} - \varphi_{\alpha\beta}, 
  \end{aligned}
\end{equation}
from which we read off the induced perturbations
\begin{equation}\label{induced_perturbations}
  u^{\alpha\beta} = 2\varphi^{\alpha\beta},\quad v^{\alpha\beta} = 0,\quad w^{\alpha\beta} = 0.
\end{equation}
If the metric perturbation is now given by the expansion of the Schwarzschild solution\cite{schwarzschild} to first order,
\begin{equation}
  A \propto \frac{1}{r}\quad\text{and}\quad \varphi^{\alpha\beta} = 2A\gamma^{\alpha\beta},
\end{equation}
the metrically induced area metric scalar fields amount to first order to
\begin{equation}
  \begin{gathered}
    V = W = \tilde V = 0, \\
    \tilde U = 4 A, \\
    A \propto \frac{1}{r}.
  \end{gathered}
\end{equation}
The condition stated above requires that the area metric deviations from these fields amount to short-ranging Yukawa corrections, i.e.~informally
\begin{equation}
  \begin{aligned}
    4A - \tilde U &{} = (\text{Yukawa corrections}), \\
                V &{} = (\text{Yukawa corrections}), \\
                W &{} = (\text{Yukawa corrections}), \\
        \tilde{V} &{} = (\text{Yukawa corrections}). \\
  \end{aligned}
\end{equation}
These conditions are equivalent to the vanishing of the linear combinations
\begin{equation}
  s_1 + 4 s_4 = 0\quad\text{and}\quad s_6 = 0,
\end{equation}
which we from now on implement, reducing the number of first-order gravitational constants by two to eight. Thus, we have ruled out the possibility of deviating \emph{too much}\footnote{In the specific sense explained above.} from Einstein gravity already in the regime of weak birefringence and restricted perturbative area metric gravity to a phenomenologically plausible sector. In this subtheory, the scalar fields around a point mass reduce to
\begin{equation}
  \begin{aligned}
    V(x) &{} = 0, \\
    W(x) &{} = 0, \\
    \tilde U(x) &{} = \frac{M}{4\pi r} \big\lbrack \alpha - (\beta + \frac{3}{4}\gamma)\mathrm e^{-\mu r}\big\rbrack, \\
    \tilde V(x) &{} = \frac{M}{4\pi r} \big\lbrack \frac{1}{4}\gamma \mathrm e^{-\mu r}\big\rbrack, \\
    A (x) &{} = \frac{M}{4\pi r} \big\lbrack \frac{1}{4} \alpha + \frac{1}{4} \beta \mathrm e^{-\mu r}\big\rbrack,
  \end{aligned}
\end{equation}
where we redefined the relevant gravitational constants using the more convenient set
\begin{equation}
  \begin{aligned}
    \mu^2 &{} = \frac{8s_1s_{39}}{9s_1^2 - 24s_1s_3 + 8s_1s_{37} + 16s_3^2}, \\
    \alpha &{} = \frac{1}{2s_1}, \\
    \beta &{} = \frac{(3s_1 + 4s_3)^2}{6s_1(9s_1^2 - 24s_1s_3 + 8s_1s_{37} + 16s_3^2)}, \\
    \gamma &{} = \frac{-8(3s_1 + 4s_3)}{6(9s_1^2 - 24s_1s_3 + 8s_1s_{37} + 16s_3^2)}.
  \end{aligned}
\end{equation}

With the reduction from ten to eight gravitational constants, the linearized field equations assume a simpler form. There are reduced scalar field equations
\begingroup\allowdisplaybreaks
\begin{align}\footnotesize
  \bigg\lbrack\frac{\delta L}{\delta u^{\alpha\beta}}\bigg\rbrack^\text{S-TF} = {} & \Delta_{\alpha\beta} \bigg\lbrack s_1 A -\frac{s_1}{4} \tilde U + s_3 \tilde V - \frac{s_1}{4} \ddot V + \frac{s_1}{12} \Delta V \bigg\rbrack, \nonumber \\
  \bigg\lbrack\frac{\delta L}{\delta v^{\alpha\beta}}\bigg\rbrack^\text{S-TF} = {} & \Delta_{\alpha\beta} \bigg\lbrack s_{11} \Box V + s_{13} V + s_{14} \Box W + s_{16} W \bigg\rbrack, \nonumber \\
  \bigg\lbrack\frac{\delta L}{\delta w^{\alpha\beta}}\bigg\rbrack^\text{S-TF} = {} & \Delta_{\alpha\beta} \bigg\lbrack s_{14} \Box V + s_{16} V - s_{11} \Box W - s_{13} W \bigg\rbrack, \nonumber \\
  \bigg\lbrack\frac{\delta L}{\delta u^{\alpha\beta}}\bigg\rbrack^\text{S-TR} = {} & \gamma_{\alpha\beta} \bigg\lbrack -\frac{2s_1}{3} \Delta A -\frac{s_1}{2} \ddot{\tilde U} + \frac{s_1}{6} \Delta\tilde U + (-\frac{3s_1}{4} + s_3) \ddot{\tilde V} -\frac{2s_3}{3} \Delta\tilde V \nonumber \\ & + \frac{s_1}{3} \Delta\ddot V - \frac{s_1}{18} \Delta\Delta V \bigg\rbrack, \label{scalar_equations_reduced} \\
  \bigg\lbrack\frac{\delta L}{\delta v^{\alpha\beta}}\bigg\rbrack^\text{S-TR} = {} & \gamma_{\alpha\beta} \bigg\lbrack (-s_1 + \frac{4s_3}{3}) \Delta A + (-\frac{3s_1}{4} + s_3) \ddot{\tilde U} - \frac{2s_3}{3} \Delta\tilde U \nonumber \\ & + s_{37} \ddot{\tilde V} - (\frac{3s_1}{2} - 2s_3 + s_{37}) \Delta\tilde V + s_{39} \tilde V \nonumber \\ & + (\frac{s_1}{2} - \frac{2s_3}{3}) \Delta\ddot V + \frac{2s_3}{9} \Delta\Delta V\bigg\rbrack, \nonumber \\
  \bigg\lbrack\frac{\delta L}{\delta b^{\alpha}}\bigg\rbrack^\text{S} = {} & \partial_\alpha \partial_t \bigg\lbrack -2s_1\tilde U + (-3s_1 + 4s_3) \tilde V + \frac{2s_1}{3} \Delta V\bigg\rbrack, \nonumber \\
  \frac{\delta L}{\delta A} = {} & -2s_1 \Delta\tilde U + (-3s_1 + 4s_3) \Delta\tilde V + \frac{2s_1}{3} \Delta\Delta V, \nonumber
\end{align}
\endgroup
vector field equations
\begingroup\allowdisplaybreaks
\begin{align}\footnotesize
  \bigg\lbrack\frac{\delta L}{\delta u^{\alpha\beta}}\bigg\rbrack^\text{V} = {} & \frac{s_1}{2} \partial_t \partial_{(\alpha} \bigg\lbrack 2 B_{\beta)} + \dot U_{\beta)}\bigg\rbrack, \nonumber \\
  \bigg\lbrack\frac{\delta L}{\delta v^{\alpha\beta}}\bigg\rbrack^\text{V} = {} & 2 \partial_{(\alpha} \bigg\lbrack  s_{11} \Box U_{\beta)} + s_{13} U_{\beta)} +  s_{14} \Box W_{\beta)} +  s_{16} W_{\beta)} \bigg\rbrack, \nonumber \\
  \bigg\lbrack\frac{\delta L}{\delta w^{\alpha\beta}}\bigg\rbrack^\text{V} = {} & 2 \partial_{(\alpha} \bigg\lbrack s_{14} \Box U_{\beta)} + s_{16} U_{\beta)} - s_{11} \Box W_{\beta)} - s_{13} W_{\beta)} \bigg\rbrack, \label{vector_equations_reduced} \\
  \bigg\lbrack\frac{\delta L}{\delta b^\alpha}\bigg\rbrack^\text{V} = {} & s_1 \Delta\bigg\lbrack 2 B_{\alpha} + \dot U_{\alpha} \bigg\rbrack, \nonumber
\end{align}
\endgroup
and traceless tensor field equations
\begingroup\allowdisplaybreaks
\begin{align}\footnotesize
  \bigg\lbrack\frac{\delta L}{\delta u^{\alpha\beta}}\bigg\rbrack^\text{TT} = {} & \frac{s_1}{4} \Box U_{\alpha\beta}, \nonumber \\
  \bigg\lbrack\frac{\delta L}{\delta v^{\alpha\beta}}\bigg\rbrack^\text{TT} = {} & s_{11} \Box V_{\alpha\beta} + s_{13} V_{\alpha\beta} + s_{14} \Box W_{\alpha\beta} + s_{16} W_{\alpha\beta}, \label{tensor_equations_reduced} \\
  \bigg\lbrack\frac{\delta L}{\delta w^{\alpha\beta}}\bigg\rbrack^\text{TT} = {} & s_{14} \Box V_{\alpha\beta} + s_{16} V_{\alpha\beta} - s_{11} \Box W_{\alpha\beta} - s_{13} W_{\alpha\beta}. \nonumber
\end{align}
\endgroup

The second observation we want to make concerns the subset
\begin{equation}
  \begin{aligned}
  \bigg\lbrack\frac{\delta L}{\delta v^{\alpha\beta}}\bigg\rbrack^\text{S-TF} = {} & \Delta_{\alpha\beta} \bigg\lbrack s_{11} \Box V + s_{13} V + s_{14} \Box W + s_{16} W \bigg\rbrack, \\
  \bigg\lbrack\frac{\delta L}{\delta w^{\alpha\beta}}\bigg\rbrack^\text{S-TF} = {} & \Delta_{\alpha\beta} \bigg\lbrack s_{14} \Box V + s_{16} V - s_{11} \Box W - s_{13} W \bigg\rbrack
  \end{aligned}
\end{equation}
of the reduced scalar equations \eqref{scalar_equations_reduced}, whose pattern is repeated in the vector equations \eqref{vector_equations_reduced} for the modes $U^\alpha$ and $W^\alpha$ as well as in the tensor equations \eqref{tensor_equations_reduced} for the modes $V^{\alpha\beta}$ and $W^{\alpha\beta}$. Linear combinations of these equations \emph{in vacuo} yield the equivalent system
\begin{equation}\label{coupled_wave_equations}
  \begin{aligned}
    0 = {} & \Box V + \nu^2 V + \sigma W, \\
    0 = {} & \Box W + \nu^2 W - \sigma V,
  \end{aligned}
\end{equation}
with constants
\begin{equation}
  \nu^2 = \frac{s_{11}s_{13}+s_{14}s_{16}}{s_{11}^2+s_{14}^2}\quad\text{and}\quad\sigma = \frac{s_{11}s_{16} - s_{13} s_{14}}{s_{11}^2 + s_{14}^2}.
\end{equation}
Performing a spatial Fourier transform of the vacuum scalar equations \ref{coupled_wave_equations}, we can translate them into a system of linear, first-order ordinary differential equations for the modes $\tilde v(t,k)$ and $\tilde w(t,k)$
\begin{equation}
  \frac{\mathrm d}{\mathrm dt}\begin{pmatrix}\tilde v \\ \tilde w \\ \dot{\tilde v} \\ \dot{\tilde w}\end{pmatrix} = \begin{pmatrix}0 & 0 & 1 & 0 \\ 0 & 0 & 0 & 1 \\ -(k^2 + \nu^2) & -\sigma & 0 & 0 \\ \sigma & -(k^2 + \nu^2) & 0 & 0\end{pmatrix}\begin{pmatrix}\tilde v \\ \tilde w \\ \dot{\tilde v} \\ \dot{\tilde w}\end{pmatrix}.
\end{equation}
What is now interesting about this system are the eigenvalues of the time evolution, which are the four complex roots
\begin{equation}
  \lambda_k = \pm \mathrm i \sqrt{(k^2 + \nu^2) \pm \mathrm i\sigma}.
\end{equation}
Most importantly, there are always $\lambda_k$ such that $\operatorname{Re}(\lambda_k)>0$ \emph{unless} $\sigma$ vanishes. As a consequence, there will always be diverging modes under time evolution if $\sigma$ is not zero. This is not restricted to the scalar modes we analyzed, but also holds for the vector and transverse traceless tensor modes that are coupled in the same way. Such a theory would not only be physically \emph{implausible}, it would be fundamentally broken. We set $\sigma$ to zero by imposing the additional condition
\begin{equation}
  s_{11} s_{16} - s_{13} s_{14} = 0
\end{equation}
and have thus reduced linearized area metric gravity to a theory parameterized by seven remaining gravitational constants, of which there are five combinations that determine the results obtained above: The two constants $\mu$ and $\nu$ appear as \emph{masses} in wave equations and screened Poisson equations, respectively, and three constants $\alpha$, $\beta$, and $\gamma$ further parameterize the linearized Schwarzschild solution.

Note that with $\sigma=0$ the wave equations for $W$, $V$, $U^\alpha$, $V^\alpha$, $U^{\alpha\beta}$, $V^{\alpha\beta}$, and $W^{\alpha\beta}$ \emph{decouple}, e.g.~the system of transverse traceless tensor equations can be transformed by taking linear combinations into
\begingroup\allowdisplaybreaks
\begin{align}\footnotesize
  \bigg\lbrack\frac{\delta L}{\delta u^{\alpha\beta}}\bigg\rbrack^\text{TT} = {} & \frac{s_1}{4} \Box U_{\alpha\beta}, \nonumber \\
  \frac{s_{11}}{s_{11}^2 + s_{14}^2} \bigg\lbrack\frac{\delta L}{\delta v^{\alpha\beta}}\bigg\rbrack^\text{TT} + \frac{s_{14}}{s_{11}^2 + s_{14}^2} \bigg\lbrack\frac{\delta L}{\delta w^{\alpha\beta}}\bigg\rbrack^\text{TT} = {} & \Box V_{\alpha\beta} + \nu^2 V_{\alpha\beta}, \label{tensor_equations_decoupled} \\
  \frac{s_{14}}{s_{11}^2 + s_{14}^2} \bigg\lbrack\frac{\delta L}{\delta v^{\alpha\beta}}\bigg\rbrack^\text{TT} - \frac{s_{11}}{s_{11}^2 + s_{14}^2} \bigg\lbrack\frac{\delta L}{\delta w^{\alpha\beta}}\bigg\rbrack^\text{TT} = {} & \Box W_{\alpha\beta} + \nu^2 W_{\alpha\beta}. \nonumber
\end{align}
\endgroup
Similar decoupled wave equations are obtained for the mentioned vector and scalar modes. It is also possible to find a linear combination of scalar field equations \eqref{scalar_equations_reduced} such that the mode $\tilde V$ obeys a massive wave equation\footnote{Not denoting linear combinations of Lagrangian variations explicitly but just referring to them as \emph{source terms}.}
\begin{equation}
  \text{(source terms)} = \Box \tilde V + \mu^2 \tilde V.
\end{equation}
Counting the wave equations we already found, there are at least 13 propagating degrees of freedom. This is already the maximum number, because our system for 17 degrees of freedom must exhibit four constraint equations arising from the gauge symmetry\cite{}. In fact, the four remaining degrees $B^\alpha$, $\tilde U$, and $A$ are determined by field equations with less than two time derivatives, as can be read off from Eqns.~\eqref{scalar_equations_reduced}--\eqref{tensor_equations_reduced}. Such equations as part of an initial value problem are usually associated with constraints, as they are not capable to \emph{evolve} initial data, but only to \emph{constrain} it.\cite{}

Summing up, the phenomenologically relevant subsector of linearized area metric gravity admits two massless propagating degrees of freedom in the form of the tensor mode $U^{\alpha\beta}$. Furthermore, there are 11 massive propagating degrees of freedom with mass $\mu$, represented by the fields $W$, $V$, $\tilde V$, $U^\alpha$, $W^\alpha$, $V^{\alpha\beta}$, and $W^{\alpha\beta}$. The remaining four degrees of freedom $A$, $\tilde U$, and $B^\alpha$ do not propagate but follow from constraints.

This again constitutes an important sanity check: The count of propagating degrees of freedom is as expected and yields $21 - 2\times 4 = 13$, just like in general relativity where we have $10 - 2\times 4 = 2$ degrees of freedom. In the latter theory, only the transverse traceless part of the spatial metric tensor propagates and does so according to a massless wave equation. For area metric gravity, the only massless propagating modes turn out to be the transverse traceless tensor $U^{\alpha\beta}$, \emph{which is exactly the perturbation induced by the propagating metric modes} (see \eqref{induced_perturbations}).

All other modes, which are not inducible by the propagating metric modes, follow massive wave equations with mass $\mu$. In the next section, it will become clear that the generation of such modes from matter distributions is suppressed, e.g.~a binary star only radiates on non-metric tensor modes or on vector or scalar modes when its angular velocity exceeds a certain threshold. This is another realization of the correspondence principle, which demands that Einstein gravity approximate area metric gravity in certain limits.

\section{Iterative solution strategy for the field equations}
Covariant constructive gravity closes matter theories by providing previously unknown dynamics for geometry to which the matter field couples. Let $\phi$ be the matter field in question, coupling locally to a geometric field $G$. Starting from the matter action\footnote{Round parentheses indicate local dependences.} $S_\text{matter}\lbrack\phi, G)$, the closure procedure yields the joint action
\begin{equation}
  S\lbrack G,\phi\rbrack = S_\text{gravity}\lbrack G\rbrack + \kappa S_\text{matter}\lbrack\phi, G),
\end{equation}
where $S_\text{gravity}$ is the action of the constructed theory compatible with the matter theory. The constant $\kappa$ controls the scale of coupling between both fields. Abbreviated as
\begin{equation}
  e\lbrack G\rbrack = \frac{\delta S_\text{grav}}{\delta G},\quad T\lbrack\phi, G) = \frac{\delta S_\text{mat}}{\delta G},\quad f\lbrack\phi,G) = \frac{\delta S_\text{mat}}{\delta \phi},
\end{equation}
the variations with respect to the matter field and the gravitational field yield the Euler-Lagrange equations
\begin{equation}\label{coupled_euler_lagrange}
  e\lbrack G\rbrack = -\kappa T\lbrack\phi, G)\quad\text{and}\quad f\lbrack\phi,G) = 0.
\end{equation}
Such a tightly coupled system is hard to solve in general. Fortunately, it is not our objective to obtain exact solutions---we have expanded the field equations up to second order and only seek to derive effects up to this finite order. Proceeding similarly as in Ref.~\ref{poisson}, a solution is constructed iteratively by expanding the geometry formally as
\begin{equation}
  G = N + \sum_{k=1}^\infty\kappa^k H_{(k)}.
\end{equation}
Truncations of the expansion at order $k$ yield approxmiations $G_{(k)}$ of the geometry. We expand the constituents $e$ and $T$ of the Euler-Lagrange equations expand as
\begin{equation}
  \begin{aligned}
    e\lbrack N + H\rbrack = {} & e_{(0)} + e_{(1)}\lbrack H\rbrack + e_{(2)}\lbrack H\rbrack H + \mathcal O(H^3), \\
    T\lbrack\phi, N + H) = {} & T_{(0)}\lbrack\phi\rbrack + T_{(1)}\lbrack\phi,H) + \mathcal O(H^2),
  \end{aligned}
\end{equation}
where $H$ contributes linearly to the first-order terms and quadratically to the second-order terms. We now solve the equations for the gravitational field up to second order by considering the orders zero to two in $\kappa$.

For the \emph{zeroth} iteration, the Euler-Lagrange equations \eqref{coupled_euler_lagrange} are evaluated at $G_{(0)} = N$, resulting in the equation
\begin{equation}\label{zeroth_order}
  e\lbrack N\rbrack = e_{(0)} = 0.
\end{equation}
This just enforces that the expansion point $N$ must solve the gravitational field equations \emph{in vacuo}. Since we explicitly consider this condition when perturbatively constructing theories, Eq.~\eqref{zeroth_order} is solved trivially.

Proceeding with the \emph{first} iteration, we evaluate at $G_{(1)} = N + \kappa H_{(1)}$. Since $e_{(0)} = 0$ already holds from the previous iteration, the first of the two equations simplifies to
\begin{equation}
  e_{(1)}\lbrack H_{(1)}\rbrack = -T_{(0)} \lbrack\phi\rbrack.
\end{equation}
Figuratively speaking, the first correction of the gravitational field is sourced by the matter content on a flat background. Having solved this equation for $H_{(1)}$, the perturbation may be used in order to solve the second equation
\begin{equation}\label{fix_phi}
  f\lbrack\phi,G_{(1)}) = 0 + \mathcal O(\kappa^2).
\end{equation}
The interpretation is similar: A deviation from the flat gravitational field, caused by the presence of matter, makes the matter field deviate from its unperturbed configuration.

The \emph{second} iteration yields an equation for the second-order perturbation $H_{(2)}$ by inserting $G_{(2)} = N + \kappa H_{(1)} + \kappa^2 H_{(2)}$ in the first field equation and simplifying using the lower-order equations. We obtain the result
\begin{equation}
  e_{(1)}\lbrack H_{(2)}\rbrack = -\kappa^{-1}T_{(0)}\lbrack\phi\rbrack - T_{(1)}\lbrack\phi,H_{(1)}) - e_{(2)}\lbrack H_{(1)}\rbrack + \mathcal O(k),
\end{equation}
where it has to be noted that $\phi$, having been fixed in Eq.~\eqref{fix_phi}, has a dependence on $\kappa H_{(1)}$. Therefore, contributions from $T_{(0)}\lbrack\phi\rbrack$ must only be considered up to order $\kappa^1$ and contributions from $T_{(1)}\lbrack\phi,H_{(1)})$ only up to order $\kappa^0$.

The second-order perturbation $H_{(2)}$ is thus sourced by both the first-order deviations of the gravitational field and the induced motion of the matter field, as will become clear when explicitly solving the binary star in the following section. Aborting the iterative solution procedure at this point, we have found the approximation
\begin{equation}
  G_{(2)} = N + \kappa H_{(1)} + \kappa^2 H_{(2)}
\end{equation}
of the geometry G coupled to $\phi$ and, as a side effect, the trajectory of the matter field $\phi$ on the linearized background $G_{(1)}$.

\section{Gravitational radiation from a binary star}
Before proceeding to make use of the iterative solution strategy and solve the binary star in area metric gravity, let us consider the same problem in Einstein gravity. We will, of course, only reproduce well-established results, but also gain confidence in the approach and become acquainted with the calculations. It is also advantageous to have the metric theory at hand in order to distinguish the uniquely area metric features later on. State-of-the-art methods derived from Einstein gravity (see e.g.~Ref.~\ref{poisson}) extend to higher perturbation orders and much more complex matter configurations than the relatively simple case considered here, but they are not applicable to area metric gravity. Rather, we make use of our hand-crafted approach that accommodates non-metric geometries just as well.

A binary star consists of two slowly moving point masses $m_i$ describing two world lines $\gamma_i\colon\mathbb R\rightarrow M$. The metric field is a section $g$ of the metric bundle and defines the matter action $S_\text{matter}$ via the length functional\footnote{From now on, we do not use geometrized units but state every occurence of the speed of light $c$ and Newton's constant $G$ explicitly.}
\begin{equation}\label{metric_matter}
  S_\text{matter}(\gamma_{(1)},\gamma_{(2)},g) = \sum_{i=1,2} m_i c \int \mathrm d\lambda \sqrt{g^{-1}(\dot\gamma_{(i)}(\lambda),\dot\gamma_{(i)}(\lambda)}.
\end{equation}
Einstein gravity completes Eq.~\eqref{metric_matter} to a predictive model by providing dynamics for the metric $g$ in terms of the Einstein-Hilbert action
\begin{equation}
  S_\text{gravity}\lbrack g\rbrack = \frac{c^3}{16\pi G}\int\mathrm d^4x \sqrt{-\operatorname{det}g}R.
\end{equation}
We use the parameterization $\gamma_{(i)}^0(\lambda) = ct$ and obtain by variation the Euler-Lagrange equations
\begin{equation}
  \sqrt{-\operatorname{det}g}\left\lbrack R^{ab} - \frac{1}{2} g^{ab} R\right\rbrack = \frac{8\pi G}{c^3} \sum_{i=1,2} m_i \delta^{(3)}(\vec x-\vec\gamma_{(i)}(t))\frac{\dot\gamma^a_{(i)}\dot\gamma^b_{(i)}{\sqrt{g^{(-1)}(\dot\gamma_{(i)},\dot\gamma_{(i)})}}
\end{equation}

\section{Spin-Up}

\begin{itemize}
\item gravitational radiation
\item spin-up
\end{itemize}

\textbf{take-home message: application to a refined matter theory yields interesting phenomenology}
