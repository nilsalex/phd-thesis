\chapter{Application: gravitational radiation from birefringent matter dynamics}
\label{chapter_weak_area}

\textit{So far, we have developed the general framework of covariant constructive gravity and derived a perturbative equivalent. A few examples illustrated the constructions, but the presentation focused on broad applicability to various geometries, without any specific bundle or matter theory in mind. In this chapter, we shift our focus and consider in depth the application of the framework to generalized linear electrodynamics, a birefringent generalization of Maxwell electrodynamics introduced in Chapter \ref{chapter_construction_algorithm}. Applying the perturbative construction procedure to third order yields gravitational field equations to second order. We will carefully analyze a $3+1$ split for the linear part of this theory and restrict to a certain sector with, in a very specific sense, physically sane phenomenology. Afterwards, we solve the two-body problem to first order and obtain the orbits of a binary system in area metric gravity. Building up on this solution, the second order of the field equations is used to derive the emission of gravitational radiation from the binary system and the radiative loss, which causes spin-up of the system. The binary star subject to area metric gravity turns out to exhibit qualitatively new behaviour as compared to Einstein gravity, e.g.~additional massive modes of gravitational radiation or a modification of Kepler's third law.}

\textit{To a large extent, the work presented in this chapter has been published as Ref.~\cite{grav_rad_paper}. The results on radiation loss are not part of the previous publication.}

\section{Area metric gravity}
The matter theory in question is generalized linear electrodynamics (GLED) as defined in Def.~\ref{def_gled} with the Lagrangian density
\begin{equation}\label{lagrangian_gled}
  L_\text{GLED} = \omega_G G^{abcd} F_{ab} F_{cd},
\end{equation}
where we choose without loss of generality the scalar density
\begin{equation}\label{area_density}
  \omega_G = \left(\frac{1}{24}\epsilon_{abcd}G^{abcd}\right)^{-1}.
\end{equation}
The principal polynomial of GLED is quartic and takes the form
\begin{equation}
  \mathcal P_\text{GLED}(k) = -\frac{1}{24} \omega_G^2 \epsilon_{mnpq} \epsilon_{rstu} G^{mnra} G^{bpsc} G^{dqtu} k_a k_b k_c k_d.
\end{equation}
As appropriate Lorentz invariant expansion point constructed from the Minkowski metric $\eta$, we already determined in Example \ref{example_expansion_points}
\begin{equation}
  N^A = J_{abcd}^A (\eta^{ac} \eta^{bd} - \eta^{ad} \eta^{bc} + \epsilon^{abcd}).
\end{equation}
Before solving the system of equivariance equations perturbatively around $N$, let us reconsider the reduced power series ansatz \eqref{ansatz_reduced}. In addition to dropping terms with a total number of derivatives that is odd or greater than 2, and dropping non-Lorentz invariant expansion coefficients, we can also discard the linear term $a_A H^A$. This term would yield a constant in the Euler-Lagrange equations, causing the flat expansion point $N$ to no longer constitute a solution to the vacuum field equations. However, the perturbation ansatz stipulates that we perturb around a solution of the field equations. Since it is obvious that Equation \eqref{prolong_0} implies from vanishing coefficients $a_A$ that also the coefficient $a$ vanishes, we readily drop both and make the further reduced ansatz
\begin{equation}\label{ansatz_area_gravity}
  \begin{aligned}
    L &{} = a_A^{\hphantom AI} H^A_{\hphantom AI} \\
      &{} \hphantom{=} + a_{AB} H^A H^B + a_{A\hphantom pB}^{\hphantom Ap\hphantom Bq} H^A_{\hphantom Ap} H^B_{\hphantom Bq} + a_{AB}^{\hphantom{AB}I} H^A H^B_{\hphantom BI} \\
      &{} \hphantom{=} + a_{ABC} H^A H^B H^C + a_{AB\hphantom pC}^{\hphantom{AB}p\hphantom Cq} H^A H^B_{\hphantom Bp} H^C_{\hphantom Cq} + a_{ABC}^{\hphantom{ABC}I} H^A H^B H^C_{\hphantom CI} \\
      &{} \hphantom{=} + \mathcal O(H^4).
  \end{aligned}
\end{equation}

\section{Solving axiom 1}
Step 1 of the perturbative construction algorithm \ref{perturbative_algorithm} consists in computing the Gotay-Marsden coefficients for the gravitational bundle. For area metric gravity, we found in Sect.~\ref{section_gled}
\begin{equation}
  \gmc{A}{B}{n}{m} = 4 I^{pqrn}_B J^A_{pqrm},
\end{equation}
which followed from the general result \eqref{gmc_contra} for purely contravariant tensor bundles.

Proceeding with step 2, we need to construct a basis for the Lorentz invariant expansion coefficients
\begin{equation}
  (a_A^{\hphantom AI}, a_{AB}, a_{A\hphantom pB}^{\hphantom Ap\hphantom Bq}, a_{AB}^{\hphantom{AB}I}, a_{ABC}, a_{AB\hphantom pC}^{\hphantom{AB}p\hphantom Cq}, a_{ABC}^{\hphantom{ABC}I})
\end{equation}
in the ansatz \eqref{ansatz_area_gravity}. This task is solved using the Haskell library \texttt{sparse-tensor} \cite{sparse-tensor} discussed in Chapter \ref{}. The result is a basis of dimension 237, enumerated in full in Appendix \ref{} and summarized in Table \ref{}. It should be emphasized that the requirement of Lorentz invariance, which is not a direct stipulation but follows via the equivariance equations from a physically motivated assumption about the expansion point, drastically reduces the dimensionality of the ansatz from
\begin{equation}
  210 + \frac{21\cdot 22}{2} + 21\cdot 210 + \frac{84\cdot 85}{2} + \frac{21\cdot 22\cdot 23}{6} + \frac{21\cdot 22}{2} \cdot  210 + 21\cdot\frac{84\cdot 85}{2} = 133672
\end{equation}
to only 237. In principle, the correctness of the ansatz can be verified by showing that it is the most generic solution to the ansatz equations \ref{}. All we have to show is that the dimensionality of the ansatz equals the corank of the linear system of ansatz equations. For the ansatz including third-order coefficients, the system is quite large---considering that the coefficient space is already of dimension 133672---such that, on standard hardware, the rank cannot be computed naively by storing the matrix in memory and using methods like singular value decomposition or fraction-free gaussian elimination. It is rather easy, however, to use the aforementioned methods and work out the corank of the linear system determining the Lorentz invariant ansatz coefficients to second order, as the dimension of this ansatz space is only $210 + \frac{21\cdot 22}{2} + 21 \cdot 210 + \frac{84\cdot 85}{2} = 8421$. Confirming the number of obtained basis ansätze up to second order, the corank of the corresponding system is indeed 40. The computer code for this computation is published as Ref.~\cite{second-order-area-repo}.
\begin{table}
  \centering
  \begin{tabular}{l r r}
    \toprule
    coefficient & dimension & gravitational constants \\
    \midrule
    $a_A^{\hphantom AI}$ & 3 & $(e_{38},\dots ,e_{40})$ \\ \addlinespace[2pt]
    $a_{AB}$ & 6 & $(e_{1},\dots ,e_{6})$ \\ \addlinespace[2pt]
    $a_{A\hphantom pB}^{\hphantom Ap\hphantom Bq}$ & 15 & $(e_{7},\dots ,e_{21})$ \\ \addlinespace[2pt]
    $a_{AB}^{\hphantom{AB}I}$ & 16 & $(e_{22},\dots ,e_{37})$ \\ \addlinespace[2pt]
    $a_{ABC}$ & 15 & $(e_{41},\dots ,e_{55})$ \\ \addlinespace[2pt]
    $a_{AB\hphantom pC}^{\hphantom{AB}p\hphantom Cq}$ & 110 & $(e_{56},\dots ,e_{165})$ \\ \addlinespace[2pt]
    $a_{ABC}^{\hphantom{ABC}I}$ & 72 & $(e_{166},\dots ,e_{237})$ \\ \addlinespace[2pt]
    \bottomrule 
  \end{tabular}
  \caption{Summary of the Lorentz invariant expansion coefficients for the area metric gravity ansatz \eqref{ansatz_area_gravity} obtained from the Haskell library \texttt{sparse-tensor} \cite{}. The dimension is the number of linearly independent basis tensors returned from the computer program. Assigning labels from $1$ to $237$ to all basis tensors, an ansatz is represented by real numbers $e_1\dots e_{237}$ using its unique basis decomposition. These numbers parameterize the gravitational theory and are thus referred to as \emph{gravitational constants}. For a complete picture of the decomposition of ansätze using basis tensors, refer to Appendix \ref{} or the computer code in Ref.~\cite{second-order-area-repo}.}
\end{table}

With the 237 ansatz coefficients at hand, solving the equivariance equations as required for step 5 is only a matter of inserting the ansatz in the system and its first two prolongations as displayed in Eqns.~\ref{prolong_0}--\ref{prolong_2}, extracting a system of linear equations for the gravitational constants, and solving this system. This task is again performed using efficient computer algebra, implemented in the Haskell library \texttt{safe-tensor}, which is introduced in Chapter \ref{}. The procedure is roughly as follows: A compatibility layer with \texttt{sparse-tensor} is used in order to construct the ansatz tensors and make them available as \texttt{Tensor} types with generalized rank (see Sect.~\ref{}). Together with predefined tensors like Kronecker deltas, intertwiners, Gotay-Marsden coefficients, or the Minkowski metric, the ansatz tensors are used in order to construct the (prolonged) equivariance equations evaluated at $N$ (Eqns.~\ref{prolong_0}--\ref{prolong_2}). Each tensorial equation is a value of type \texttt{Tensor} and, as such, can be evaluated into a list of its components. Every component is a linear equation for the 237 gravitational constants. Collecting all components for all tensorial equations, we obtain a matrix representing the linear system for the constants $e_1\dots e_{237}$. The system is small enough to be brought into reduced row echelon form applying fraction-free gaussian elimination and backward substitution using 64-bit integers\footnote{Exploiting the observation we made earlier that, using intertwiners with purely rational components, all coefficients in the system remain rational.}, which yields a solution that parameterizes the constants with a few remaining indeterminate gravitational constants. As an example for the process, let us walk through the solution for the linear expansion coefficient $a_A^{\hphantom AI}$.

\begin{example}[solution of the equivariance equations to first order]
  Having set $a_A = 0$, the remaining expansion coefficient for the linear order is $a_A^{\hphantom AI}$, which is determined in part by the second unprolonged equation \eqref{prolong_0}. A suitable basis for this coefficient is
  \begin{equation}
    a_A^{\hphantom AI} = J_A^{abcd} J_{pq}^I\lbrack e_1 \cdot \eta_{ac} \eta_{bd} \eta^{pq} + e_2 \cdot \eta_{ac} \delta_b^p \delta_d^q + e_3 \cdot \epsilon_{abcd} \eta^{pq} \rbrack
  \end{equation}
  with three gravitational constants $e_1,e_2,e_3$. Inserting this ansatz into the unprolonged equation
  \begin{equation}
    0 = a_A^{\hphantom AI} \gmc{A}{B}{(n}{m} J^{pq)}_I N^B =\vcentcolon T^{npq}_m
  \end{equation}
  yields a tensorial equation $0 = T^{npq}_m$ with 256 components. Each component is of the form
  \begin{equation}
    0 = c_1\cdot e_1  + c_2\cdot e_2 + c_3\cdot e_3.
  \end{equation}
  The collection of all components is a system of 256 linear equations for three variables. A lot of these equations are redundant, because they are trivial or linearly dependent. A naive reduction by eliminating trivial equations and choosing only one representative for equations that are multiples of each other already reduces the system to the single equation
  \begin{equation}
    0 = 2 e_1 + e_2 + 4 e_3.
  \end{equation}
  Setting e.g.~$e_2 = -2 e_1 -4 e_3$ solves the equivariance equation for the coefficient $a_A^{\hphantom AI}$, leaving it parameterized by two gravitational constants $e_1$ and $e_3$.
\end{example}

Applied to the whole system of equivariance equations, we obtain a parameterization of the solution (displayed in full in Appendix \ref{}) by 50 independent gravitational constants. A subset of 16 constants governs \emph{linearized} area metric gravity via the quadratic Lagrangian density, from which---as we will encounter later---only 11 independent linear combinations play a role for the Euler-Lagrange equations. The procedure outlined here is implemented in Haskell using the library \texttt{sparse-tensor} for ansatz generation as well as the library \texttt{safe-tensor} for constructing and solving the equivariance equations, with the source code and results published as Ref.~\cite{second-order-area-repo}.

\section{Solving axiom 2}
The pedestrian approach towards implementing causal compatibility of the just constructed gravitational theory with GLED is to carefully execute steps 6--12 of the perturbative construction algorithm. This way, we obtain an approximation of the area metric gravity principal polynomial and have to match the causal structure with a first-order expansion of the GLED principal polynomial. While entirely feasible, this approach is less illustrive than the \emph{constructive} approach we employ instead. The underlying realization behind this technique is that the diffeomorphism invariance of the gravitational theory dramatically restricts the possible principal polynomials. In fact, we will see that for third-order area metric Lagrangians, the admissible principal polynomials are already causally compatible with the corresponding expansion of the GLED polynomial. \emph{There is no causality mismatch left to be fixed.}

To this end, recall the GLED polynomial \eqref{gled_polynomial}, which using the scalar density \eqref{area_density} assumes the form
\begin{equation}
  \mathcal P_\text{GLED}(k) = -\frac{1}{\frac{1}{24}(\epsilon_{abcd}G^{abcd})^2} \epsilon_{mnpq} \epsilon_{rstu} G^{mnra} G^{bpsc} G^{dqtu} k_a k_b k_c k_d.
\end{equation}
Expanding this expression to linear order in the perturbation yields
\begin{equation}\label{gled_poly_first_order}
  \begin{aligned}
    \mathcal P_\text{GLED}(k) &{} = \left\{ \left\lbrack 1 - \frac{1}{24} \epsilon(H) \right\rbrack \eta(k,k) + \frac{1}{2} H(k,k) \right\}^2 + \mathcal O(H^2) \\
                              &{} = \lbrack P^{(\leq 1)}_\text{GLED} \rbrack^2 + \mathcal O(H^2),
  \end{aligned}
\end{equation}
where the abbreviations
\begin{equation}
  \epsilon(H) = \epsilon_{abcd} H^{abcd}\quad\text{and}\quad H(k,k) = \eta_{ac} H^{abcd} k_b k_d
\end{equation}
have been introduced. In the following, we will also make use of
\begin{equation}
  \eta(H) = \eta_{ac} \eta_{bd} H^{abcd}.
\end{equation}
Up to first order, we find that the GLED polynomial factors into the square of a metric polynomial $P^{(\leq 1)}_\text{GLED}$. This has a remarkable consequence: For weak gravitational fields, where the approxmiation to first order is sufficiently good, the physics of point particles adhering to GLED dynamics is indistinguishable from the Maxwellian setting with a metric perturbation $h$ by virtue of the identification
\begin{equation}
  h^{ab} = \left\lbrack 1 - \frac{1}{24} \epsilon(H) \right\rbrack \eta^{ab} + \frac{1}{2} \eta_{cd}H^{acbd} = (P^{(\leq 1)}_\text{GLED})^{ab}.
\end{equation}
This effect only holds in the limit of geometric optics---the GLED field equations do \emph{not} reduce to Maxwell equations with a metric perturbation. Consequently, even to first order in the area metric perturbation, non-metric effects can be observed. An in-depth study of classical and quantum electrodynamics on weakly birefringent backgrounds based on exactly this realization has been conducted in Ref.~\cite{quantum_gled}.

We will now proceed to show that the possible principal polynomials arising from third-order area metric gravity Lagrangians as constructed in the previous section are only mildly more general than the effectively quadratic first-order GLED polynomial \eqref{gled_poly_first_order}. This issue is approached by first considering the corresponding Euler-Lagrange equations.

\begin{proposition}

\end{proposition}


\section{3+1 decomposition}
\begin{itemize}
\item 3+1 split
\item observer frame
\item spatial fields
\item gauge fixing
\item reduction: schwarzschild solution + mode decoupling
\end{itemize}

\section{binary pulsar}
\begin{itemize}
\item gravitational radiation
\item spin-up
\end{itemize}

\textbf{take-home message: application to a refined matter theory yields interesting phenomenology}
