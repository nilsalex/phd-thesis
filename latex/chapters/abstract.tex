\begin{abstract}
  Matter theories are not predictive if they couple to geometry with unknown dynamics: it is not possible to anticipate how matter behaves in the future without knowing how the geometry evolves. This thesis studies the completion of such matter theories to predictive theories of matter and gravity. Einstein solved this problem for Maxwell's electrodynamics by providing the Einstein equations. Indeed, a recurring theme of the work presented here is that general relativity is recovered as far as metric theories are concerned.

  We start with the definition of two axioms that guide the completion of matter theories to predictive theories: gravity must be generally covariant and causally compatibility with the matter theory. Both axioms are brought into precise mathematical form. The foundation for the mathematical formulation is Lagrangian field theory on the jet bundle, where the geometry may be given by fields of arbitrary tensorial nature or, by extension of the approach, even nontensorial fields. In particular, the geometry need not be metric.

  From the mathematical definitions follow partial differential equations and algebraic equations whose solutions yield candidate gravitational Lagrangians. This finding reduces the task of completing a matter theory with a gravitational theory to a computational problem, which we state in the condensed form of an algorithm. Since the algorithm provides a construction procedure for gravitational theories on the basis of general covariance, it shall bear the name \emph{covariant constructive gravity}.

  Applying the construction algorithm to Maxwell's electrodynamics reproduces general relativity. Theories beyond Maxwell and Einstein turn out harder to construct, such that we need to reduce the complexity of the problem in order to arrive at \emph{some} physical implications. One possibility is to make a perturbation ansatz, which transforms the problem into simple linear algebra. Using this ansatz, we derive the second-order gravitational field equations for a birefringent generalisation of Maxwell's electrodynamics and consider the binary star as a prototypical example. Interesting phenomenology is obtained as result: a modification of Kepler's third law, the emission of massive gravitational waves, and a modified inspiral curve. These predictions demonstrate the predictive power of covariant constructive gravity---given a generalisation of Maxwell's electrodynamics, it is possible to derive gravitational implications.

  The second approach is symmetry reduction, which is shown to yield the Friedmann equations if applied to a metric theory with cosmological symmetry. We sketch the application to nonmetric theories, but leave the implementation open for future research.

\end{abstract}

\newpage
\begin{ngerman}
\renewcommand{\abstractname}{Kurzzusammenfassung}
\begin{abstract}
  Materietheorien, denen Geometrie mit unbekannter Dynamik zugrunde liegt, sind nicht prädiktiv. Es fehlt ihnen die Möglichkeit, das Verhalten von Materie in der Zukunft vorherzusagen, denn die Entwicklung der Geometrie ist nicht bekannt. In dieser Dissertation soll die Vervollständigung von solchen Materietheorien zu prädiktiven Theorien von Materie \emph{und} Gravitation untersucht werden. Einstein hat dieses Problem für die Maxwellsche Elektrodynamik gelöst, indem er die Einsteinschen Feldgleichungen postuliert hat. Auch im Folgenden wird die Allgemeine Relativitätstheorie erneut hergeleitet werden, wann immer metrische Theorien besprochen werden.

  Zu Beginn werden die beiden Axiome präsentiert, welche die Vervollständigung von Materietheorien leiten: Die Gravitationstheorie muss allgemein kovariant sein und eine zur Materietheorie kompatible Kausalität aufweisen. Mittels Lagrange-Feldtheorie auf Jetbündeln gelingt eine präzise mathematische Definition beider Axiome, wobei die Geometrie durch beliebige tensorielle Felder gegeben sein kann -- sogar eine Erweiterung zu nicht-tensoriellen Feldern ist möglich. Hervorzuheben ist, dass die Geometrie nicht zwingend metrisch sein muss.

  Die mathematische Formulierung der Axiome impliziert sowohl partielle Differentialgleichungen als auch algebraische Gleichungen, deren Lösungen potentielle Lagrange-Dichten der Gravitation sind. Damit reduziert sich das Problem der Vervollständigung von Materietheorien mittels einer geeigneten Gravitationstheorie auf ein reines Rechenproblem, welches in Form eines Algorithmus angegeben werden kann. Dieser konstruktive Zugang zu modifizierten Gravitationstheorien, der auf dem kovarianten Lagrange-Formalismus basiert, wird \emph{Kovariante Konstruktive Gravitation} genannt.

  Angewandt auf die Maxwellsche Elektrodynamik, reproduziert der Algorithmus die Allgemeine Relativitätstheorie. Theorien jenseits von Maxwell und Einstein sind weniger trivial zu konstruieren, weshalb Methoden zur Reduktion der Komplexität erforderlich sind, um überhaupt physikalische Schlüsse ziehen zu können. Ein Störungsansatz reduziert das Problem auf Lineare Algebra. Mittels dieses Ansatzes lässt sich die zweite Störungsordnung der gravitativen Feldgleichungen für eine doppelbrechende Erweiterung der Maxwellschen Elektrodynamik herleiten. In diesem Beispiel stellt ein Doppelsternsystem interessante Phänomenologie zur Schau: ein modifiziertes drittes Keplersche Gesetz, die Abstrahlung von massiven Gravitationswellen, sowie eine veränderte Dynamik der Orbitperiode aufgrund der Strahlungsverluste. Solche Ergebnisse verdeutlichen die Vorhersagekraft der Kovarianten Konstruktiven Gravitation -- aus einer Verallgemeinerung der Elektrodynamik folgen gravitative Phänomene.

  Des Weiteren ist es möglich, mittels Symmetriereduktion eine Gravitationstheorie mit begrenzter Gültigkeit herzuleiten. Am Beispiel einer metrischen Theorie mit kosmologischer Symmetrie werden die Friedmann-Gleichungen wiederentdeckt. Die Anwendung dieser Vorgehensweise auf nichtmetrische Theorien wird nur skizziert, ihre Implementierung bleibt ein offenes Forschungsgebiet.

\end{abstract}
\end{ngerman}
