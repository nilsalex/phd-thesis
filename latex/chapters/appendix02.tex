\chapter{Solution of the equivariance equations}
\label{appendix_reduction}

The following relations for the ansatz coefficients $e_{1},\dots,e_{237}$ solve the perturbative equivariance equations \eqref{prolong_0}--\eqref{prolong_2} in terms of 50 indeterminate constants $k_{1},\dots,k_{50}$. See \cite{Alex_2020_area-metric-gravity} for Haskell code that yields this result.

\begin{itemize}
  \item first and second order (constants $e_1,\dots,e_{40}$):
\begingroup
\allowdisplaybreaks
\begin{align}\label{eq_second_order_constants}
  e_{1} = {} &  k_{1} \nonumber \\
  e_{2} = {} &  k_{2} \nonumber \\
  e_{3} = {} & -2 k_{1} -\frac{2}{3} k_{2} \nonumber \\
  e_{4} = {} & 4 k_{1} + \frac{1}{3} k_{2} \nonumber \\
  e_{5} = {} &  k_{3} \nonumber \\
  e_{6} = {} & -3 k_{1} -\frac{1}{2} k_{2} -3 k_{3} \nonumber \\
  e_{7} = {} &  k_{4} \nonumber \\
  e_{8} = {} &  k_{5} \nonumber \\
  e_{9} = {} &  k_{6} \nonumber \\
  e_{10} = {} &  k_{7} \nonumber \\
  e_{11} = {} &  k_{8} \nonumber \\
  e_{12} = {} & \frac{1}{2} k_{6} + \frac{5}{8} k_{7} \nonumber \\
  e_{13} = {} & -\frac{16}{3} k_{4} + 16 k_{5} -\frac{7}{3} k_{6} -\frac{5}{12} k_{7} + \frac{4}{3} k_{8} \nonumber \\
  e_{14} = {} & -\frac{8}{3} k_{4} + 8 k_{5} -\frac{13}{6} k_{6} -\frac{11}{24} k_{7} + \frac{2}{3} k_{8} \nonumber \\
  e_{15} = {} &  k_{4} -\frac{1}{8} k_{6} -\frac{23}{32} k_{7} -\frac{1}{2} k_{8} \nonumber \\
  e_{16} = {} &  k_{9} \nonumber \\
  e_{17} = {} &  k_{10} \nonumber \\
  e_{18} = {} & \frac{3}{2} k_{4} + \frac{3}{4} k_{6} -\frac{3}{16} k_{7} + 3 k_{9} \nonumber \\
  e_{19} = {} & \frac{1}{2} k_{4} + \frac{1}{4} k_{6} -\frac{1}{16} k_{7} +  k_{9} \nonumber \\
  e_{20} = {} & -\frac{1}{4} k_{4} -\frac{1}{8} k_{6} + \frac{1}{32} k_{7} -\frac{1}{2} k_{9} \nonumber \\
  e_{21} = {} &  k_{4} -3 k_{5} + \frac{1}{4} k_{6} -\frac{3}{16} k_{7} -\frac{1}{2} k_{8} +  k_{9} -3 k_{10} \nonumber \\
  e_{22} = {} &  k_{11} \nonumber \\
  e_{23} = {} &  k_{12} \nonumber \\
  e_{24} = {} &  k_{13} \nonumber \\
  e_{25} = {} &  k_{14} \nonumber \\
  e_{26} = {} &  k_{6} + \frac{3}{4} k_{7} - k_{14} \nonumber \\
  e_{27} = {} & - k_{4} + \frac{1}{2} k_{7} \nonumber \\
  e_{28} = {} & \frac{5}{3} k_{4} + \frac{5}{12} k_{6} -\frac{25}{48} k_{7} -2 k_{11} - k_{12} -\frac{2}{3} k_{13} -\frac{1}{4} k_{14} \nonumber \\
  e_{29} = {} &  k_{6} + \frac{3}{4} k_{7} - k_{14} \nonumber \\
  e_{30} = {} & -\frac{4}{3} k_{4} -\frac{5}{6} k_{6} + \frac{1}{24} k_{7} + 4 k_{11} + 2 k_{12} + \frac{1}{3} k_{13} + \frac{1}{2} k_{14} \nonumber \\
  e_{31} = {} &  k_{15} \nonumber \\
  e_{32} = {} &  k_{16} \nonumber \\
  e_{33} = {} &  k_{4} -\frac{1}{2} k_{7} -3 k_{11} -\frac{1}{2} k_{13} -6 k_{15} \nonumber \\
  e_{34} = {} & \frac{1}{2} k_{6} + \frac{3}{8} k_{7} -\frac{3}{2} k_{12} -\frac{1}{2} k_{14} -3 k_{16} \nonumber \\
  e_{35} = {} & -2 k_{4} - k_{6} + \frac{1}{4} k_{7} \nonumber \\
  e_{36} = {} & - k_{4} + \frac{1}{2} k_{7} -\frac{3}{2} k_{12} -\frac{1}{2} k_{14} -3 k_{16} \nonumber \\
  e_{37} = {} & \frac{1}{12} k_{4} + \frac{1}{12} k_{6} + \frac{1}{48} k_{7} -\frac{1}{8} k_{12} -\frac{1}{24} k_{14} +  k_{15} + \frac{1}{4} k_{16} \nonumber \\
  e_{38} = {} & -2 k_{4} +  k_{7} \nonumber \\
  e_{39} = {} & -2 k_{6} -\frac{3}{2} k_{7} \nonumber \\
e_{40} = {} &  k_{4} + \frac{1}{2} k_{6} -\frac{1}{8} k_{7}
\end{align}
\endgroup
  \item third order (constants $e_{41},\dots,e_{237}$): see \cite{Alex_2020_area-metric-gravity}.
\end{itemize}

From the 16 constants $k_{1},\dots,k_{16}$ that govern the second-order expansion of the area metric Lagrangian, only 10 linearly independent combinations contribute to the linearized Euler-Lagrange equations. A possible basis is given by the 10 gravitational constants $s_{i}$ below\footnote{The constants are not labelled with consecutive numbers, because the labels reflect how they have been calculated: each constant from $s_1$ to $s_{46}$ is the prefactor of a certain term in the scalar field equation, but also a linear combination of the 16 constants $k_{i}$. The subset below is a basis; every coefficient of the linearized field equations is a linear combination of the $s_{i}$.}. These are obtained from a $3+1$ and subsequent scalar-vector-tensor decomposition of the linearized field equations (see \cite{Alex_2020_area-metric-gravity}).
\begin{align}\label{eq_linearized_eom_constants}
    s_{1} = {} &  2 k_{6} + \frac{3}{2} k_{7} \nonumber \\
    s_{3} = {} &  \frac{3}{2} k_{6} + \frac{9}{8} k_{7} - 6 k_{12} - 2 k_{14} \nonumber \\
    s_{4} = {} &  -\frac{1}{2} k_{6} - \frac{3}{8} k_{7} - \frac{1}{2} k_{14} \nonumber \\
    s_{6} = {} &  k_{6} + \frac{3}{4} k_{7} - 3 k_{12} - k_{14} - 6 k_{16} \nonumber \\
    s_{11} = {} &  \frac{1}{2} k_{6} + \frac{11}{8} k_{7} + 2 k_{8} - 2 k_{13} - \frac{1}{2} k_{14} \nonumber \\
    s_{13} = {} &  -2 k_{2} \nonumber \\
    s_{14} = {} &  -2 k_{4} + 24 k_{5} - k_{6} - \frac{3}{4} k_{7} + 4 k_{8} - 12 k_{9} + 24 k_{10} - 24 k_{11} - 6 k_{12} - 4 k_{13} \nonumber \\ & - 2 k_{14} - 48 k_{15} - 12 k_{16} \nonumber \\
    s_{16} = {} &  -24 k_{1} - 4 k_{2} - 24 k_{3} \nonumber \\
    s_{37} = {} &  -24 k_{5} + 2 k_{6} + \frac{5}{2} k_{7} - 4 k_{8} + 24 k_{11} - 12 k_{12} + 4 k_{13} - 4 k_{14} \nonumber \\
    s_{39} = {} &  24 k_{1} + 4 k_{2}
\end{align}
