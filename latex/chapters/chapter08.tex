\chapter{Conclusions}

In this thesis, the concept of covariant constructive gravity has been put on a solid mathematical footing. Lagrangian field theory on jet bundles turned out to be ideally suited for the definition of the general covariance axiom. The equivariance equations that follow from this axiom transform the implementation into a computational task, opening up the \emph{constructive} pathway towards modified theories of gravity. Using the Cartan form that corresponds to a diffeomorphism invariant Lagrangian density, we have seen how general covariance implies a version of the first and second Noether theorem---important results that have proven very useful further down the line. Within this framework, the axiom of causal compatibility has been formulated as additional algebraic conditions on the gravitational field equations.

From the equivariance equations and causality conditions, we derived a concise algorithm which guides the construction of novel gravitational theories that implement general covariance and are causally compatible to a given matter theory. Because this can be seen as generalization of Lovelock's uniqueness theorems for Einstein gravity \cite{Lovelock_1969,Lovelock_1971,Lovelock_1972}, we could show that this construction procedure applied to Maxwell's electrodynamics indeed reproduces metric gravity as derived by Einstein.

Of course, covariant constructive gravity would not be that interesting if it were just another tool that reproduces Einstein gravity. Its \emph{raison d'\^etre} is the derivation of \emph{modified} theories of gravity that complete novel matter theories to predictive models of the universe. The remainder of the thesis was dedicated towards achieving this goal. First, we have discussed three examples of novel matter theories: birefringent electrodynamics and two bimetric theories. While it is straightforward to set up the construction procedure and derive general results concerning the solution space, \emph{finding} these solutions in practice is notoriously hard and turned out not to be feasible for the examples in question.

Therefore, we investigated possibilities to arrive at results that are valid in certain specific settings, without the need to know the ``full'' solutions. Our main strategy was perturbation theory, which seeks to find solutions that are valid for small deviations of the gravitational field from a Lorentz invariant background. With a corresponding perturbation ansatz, the equivariance equations transform into a system of linear equations for the expansion coefficients. As a consequence of the Lorentz invariance of the background geometry, the expansion coefficients themselves exhibit Lorentz invariance, which reduces their dimensionality a lot---before solving any equivariance equation.

Many of the computations that are necessary in order to construct Lorentz invariant ansätze and solve the perturbative equivariance equations have been delegated to the computer. For this purpose, two Haskell packages have been developed and presented in this thesis---with a focus on the package \texttt{sparse-tensor} which composes and solves the equivariance equations. Methods from functional programming lend themselves for an efficient and safe implementation of tensor algebra, enabling us to repeat and modify calculations whenever required, without having to redo them by hand.

\begin{itemize}
\item complements canonical approach
\item impulses for canonical constructive gravity: causality, ansätze
\item from new concept to prediction of novel effects
\item conceptual, more work needed for prediction of outcome of high-precision experiments
\item relevance is given regardless of the status of quantum gravity
\end{itemize}

coming back to the quote
