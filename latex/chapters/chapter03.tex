\chapter{The construction algorithm}

\textit{Having introduced the axioms of covariant constructive gravity and cast them in precise mathematical language, we consolidate the results and state the algorithm for the construction of modified gravity Lagrangians from novel matter theories. After a discussion about practical implications in general, we proceed with sketching the application to a few examples.}

\section{General formulation}

The results obtained so far allow us to formulate a comprehensive algorithm for the construction of gravitational Lagrangians. These Lagrangians are the most general conceivable Lagrangians within our formalism that satisfy both axioms of covariant constructive gravity. All that has to be provided is a matter theory that couples to geometry and the algorithm will yield all candidates for gravitational theories that determine the so far undetermined dynamics of the gravitational field, finally giving the theory predictive power. In this sense, the task of searching for modified gravitational theories boils down to the solution of specific PDE systems of a simple form and algebraic equations concerning the causal structure of gravity.

\begin{algorithm}[H]\label{algorithm}
  \DontPrintSemicolon
  \KwData{Geometry bundle $E_\text{grav}\overset{\pi_\text{grav}}{\longrightarrow}M$, matter bundle $E_\text{mat}\overset{\pi_\text{mat}}{\longrightarrow}M$,
  Lagrangian matter field theory on $E_\text{grav} \oplus_M J^1E_\text{mat}$ with linear field equations}
  \KwResult{Most general diffeomorphism invariant and causally compatible gravitational Lagrangian field theory on $J^2E_\text{grav}$}
  compute the Gotay-Marsden coefficients \eqref{gmc_contra} for $E_\text{grav}$ \;
  set up the equivariance equations \eqref{equivariance_eqn_1}--\eqref{equivariance_eqn_4} \;
  solve the equivariance equations for the gravitational Lagrangian density $L_\text{grav}$ \;
  compute the Euler-Lagrange equations \eqref{euler_lagrange_local} corresponding to $L_\text{grav}$ \;
  restrict the gravitational theory to second-derivative-order field equations \;
  calculate the principal polynomials \eqref{principal_polynomial} $\mathcal P_\text{grav}$ and $\mathcal P_\text{mat}$ \;
  solve the causal compatibility conditions $C_\text{grav} = C_\text{mat}$ and $V_\text{grav} \subseteq V_\text{mat}$
  \caption{Gravitational closure using covariant constructive gravity}
\end{algorithm}

Let us comment on the algorithm step by step: The first step, calculating the Gotay-Marsden coefficients, is trivial. The coefficients follow from the prescribed or inherited action of diffeomorphisms on the geometry bundle. For purely covariant or contravariant tensor bundles, Prop.~\ref{prop_gmc_intertwiner} already gives the final expression. These coefficients have to be inserted into Eqns.~\ref{equivariance_eqn_1}--\ref{equivariance_eqn_4} in order to execute step 2. As a result, we obtain a system of linear, first-order partial differential equations for the Lagrangian density $L$ with coefficients that are linear in the independent variables. More precisely, the PDEs are of the form
\begin{equation}
  0 = A^j_i x^i u_{,j} + B u,
\end{equation}
$u$ is the dependent variable, $x^i$ are the independent variables and the coefficients $A^j_i$ and $B$ are constants. Conceptually, it is clear how to solve such PDEs\cite{seiler}, albeit in most cases practically infeasible, due to the sheer size of the system and the number of independent variables. However, as we will see in a later chapter, the system admits a property called \emph{involutivity}, from which we can infer strong results about solutions and derive a perturbative solution strategy.

For a known solution, it is only a matter of applying Eq.~\ref{euler_lagrange_local} to the Lagrangian in order to compute the Euler-Lagrange equations for step 4. Restricting to second-derivative-order field equations, as required by step 5, could be done now by imposing that higher-derivative-order terms vanish. In practice, however, such restrictions will be placed at an earlier stage, in order to rule out higher orders from the beginning. A similar pattern emerges for steps 6 and 7: placing restrictions on the computed entities is possible, but may be hard to enforce after the fact. So it is worth keeping this requirement in mind early on.

\section{Example: Einstein gravity}

As already outlined in the introduction, Einstein gravity can be thought of as the gravitational closure of Maxwell electrodynamics in four dimensions. This theory provides dynamics for sections $A$ in the bundle $T^\ast M$ of one-forms, parameterized with sections $g$ in the bundle $S(T^2_0M)$ of contravariant\footnote{Many descriptions regard the covariant inverse metric as fundamental. In this case, the contravariant metric tensor appearing in the Lagrangian density \eqref{maxwell_lagrangian} is the inverse of the metric field. Both descriptions will yield slightly differing intermediate results during the construction procedure, but are fundamentally equivalent.} symmetric tensors of rank two. $A$ is commonly known as the \emph{electromagnetic potential}, $g$ as the \emph{metric tensor}. The dynamics of the electromagnetic field is given by the Lagrangian density
\begin{equation}
  L_\text{Maxwell} = \sqrt{-\operatorname{det} g} g^{ac}g^{bd} F_{ab} F_{cd},
\end{equation}
where the electromagnetic potential enters via the field strength tensor $F = \mathrm dA$ and we write ``$\operatorname{det}g$'' for the determinant of the \emph{covariant} metric tensor, which is the inverse of the metric tensor chosen here as fundamental field.

We now collect the ingredients for the construction algorithm. The fibre dimension of the bundle $S(T^2_0M)$ is 10, such that a suitable pair of intertwiners $(I,J)$ between this bundle and the unrestricted tensor bundle $T^2M$ is given by
\begin{equation}
  \footnotesize
  \begin{gathered}
  I^{ab}_1 = \begin{pmatrix}1 & 0 & 0 & 0 \\ 0 & 0 & 0 & 0 \\ 0 & 0 & 0 & 0 \\ 0 & 0 & 0 & 0\end{pmatrix}^{ab}, I^{ab}_2 = \begin{pmatrix}0 & 1 & 0 & 0 \\ 1 & 0 & 0 & 0 \\ 0 & 0 & 0 & 0 \\ 0 & 0 & 0 & 0\end{pmatrix}^{ab}, I^{ab}_3 = \begin{pmatrix}0 & 0 & 1 & 0 \\ 0 & 0 & 0 & 0 \\ 1 & 0 & 0 & 0 \\ 0 & 0 & 0 & 0\end{pmatrix}^{ab} I^{ab}_4 = \begin{pmatrix}0 & 0 & 0 & 1 \\ 0 & 0 & 0 & 0 \\ 0 & 0 & 0 & 0 \\ 1 & 0 & 0 & 0\end{pmatrix}^{ab}, \\
  I^{ab}_5 = \begin{pmatrix}0 & 0 & 0 & 0 \\ 0 & 1 & 0 & 0 \\ 0 & 0 & 0 & 0 \\ 0 & 0 & 0 & 0\end{pmatrix}^{ab}, I^{ab}_6 = \begin{pmatrix}0 & 0 & 0 & 0 \\ 0 & 0 & 1 & 0 \\ 0 & 1 & 0 & 0 \\ 0 & 0 & 0 & 0\end{pmatrix}^{ab}, I^{ab}_7 = \begin{pmatrix}0 & 0 & 0 & 0 \\ 0 & 0 & 0 & 1 \\ 0 & 0 & 0 & 0 \\ 0 & 1 & 0 & 0\end{pmatrix}^{ab}, I^{ab}_8 = \begin{pmatrix}0 & 0 & 0 & 0 \\ 0 & 0 & 0 & 0 \\ 0 & 0 & 1 & 0 \\ 0 & 0 & 0 & 0\end{pmatrix}^{ab}, \\
  I^{ab}_9 = \begin{pmatrix}0 & 0 & 0 & 0 \\ 0 & 0 & 0 & 0 \\ 0 & 0 & 0 & 1 \\ 0 & 0 & 1 & 0\end{pmatrix}^{ab}, I^{ab}_{10} = \begin{pmatrix}0 & 0 & 0 & 0 \\ 0 & 0 & 0 & 0 \\ 0 & 0 & 0 & 0 \\ 0 & 0 & 0 & 1\end{pmatrix}^{ab},
  \end{gathered}
\end{equation}
and
\begin{equation}
  \footnotesize
  \begin{gathered}
    J_{ab}^1 = \begin{pmatrix}1 & 0 & 0 & 0 \\ 0 & 0 & 0 & 0 \\ 0 & 0 & 0 & 0 \\ 0 & 0 & 0 & 0\end{pmatrix}_{ab}, J_{ab}^2 = \begin{pmatrix}0 & \frac{1}{2} & 0 & 0 \\ \frac{1}{2} & 0 & 0 & 0 \\ 0 & 0 & 0 & 0 \\ 0 & 0 & 0 & 0\end{pmatrix}_{ab}, J_{ab}^3 = \begin{pmatrix}0 & 0 & \frac{1}{2} & 0 \\ 0 & 0 & 0 & 0 \\ \frac{1}{2} & 0 & 0 & 0 \\ 0 & 0 & 0 & 0\end{pmatrix}_{ab} J_{ab}^4 = \begin{pmatrix}0 & 0 & 0 & \frac{1}{2} \\ 0 & 0 & 0 & 0 \\ 0 & 0 & 0 & 0 \\ \frac{1}{2} & 0 & 0 & 0\end{pmatrix}_{ab}, \\
    J_{ab}^5 = \begin{pmatrix}0 & 0 & 0 & 0 \\ 0 & 1 & 0 & 0 \\ 0 & 0 & 0 & 0 \\ 0 & 0 & 0 & 0\end{pmatrix}_{ab}, J_{ab}^6 = \begin{pmatrix}0 & 0 & 0 & 0 \\ 0 & 0 & \frac{1}{2} & 0 \\ 0 & \frac{1}{2} & 0 & 0 \\ 0 & 0 & 0 & 0\end{pmatrix}_{ab}, J_{ab}^7 = \begin{pmatrix}0 & 0 & 0 & 0 \\ 0 & 0 & 0 & \frac{1}{2} \\ 0 & 0 & 0 & 0 \\ 0 & \frac{1}{2} & 0 & 0\end{pmatrix}_{ab}, J_{ab}^8 = \begin{pmatrix}0 & 0 & 0 & 0 \\ 0 & 0 & 0 & 0 \\ 0 & 0 & 1 & 0 \\ 0 & 0 & 0 & 0\end{pmatrix}_{ab}, \\
    J_{ab}^9 = \begin{pmatrix}0 & 0 & 0 & 0 \\ 0 & 0 & 0 & 0 \\ 0 & 0 & 0 & \frac{1}{2} \\ 0 & 0 & \frac{1}{2} & 0\end{pmatrix}_{ab}, J_{ab}^{10} = \begin{pmatrix}0 & 0 & 0 & 0 \\ 0 & 0 & 0 & 0 \\ 0 & 0 & 0 & 0 \\ 0 & 0 & 0 & 1\end{pmatrix}_{ab}.
  \end{gathered}
\end{equation}
The intertwiner $I$ distributes the ten degrees of freedom for a symmetric tensor of dimension 4 across the components of a generic rank-2 tensor
\begin{equation}
  I^{ab}(c_Au^A) = \begin{pmatrix} c_1 & c_2 & c_3 & c_4 \\ c_2 & c_5 & c_6 & c_7 \\ c_3 & c_6 & c_8 & c_9 \\ c_4 & c_7 & c_9 & c_{10}\end{pmatrix}^{ab},
\end{equation}
while $J$ projects such symmetrically distributed components back to the ten degrees of freedom, discarding possible antisymmetric contributions. Note that $I$ and $J$ could also be chosen such that the matrix representations coincide by using factors of $\frac{1}{\sqrt 2}$ for off-diagonal entries in both intertwiners. This has the apparent advantage that $I$ and $J$ do not need to be distinguished from each other. However, a disadvantage of using them interchangeably is that this would obscure the different roles that $I$ and $J$ play, especially if they are used not only in setting up the equivariance equations, but also for manipulating them. The irrational coefficients like $\frac{1}{\sqrt 2}$ would also further complicate the computer-aided treatment introduced in Chap.~\ref{chapter_perturbation}, which for purely rational intertwiners yields purely rational results.

Prop.~\ref{prop_gmc_intertwiner} yields the Gotay-Marsden coefficients from $(I,J)$ as
\begin{equation}
  \gmc{A}{B}{n}{m} = 2 I^{pn}_B J^A_{pm}.
\end{equation}
Contracting with $I$ and $J$ yields the spacetime expression
\begin{equation}
  \gmc{ab}{cd}{n}{m} = 2 \delta^{(a}_m \delta^{b)}_{(c} \delta^n_{d)},
\end{equation}
which serves as a good sanity check: Contracting again with a metric $g$ and the derivatives of a vector field $\xi$ results in the well-known transformation of $g$ w.r.t.~infinitesimal diffeomorphism generated by $\xi$,
\begin{equation}
  \gmc{ab}{cd}{n}{m} g^{cd} \xi^m_n = 2g^{n(a} \xi^{b)}_{,n}.
\end{equation}

The second ingredient is the principal polynomial of electrodynamics, which reduces to\footnote{Computing the principal polynomial may lead to a result of higher degree than \eqref{maxwell_poly}. For the second axiom of covariant constructive gravity, however, only the reduced form without repeating factors is of relevance---because the causal structure is determined by the reduced polynomial alone\cite{sergio}.} the homogeneous quadratic polynomial\cite{sergio?}
\begin{equation}\label{maxwell_poly}
  \mathcal P_\text{Maxwell}(k) = g(k,k).
\end{equation}
From this result follows the standard notion of causality in relativity: Light rays with codirection $k$ are constrained to the vanishing set $V$ and, thus, satisfy $g(k,k)=0$. The wave covectors related to massive observers lie within the hyperbolicity cone $C$, which restricts them to $g(k,k) > 0$ (admitting the \emph{mostly minus} convention $(+---)$ for the signature of the metric). For more details, we refer the reader to the theory developed in Ref.~\cite{sergio,closure paper,?} and the corresponding examples.

Before proceeding, let us emphasize that there are only two things needed from the matter theory (which is Maxwell electrodynamics in this case):
\begin{enumerate}
  \item the Gotay-Marsden coefficients $\gmc{A}{B}{n}{m} = 2 I_B^{pn} J^A_{pm}$ and
  \item the principal polynomial $\mathcal P_\text{Maxwell}(k) = g(k,k)$.
\end{enumerate}

The equivariance equations \eqref{equivariance_eqn_1}--\eqref{equivariance_eqn_4} for the metric gravitational Lagrangian are a system of 140 PDEs for one variable dependent on 154 independent variables\footnote{The dimension of the second jet bundle over the metric bundle is $4+10+4\times 10 + \binom{4}{2}\times 10=154$.}. Because the system admits the aforementioned property called involutivity, which will play a major role and therefore will actually be proven in Chap.~\ref{chapter_perturbation}, we can make use of a very strong result about the solutions of this system\cite{seiler}: There are $154-140=14$ functions $\psi_\alpha$ of the independent variables, such that any solution of the homogeneous system
\begin{equation}
  0 = A^{Ij} u_{,j}
\end{equation}
is of the form $f(\psi_1,\dots,\psi_{14})$ for any (suitably differentiable) function $f$. Any particular solution $\omega$ of the inhomogeneous system
\begin{equation}
  0 = A^{Ij} u_{,j} + B^I
\end{equation}
yields, by virtue of the product rule, the general form of a solution,
\begin{equation}\label{general_metric_solution}
  u = \omega\cdot f(\psi_1,\dots,\psi_{14}).
\end{equation}

Now, the dynamics of general relativity as derived by Einstein are given by the manifestly diffeomorphism equivariant Einstein-Hilbert Lagrangian density
\begin{equation}\label{einstein_hilbert}
  L_\text{Einstein-Hilbert} = \frac{1}{2\kappa} \sqrt{-\operatorname{det}g}(R - 2\Lambda),
\end{equation}
from which we readily recognise two solutions,
\begin{equation}
  \omega = \sqrt{-\operatorname{det}g}\quad\text{and}\quad \psi_1 = R.
\end{equation}
The constants $\kappa$ and $\Lambda$ are known as gravitational constant and cosmological constant, respectively, and $R$ is the Ricci scalar curvature. Together with the homogeneous solution $\psi_1=R$, the remaining 13 homogeneous solutions $\psi_2,\dots,\psi_{14}$ are known in the literature as the fourteen \emph{curvature invariants}\cite{curvature_invariants}.

While the system of equivariance equations alone admits a multitude of solutions \eqref{general_metric_solution}, it has been shown by Lovelock\cite{} that only Einstein general relativity \eqref{einstein_hilbert} admits second-order-derivative field equations. Step 5 of the construction algorithm therefore restricts the gravitational theory closing Maxwell electrodynamics to general relativity with its two undetermined constants \emph{exactly}. The causality conditions do not have to be implemented anymore, since they follow trivially---the causal structures of Maxwell electrodynamics and Einstein gravity coincide\cite{??}.

\section{Example: area metric gravity}
\begin{itemize}
\item introduction to area metric gravity
\item principal polynomial
\item Gotay-Marsden coefficients, equivariance equations
\end{itemize}

\section{Example: bimetric gravity}
\begin{itemize}
\item what is bimetric gravity?
\item principal polynomial(s)
\item Gotay-Marsden coefficients, equivariance equations
\end{itemize}

