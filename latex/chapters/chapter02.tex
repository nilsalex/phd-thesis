\chapter{Covariant constructive gravity}

\textbf{take-home message: two conditions on gravitational dynamics determined by coupled matter theory}

\section{Lagrangian field theory}
\begin{itemize}
\item Lagrangian as differential form on a jet bundle
\item variational problem
\item cartan form
\item noether theorem
\item energy momentum tensor
\end{itemize}

\section{Axiom I: diffeomorphism invariance}
\begin{itemize}
\item justification, history, general relativity
\item geometric formulation
\item equivariance equations
\end{itemize}

\section{Axiom II: causal compatibility}
\begin{itemize}
\item derivation (consistent co-evolution)
\item principal symbol, principal polynomial
\item geometric formulation
\end{itemize}

\section{Relation to canonical constructive gravity}
\begin{itemize}
\item relation between diffeo invariance and constraint algebra
\item the role of causality in the constraint algebra
\end{itemize}

\section{Example: Einstein gravity}
\begin{itemize}
\item Gotay-Marsden coefficients, equivariance equations
\item GR as solution
\item causality is already solved
\end{itemize}

\section{Example: area metric gravity}
\begin{itemize}
\item introduction to area metric gravity
\item principal polynomial
\item Gotay-Marsden coefficients, equivariance equations
\end{itemize}

\section{Example: bimetric gravity}
\begin{itemize}
\item what is bimetric gravity?
\item principal polynomial(s)
\item Gotay-Marsden coefficients, equivariance equations
\end{itemize}

