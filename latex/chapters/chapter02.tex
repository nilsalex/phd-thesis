\chapter{The axioms of covariant constructive gravity}\label{chapter_axioms}

\textbf{take-home message: two conditions on gravitational dynamics determined by coupled matter theory}

This chapter is concerned with the two axioms of covariant constructive gravity. Their informal formulation is as follows:
\begin{axiom}[Diffeomorphism invariance]
  The dynamical laws that govern gravity are invariant under spacetime diffeomorphisms. \cite{Alex_2020}
\end{axiom}
\begin{axiom}[Causal compatibility]
  Provided that spacetime is additionally inhabited by matter fields, their dynamics is causally compatible with the gravitational dynamics. \cite{Alex_2020}
\end{axiom}

For a more precise formulation of the axioms, which will enable us to derive their consequences for gravitational theories coupled to novel matter, an introduction of the basic concepts of Lagrangian field theory is in order.

\section{Lagrangian field theory}
For the purpose of the present work, a \emph{Lagrangian field theory} will be a geometric formulation of certain conditions on sections $\sigma\in\Gamma(\pi)$---called \emph{fields}---of some bundle $E\overset{\pi}{\longrightarrow} M$. These conditions select the physical realizations of fields admissible by the theory and constitute the \emph{dynamical laws}. The bundle $\pi$ shall be constructed from a tensor bundle, i.e.~be a sub-bundle of some tensor bundle $T^r_sM$. It is possible to extend the framework to include other bundles, with the caveat that a lift of the action of the diffeomorphism group on $M$ to $E$ may have to be specified manually. Although not relevant for much of the development of the theory, the base manifolds to be considered later for concrete examples will be spacetime manifolds of dimension 4.

\begin{example}
  Two examples for Lagrangian field theories are
  \begin{itemize}
    \item Einstein gravity on the symmetric sub-bundle of $T^2_0M$ of inverse metric tensors\footnote{The equivalent formulation as a theory for a metric tensor fields takes place on the bundle $T^0_2M$.} with dynamical laws given by the Einstein equations action, and
    \item Maxwell electrodynamics on the bundle $T^\ast M$ of potential one-forms with dynamical laws given by the Maxwell equations.
  \end{itemize}
\end{example}

Both theories are \emph{Lagrangian} because they derive their dynamical laws in a certain geometric manner. The mechanism will be explained in the following, but first, let us fix some of the notation involved.

A bundle is denoted as $E\overset{\pi}{\longrightarrow}M$, where $E$ is the total space, $M$ is the base manifold $M$, $\pi$ is the submersion. As a shorthand, it is common to write just $\pi$---it is then understood that total space and base manifold are domain and co-domain of $\pi$, respectively. The dimension of $M$ is written as $n$, the dimension of a typical fibre $F$ of $\pi$ as $m$. Coordinate functions on $E$ are denoted by $(x^i,u^A)$. Such coordinates extend to the $k$th jet bundle $J^kE \overset{\pi_k}{\longrightarrow}M$ over $\pi$ as $(x^i,\allowbreak u^A,\allowbreak u^A_{i_1},\allowbreak u^A_{i_1i_2},\allowbreak \dots,\allowbreak u^A_{i_1\dots i_k})$. The literature on jet bundles mostly employs multi-indices for higher-order jet bundles (see e.g.~Ref.~\cite{saunders}), which is certainly the right approach for studying the properties of jet bundles, but for practical calculations on the second-order jet bundle performed below the intertwiner technique (see \ref{def_intertwiner}) will be used which also takes care of ambiguities regarding symmetric indices. Prolongations of sections $\sigma$ are denoted with $j^k(\sigma)$, projections between jet bundles of different order with $\pi_{k,k^\prime}(\sigma)$. The latter are submersions in their own right and thus also define bundle projections. Throughout, the Einstein summation convention is used.

\begin{definition}[Lagrangian]
  Let $M$ be a smooth manifold of dimension $n$ and $E \overset{\pi}{\longrightarrow} M$ a smooth fibre bundle over $M$ with typical fibre $F$. A \textbf{Lagrangian} $\mathscr L$ of order $k$ is an element
  \begin{equation}
    \mathscr L \in \textstyle\bigwedge^n_0\pi_k.
  \end{equation}
In other words, $\mathscr L$ is a horizontal $n$-form on the $k$-th-order jet bundle $\pi_k$ of $\pi$.

Assuming $M$ to be orientable with volume form $\Omega\in\Lambda^nM$, a Lagrangian $\mathscr L$ is equivalently characterized by its \textbf{Lagrangian density} $L\in C^\infty(J^k\pi)$,
  \begin{equation}\label{lagrangian_density}
  \mathscr L = L\pi_k^\ast\Omega.
\end{equation}
\end{definition}

The claim of \eqref{lagrangian_density} becomes apparent in local coordinates, where a horizontal $n$-form on $\pi_k$ appears as
\begin{equation}\label{lagrangian_local}
  \textstyle\bigwedge^n_0\pi_k\ni \mathscr L = \mathscr L(x^i,u^A,u^A_i,\dots) dx^1\wedge \dots \wedge dx^n.
\end{equation}
From \eqref{lagrangian_local} and \eqref{lagrangian_density}, it is also clear how the notion of a Lagrangian as a horizontal $n$-form captures in a geometric way the notion used elsewhere as a bundle map $J^kE \rightarrow \Lambda^nM$ (see Refs.~\cite{saunders,Gotay_1992,Alex_2020}).


From now on, we will consider smooth, orientable base manifolds $M$ and smooth bundles over the base manifold. Depending on the context, the symbol $\Omega$ will denote either the form on $M$ or the pullback to various bundles over $M$.

\begin{definition}{(local action functional)}
  Given a Lagrangian $\mathscr L \in \bigwedge^n_0\pi_k$ and a compact $n$-dimensional submanifold $C \subset M$, the \textbf{local action functional} is defined as the map
  \begin{equation}
    \sigma \mapsto S\lbrack\sigma\rbrack = \int_C (j^k\sigma)^\ast\mathscr L
  \end{equation}
for all local sections $\sigma$ of $\pi$ with support on $C$.
\end{definition}

Lagrangian field theory now stipulates that sections $\sigma\in\Gamma(\pi)$ are physical if they are extremals of the action functional\cite{}. The well-known \emph{Euler-Lagrange} equations from the calculus of variations provide a necessary condition in local coordinates which such sections extremals must satisfy.
\begin{proposition}
  Let $L\in C^\infty(J^k\pi)$ be a Lagrangian density. Let $C$ be a compact submanifold of $M$ and $\sigma$ be a local section of $\pi$ such that the local action functional $S\lbrack\sigma\rbrack$ is defined. If $\sigma$ is an extremal of $S$, it satisfies the \textbf{Euler-Lagrange equations}
  \begin{equation}\label{euler_lagrange_local}
    (j^{2k}\sigma)^\ast\left( \sum_{l=0}^{k} (-1)^l D_{i_1} \cdots D_{i_l} \frac{\partial L}{\partial u^A_{i_1\cdots i_l}}\right) = 0.
  \end{equation}
\end{proposition}
\begin{proof}
  See Saunders 1989 or older, original work
\end{proof}

The intrinsic equivalent to the Euler-Lagrange equations in local coordinates introduces a new object, the \emph{Cartan form}, which plays a central role in the geometrization of Lagrangian field theory.
\begin{proposition}
  Given a Lagrangian $\mathscr L = L\Omega \in \bigwedge^n_0\pi_k$, there exists an $n+1$ form $\Theta_L\in\bigwedge^{n}_0\pi_{2k-1,k-1}\cap\bigwedge^{n}_{1}\pi_{2k-1}$, such that, globally, the variation of the Lagrangian is given by
  \begin{equation}\label{global_lagrange_form}
    \delta L = \pi^\ast_{2k,k} \left( dL\wedge\Omega\right) + d_h \Theta_L
  \end{equation}
  and extremals of $\mathscr L$ are extremals of $\Theta_L$ in the sense that
  \begin{equation}
    \pi_{2k-1,k}^\ast (j^k\sigma)^\ast\mathscr L = (j^{2k-1}\sigma)^\ast\Theta_L.
  \end{equation}
  Such a form $\Theta_L$ is called a \textbf{Cartan form}.
\end{proposition}
\begin{proof}
  See Saunders 1989 or older, original work
\end{proof}

This definition generalizes the local derivation of \eqref{euler_lagrange_local}: The variation $\delta L$ is obtained by lifting $d\mathscr L$ to $\pi_{2k}$ and cancelling non-horizontal terms (over $E$) by adding a derivative, which corresponds to repeated integrations by parts.

A possible\footnote{Generally, $\Theta_L$ is not uniquely defined. \cite{saunders}} coordinate expression for $\Theta_L$ is \ref{}
\begin{equation}\label{local_cartan_form}
  \Theta_L = L\Omega + \sum_{s=0}^{k-1} \sum_{l=0}^{k-s-1} (-1)^l D_{i_1} \cdots D_{i_l} \left(\frac{\partial L}{\partial u^A_{j i_1 \dots i_{l} p_1 \dots p_s}}\right) \psi^A_{p_1\dots p_s} \wedge \left(i_{\partial_j}\Omega\right),
\end{equation}
where the forms $\psi^A_{p_1\dots p_s} = du^A_{p_1\dots p_s} - u^A_{p_1\dots p_sq} dx^q$ span the contact system of $\pi_k$ (see \ref{}).

Straight-forward application of \eqref{global_lagrange_form} to \eqref{local_cartan_form} yields the well-known coordinate expression
\begin{equation}
  \delta L = \left( \sum_{l=0}^{k} (-1)^l D_{i_1} \cdots D_{i_l} \frac{\partial L}{\partial u^A_{i_1\cdots i_l}}\right) u^A \wedge \Omega
\end{equation}
for the Euler-Lagrange form, reconciling the intrinsic formulation using the Cartan form with the explicitly coordinate-dependent\footnote{Which is not to say \emph{ill-defined}.} formulation using the Euler-Lagrange equations.

In later sections, we will restrict our attention to Lagrangians of second derivative order. As it turns out, the Cartan form for such a theory is \emph{unique}.
\begin{proposition}\label{prop_cartan_unique}
  The Cartan form is unique for second-order Lagrangians.
\end{proposition}
\begin{proof}
  See Saunders 1989 or older, original work
\end{proof}

\section{Axiom I: diffeomorphism invariance}

In the language of jet bundles, the first axiom can be formalized as equivariance condition under a certain group action on the Lagrangian. The group in question is the diffeomorphism group $\mathrm{Diff}(M)$, acting on $M$ by function application. By virtue of the pushforward-pullback construction, sub-bundles of tensor bundles carry a canonical action of $\mathrm{Diff}(M)$ as bundle automorphisms, denoted as $\varphi_E\in \mathrm{Aut}(E)$ for every $\varphi\in\mathrm{Diff}(M)$. We call this the \emph{lift} of the diffeomorphism $\varphi$. This action, in turn, lifts naturally to the jet bundles over $E$.

\begin{definition}[prolongation of morphisms]
  Let $E \overset{\pi_E}{\longrightarrow} M$ and $H \overset{\pi_H}{\longrightarrow} N$ be two bundles. The $k$th-order jet bundle lift of a bundle morphism $(F,f)$ from $\pi_E$ to $\pi_H$ is the unique bundle morphism $(j^k(F),f)$ from $J^k\pi_E$ to $J^k\pi_H$ that lets the diagram in Fig.~(??) commute.
\end{definition}
A proof for the existence and uniqueness of this construction can be found in Ref.~\cite{saunders}. With the notion of the \emph{lift} of a bundle automorphism at hand, we now give the first axiom a precise meaning
\begin{definition}[diffeomorphism invariant theory]
  A Lagrangian field theory is called \textbf{diffeomorphism invariant} if its Lagrangian $\mathscr L\in\textstyle\bigwedge_0^n\pi_k$ is invariant with respect to the lifted action of $\mathrm{Diff}(M)$ on $J^kE$, i.e.~if for all $\varphi\in\mathrm{Diff}(M)$
  \begin{equation}\label{lagrangian_diffeo_invariance}
    j^k(\varphi_E)^\ast \mathscr L = \mathscr L.
  \end{equation}
\end{definition}
This definition applies not only to tensor bundles and the diffeomorphism group---all we need is a well-defined action as bundle automorphism. For tensor bundles, however, there is always the canonical action built from the pushforward action on tangent vectors
\begin{equation}
  \varphi_\ast\colon T_pM\rightarrow T_{\varphi(p)}M
\end{equation}
and the pullback action on cotangent vectors\footnote{The action is inverted in order to still define a covariant functor. \textbf{explain!}\cite{}}
\begin{equation}
   (\varphi^\ast)^{-1}\colon T^\ast_pM\rightarrow T^\ast_{\varphi(p)}M.
\end{equation}
Using a coordinate $(U,x)$ containing $p$ and $(V,y)$ containing $\varphi(p)$, the bundle automorphisms act on coordinate-induced component functions of vector fields as
\begin{equation}\label{diffeo_vector_coordinates}
(\varphi_\ast X)_{(y)}^j(y(p)) = \frac{\partial (y^j\circ \varphi)}{\partial x^i} \Bigg\rvert_{\varphi^{-1}(p)} \cdot X_{(x)}^i(x(\varphi^{-1}(p)))
\end{equation}
and on component functions of covector fields as\footnote{As for any group homomorphism, we have $(\varphi^\ast)^{-1}=(\varphi^{-1})^\ast$.}
\begin{equation}\label{diffeo_covector_coordinates}
((\varphi^{-1})^\ast\omega)_{(y)j}(y(p)) = \frac{\partial (x^i\circ \varphi^{-1})}{\partial y^j}\Bigg\rvert_p \cdot \omega_{(x)i}(x(\varphi^{-1}(p))).
\end{equation}

We wish to encode \eqref{lagrangian_diffeo_invariance} as local conditions on the Lagrangian density. To this end, consider a coordinate representation $\mathscr L = Ld^nx$. 
\begin{proposition}
  Let $\mathscr L = L_{(x)}d^nx$ be a coordinate representation of a diffeomorphism invariant Lagrangian, induced by a coordinate chart $(U,x)$ on $M$. It follows from the invariance condition \eqref{lagrangian_diffeo_invariance} that $L_{(x)}$ is \textbf{diffeomorphism equivariant}, i.e.~it holds for all $\varphi\in\mathrm{Diff}(M)$ that, over the intersection of $U$ and $\varphi(U)$,
  \begin{equation}\label{lagrangian_density_diffeo_equivariance}
    L_{(x)}\circ j^k_{(x)}(\varphi_E) = \lvert d\varphi_{(x)}\rvert^{-1} L_{(x)}.
  \end{equation}
  $\lvert d\varphi_{(x)}\rvert$ denotes the determinant of the Jacobian of $\varphi$ in terms of the coordinate chart $(U,x)$.
\end{proposition}
\begin{proof}
  The result follows from the coordinate expression \eqref{diffeo_covector_coordinates} for the pullback of one forms, which extends to horizontal forms on the jet bundle.
\end{proof}

Covariant constructive gravity derives its calculational power from the observation that the infinitesimal version of \eqref{lagrangian_density_diffeo_equivariance} is equivalent to a system of linear partial differential equations (PDEs) for the Lagrangian density $L$. For the derivation of this theorem and the remainder of the chapter, we will work on the second jet bundle, as we are ultimately interested in investigating field theories of second derivative order. We will also drop the chart label from coordinate-dependent quantities. In order to lighten the notation, partial derivatives of functions on $J^2M$ are denoted $L_{,m}$ for derivatives with respect to coordinates on $M$ and $L_{:A},\,L_{:A}^{\hphantom{:A}p},\,L_{:A}^{\hphantom{:A}pq}$ for derivatives with respect to fibre coordinates.

Since the bundles in question are (sub-bundles of) tensor bundles $T^m_nM$, it is possible to restrict to coordinates which are linear on the fibres. The \emph{intertwiner technique}\footnote{First described for a similar setting in Ref.~\cite{david_bachelor}, later introduced in the context of covariant constructive gravity\cite{Alex_2020}} relates such coordinates to coordinates on $T^m_nM$ itself by virtue of two special bundle morphisms.
\begin{definition}[intertwiners]
  Let $E\overset{\pi}{\longrightarrow}M$ be a sub-bundle of $T^m_nM$. A pair of vector bundle morphisms $(I,J)$,
  \begin{equation}
    \begin{aligned}
      I&{} \colon E\rightarrow T^m_nM, \\
      J&{} \colon T^m_nM\rightarrow E,
    \end{aligned}
  \end{equation}
  which cover $\mathrm{id}_M$ and satisfy $J\circ I = \mathrm{id}_E$ is called a pair of \textbf{intertwiners} for $\pi$.
\end{definition}
It follows from the property $J\circ I=\mathrm{id}_E$ that $J$ is a \emph{surjection} and $I$ is an \emph{injection}. Expressed in adapted coordinates, it is clear how the coordinate representations of $I$ and $J$ relate fibre coordinates to each other,
\begin{equation}
  \begin{aligned}
  u^{a_1\dots a_m}_{b_1\dots b_n} &= I^{a_1\dots a_m}_{b_1\dots b_nA} \cdot u^A, \\
  u^A &= J^{Ab_1\dots b_n}_{a_1\dots a_m} \cdot u^{a_1\dots a_m}_{b_1\dots b_n}, \\
  \delta^B_A &{} = I^{a_1\dots a_m}_{b_1\dots b_nA} \cdot J^{Ab_1\dots b_n}_{a_1\dots a_m}.
  \end{aligned}
\end{equation}
Concrete implementations of $I$ and $J$ will be introduced in Sect.~\ref{}. Intertwiners for the symmetric sub-bundle of $T^0_2M$ are used to deduplicate second-order derivative indices by defining
\begin{equation}
  \begin{aligned}
    u^A_{I} &{} = J^{ij}_I u^A_{ij}, \\
    u^A_{ij} &{} = I^I_{ij} u^A_I.
  \end{aligned}
\end{equation}

Proceeding to derive the infinetismal version of axiom 1, we first need to specify what is meant by \emph{infinitesimal}. As the symmetry group in question is the diffeomorphism group on the base manifold, the infinitesimal equivalent is the corresponding Lie algebra $\Gamma(TM)$ of sections of the tangent bundle over $M$.\footnote{not a Lie algebra, etc.~\cite{}} The Lie bracket is given by the Lie bracket of vector fields. In a given coordinate chart, an element $\xi\in\Gamma(TM)$ defines an infinitesimal diffeomorphism as
\begin{equation}\label{diffeo_infinitesimal}
  x^i \mapsto x^i + \xi^i.
\end{equation}
From \eqref{diffeo_vector_coordinates} and \eqref{diffeo_covector_coordinates}, we know how an infinitesimal diffeomorphism \eqref{diffeo_infinitesimal} acts on vectors and covectors. Dropping chart labels because everything takes place in the chart $(U,x)$, the actions are given by
\begin{equation}
  X^i \mapsto X^i + X^j \xi^i_{,j}
\end{equation}
and
\begin{equation}
  \omega_i \mapsto \omega_i - \omega_j \xi^j_{,i}.
\end{equation}
On higher-ranked tensor bundles, the action generalizes to
\begin{equation}
  T^A \mapsto T^A + C^{A\hphantom Bn}_{\hphantom AB\hphantom nm}T^B \xi^m_{,n},
\end{equation}
with $C^{i\hphantom jn}_{\hphantom ij\hphantom nm} = \delta^i_m \delta^n_j$ for the special case of vectors and $C_{i\hphantom{jn}m}^{\hphantom ijn}=-\delta^n_i\delta^j_m$ for covectors. From this, we can read off the Lie algebra morphism which maps vector fields on $M$ to vector fields on $E$ and is induced by the group homomorphism from $\operatorname{Diff}(M)$ to $\operatorname{Aut}(E)$:
\begin{equation}\label{algebra_morphism}
  \begin{aligned}
    \Gamma(TM) &{} \rightarrow \Gamma(TE) \\
    \xi &{} \mapsto \xi_E \vcentcolon= \xi^m\partial_m + C^{A\hphantom Bn}_{\hphantom AB\hphantom nm}u^B \xi^m_{,n} \partial_A.
  \end{aligned}
\end{equation}
The constant coefficients $C^{A\hphantom Bn}_{\hphantom AB\hphantom nm}$ will be called \emph{Gotay-Marsden coefficients} after the authors of Ref.~\cite{Gotay_1992}, where this formalism is developed in a more general setting---but only for the first jet bundle. In the language of this reference, a tensor field theory is of \emph{differential index} 1.

As it turns out, the map $\xi\mapsto \xi_E$ indeed defines a homomorphism between Lie algebras.
\begin{proposition}
  The map \eqref{algebra_morphism} is a Lie algebra homomorphism, i.e.~it holds for all $\xi,\psi\in\Gamma(TM)$ that
  \begin{equation}\label{lie_algebra_homo}
    \lbrack \xi_E,\psi_E\rbrack = \lbrack \xi,\psi\rbrack_E.
  \end{equation}
\end{proposition}
\begin{proof}
  To be worked out. Idea: Arises from the definition via the group homomorphism from $\operatorname{Diff}(M)$ to $\operatorname{Aut}(E)$. Caution: The latter is not a finite dimensional Lie group.
\end{proof}
A useful corollary of the homomorphism property of the Lie algebra lift $\xi\mapsto\xi_E$, which will play a role in Sect.~\ref{}, is the following fact about Gotay-Marsden coefficients.
\begin{corollary}
  The Gotay-Marsden coefficients \gmc{A}{B}{n}{m} corresponding to a tensor field theory satisfy the relation
  \begin{equation}
    \gmc{A}{B}{n}{m} \gmc{B}{C}{q}{p} - \gmc{A}{B}{q}{p} \gmc{B}{C}{n}{m} = \gmc{A}{B}{q}{m} - \gmc{A}{B}{n}{p}.
  \end{equation}
\end{corollary}
\begin{proof}
  Expanding \eqref{lie_algebra_homo} and making use of the coordinate expression \eqref{algebra_morphism} yields the identity
  \begin{equation}
    \left\lbrack\gmc{A}{B}{n}{m} \gmc{B}{C}{q}{p} - \gmc{A}{B}{q}{p} \gmc{B}{C}{n}{m} - (\gmc{A}{B}{q}{m} - \gmc{A}{B}{n}{p})\right\rbrack \xi^m_{,n} \psi^p_{,q} = 0,
  \end{equation}
  from which the result follows, as $\xi$ and $\psi$ can be chosen arbitrarily.
\end{proof}

If required for calculations, Gotay-Marsden coefficients are easily expressed using intertwiners.
\begin{proposition}\label{prop_gmc_intertwiner}
  Let $E\overset{\pi}{\longrightarrow} M$ be a sub-bundle of the tensor bundle $T^r_sM$ with a pair $(I,J)$ of intertwiners. If the tensors in $E$ are purely contravariant, i.e.~$s=0$, the Gotay-Marsden coefficients are
  \begin{equation}\label{gmc_contra}
    \gmc{A}{B}{n}{m} = r \cdot I^{a_1\dots a_{r-1}n}_B J^A_{a_1\dots a_{r-1}m}.
  \end{equation}
  If the tensors in $E$ are purely covariant, i.e.~$r=0$, the Gotay-Marsden coefficients are
  \begin{equation}
    \gmcd{A}{B}{n}{m} = -s \cdot I^B_{a_1\dots a_{s-1}m} J^{a_1\dots a_{s-1}n}_A.
  \end{equation}
  Note that in the latter case of a purely covariant tensor bundle a fibre coordinate function is denoted $u_A$ with a lower index.
\end{proposition}

The Gotay-Marsden coefficients are the defining objects for the PDE version of the invariance of $L$ under infinitesimal diffeomorphisms. This constitutes the central result concerning the first axiom and shall be proved in the following.

\begin{theorem}\label{equivariance_eqns_thm}
  Let $L$ be the local Lagrangian density of a second derivative order Lagrangian $\mathscr L = Ld^4x \in \textstyle\bigwedge_0^n\pi_2$. If $\mathscr L$ is diffeomorphism invariant, i.e.~satisfies the invariance condition \eqref{lagrangian_diffeo_invariance} of the first axiom of covariant constructive gravity, its local representation $L$ satisfies a system of first-order linear partial differential equations given by
  \begin{subequations}\label{equivariance_eqns}
    \begin{align}
      0 &{} = L_{,m} \label{equivariance_eqn_1}\\
      0 &{} = L_{:A} \gmc{A}{B}{n}{m}u^B + L_{:A}^{\hphantom{:A}p}\left\lbrack\gmc{A}{B}{n}{m} \delta^q_p - \delta^A_B\delta^q_m\delta^n_p\right\rbrack u^{B}_{\hphantom Bq} \nonumber \\
        &{} \hphantom{=} + L_{:A}^{\hphantom{:A}I} \left\lbrack\gmc{A}{B}{n}{m} \delta^J_I - 2 \delta^A_B J^{pn}_{I} I^J_{pm}\right\rbrack u^B_{\hphantom BJ} + L\delta^n_m \label{equivariance_eqn_2}\\
      0 &{} = L_{:A}^{\hphantom{:A}(p\mid} \gmc{A}{B}{\mid n)}{m} u^B + L_{:A}^{\hphantom{:A}I}\left\lbrack\gmc{A}{B}{(n}{m} 2J^{p)q}_I - \delta^A_B J^{pn}_I \delta^q_m\right\rbrack u^B_{\hphantom Bq} \label{equivariance_eqn_3}\\
      0 &{} = L_{:A}^{\hphantom{:A}I} \gmc{A}{B}{(n}{m} J^{pq)}_I u^B.\label{equivariance_eqn_4}
    \end{align}
  \end{subequations}
\end{theorem}
\begin{proof}
  Given a vector field $X$ on $E$, the lift to the total space $J^2E$ of the second jet bundle is uniquely defined\cite{saunders_jet_bundles}. Applying this lift to the vector field $\xi_E$ corresponding to $\xi\in \Gamma(TM)$ yields the vector field
  \begin{equation}\label{vector_field_2nd_prolongation}
    \begin{aligned}
      \xi_{J^2E} &{} \vcentcolon= \xi^m\partial_m \\
                 &{} \hphantom{\vcentcolon=} + \gmc{A}{B}{n}{m} u^B \xi^m_{,n}\partial_A + \gmc{A}{B}{n}{m} u^B_p\xi^m_{,n}\partial_A^{\hphantom Ap} - u^A_m\xi^m_{,p}\partial_A^{\hphantom Ap} \\
                 &{} \hphantom{\vcentcolon=} + \gmc{A}{B}{n}{m} u^B_I \xi^m_{,n} \partial_A^{\hphantom AI} - 2 J_I^{nr}I^J_{mr}u^A_J\xi^m_{,n}\partial_A^{\hphantom AI} \\
                 &{} \hphantom{\vcentcolon=} + \gmc{A}{B}{n}{m} u^B \xi^m_{,np} \partial_A^{\hphantom Ap} + 2 \gmc{A}{B}{n}{m} J_I^{pq} u^B_p \xi^m_{,nq} \partial_A^{\hphantom AI} - J_I^{pq} u^A_m \xi^m_{,pq} \partial_A^{\hphantom AI} \\
                 &{} \hphantom{\vcentcolon=} + \gmc{A}{B}{n}{m} J_I^{pq} u^B \xi^m_{,npq} \partial_A^{\hphantom AI}.
    \end{aligned}
  \end{equation}
  Like before, the map $\xi\mapsto\xi_{J^2E}$ constitutes a Lie algebra morphism from $\Gamma(TM)$ to $\Gamma(\pi_2)$. \textbf{proof/citation needed}

  Assuming that \eqref{lagrangian_density_diffeo_equivariance} holds, we obtain the infinitesimal version by acting on $L$ with $\xi_{J^2E}$ for the left-hand side and approximating $\lvert d\varphi\rvert^{-1}$ as $1 - \xi^m_{,m}$ for the right-hand side. Equating both sides yields
  \begin{equation}\label{infinitesimal_with_vector}
    \begin{aligned}
      0 &{} = L_{,m} \xi^m \\
        &{} \hphantom{=} + \big\{ L_{:A} \gmc{A}{B}{n}{m}u^B + L_{:A}^{\hphantom{:A}p}\left\lbrack\gmc{A}{B}{n}{m} \delta^q_p - \delta^A_B\delta^q_m\delta^n_p\right\rbrack u^{B}_{\hphantom Bq} \\
        &{} \hphantom{=+\{} + L_{:A}^{\hphantom{:A}I} \left\lbrack\gmc{A}{B}{n}{m} \delta^J_I - 2 \delta^A_B J^{pn}_{I} I^J_{pm}\right\rbrack u^B_{\hphantom BJ} + L\delta^n_m \big\} \xi^m_{,n} \\
        &{} \hphantom{=} + \big\{ L_{:A}^{\hphantom{:A}p} \gmc{A}{B}{n}{m} u^B + L_{:A}^{\hphantom{:A}I}\left\lbrack\gmc{A}{B}{n}{m} 2J^{pq}_I - \delta^A_B J^{pn}_I \delta^q_m\right\rbrack u^B_{\hphantom Bq} \big\} \xi^m_{,np} \\
        &{} \hphantom{=} + L_{:A}^{\hphantom{:A}I} \gmc{A}{B}{n}{m} J^{pq}_I u^B \xi^m_{,npq}.
    \end{aligned}
  \end{equation}
Since \eqref{infinitesimal_with_vector} holds for any $\xi\in\Gamma(TM)$, the individual contributions for $\xi$, $\partial\xi$, $\partial\partial\xi$, $\partial\partial\partial\xi$ are satisfied separately.
\end{proof}

On a four-dimensional spacetime manifold, Thm.~\ref{equivariance_eqns_thm} yields a system of 140 linear PDEs of first order for the Lagrangian density $L$. Any diffeomorphism invariant tensor field theory of second derivative order must satisfy this system and, conversely, any solution to the system provides a candidate for a diffeomorphism invariant theory. Thus, the search for such theories has been reduced to the mathematical task of solving PDEs of a certain (simple!) form. The only ingredients which depend on the specific theory at hand are the Gotay-Marsden coefficients, such that it is possible to derive certain properties of the system without knowlegde of the concrete tensor bundle.

The literature on this kind of PDEs is very extensive\cite{} with many applications throughout science\cite{}. There are strong results on the properties and solutions which provide a good basis for our work with Eqns.~\ref{equivariance_eqn_1}--\ref{equivariance_eqn_4} in the following.

The PDE system will be referred to as \emph{equivariance equations} from now on. In a similar form, these equations already appear in Ref.~\cite{Gotay_1992} during the derivation of conservation laws arising from diffeomorphism invariance. As they were not meant to be solved for the Lagrangian density, the presentation is not as explicit as here. Also note that Ref.~\cite{Gotay_1992} considers theories of arbitrary differential index but only the first jet bundle, whereas the present derivation takes place on the second jet bundle but is restricted to a differential index of 1, i.e.~tensor field theories of second derivative order. The extension to theories with arbitrary differential index is possible---there will be a series of Gotay-Marsden coefficients
\begin{equation}
  \xi_E = \xi^m\partial_m + C^A_{\hphantom ABm} \xi^m + C^{A\hphantom Bn}_{\hphantom AB\hphantom nm} \xi^m_{,n} + C^{A\hphantom Bnp}_{\hphantom AB\hphantom {np}m} \xi^m_{,np} + \dots
\end{equation}
which follow from the action of the diffeomorphism group on the bundle. \cite{Gotay_1992}

\section{Noether theorems}
Diffeomorphism invariance of the Lagrangian as required by the first axiom results in a number of interesting properties of the theory. Among these are identities for the Euler-Lagrange equations and conservation laws for the dynamics given by the Euler-Lagrange equations, which are examples for the well-known Noether theorems\cite{}. In analogy to the derivation for theories of first derivative order in Ref.~\cite{Gotay_1992}, we shall now prove a version of the Noether theorems for the second-order formalism developed above.

The first step is to realize that the Cartan form for a diffeomorphism invariant Lagrangian is itself diffeomorphism invariant.
\begin{proposition}[\cite{}]\label{prop_cartan_diffeo}
  Let $\mathscr L$ be a diffeomorphism invariant Lagrangian, i.e.~$j^k(\varphi_E)^\ast \mathscr L=\mathscr L$. Any corresponding Cartan form $\Theta_L$ satisfies the diffeomorphism invariance condition
  \begin{equation}
    j^{2k-1}(\varphi_E)^\ast \Theta_L = \Theta_L.
  \end{equation}
\end{proposition}
\begin{proof}
  Find one in the literature/derive my own version
\end{proof}
Using the infinitesimal version $\mathcal L_{\xi_{J^{2k-1}E}} \Theta_L = 0$ of the diffeomorphism invariance of $\Theta_L$, the first Noether theorem follows as a direct consequence.
\begin{theorem}[First Noether theorem\cite{}]\label{thm_first_noether}
  Let $\mathscr L$ be a diffeomorphism invariant Lagrangian and $\Theta_L$ a corresponding Cartan form. For any lifted generator $\xi_{J^{2k-1}E}$ of the diffeomorphism action and any section $\sigma\in\Gamma(\pi)$ satisfying the Euler-Lagrange equations $j^{2k}(\sigma)^\ast(\delta L)=0$, it follows that the current defined as
  \begin{equation}
    \boldsymbol{j}(\sigma) = (j^{2k-1}\sigma)^\ast \iota_{\xi_{J^{2k-1}E}} \Theta_L
  \end{equation}
  is a closed differential form, i.e.~
  \begin{equation}\label{first_noether_eq}
    0 = d \boldsymbol{j}(\sigma).
  \end{equation}
\end{theorem}
\begin{proof}
  Applying the Cartan formula to the infinitesimal diffeomorphism invariance condition for $\Theta_L$ (which follows from Prop.~\ref{prop_cartan_diffeo}) gives
  \begin{equation}
    0 = \mathcal L_{\xi_{J^{2k-1}E}} \Theta_L = d\,\iota_{\xi_{J^{2k-1}E}} \Theta_L + \iota_{\xi_{J^{2k-1}E}} d \Theta_L
  \end{equation}
  such that
  \begin{equation}
    \begin{aligned}
      d \left((j^{2k-1}\sigma)^\ast \iota_{\xi_{J^{2k-1}E}} \Theta_L \right) &{} = (j^{2k-1}\sigma)^\ast ( d\,\iota_{\xi_{J^{2k-1}E}} \Theta_L ) \\
                                                                             &{} = -(j^{2k-1}\sigma)^\ast (\iota_{\xi_{J^{2k-1}E}} d \Theta_L ).
    \end{aligned}
  \end{equation}
  One of the defining properties of the Cartan form $\Theta_L$ is that extremals of $\mathscr L$ are extremals of $\Theta_L$. Because $\sigma$, satisfying the Euler-Lagrange equations, is an extremal of $\mathscr L$, it also satisfies the condition\cite{}
  \begin{equation}
    0 = (j^{2k-1}\sigma)^\ast (\iota_\Xi d \Theta_L)
  \end{equation}
  for extremals of $\Theta_L$. The condition holds for arbitrary vector fields $\Xi$ on $J^{2k-1}E$, including the vector fields $\xi_{J^{2k-1}E}$.
\end{proof}

Eq.~\ref{first_noether_eq} defines a current which is conserved \emph{on shell}, i.e.~whenever the Euler-Lagrange equations hold. Thm.~3.1 of Ref.~\cite{Gotay_1992} already shows---for theories defined on the first jet bundle---how the current arises from the so-called \emph{stress-energy-momentum} tensor. This result can now be generalized to tensor field theories of second derivative order.
\begin{theorem}[Gotay-Marsden stress-energy-momentum tensor]\label{sem_thm}
  Let $\Theta_L$ be the diffeomorphism invariant Cartan form corresponding to a diffeomorphism invariant Lagrangian of second derivative order ($k=2$). For any local section $\sigma$ of the underlying bundle, there exists a unique $(1,1)$-tensor density $\mathcal T(\sigma)$ on the base manifold $M$ such that for all vector fields $\xi$ with compact support on $M$ and embedded hypersurfaces $i_\Sigma \colon \Sigma \rightarrow M$
  \begin{equation}\label{sem_definition}
    \int_\Sigma i_\Sigma^\ast\boldsymbol{j}(\sigma) = \int_\Sigma \mathcal T^{\,n}_{\,m}(\sigma)\xi^m \omega_n.
  \end{equation}
  The tensor density $\mathcal T(\sigma)$ is called the \textbf{Gotay-Marsden stress-energy-momentum (SEM) tensor}.
\end{theorem}
\begin{proof}
  With the Cartan form for a second-derivative-order theory being uniquely defined (see Prop.~\ref{prop_cartan_unique}), there is always the coordinate expression \eqref{local_cartan_form}. Setting $k=2$ and making use of intertwiners for second-derivative indices, this expression reads
  \begin{equation}
    \begin{aligned}
      \Theta_L &{} = L\Omega \\
               &{} \hphantom{=} + \frac{\partial L}{\partial u^A_j}(du^A - u^A_q dx^q)\wedge\omega_j - D_i \frac{\partial L}{\partial u^A_I} J_I^{ji} (du^A-u^A_qdx^q)\wedge\omega_j \\
               &{} \hphantom{=} + \frac{\partial L}{\partial u^A_I} J_I^{jp} (du^A_p - u^A_J I_{pq}^J dx^q)\wedge\omega_j.
  \end{aligned}
  \end{equation}
  Note that, because the Cartan form is horizontal over the first jet bundle, there is no appearance of the forms $du^A_{ij}$ and $du^A_{ijk}$ in the coordinate expression for $\Theta_L$. Thus, the pairing with $\xi_{J^3E}$ for the calculation of $\boldsymbol{j}(\sigma)$ makes use only of the coefficients $\xi^m$, $\xi^A$, and $\xi^A_i$. Performing the pairing and the subsequent pullback with respect to the prolongation of $\sigma$, the current is obtained as
  \begin{equation}
    \begin{aligned}
      \boldsymbol{j}(\sigma) &{} = L \xi^j \omega_j \\
                             &{} \hphantom{=} + L_{:A}^{\hphantom{:A}j} (\xi^A - \sigma^A_{,q} \xi^q) \omega_j - D_i L_{:A}^{\hphantom{:A}I} J_I^{ji} (\xi^A - \sigma^A_{,q} \xi^q) \omega_j \\
                             &{} \hphantom{=} + L_{:A}^{\hphantom{:A}I} J_I^{jp} (\xi^A_p - \sigma^A_{,J}I^J_{pq}\xi^q) \omega_j,
    \end{aligned}
  \end{equation}
  where $L$ and its derivatives are to be understood as being evaluated at prolongations of the section $\sigma$.

  Using $\xi^A = \gmc{A}{B}{n}{m}u^B\xi^m_{,n}$ and $\xi^A_p = D_p(\gmc{A}{B}{n}{m}u^B\xi^m_{,n})-u^A_m\xi^m_{,p}$ from Eq.~\ref{vector_field_2nd_prolongation} yields the current in its expanded form, which is
  \begin{equation}\label{sem_calculation_current}
    \begin{aligned}
      \boldsymbol{j}(\sigma) &{} = \big\lbrack L\delta^n_m - L_{:A}^{\hphantom{:A}n}\sigma^A_{,m} + D_iL_{:A}^{\hphantom{:A}I}J_I^{in}\sigma^A_{,m}-L_{:A}^{\hphantom{:A}I}J_I^{np}I^J_{mp}\sigma^A_{,J}\big\rbrack \xi^m \omega_n \\
                             &{} \hphantom{=} + \big\lbrack \gmc{A}{B}{n}{m}(L_{:A}^{\hphantom{:A}j}\sigma^B - D_iL_{:A}^{\hphantom{:A}I}J_{I}^{ij}\sigma^B + L_{:A}^{\hphantom{:A}^I}J_{I}^{jp}\sigma^B_{,p}) - L_{:A}^{\hphantom{:A}I} J_I^{jn} \sigma^A_{,m}\big\rbrack \xi^m_{,n}\omega_j \\
                             &{} \hphantom{=} + \big\lbrack L_{:A}^{\hphantom{:A}I} J_I^{jp} \gmc{A}{B}{n}{m}\sigma^B\big\rbrack \xi^m_{,np}\omega_j.
    \end{aligned}
  \end{equation}
  The key to proving the identity \eqref{sem_definition} is to express the integrals with contributions from $\xi^m_{,n}$ and $\xi^m_{,np}$ as volume integrals using Gauss's theorem and to then repeatedly simplify the integrand by employing the diffeomorphism equivariance equations \eqref{equivariance_eqns} and a variant of integration by parts, in analogy to the operations performed in Ref.~\cite{Gotay_1992} for first-order theories.

  Applying this procedure to the terms containing two derivatives of $\xi$ eliminates these terms, at the cost of new terms containing lower derivatives:
  \begin{equation}\label{sem_calculation_2nd_deriv}
    \begin{multlined}
      \int_\Sigma L_{:A}^{\hphantom{:A}I} J_I^{jp} \gmc{A}{B}{n}{m}\sigma^B \xi^m_{,np}\omega_j \\
      \begin{aligned}
          &{} = \int_V D_j \big\lbrack L_{:A}^{\hphantom{:A}I} J_I^{jp} \gmc{A}{B}{n}{m}\sigma^B \xi^m_{,np}\big\rbrack \omega \\
          &{} = \int_V \big\{ D_j \big\lbrack L_{:A}^{\hphantom{:A}I} J_I^{jp} \gmc{A}{B}{n}{m}\sigma^B\big\rbrack \xi^m_{,np} + \underbrace{L_{:A}^{\hphantom{:A}I} J_I^{jp} \gmc{A}{B}{n}{m}\sigma^B \xi^m_{,npj}}_{\mathrlap{\!=0\ \eqref{equivariance_eqn_4}}} \big\}\omega \\
          &{} = \int_V \big\{ D_p \big( D_j \big\lbrack L_{:A}^{\hphantom{:A}I} J_I^{jp} \gmc{A}{B}{n}{m}\sigma^B\big\rbrack \xi^m_{,n} \big) - D_p D_j \big\lbrack L_{:A}^{\hphantom{:A}I} J_I^{jp} \gmc{A}{B}{n}{m}\sigma^B\big\rbrack \xi^m_{,n}\big\}\omega \\
          &{} = \int_V \big\{ D_p \big( D_j \big\lbrack L_{:A}^{\hphantom{:A}I} J_I^{jp} \gmc{A}{B}{n}{m}\sigma^B\big\rbrack \xi^m_{,n} \big) - D_n\big(D_p D_j \big\lbrack L_{:A}^{\hphantom{:A}I} J_I^{jp} \gmc{A}{B}{n}{m}\sigma^B\big\rbrack \xi^m\big) \\
          &{} \hphantom{= \int_V \big\{} + \underbrace{D_n D_p D_j \big\lbrack L_{:A}^{\hphantom{:A}I} J_I^{jp} \gmc{A}{B}{n}{m}\sigma^B\big\rbrack}_{\mathrlap{\!=0\ \eqref{equivariance_eqn_4}}} \xi^m\big\}\omega \\
          &{} = \int_\Sigma \big\{ D_j \big\lbrack L_{:A}^{\hphantom{:A}I} J_I^{jp} \gmc{A}{B}{n}{m}\sigma^B\big\rbrack \xi^m_{,n} \omega_p - D_p D_j \big\lbrack L_{:A}^{\hphantom{:A}I} J_I^{jp} \gmc{A}{B}{n}{m}\sigma^B\big\rbrack \xi^m \omega_n\big\}
      \end{aligned}
    \end{multlined}
  \end{equation}
  The original contributions from \eqref{sem_calculation_current} together with the new contributions from \eqref{sem_calculation_2nd_deriv} containing first derivatives of $\xi$ combine to
  \begin{equation}\label{sem_calculation_1st_deriv}
    \begin{multlined}
      \int_\Sigma \big\lbrack \underbrace{L_{:A}^{\hphantom{:A}j} \gmc{A}{B}{n}{m}\sigma^B + 2 L_{:A}^{\hphantom{:A}I} J_I^{jp} \gmc{A}{B}{n}{m}\sigma^B_{,p} - L_{:A}^{\hphantom{:A}I} J_I^{jn}\sigma^A_{,m}}_{\vcentcolon=S_m^{jn}}\big\rbrack \xi^m_{,n}\omega_j \\
      \begin{aligned}
        &{} = \int_V D_j \big\lbrack S_m^{jn} \xi^m_{,n}\big\rbrack \omega \\
        &{} = \int_V \big\{ D_j S_m^{jn} \xi^m_{,n} + \underbrace{S_m^{jn} \xi^m_{,jn}}_{\mathrlap{\!=0\ \eqref{equivariance_eqn_3}}}\big\}\omega\\
        &{} = \int_V \big\{ D_n\big( D_j S_m^{jn} \xi^m\big) - \underbrace{D_n D_j S_m^{jn}}_{\mathrlap{\!=0\ \eqref{equivariance_eqn_3}}} \xi^m \big\}\omega\\
        &{} = \int_\Sigma D_j S_m^{jn} \xi^m \omega_n.
      \end{aligned}
    \end{multlined}
  \end{equation}
  Putting together \eqref{sem_calculation_current}--\eqref{sem_calculation_1st_deriv} finally gives
  \begin{equation}\label{sem_last_calc}
    \begin{aligned}
      \int_\Sigma i_\Sigma^\ast\boldsymbol{j}(\sigma) &{} =\, \int_\Sigma \big\{ L\delta^n_m - L_{:A}^{\hphantom{:A}n}\sigma^A_{,m} - 2 L_{:A}^{\hphantom{:A}I}J_I^{np} I^J_{mp} \sigma^A_{,J} + \big(L_{:A}^{\hphantom{:A}I}\sigma^B_{,I} \\
                                                        &{} \hphantom{=\, \int_\Sigma\big\{} + L_{:A}^{\hphantom{:A}j}\sigma^B_{,j} + D_j L_{:A}^{\hphantom{:A}j}\sigma^B - D_j D_p L_{:A}^{\hphantom{:A}I}J_I^{jp} \sigma^B \big)\gmc{A}{B}{n}{m} \big\} \xi^m \omega_n \\
                                                        &{} \stackrel{\mathclap{\eqref{equivariance_eqn_2}}}{=}\, \int_\Sigma \big\{{-\big\lbrack} L_{:A} - D_p L_{:A}^{\hphantom{:A}p} + D_p D_q L_{:A}^{\hphantom{:A}I} J_I^{pq} \big\rbrack \gmc{A}{B}{n}{m} \sigma^B \big\} \xi^m \omega_n,
    \end{aligned}
  \end{equation}
  from which the Gotay-Marsden stress-energy-momentum tensor density is easily read off as
  \begin{equation}\label{sem_tensor_result}
    \mathcal T^{\,n}_{\,m}(\sigma) = -\frac{\delta L}{\delta u^A} \gmc{A}{B}{n}{m} \sigma^B,
  \end{equation}
  where $\frac{\delta L}{\delta u^A} =  L_{:A} - D_p L_{:A}^{\hphantom{:A}p} + D_p D_q L_{:A}^{\hphantom{:A}I} J_I^{pq}$ denotes the variational derivative. 
\end{proof}

According to \eqref{sem_tensor_result}, the Gotay-Marsden stress-energy-momentum tensor density vanishes identically \emph{on-shell}, i.e.~for sections solving the Euler-Lagrange equations---a recurring theme in the analysis of generally covariant theories. The first-order version of Thm.~\ref{sem_thm}, proven in Ref.~\cite{Gotay_1992}, settled a long-standing debate about SEM tensors densities by providing a definition which is based on Noether theory and naturally satisfies a \emph{generalized Belinfante-Rosenfeld formula}\footnote{For general relativity, the Belinfante-Rosenfeld formula\cite{Belinfante,Rosenfeld} relates the SEM tensor density obtained from Noether theory by considering translations to the Hilbert SEM tensor density, which is defined as the source density of the Einstein equations.\cite{Gotay_1992} This comes with a seemingly \emph{ad hoc} symmetrization of the Noether SEM tensor density. The generalized Belinfante-Rosenfeld formula\cite{Gotay_1992} relates the Gotay-Marsden SEM tensor density \eqref{sem_tensor_result} to the Noether SEM tensor density without such choices, just by considering currents and spacetime diffeomorphisms.}.

In addition, the Gotay-Marsden SEM tensor density lends itself for a concise formulation of Noether's second theorem, based on a previous result \cite{Gotay_1992} for first-order theories.
\begin{theorem}[Second Noether theorem]\label{thm_second_noether}
  Consider a second-derivative-order Lagrangian with local representative $L$ on a tensor field bundle. If the Lagrangian is invariant with respect to diffeomorphisms, the corresponding Gotay-Marsden stress-energy-momentum tensor density $\mathcal T^n_m$ satisfies the differential relation
  \begin{equation}\label{second_noether}
    D_n \mathcal T^n_m = \frac{\delta L}{\delta u^A} u^A_{\hphantom Am}.
  \end{equation}
\end{theorem}
\begin{proof}
  Starting from the first expression for $\mathcal T^n_m$ obtained in Eq.~\eqref{sem_last_calc}, which is
  \begin{equation}
    \begin{aligned}
      \mathcal T^n_m = {} & L\delta^n_m - L_{:A}^{\hphantom{:A}n}u^A_{\hphantom Am} - 2 L_{:A}^{\hphantom{:A}I}J_I^{np} I^J_{mp} u^A_{\hphantom AJ} \\
      {} & + \big\lbrack L_{:A}^{\hphantom{:A}I}u^B_{\hphantom BI} + L_{:A}^{\hphantom{:A}j}u^B_{\hphantom Bj} + D_j L_{:A}^{\hphantom{:A}j}u^B - D_j D_p L_{:A}^{\hphantom{:A}I}J_I^{jp} u^B \big\rbrack\gmc{A}{B}{n}{m},
    \end{aligned}
  \end{equation}
  the identity \eqref{second_noether} follows via a direct computation of the divergence. Two terms in the intermediate result are reduced using the equivariance equations \eqref{equivariance_eqn_3} and \eqref{equivariance_eqn_4}.
\end{proof}
An expansion the Gotay-Marsden SEM tensor density in Eq.~\eqref{second_noether} reveals indeed a differential relation
\begin{equation}
  -D_n \left(\frac{\delta L}{\delta u^A}\gmc{A}{B}{n}{m}u^B\right) = \frac{\delta L}{\delta u^A} u^A_{\hphantom Am}
\end{equation}
for the \emph{Euler-Lagrange equations}, which is exactly the statement of Noether's second theorem \cite{}. The identity holds \emph{off-shell}, i.e.~for any section of the tensor bundle regardless of whether it satisfies Euler-Lagrange equations.

\section{Axiom II: causal compatibility}\label{sect_causal_compatibility}
\textit{This section follows very closely Sect.~II.B of Ref.~\cite{Alex_2020}.}

For the mathematical formulation of the second axiom, we utilize the close relation of the causal structure of field equations to the short-wavelength limit of the theory.\cite{D_ll_2018} First, we restrict to Lagrangians which are \emph{degenerate} in the sense that the Euler-Lagrange equations---although defined on $J^4E$---depend only on second derivatives and lower, i.e.
\begin{equation}
  \delta L = \pi^\ast_{2k,k}\delta \tilde L
\end{equation}
for $k=2$. This makes the theory immune from Ostrogradsky instabilities\cite{ostrogradsky}, which afflict theories of higher derivative orders. In addition, the formalism is being kept very close to Einstein gravity, whose Lagrangian is likewise degenerate---so we are still right on track in sticking closely to the established formalism and just inject different matter dynamics at the very beginning. Given the Euler-Lagrange equations $E_A=0$ (henceforth called \emph{field equations}) of the degenerate second-order theory, we enter the limit of short wavelengths by considering the Wentzel-Kramers-Brillouin (WKB) ansatz for a local section $\sigma$ of $\pi$
\begin{equation}\label{wkb_ansatz}
  \sigma^A(x^m) = \mathfrak{Re}\{ \mathrm e^\frac{\mathrm iS(x^m)}{\lambda} \lbrack a^A(x^m) + \mathcal O(\lambda)\rbrack \}.
\end{equation}
Evaluating the field equations at this ansatz and taking the limit $\lambda\rightarrow 0$ gives to leading order
\begin{equation}\label{wkb_section}
  \underbrace{\left(\frac{\partial E_A}{\partial u^B_I}\right)J^{ij}_Ik_ik_j}_{T_{AB}(k)}a^B = 0,
\end{equation}
which depends on the wavefront $S$ only via the wave covector $k=-\mathrm dS$. Eq.~\ref{wkb_section} is a linear equation for the amplitudes $a^B$ with coefficients from the $r\times r$ matrix $T_{AB}(k)$, where $r$ denotes the fibre dimension of the theory. This matrix, called the \emph{principal symbol} of the field equations, plays an important role in the short-wavelength limit: If the theory admits solutions with nontrivial amplitudes $a^B$, the principal symbol $T_{AB}(k)$ must necessarily be noninjective. By virtue of this condition, the principal symbol selects the physically admissible wave covectors in the WKB ansatz. As a square matrix is noninjective if and only if its determinant vanishes, admissible wave covectors can equivalently be characterized by a vanishing condition on the determinant of $T_{AB}(k)$.

There is, however, a problem with this approach: In the presence of gauge symmetries, there are nontrivial solutions equivalent to the trivial solution $a^B=0$. These solutions will also be contained in the kernel of the principal symbol, rendering the naive conditions on wave covectors formulated above meaningless. More specifically, assuming a gauge symmetry with $s$-dimensional gauge orbits, there are exactly $s$ independent functions $\chi_{(i)}^A(k)$ which are equivalent to the trivial solution and span an $s$-dimensional subspace of the kernel of $T_{AB}(k)$. In order to allow for solutions which are \emph{not} equivalent to the trivial solution, the kernel needs to be of dimension greater or equal than $s+1$.

In the case of a diffeomorphism invariant theory, we have $s=4$ and it follows from the equivariance equation \eqref{equivariance_eqn_4} that
\begin{equation}
  0 = T_{AB}(k) \gmc{B}{C}{n}{i} u^C k_n =\vcentcolon T_{AB}(k) \chi^B_{(i)}(k).
\end{equation}
\textbf{proof missing}

The condition that the kernel of the principal symbol be of dimension $s+1$ or higher is equivalent to imposing that the order-$s$ adjugate matrix
\begin{equation}
  Q^{(A_1\dots A_s)(B_1\dots B_s)}(k) \vcentcolon= \frac{\partial^s \operatorname{det}(T_{AB}(k))}{\partial T_{A_1B_1}(k)\dots T_{A_sB_s}(k)}
\end{equation}
vanish.\footnote{The vanishing of the order-$s$ adjugate matrix is equivalent to the vanishing of all order-$s$ subdeterminants, which are obtained by removing all possible combinations of $s$ rows and and $s$ columns from the matrix and calculating the determinant of each such reduced matrix. This is why the adjugate matrix is of dimension $\binom{r}{s} \times \binom{r}{s}$ for theories with fibre dimension $r$ and $s$-dimensional gauge symmetries.} In this situation, where we have a square $r\times r$ matrix with $s$ vectors $(\chi_{(i)})_{i=1\dots s}$ spanning a subspace of the kernel, we can use the general result\cite{Itin_2009,D_ll_2018} 
\begin{equation}
  Q^{(A_1\dots A_s)(B_1\dots B_s)}(k) = \epsilon^{\mu_1\dots\mu_s}\epsilon^{\nu_1\dots\nu_s}\left\lbrack\prod_{i=1}^s\chi_{(\mu_i)}^{A_i}\right\rbrack\left\lbrack\prod_{j=1}^s\chi_{(\nu_j)}^{B_j}\right\rbrack \mathcal P(k)
\end{equation}
to arrive at the so-called \emph{principal polynomial} $\mathcal P(k)$.
\begin{definition}[principal polynomial]\label{def_principal_polynomial}
  Consider a bundle $E\overset{\pi}{\longrightarrow}M$ with fibre dimension $r$ and a Lagrangian field theory on a jet bundle over $\pi$ that results in Euler-Lagrange equations of second derviative order. Assume the $s$ vectors $(\chi^A_{(i)}(k))_{i=1\dots s}$ to be generators of the gauge transformations of the theory. In particular, the $\chi^A_{(i)}$ span the left and right kernel of the principal symbol $T_{AB}(k)$. Choosing $s$ rows and columns of $T$ such that the order-$s$ adjugate matrix entry $Q^{(A_1\dots A_s)(B_1\dots B_s)}$ does not vanish, we define the \textbf{principal polynomial} as the quotient
  \begin{equation}\label{principal_polynomial}
    \mathcal P(k) = \frac{Q^{(A_1\dots A_s)(B_1\dots B_s)}}{\epsilon^{\mu_1\dots\mu_s}\epsilon^{\nu_1\dots\nu_s}\left\lbrack\prod_{i=1}^s\chi_{(\mu_i)}^{A_i}\right\rbrack\left\lbrack\prod_{j=1}^s\chi_{(\nu_j)}^{B_j}\right\rbrack}.
  \end{equation}
\end{definition}
The principal polynomial is a homogeneous polynomial of order $2r-4s$ in the components $k_a$ of the wave covector and has---as is clear from the derivation above---the important property that in order for an ansatz \eqref{wkb_ansatz} to describe a nontrivial solution in the short-wavelength limit the wave covector $k=-\mathrm dS$ must be a root of $\mathcal P$. Thus, the complete information about the propagation of waves in the infinite frequency limit is encoded in the principal symbol. This is an example for the more general result that the eligibility of a theory as a \emph{physically relevant} theory hinges on properties of $\mathcal P$. More specifically, it has been shown that a theory can only be \emph{predictive}, \emph{interpretable}, and \emph{quantizable} if the principal polynomial satisfies certain algebraic conditions, which further propagate to conditions on the underlying geometry.\cite{R_tzel_2011,Rivera_2012}

The principal polynomial is also closely related to the Cauchy problem of the field equations, as a Cauchy problem can only be well-posed within a region of $M$ if $\mathcal P$ restricts to a \emph{hyperbolic}\footnote{A homogeneous polynomial $\mathcal P$ of degree $d$ is hyperbolic if there exists a covector $h$ such that $\mathcal P(h)=0$ and any shifted covector $h+\lambda w$ intersects the vanishing set of $\mathcal P$ exactly $d$ times. Such a covector $h$ is said to be hyperbolic with respect to $\mathcal P$.} polynomial in this region. Furthermore, given a theory with hyperbolic principal polynomial, admissible initial data hypersurfaces are characterized by the condition that the surface normal be hyperbolic with respect to $\mathcal P$.\cite{hörmander,ivrii} Predictivity if the \emph{raison d'\^etre} for physical theories, which is why we will restrict our attention to tensor field theories with hyperbolic principal polynomials.

Two geometric objects are important for the formulation of the axiom of causal compatibility: The vanishing set $V_p\in T_p^\ast M$ of $\mathcal P$ and the set $C_p\in T_p^\ast M$ of all hyperbolic covectors with respect to $\mathcal P$. Both sets are defined at each point and thus form distributions $V$ and $C$ on $M$. The vanishing set $V_p$ consists of all admissible wave covectors in the infinite frequency limit, restricting the propagation directions of fields in spacetime. The set $C_p\in T_p^\ast M$, on the other hand, contains the information about possible choices of initial data hypersurfaces. It constitutes a convex cone\cite{garding} and is commonly called the \emph{hyperbolicity cone}\cite{R_tzel_2011,Rivera_2012}.

Let us now consider the situation where a theory for some matter field coupled to geometry has been prescribed, say on a bundle $E_\text{grav} \oplus_M J^1 E_\text{mat}$, and the principal polynomial $\mathcal P_\text{mat}$ is hyperbolic. Both distributions $V_\text{mat}$ and $C_\text{mat}$ exist and they contain all relevant information about the causality of the matter theory. The objective of covariant constructive gravity is to close the matter theory by providing a dynamical theory of the geometry, defined on the bundle $J^2E_\text{grav}$. As a result, we obtain distributions $V_\text{grav}$ and $C_\text{grav}$ of vanishing sets and hyperbolicity cones for the gravitational theory. The principle of causal compatibility between matter theory and gravitational theory now mandates following relation between both pairs of distributions.
\begin{definition}[causally compatible gravitational closure]\label{causal_compatibility_def}
  Consider two bundles $E_\text{grav}\overset{\pi_\text{grav}}{\longrightarrow}M$ and $E_\text{mat}\overset{\pi_\text{mat}}{\longrightarrow}M$ and a Lagrangian matter field theory on $E_\text{grav} \oplus_M J^1E_\text{mat}$ whose Euler-Lagrange equations are linear in the matter field. The corresponding principal polynomial $\mathcal P_\text{mat}$ shall be hyperbolic and thus defines the vanishing set distribution $V_\text{mat}$ and the hyperbolicity cone distribution $C_\text{mat}$. We say that a gravitational Lagrangian field theory on $J^2E_\text{grav}$ with Euler-Lagrange equations of second derivative order, a principal polynomial $\mathcal P_\text{grav}$, and distributions $V_\text{grav},C_\text{grav}$ is \textbf{causally compatible} with the matter field theory if
  \begin{equation}\label{axiom2_eq}
    C_\text{grav} = C_\text{mat}\quad \text{and}\quad V_\text{mat} \subseteq V_\text{grav}.
  \end{equation}
\end{definition}
The first condition immediately implies that $\mathcal P_\text{grav}$ is hyperbolic as well. Furthermore, it ensures that both theories share their initial value surfaces and allow for a unified observer definition\cite{R_tzel_2011,Rivera_2012}. As recent measurements showed with a high degree of certainty that gravitational waves propagate at the speed of light, we include the second condition of the distribution of vanishing sets into the definition of causal compatibility. It requires that wave covectors of the matter theory are possible wave covectors of the gravitational theory, but leaves open the possibility for different modes of propagation.

Before closing this section about the second axiom, a remark about its practical implications is in order. As we will see during the perturbative implementation of the covariant constructive gravity programme, the requirement of diffeomorphism invariance alone already restricts the principal polynomial of the gravitational field equations quite a lot, such that up to the third iteration of the perturbative construction procedure we will not need to enforce Eq.~\ref{axiom2_eq} explicitly---at least for our chosen example. For the nonperturbative construction of general relativity, the condition will not be needed at all. This hints at the promising possibility that the second axiom may actually weakened by the extent to which it may already be implied by the first axiom.

