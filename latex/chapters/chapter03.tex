\chapter{The construction algorithm}

\textit{Having introduced the axioms of covariant constructive gravity and cast them in precise mathematical language, we consolidate the results and state the algorithm for the construction of modified gravity Lagrangians from novel matter theories. After a discussion about practical implications in general, we proceed with sketching the application to a few examples.}

\section{General formulation}

The results obtained so far allow us to formulate a comprehensive algorithm for the construction of gravitational Lagrangians. These Lagrangians are the most general conceivable Lagrangians within our formalism that satisfy both axioms of covariant constructive gravity. All that has to be provided is a matter theory that couples to geometry and the algorithm will yield all candidates for gravitational theories that determine the so far undetermined dynamics of the gravitational field, finally giving the theory predictive power. In this sense, the task of searching for modified gravitational theories boils down to the solution of specific PDE systems of a simple form and algebraic equations concerning the causal structure of gravity.

\begin{algorithm}[H]\label{algorithm}
  \DontPrintSemicolon
  \KwData{Geometry bundle $E_\text{grav}\overset{\pi_\text{grav}}{\longrightarrow}M$, matter bundle $E_\text{mat}\overset{\pi_\text{mat}}{\longrightarrow}M$,
  Lagrangian matter field theory on $E_\text{grav} \oplus_M J^1E_\text{mat}$ with linear field equations}
  \KwResult{Most general diffeomorphism invariant and causally compatible gravitational Lagrangian field theory on $J^2E_\text{grav}$}
  compute the Gotay-Marsden coefficients \eqref{gmc_contra} for $E_\text{grav}$ \;
  set up the equivariance equations \eqref{equivariance_eqn_1}--\eqref{equivariance_eqn_4} \;
  solve the equivariance equations for the gravitational Lagrangian density $L_\text{grav}$ \;
  compute the Euler-Lagrange equations \eqref{euler_lagrange_local} corresponding to $L_\text{grav}$ \;
  restrict the gravitational theory to second-derivative-order field equations \;
  calculate the principal polynomials \eqref{principal_polynomial} $\mathcal P_\text{grav}$ and $\mathcal P_\text{mat}$ \;
  solve the causal compatibility conditions $C_\text{grav} = C_\text{mat}$ and $V_\text{grav} \subseteq V_\text{mat}$
  \caption{Gravitational closure using covariant constructive gravity}
\end{algorithm}

Let us comment on the algorithm step by step: The first step, calculating the Gotay-Marsden coefficients, is trivial. The coefficients follow from the prescribed or inherited action of diffeomorphisms on the geometry bundle. For purely covariant or contravariant tensor bundles, Prop.~\ref{prop_gmc_intertwiner} already gives the final expression. These coefficients have to be inserted into Eqns.~\ref{equivariance_eqn_1}--\ref{equivariance_eqn_4} in order to execute step 2. As a result, we obtain a system of linear, first-order partial differential equations for the Lagrangian density $L$ with coefficients that are linear in the independent variables. More precisely, the PDEs are of the form
\begin{equation}
  0 = A^j_i x^i u_{,j} + B u,
\end{equation}
$u$ is the dependent variable, $x^i$ are the independent variables and the coefficients $A^j_i$ and $B$ are constants. Conceptually, it is clear how to solve such PDEs\cite{seiler}, albeit in most cases practically infeasible, due to the sheer size of the system and the number of independent variables. However, as we will see in a later chapter, the system admits a property called \emph{involutivity}, from which we can infer strong results about solutions and derive a perturbative solution strategy.

For a known solution, it is only a matter of applying Eq.~\ref{euler_lagrange_local} to the Lagrangian in order to compute the Euler-Lagrange equations for step 4. Restricting to second-derivative-order field equations, as required by step 5, could be done now by imposing that higher-derivative-order terms vanish. In practice, however, such restrictions will be placed at an earlier stage, in order to rule out higher orders from the beginning. A similar pattern emerges for steps 6 and 7: placing restrictions on the computed entities is possible, but may be hard to enforce after the fact. So it is worth keeping this requirement in mind early on.

\section{Example: Einstein gravity}

As already outlined in the introduction, Einstein gravity can be thought of as the gravitational closure of Maxwell electrodynamics in four dimensions. This theory provides dynamics for sections $A$ in the bundle $T^\ast M$ of one-forms, parameterized with sections $g$ in the bundle $S(T^2_0M)$ of contravariant\footnote{Many descriptions regard the covariant inverse metric as fundamental. In this case, the contravariant metric tensor appearing in the Lagrangian density \eqref{maxwell_lagrangian} is the inverse of the metric field. Both descriptions will yield slightly differing intermediate results during the construction procedure, but are fundamentally equivalent.} symmetric tensors of rank two. $A$ is commonly known as the \emph{electromagnetic potential}, $g$ as the \emph{metric tensor}. The dynamics of the electromagnetic field is given by the Lagrangian density
\begin{equation}\label{lagrangian_maxwell}
  L_\text{Maxwell} = \sqrt{-\operatorname{det} g} g^{ac}g^{bd} F_{ab} F_{cd},
\end{equation}
where the electromagnetic potential enters via the field strength tensor $F = \mathrm dA$ and we write ``$\operatorname{det}g$'' for the determinant of the \emph{covariant} metric tensor, which is the inverse of the metric tensor chosen here as fundamental field.

We now collect the ingredients for the construction algorithm. The fibre dimension of the bundle $S(T^2_0M)$ is 10, such that a suitable pair of intertwiners $(I,J)$ between this bundle and the unrestricted tensor bundle $T^2M$ is given by
\begin{equation}
  \footnotesize
  \begin{gathered}
  I^{ab}_1 = \begin{pmatrix}1 & 0 & 0 & 0 \\ 0 & 0 & 0 & 0 \\ 0 & 0 & 0 & 0 \\ 0 & 0 & 0 & 0\end{pmatrix}^{ab}, I^{ab}_2 = \begin{pmatrix}0 & 1 & 0 & 0 \\ 1 & 0 & 0 & 0 \\ 0 & 0 & 0 & 0 \\ 0 & 0 & 0 & 0\end{pmatrix}^{ab}, I^{ab}_3 = \begin{pmatrix}0 & 0 & 1 & 0 \\ 0 & 0 & 0 & 0 \\ 1 & 0 & 0 & 0 \\ 0 & 0 & 0 & 0\end{pmatrix}^{ab} I^{ab}_4 = \begin{pmatrix}0 & 0 & 0 & 1 \\ 0 & 0 & 0 & 0 \\ 0 & 0 & 0 & 0 \\ 1 & 0 & 0 & 0\end{pmatrix}^{ab}, \\
  I^{ab}_5 = \begin{pmatrix}0 & 0 & 0 & 0 \\ 0 & 1 & 0 & 0 \\ 0 & 0 & 0 & 0 \\ 0 & 0 & 0 & 0\end{pmatrix}^{ab}, I^{ab}_6 = \begin{pmatrix}0 & 0 & 0 & 0 \\ 0 & 0 & 1 & 0 \\ 0 & 1 & 0 & 0 \\ 0 & 0 & 0 & 0\end{pmatrix}^{ab}, I^{ab}_7 = \begin{pmatrix}0 & 0 & 0 & 0 \\ 0 & 0 & 0 & 1 \\ 0 & 0 & 0 & 0 \\ 0 & 1 & 0 & 0\end{pmatrix}^{ab}, I^{ab}_8 = \begin{pmatrix}0 & 0 & 0 & 0 \\ 0 & 0 & 0 & 0 \\ 0 & 0 & 1 & 0 \\ 0 & 0 & 0 & 0\end{pmatrix}^{ab}, \\
  I^{ab}_9 = \begin{pmatrix}0 & 0 & 0 & 0 \\ 0 & 0 & 0 & 0 \\ 0 & 0 & 0 & 1 \\ 0 & 0 & 1 & 0\end{pmatrix}^{ab}, I^{ab}_{10} = \begin{pmatrix}0 & 0 & 0 & 0 \\ 0 & 0 & 0 & 0 \\ 0 & 0 & 0 & 0 \\ 0 & 0 & 0 & 1\end{pmatrix}^{ab},
  \end{gathered}
\end{equation}
and
\begin{equation}
  \footnotesize
  \begin{gathered}
    J_{ab}^1 = \begin{pmatrix}1 & 0 & 0 & 0 \\ 0 & 0 & 0 & 0 \\ 0 & 0 & 0 & 0 \\ 0 & 0 & 0 & 0\end{pmatrix}_{ab}, J_{ab}^2 = \begin{pmatrix}0 & \frac{1}{2} & 0 & 0 \\ \frac{1}{2} & 0 & 0 & 0 \\ 0 & 0 & 0 & 0 \\ 0 & 0 & 0 & 0\end{pmatrix}_{ab}, J_{ab}^3 = \begin{pmatrix}0 & 0 & \frac{1}{2} & 0 \\ 0 & 0 & 0 & 0 \\ \frac{1}{2} & 0 & 0 & 0 \\ 0 & 0 & 0 & 0\end{pmatrix}_{ab} J_{ab}^4 = \begin{pmatrix}0 & 0 & 0 & \frac{1}{2} \\ 0 & 0 & 0 & 0 \\ 0 & 0 & 0 & 0 \\ \frac{1}{2} & 0 & 0 & 0\end{pmatrix}_{ab}, \\
    J_{ab}^5 = \begin{pmatrix}0 & 0 & 0 & 0 \\ 0 & 1 & 0 & 0 \\ 0 & 0 & 0 & 0 \\ 0 & 0 & 0 & 0\end{pmatrix}_{ab}, J_{ab}^6 = \begin{pmatrix}0 & 0 & 0 & 0 \\ 0 & 0 & \frac{1}{2} & 0 \\ 0 & \frac{1}{2} & 0 & 0 \\ 0 & 0 & 0 & 0\end{pmatrix}_{ab}, J_{ab}^7 = \begin{pmatrix}0 & 0 & 0 & 0 \\ 0 & 0 & 0 & \frac{1}{2} \\ 0 & 0 & 0 & 0 \\ 0 & \frac{1}{2} & 0 & 0\end{pmatrix}_{ab}, J_{ab}^8 = \begin{pmatrix}0 & 0 & 0 & 0 \\ 0 & 0 & 0 & 0 \\ 0 & 0 & 1 & 0 \\ 0 & 0 & 0 & 0\end{pmatrix}_{ab}, \\
    J_{ab}^9 = \begin{pmatrix}0 & 0 & 0 & 0 \\ 0 & 0 & 0 & 0 \\ 0 & 0 & 0 & \frac{1}{2} \\ 0 & 0 & \frac{1}{2} & 0\end{pmatrix}_{ab}, J_{ab}^{10} = \begin{pmatrix}0 & 0 & 0 & 0 \\ 0 & 0 & 0 & 0 \\ 0 & 0 & 0 & 0 \\ 0 & 0 & 0 & 1\end{pmatrix}_{ab}.
  \end{gathered}
\end{equation}
The intertwiner $I$ distributes the ten degrees of freedom for a symmetric tensor of dimension 4 across the components of a generic rank-2 tensor
\begin{equation}
  I^{ab}(c_Au^A) = \begin{pmatrix} c_1 & c_2 & c_3 & c_4 \\ c_2 & c_5 & c_6 & c_7 \\ c_3 & c_6 & c_8 & c_9 \\ c_4 & c_7 & c_9 & c_{10}\end{pmatrix}^{ab},
\end{equation}
while $J$ projects such symmetrically distributed components back to the ten degrees of freedom, discarding possible antisymmetric contributions. Note that $I$ and $J$ could also be chosen such that the matrix representations coincide by using factors of $\frac{1}{\sqrt 2}$ for off-diagonal entries in both intertwiners. This has the apparent advantage that $I$ and $J$ do not need to be distinguished from each other. However, a disadvantage of using them interchangeably is that this would obscure the different roles that $I$ and $J$ play, especially if they are used not only in setting up the equivariance equations, but also for manipulating them. The irrational coefficients like $\frac{1}{\sqrt 2}$ would also further complicate the computer-aided treatment introduced in Chap.~\ref{chapter_perturbation}, which for purely rational intertwiners yields purely rational results.

Prop.~\ref{prop_gmc_intertwiner} yields the Gotay-Marsden coefficients from $(I,J)$ as
\begin{equation}
  \gmc{A}{B}{n}{m} = 2 I^{pn}_B J^A_{pm}.
\end{equation}
Contracting with $I$ and $J$ yields the spacetime expression
\begin{equation}
  \gmc{ab}{cd}{n}{m} = 2 \delta^{(a}_m \delta^{b)}_{(c} \delta^n_{d)},
\end{equation}
which serves as a good sanity check: Contracting again with a metric $g$ and the derivatives of a vector field $\xi$ results in the well-known transformation of $g$ w.r.t.~infinitesimal diffeomorphism generated by $\xi$,
\begin{equation}
  \gmc{ab}{cd}{n}{m} g^{cd} \xi^m_n = 2g^{n(a} \xi^{b)}_{,n}.
\end{equation}

The second ingredient is the principal polynomial of electrodynamics, which reduces to\footnote{Computing the principal polynomial may lead to a result of higher degree than \eqref{maxwell_poly}. For the second axiom of covariant constructive gravity, however, only the reduced form without repeating factors is of relevance---because the causal structure is determined by the reduced polynomial alone\cite{sergio}.} the homogeneous quadratic polynomial\cite{sergio?}
\begin{equation}\label{maxwell_poly}
  \mathcal P_\text{Maxwell}(k) = g(k,k).
\end{equation}
From this result follows the standard notion of causality in relativity: Light rays with codirection $k$ are constrained to the vanishing set $V$ and, thus, satisfy $g(k,k)=0$. The wave covectors related to massive observers lie within the hyperbolicity cone $C$, which restricts them to $g(k,k) > 0$ (admitting the \emph{mostly minus} convention $(+---)$ for the signature of the metric). For more details, we refer the reader to the theory developed in Ref.~\cite{sergio,closure paper,?} and the corresponding examples.

Before proceeding, let us emphasize that there are only two things needed from the matter theory (which is Maxwell electrodynamics in this case):
\begin{enumerate}
  \item the Gotay-Marsden coefficients $\gmc{A}{B}{n}{m} = 2 I_B^{pn} J^A_{pm}$ and
  \item the principal polynomial $\mathcal P_\text{Maxwell}(k) = g(k,k)$.
\end{enumerate}

The equivariance equations \eqref{equivariance_eqn_1}--\eqref{equivariance_eqn_4} for the metric gravitational Lagrangian are a system of 140 PDEs for one variable dependent on 154 independent variables\footnote{The dimension of the second jet bundle over the metric bundle is $4+10+4\times 10 + \binom{4}{2}\times 10=154$.}. Because the system admits the aforementioned property called involutivity, which will play a major role and therefore will actually be proven in Chap.~\ref{chapter_perturbation}, we can make use of a very strong result about the solutions of this system\cite{seiler}: There are $154-140=14$ functions $\psi_\alpha$ of the independent variables, such that any solution of the homogeneous system
\begin{equation}
  0 = A^{Ij} u_{,j}
\end{equation}
is of the form $f(\psi_1,\dots,\psi_{14})$ for any (suitably differentiable) function $f$. Any particular solution $\omega$ of the inhomogeneous system
\begin{equation}
  0 = A^{Ij} u_{,j} + B^I
\end{equation}
yields, by virtue of the product rule, the general form of a solution,
\begin{equation}\label{general_metric_solution}
  u = \omega\cdot f(\psi_1,\dots,\psi_{14}).
\end{equation}

Now, the dynamics of general relativity as derived by Einstein are given by the manifestly diffeomorphism equivariant Einstein-Hilbert Lagrangian density
\begin{equation}\label{einstein_hilbert}
  L_\text{Einstein-Hilbert} = \frac{1}{2\kappa} \sqrt{-\operatorname{det}g}(R - 2\Lambda),
\end{equation}
from which we readily recognise two solutions,
\begin{equation}
  \omega = \sqrt{-\operatorname{det}g}\quad\text{and}\quad \psi_1 = R.
\end{equation}
The constants $\kappa$ and $\Lambda$ are known as gravitational constant and cosmological constant, respectively, and $R$ is the Ricci scalar curvature. Together with the homogeneous solution $\psi_1=R$, the remaining 13 homogeneous solutions $\psi_2,\dots,\psi_{14}$ are known in the literature as the fourteen \emph{curvature invariants}\cite{curvature_invariants}.

While the system of equivariance equations alone admits a multitude of solutions \eqref{general_metric_solution}, it has been shown by Lovelock\cite{} that only Einstein general relativity \eqref{einstein_hilbert} admits second-order-derivative field equations. Step 5 of the construction algorithm therefore restricts the gravitational theory closing Maxwell electrodynamics to general relativity with its two undetermined constants \emph{exactly}. The causality conditions do not have to be implemented anymore, since they follow trivially---the causal structures of Maxwell electrodynamics and Einstein gravity coincide\cite{??}.

There is another interesting result that follows from the equivariance equations. Restricting to the zeroth jet bundle and switching from abstract indices to indices inherited from the tangent bundle, the equations are
\begin{equation}\label{metric_order_0_eq}
  0 = \omega_{,m}\quad\text{and}\quad 0 = 2\frac{\partial\omega}{\partial g^{am}} g^{an} + \delta^n_m \omega.
\end{equation}
If we solve the first equation by restricting further to $\omega=\omega(g^{ab})$ and manipulate the second equation by contraction with the covariant metric, we obtain
\begin{equation}
  \frac{\partial\omega}{\partial g^{ab}} = -\frac{1}{2} g_{ab} \omega.
\end{equation}
This equation is obviously symmetric in the indices and therefore boils down to a system of 10 PDEs for the function $\omega$ of the 10 independent variables $g^{ab}$. As the system is completely determined, the known solution $\omega=\sqrt{-\operatorname{det}g}$, which can be easily verified by straightforward differentiation, is the \emph{unique} solution. Using our framework, we thus have provided a derivation of the well-known fact that the only scalar densities that can be constructed from the metric tensor are powers of the metric determinant.

The same result holds for the equivariance equations restricted to the first jet bundle, which are
\begin{subequations}
  \begin{align}
    0 &{} = L_{,m}, \label{metric_order_1_eq_1}\\
    0 &{} = 2 L_{:am}g^{an} + 2 L_{:am}^{\hphantom{:am}p}g^{an}_{,p} - L_{:ab}^{\hphantom{:ab}n}g^{ab}_{,m}, \label{metric_order_1_eq_2}\\
    0 &{} = L_{:am}^{\hphantom{:am}(p}g^{n)a}. \label{metric_order_1_eq_3}
  \end{align}
\end{subequations}
Equation \ref{metric_order_1_eq_3} is a system of 40 individual equations for the 40 derivatives of $L$ with respect to the first derivatives of the metric tensor. The rank of this subsystem is full\footnote{If in doubt, such statements concerning our linear PDE systems can be verified without much computational effort by evaluation at randomly chosen points in the jet bundle. At worst, the rank at such points will be \emph{less} than at a generic point.}, which completely eliminates any possible dependency of $L$ on the first derivatives of $g$. The remaining system is equivalent to the zeroth-order system \eqref{metric_order_0_eq} with the unique solution $L=\sqrt{-\operatorname{det}g}$, demonstrating with a very quick derivation that there is no non-trivial diffeomorphism equivariant Lagrangian density of first derivative order for the metric tensor.

Before proceeding with the next example, it should that the insights gained about metric gravitational theory compatible with Maxwell electrodynamics are not new as fare as the \emph{results} are concerned. Rather, we have seen how the developed framework readily reproduces the known results without much effort and yet again confirms earlier derivations.

\section{Example: area metric gravity}

As a first example for a modified theory of gravity that follows from covariant constructive gravity, we consider \emph{area metric gravity}. The starting point is a generalization of Maxwell electrodynamics.
\begin{definition}[generalized linear electrodynamics]
  Let $M$ be a four-dimensional spacetime manifold. The bundle $E_\text{area}$, constructed as a subbundle of $T^4_0M$ by imposing the linear conditions
  \begin{equation}
    G^{abcd} = G^{cdab} = -G^{bacd}
  \end{equation}
  on the tensor components, is called the area metric bundle. Given a scalar density $\omega$ of weight 1, sections $G$ of this bundle serve as coefficients for the Lagrangian density of generalized linear electrodynamics (GLED),
  \begin{equation}\label{lagrangian_gled}
    L_\text{GLED} = \omega_G G^{abcd} F_{ab} F_{cd}.
  \end{equation}
\end{definition}
It is easy to see that GLED is a generalization of Maxwell electrodynamics by setting\footnote{Note that $\omega_G$ as defined above is a valid scalar density of weight 1 not only for this special choice of $G$, but also for general area metric fields. Without loss of generality, we will keep making use of this density---any other density is obtained by multiplication of $\omega_G$ with a scalar. See Sect.~\ref{??} for details.}
\begin{equation}\label{metric_induced_area}
  G^{abcd} = g^{ac} g^{bd} - g^{ad} g^{bc} + \frac{1}{\sqrt{-\operatorname{det}g}} \epsilon^{abcd}\quad\text{and}\quad \omega_G = \left(\frac{1}{24}\epsilon_{abcd}G^{abcd}\right)^{-1}
\end{equation}
in the GLED Lagrangian density \eqref{lagrangian_gled}, which reproduces the Maxwell Lagrangian density \eqref{lagrangian_maxwell}. Not restricting the area metric field to the specific form \eqref{metric_induced_area} but leaving all 21 independent components unconstrained yields, of course, a more general theory.

GLED as generalization of Maxwell electrodynamics is the result of an axiomatic approach to classical eletrodynamics called \emph{premetric electrodynamics}\cite{hehl}. This approach makes a few assumptions like conversation of charge and magnetic flux, the existence of a Lorentz force law, and a superposition principle. As a consequence, the indeterminates of such a theory are reduced to the so-called \emph{constitutive tensor} $\chi$, which is already known from electrodynamics in media, but now also determines the behaviour of the electromagnetic field in \emph{in vacuo}. In our language, $\chi$ is the area metric $G$.

The causality of GLED crucially depends on the area metric via the principle polynomial\cite{}
\begin{equation}
  \mathcal P_\text{GLED}(k) = -\frac{1}{24} \omega_G^2 \epsilon_{mnpq} \epsilon_{rstu} G^{mnra} G^{bpsc} G^{dqtu} k_a k_b k_c k_d,
\end{equation}
which is generally irreducible and of rank 4. Consequently, the null surfaces are no longer metric light cones, but more complex quartic surfaces. For example, $\mathcal P_\text{GLED}$ could factor into the product of two metrics, in which case the vanishing set at a point would be the union of two metric light cones with different opening angles. In this example, the phase velocity of a wave depends on the light cone in which the wave covector lies. The two options can be seen as new polarization degree of freedom, such that the speed of light is determined by the polarization---an effect commonly known as \emph{birefringence}. While in classical electrodynamics this is only possible in nonlinear media, GLED allows for birefringence \emph{in vacuo}.

Much of the remainder of this thesis is dedicated to the application of the construction algorithm to GLED, which should yield the gravitational theory completing general linear electrodynamics to a predictive theory of matter \emph{and} gravity. This new theory shall bear the name \emph{area metric gravity}.

We again start with the definition of suitable intertwiners. It is often useful to interpret the components $G^{abcd}$ of an area metric, which consist of two antisymmetric pairs and is symmetric in these pairs, as symmetric 6 by 6 matrix
\begin{equation}\label{petrov}
  G^{[ab][cd]} =
  \begin{pmatrix}
    G^{0101} & G^{0102} & G^{0103} & G^{0112} & G^{0113} & G^{0123} \\
    \cdot & G^{0202} & G^{0203} & G^{0212} & G^{0213} & G^{0223} \\
    \cdot & \cdot & G^{0303} & G^{0312} & G^{0313} & G^{0323} \\
    \cdot & \cdot & \cdot & G^{1212} & G^{1213} & G^{1223} \\
    \cdot & \cdot & \cdot & \cdot & G^{1313} & G^{1323} \\
    \cdot & \cdot & \cdot & \cdot & \cdot & G^{2323}
  \end{pmatrix}^{[ab][cd]}.
\end{equation}
Intertwiners can then be chosen such that $I$ distributes abstract components $G^1,\dots,G^{21}$ over such a matrix, i.e.
\begin{equation}
  I^{[ab][cd]}_1 = \begin{pmatrix} 1 & 0 & 0 & 0 & 0 & 0 \\ 0 & 0 & 0 & 0 & 0 & 0 \\ 0 & 0 & 0 & 0 & 0 & 0 \\ 0 & 0 & 0 & 0 & 0 & 0 \\ 0 & 0 & 0 & 0 & 0 & 0 \\ 0 & 0 & 0 & 0 & 0 & 0 \end{pmatrix}^{[ab][cd]},\quad
  I^{[ab][cd]}_2 = \begin{pmatrix} 0 & 1 & 0 & 0 & 0 & 0 \\ 1 & 0 & 0 & 0 & 0 & 0 \\ 0 & 0 & 0 & 0 & 0 & 0 \\ 0 & 0 & 0 & 0 & 0 & 0 \\ 0 & 0 & 0 & 0 & 0 & 0 \\ 0 & 0 & 0 & 0 & 0 & 0 \end{pmatrix}^{[ab][cd]},
\end{equation}
and so on. The surjections $J$ project back to abstract indices, where the multiplicities are either 4 for components like $G^{0123}$ or 8 for components like $G^{0101}$:
\begin{equation}
  J_{[ab][cd]}^1 = \begin{pmatrix} \frac{1}{4} & 0 & 0 & 0 & 0 & 0 \\ 0 & 0 & 0 & 0 & 0 & 0 \\ 0 & 0 & 0 & 0 & 0 & 0 \\ 0 & 0 & 0 & 0 & 0 & 0 \\ 0 & 0 & 0 & 0 & 0 & 0 \\ 0 & 0 & 0 & 0 & 0 & 0 \end{pmatrix}_{[ab][cd]},\quad
  J_{[ab][cd]}^2 = \begin{pmatrix} 0 & \frac{1}{8} & 0 & 0 & 0 & 0 \\ \frac{1}{8} & 0 & 0 & 0 & 0 & 0 \\ 0 & 0 & 0 & 0 & 0 & 0 \\ 0 & 0 & 0 & 0 & 0 & 0 \\ 0 & 0 & 0 & 0 & 0 & 0 \\ 0 & 0 & 0 & 0 & 0 & 0 \end{pmatrix}_{[ab][cd]},
\end{equation}
et cetera.

As usual, the Gotay-Marsden coefficients follow from Prop.~\ref{prop_gmc_intertwiner}. Since the fibre dimension of $T^4_0M$ is 4 and the tensors are purely contravariant, the coefficients are
\begin{equation}
  \gmc{A}{B}{n}{m} = 4 I^{pqrn}_B J^A_{pqrm},
\end{equation}
ore, using the spacetime representation,
\begin{equation}
\gmc{abcd}{efgh}{n}{m} = 4 \delta^{[a\mid}_{\vphantom{f}m} \delta^{n}_{[e} \delta^{\mid b]}_{f]} \delta^{[c}_{[g} \delta^{d]}_{h]} \Bigg\rvert_{[ab] \leftrightarrow [cd]\atop [ef] \leftrightarrow [gh]},
\end{equation}
where $e_{XY} \big\rvert_{X \leftrightarrow Y} = \frac{1}{2} e_{XY} + \frac{1}{2} e_{YX}$ denotes idempotent symmetrization of the expression $e$ in $X$ and $Y$.

Having computed the intertwiners and Gotay-Marsden coefficients, the equivariance equations \eqref{equivariance_eqn_1}--\eqref{equivariance_eqn_4} are ready to be set up. Since the second area metric jet bundle is of dimension
\begin{equation}
  \operatorname{dim}(J^2E_\text{area}) = 4 + 21 + 4\times 21 + \binom{4}{2}\times 21 = 319,
\end{equation}
the system of equivariance equations consists of 140 linear, first-order PDE for one function of 319 independent variables. The claim of covariant constructive gravity is that solutions to this system are candidates for gravitational Lagrangians. Unfortunately, it is computationally infeasible to present such a solution\footnote{Strictly speaking, two solutions are known: The scalar density $\omega_G$ defined in Eq.~\ref{metric_induced_area} and a different choice for $\omega_G$ which is computed from the determinant of the $6\times 6$ matrix \eqref{petrov}, which is also a valid scalar density. However, these solutions do not depend on derivatives of the area metric and as such would not yield \emph{dynamic} field equations if used as Lagrangian density.} and, unlike for metric gravity, there are no known curvature invariants for us to rely on.

Therefore, we will resort to perturbation theory in order to derive results for weak gravitational fields in Chap.~\ref{chapter_weak_area} and also shortly explore the possibility of directly solving the cosmological sector of area metric gravity in Chap.~\ref{chapter_cosmo}.

\section{Example: bimetric gravity}

A lot of work has already been done in order to answer the question: \emph{What would gravity look like if there were two metrics instead of one?} From the perspective of covariant constructive gravity (and gravitational closure in general), this question is meaningless without reference to a bimetric matter action. The question should rather be: \emph{How can matter theories that couple to two different metrics be completed by a bimetric gravitational theory?}

An example for such a matter theory is the bimetric Klein-Gordon theory for two scalar fields.
\begin{definition}[bimetric Klein-Gordon theory]
  Let $M$ be a four-dimensional spacetime manifold. The bundle $E_\text{bimetric} = S(T^2_0M) \oplus S(T^2_0M)$ constructed as the direct sum of two metric bundles is called the bimetric bundle. Section $(g,h)$ of this bundle serve as coefficients for the Lagrangian density of the bimetric Klein-Gordon theory
  \begin{equation}
    L_\text{2KG} = \sqrt{-\operatorname{det}g}g^{ab} \phi_{,a}\phi_{,b} - m_\phi^2 \phi^2 + \sqrt{-\operatorname{det}h}h^{ab} \psi_{,a}\psi_{,b} - m_\psi^2 \psi^2,
  \end{equation}
  where $\phi$ and $\psi$ are scalar fields, i.e.~smooth functions $M\rightarrow \mathbb R$, with nonnegative masses $m_\phi$ and $m_\psi$.
\end{definition}


\begin{itemize}
\item what is bimetric gravity?
\item principal polynomial(s)
\item Gotay-Marsden coefficients, equivariance equations
\end{itemize}


