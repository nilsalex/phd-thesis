\chapter{Covariant constructive gravity}

\textbf{take-home message: two conditions on gravitational dynamics determined by coupled matter theory}

\section{Lagrangian field theory}

\begin{definition}{(Lagrangian)}
  Let $M$ be a smooth manifold of dimension $n$ and $E \overset{\pi}{\longrightarrow} M$ a smooth fibre bundle over $M$ with typical fibre $F$. A \textbf{Lagrangian} $\mathcal L$ of order $k$ is an element
  \begin{equation}
    \mathcal L \in \Lambda^n_0(\pi_k),
  \end{equation}
the space of horizontal $n$-forms on the $k$-th-order jet bundle $\pi_k$ of $\pi$.

  Assuming $M$ to be orientable with volume form $\Omega\in\Lambda^n(\tau_M^\ast)$, a Lagrangian $\mathcal L$ is equivalently characterized by its \textbf{Lagrangian density} $L\in C^\infty(J^k\pi)$,
\begin{equation}
  \mathcal L = L\pi_k^\ast\Omega.
\end{equation}
\end{definition}

From now on, we will consider smooth, orientable base manifolds $M$ and smooth bundles over the base manifold. Depending on the context, the symbol $\Omega$ will denote either the form on $M$ or the pullback to various bundles over $M$.

\begin{definition}{(local action functional)}
  Given a Lagrangian density $\mathcal L \in \Lambda^n_0(\pi_k)$ and a compact $n$-dimensional submanifold $C \subset M$, the \textbf{local action functional} is defined as the map
  \begin{equation}
    \phi \mapsto S\lbrack\phi\rbrack = \int_C (j^k\phi)^\ast\mathcal L
  \end{equation}
for all local sections $\phi$ of $\pi$ with support on $C$.
\end{definition}

Lagrangian field theory now stipulates that sections $\phi\in\Gamma(\pi)$ are physical if they are stationary points of the action functional. A well-known result from the calculus of variations is the following:

\begin{proposition}
  Let $L\in C^\infty(J^k\pi)$ be a Lagrangian density. Let $C$ be a compact submanifold of $M$ and $\phi$ be a local section of $\pi$ such that the local action functional $S\lbrack\phi\rbrack$ is defined. If $\phi$ is an extremal of $S$, it satisfies the Euler-Lagrange equations
  \begin{equation}
    (j^{2k}\phi)^\ast\left( \sum_{l=0}^{k} (-1)^l D_{i_1} \cdots D_{i_l} \frac{\partial L}{\partial u^A_{i_1\cdots i_l}}\right) = 0.
  \end{equation}
\end{proposition}

\begin{proposition}
  Given a Lagrangian $\mathcal L = L\Omega \in \Lambda^n_0(\pi_k)$, there exists an $n+1$ form $\Theta_{\mathcal L}\in\Lambda^{n+1}_1(\pi_k)$, such that, globally, the variation of the Lagrangian is given by
  \begin{equation}\label{global_lagrange_form}
    \delta L = \pi^\ast_{2k,k} \left( dL\wedge\Omega\right) + d_h \Theta_{\mathcal L}
  \end{equation}
  Such a form $\Theta_{\mathcal L}$ is called a \textbf{Cartan form}.
\end{proposition}

A coordinate expression for $\Theta_{\mathcal L}$ is
\begin{equation}\label{local_cartan_form}
  \Theta_{\mathcal L} = L\Omega + \sum_{s=0}^{k-1} \sum_{l=0}^{k-s-1} (-1)^l D_{i_1} \cdots D_{i_l} \left(\frac{\partial L}{\partial u^A_{j i_1 \dots i_{l} p_1 \dots p_s}}\right) \psi^A_{p_1\dots p_s} \wedge \left(i_{\partial_j}\Omega\right).
\end{equation}
Straight-forward application of (\ref{global_lagrange_form}) to (\ref{local_cartan_form}) yields the well-known coordinate expression
\begin{equation}
  \delta L = \left( \sum_{l=0}^{k} (-1)^l D_{i_1} \cdots D_{i_l} \frac{\partial L}{\partial u^A_{i_1\cdots i_l}}\right) u^A \wedge \Omega
\end{equation}
for the Euler-Lagrange form.

\begin{itemize}
\item Lagrangian as differential form on a jet bundle
\item variational problem
\item cartan form
\item noether theorem
\item energy momentum tensor
\end{itemize}

\section{Axiom I: diffeomorphism invariance}
\begin{itemize}
\item justification, history, general relativity
\item geometric formulation
\item equivariance equations
\end{itemize}

\section{Axiom II: causal compatibility}
\begin{itemize}
\item derivation (consistent co-evolution)
\item principal symbol, principal polynomial
\item geometric formulation
\end{itemize}

\section{Relation to canonical constructive gravity}
\begin{itemize}
\item relation between diffeo invariance and constraint algebra
\item the role of causality in the constraint algebra
\end{itemize}

\section{Example: Einstein gravity}
\begin{itemize}
\item Gotay-Marsden coefficients, equivariance equations
\item GR as solution
\item causality is already solved
\end{itemize}

\section{Example: area metric gravity}
\begin{itemize}
\item introduction to area metric gravity
\item principal polynomial
\item Gotay-Marsden coefficients, equivariance equations
\end{itemize}

\section{Example: bimetric gravity}
\begin{itemize}
\item what is bimetric gravity?
\item principal polynomial(s)
\item Gotay-Marsden coefficients, equivariance equations
\end{itemize}

