\chapter{Contributions}

Where not stated otherwise, this thesis contains original research. Parts have been researched in close collaboration with Tobias Reinhart and published as Ref.\ \cite{Alex_2020}, building up on previous results \cite{Reinhart_2018}.

Some of the calculations involved in Chap.~\ref{chapter_weak_area} have already been performed for Ref.\ \cite{Alex_2019}. The remainder of this chapter, excluding the results on radiation loss, is published as Ref.\ \cite{Alex_2020_2}, albeit in a more compact form.

Many calculations rely on heavy use of computer algebra. Two Haskell packages have been developed specifically with this purpose in mind: \texttt{sparse-tensor} \cite{Reinhart_2019_sparse-tensor}, which originates from joint development with Tobias Reinhart, and \texttt{safe-tensor} \cite{Alex_2020_safe-tensor}. Haskell code for the example in Chap.~\ref{chapter_weak_area} that makes use of these packages is available as Ref.\ \cite{Alex_2020_area-metric-gravity}.

Complete sections paraphrasing content of Refs.~\cite{Alex_2020} and \cite{Alex_2020_2} are introduced as such and typeset in \textit{italic style}. They may, however, contain additional, previously unpublished details.

\noindent\rule{\textwidth}{0.4pt}

\cite{Alex_2020}~\fullcite{Alex_2020}

\cite{Reinhart_2018}~\fullcite{Reinhart_2018}

\cite{Alex_2019}~\fullcite{Alex_2019}

\cite{Alex_2020_2}~\fullcite{Alex_2020_2}

\cite{Reinhart_2019_sparse-tensor}~\fullcite{Reinhart_2019_sparse-tensor}

\cite{Alex_2020_safe-tensor}~\fullcite{Alex_2020_safe-tensor}

\cite{Alex_2020_area-metric-gravity}~\fullcite{Alex_2020_area-metric-gravity}

