\chapter{Conclusions}

In this thesis, the concept of covariant constructive gravity has been put on a solid mathematical footing. Lagrangian field theory on jet bundles turned out to be ideally suited for the definition of the general covariance axiom. The equivariance equations that follow from this axiom transform the implementation into a computational task, opening up the \emph{constructive} pathway towards modified theories of gravity. Using the Cartan form that corresponds to a diffeomorphism invariant Lagrangian density, we have seen how general covariance implies a version of the first and second Noether theorem---important results that have proven very useful further down the line. Within this framework, the axiom of causal compatibility has been formulated in terms of additional algebraic conditions on the gravitational field equations.

From the equivariance equations and causality conditions, we derived a concise algorithm which guides the construction of novel gravitational theories that implement general covariance and are causally compatible to a given matter theory. Because it can be seen as generalisation of Lovelock's uniqueness theorem for Einstein gravity \cite{Lovelock_1969,Lovelock_1971,Lovelock_1972}, we could show that this construction procedure applied to Maxwell's electrodynamics indeed reproduces metric gravity as derived by Einstein.

Of course, covariant constructive gravity would not be that interesting if it were just another tool that reproduces Einstein gravity. Its \emph{raison d'\^etre} is the derivation of \emph{modified} theories of gravity that complete novel matter theories to predictive models of the universe. The remainder of the thesis was dedicated towards achieving this goal. First, we have discussed three examples of novel matter theories: birefringent electrodynamics and two bimetric theories. While it is straightforward to set up the construction procedure and derive general results concerning the solution space, \emph{finding} these solutions in practice is notoriously hard and turned out not to be feasible for the examples in question.

Therefore, we investigated possibilities to arrive at results that are valid in certain specific settings, without the need to know the ``full'' solutions. Our main strategy was perturbation theory, which seeks to find solutions that are valid for small deviations of the gravitational field from a Lorentz invariant background. With a corresponding perturbation ansatz, the equivariance equations transform into a system of linear equations for the expansion coefficients. As a consequence of the Lorentz invariance of the background geometry, the expansion coefficients themselves are Lorentz invariant, which reduces their dimensionality a lot---before solving any equivariance equation.

Many of the computations that are necessary in order to construct Lorentz invariant ansätze and solve the perturbative equivariance equations have been delegated to the computer. For this purpose, two Haskell packages have been developed and presented in this thesis---with a focus on the package \texttt{sparse-tensor} which composes and solves the equivariance equations. Methods from functional programming lend themselves for an efficient and safe implementation of tensor algebra, enabling us to repeat and modify calculations whenever required, without having to redo them by hand.

Chapter \ref{chapter_weak_area} was the culmination of this thesis, where we put all pieces together and derived perturbative area metric gravity up to third perturbation order in the Lagrangian density. From this Lagrangian, the linearised gravitational field equations and their second perturbation order follows. Quite remarkably, the linearised field equations coincide with the equations derived in the canonical framework \cite{Schneider_2017,Alex_2019}---with an important caveat: the field equations in the canonical picture as obtained by solving the canonical closure equations \cite{D_ll_2018,Schneider_2017} are \emph{not} causally compatible with the matter theory, i.e.\ their principal polynomial does not reduce to the Minkowski metric.

In order to cure the causality, one of the \emph{eleven} gravitational constants had to be fixed \cite{Alex_2019}, reducing their number to \emph{ten}, which then equals the number of constants obtained in the covariant framework. The reason for this mismatch is believed to lie in the so-called ansatz equations, which enforce Lorentz invariance of the perturbation ansatz. In perturbative covariant constructive gravity, these are already solved by considering only Lorentz invariant ansätze to begin with. Canonical constructive gravity, on the other hand, makes ansätze \emph{after} performing the 3+1 split---effectively implementing a spatial $\mathrm{SO}(3)$ symmetry. But this is a \emph{weaker} requirement than the spatiotemporal $\mathrm{SO}(1,3)$ symmetry that follows from the equivariance equations. In the case of the linearised field equations with causality mismatch, not all of the ansatz equations seem to be solved---one condition on the gravitational constants is not yet implemented. To remedy this, one has to find the equivalent of the ansatz equations in the canonical picture by prolonging and projecting the PDE or otherwise ensure Lorentz invariance of the ansätze.

The comparison demonstrates that the presence of matter causality in the canonical constraint algebra is not responsible for the causality of the gravitational theory. For the linearised field equations\footnote{But also for the second-order equations, as shown in Sect.~\ref{section_area_construction}.}, diffeomorphism invariance actually constrains the gravity causality. Whether diffeomorphism invariance is enforced by imposing it directly on the Lagrangian density or by requiring the canonical constraint algebra to implement the hypersurface deformation algebra---in whichever frame---is secondary.

The ability to reason about the canonical closure programme using insights from covariant constructive gravity shows how both approaches complement each other. Comparing the results of covariant and canonical constructive gravity on the one hand increases the confidence, as they are so similar, but also provides impulses for improvements: the canonical approach should embrace Lorentz invariance and also reconsider its claims concerning causality, while the covariant approach could benefit from a canonical formulation. The multisymplectic framework based on the Cartan form \cite{Gotay_1991} seems ideal for this task.

Building up on the third-order area metric Lagrangian, we inspected the binary star with circular orbits, one of the simplest conceivable systems. Thanks to this simplicity, however, it was possible to derive second-order effects that proved to be quite rich. A binary star in area metric gravity emits massive gravitational waves---in addition to the radiation already known from Einstein gravity. These new modes of radiation have the potential to induce novel deformation patterns in test matter distributions and to alter the spin-up behaviour of the binary star. For the massless modes of radiation that are also observed in metric gravity, we made use of the second Noether theorem and obtained a quantitative description of how a binary star is expected to decrease its orbital period as it emits gravitational waves. Fig.~\ref{figure_spinup} shows a few exemplary cases, which demonstrate the deviations from Einstein gravity that are expected in area metric gravity.

These results should be understood as \emph{conceptual}, because much more work would be needed for the prediction of the outcome of high-precision experiments. The ambition of this thesis was to demonstrate \emph{in principle} the predictive power of covariant constructive gravity. Starting from a modification of Maxwell's electrodynamics---by allowing for birefringence in vacuum---it is possible to derive a compatible theory of gravity that prescribes the dynamics of the new geometry used by the such refined matter theory. The resulting gravitational theory has a limit where it corresponds to Einstein gravity, but it also allows for interesting deviations: massive gravitational waves that are emitted from a binary star which exceeds a certain angular velocity threshold, a modification of Kepler's third law, or a refined inspiral curve.

We finally explored the possibility of making similar predictions for symmetry-reduced theories---proposing an approach that meets the minimal requirement of reproducing metric cosmology. It will be exciting to see the application to novel matter theories.

The famous words by John Archibald Wheeler quoted at the beginning of Chapter \ref{chapter_introduction} seem to apply not only at the level of field equations---where matter fields source gravitational fields, while gravitational fields determine the motion of matter fields---but also at the level of \emph{theories}. The gravitational field equations themselves are, to a certain degree, determined by the dynamics of matter fields. Considering novel matter theories that couple to nonmetric geometries has gravitational implications, which covariant constructive gravity is able to quantify. Improving the predictions in order to make the constructed theories testable in practice should be at the centre of upcoming research. The standard model of particle physics and general relativity are not the definite models of the universe---covariant constructive gravity can further the search for alternatives.
