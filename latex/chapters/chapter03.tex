\chapter{The construction algorithm}

\textit{Having introduced the axioms of covariant constructive gravity and cast them in precise mathematical language, we consolidate the results and state the algorithm for the construction of modified gravity Lagrangians from novel matter theories. After a discussion about practical implications in general, we proceed with sketching the application to a few examples.}

\section{General formulation}

The results obtained so far allow us to formulate a comprehensive algorithm for the construction of gravitational Lagrangians. These Lagrangians are the most general conceivable Lagrangians within our formalism that satisfy both axioms of covariant constructive gravity. All that has to be provided is a matter theory that couples to geometry and the algorithm will yield all candidates for gravitational theories that determine the so far undetermined dynamics of the gravitational field, finally giving the theory predictive power. In this sense, the task of searching for modified gravitational theories boils down to the solution of specific PDE systems of a simple form and algebraic equations concerning the causal structure of gravity.

\begin{algorithm}[H]\label{algorithm}
  \DontPrintSemicolon
  \KwData{Geometry bundle $E_\text{grav}\overset{\pi_\text{grav}}{\longrightarrow}M$, matter bundle $E_\text{mat}\overset{\pi_\text{mat}}{\longrightarrow}M$,
  Lagrangian matter field theory on $E_\text{grav} \oplus_M J^1E_\text{mat}$ with linear field equations}
  \KwResult{Most general diffeomorphism invariant and causally compatible gravitational Lagrangian field theory on $J^2E_\text{grav}$}
  compute the Gotay-Marsden coefficients \eqref{gmc_contra} for $E_\text{grav}$ \;
  set up the equivariance equations \eqref{equivariance_eqn_1}--\eqref{equivariance_eqn_4} \;
  solve the equivariance equations for the gravitational Lagrangian density $L_\text{grav}$ \;
  compute the Euler-Lagrange equations \eqref{euler_lagrange_local} corresponding to $L_\text{grav}$ \;
  restrict the gravitational theory to second-derivative-order field equations \;
  calculate the principal polynomials \eqref{principal_polynomial} $\mathcal P_\text{grav}$ and $\mathcal P_\text{mat}$ \;
  solve the causal compatibility conditions $C_\text{grav} = C_\text{mat}$ and $V_\text{grav} \subseteq V_\text{mat}$
  \caption{Gravitational closure using covariant constructive gravity}
\end{algorithm}

Let us comment on the algorithm step by step: The first step, calculating the Gotay-Marsden coefficients, is trivial. The coefficients follow from the prescribed or inherited action of diffeomorphisms on the geometry bundle. For purely covariant or contravariant tensor bundles, Prop.~\ref{prop_gmc_intertwiner} already gives the final expression. These coefficients have to be inserted into Eqns.~\ref{equivariance_eqn_1}--\ref{equivariance_eqn_4} in order to execute step 2. As a result, we obtain a system of linear, first-order partial differential equations for the Lagrangian density $L$ with coefficients that are linear in the independent variables. More precisely, the PDEs are of the form
\begin{equation}
  0 = A^j_i x^i u_{,j} + B u,
\end{equation}
$u$ is the dependent variable, $x^i$ are the independent variables and the coefficients $A^j_i$ and $B$ are constants. Conceptually, it is clear how to solve such PDEs\cite{seiler}, albeit in most cases practically infeasible, due to the sheer size of the system and the number of independent variables. However, as we will see in a later chapter, the system admits a property called \emph{involutivity}, from which we can infer strong results about solutions and derive a perturbative solution strategy.

For a known solution, it is only a matter of applying Eq.~\ref{euler_lagrange_local} to the Lagrangian in order to compute the Euler-Lagrange equations for step 4. Restricting to second-derivative-order field equations, as required by step 5, could be done now by imposing that higher-derivative-order terms vanish. In practice, however, such restrictions will be placed at an earlier stage, in order to rule out higher orders from the beginning. A similar pattern emerges for steps 6 and 7: placing restrictions on the computed entities is possible, but may be hard to enforce after the fact. So it is worth keeping this requirement in mind early on.

\section{Example: Einstein gravity}
\begin{itemize}
\item Gotay-Marsden coefficients, equivariance equations
\item GR as solution
\item causality is already solved
\end{itemize}

\section{Example: area metric gravity}
\begin{itemize}
\item introduction to area metric gravity
\item principal polynomial
\item Gotay-Marsden coefficients, equivariance equations
\end{itemize}

\section{Example: bimetric gravity}
\begin{itemize}
\item what is bimetric gravity?
\item principal polynomial(s)
\item Gotay-Marsden coefficients, equivariance equations
\end{itemize}

